\paragraph{Diagrammi UML}\mbox{}\\
Con l'obiettivo di rendere chiare le soluzioni progettuali utilizzate, è necessario l'utilizzo di {diagrammi UML}\ped{G}. Quest'ultimi devono essere realizzati utilizzando lo {standard}\ped{G} 2.0.

È richiesta la realizzazione di:
\begin{itemize}
	\item[•] \textbf{Diagrammi delle attività}: descrivono un processo o un algoritmo;
	\item[•] \textbf{Diagrammi dei package}: per raggruppare elementi e fornire un {namespace}\ped{G} per gli elementi raggruppati;
	\item[•] \textbf{Diagrammi di sequenza}: rappresentano una sequenza di processi o funzioni;
	\item[•] \textbf{Diagrammi di classi}: rappresentano le classi utilizzate e le loro relazioni.
\end{itemize}
In caso vengano utilizzati dei {Design Pattern}\ped{G} sarà necessario accompagnarli con una descrizione ed un diagramma UML.

\subsubsection{Progettazione}\mbox{}\\
Dopo aver terminato il periodo di Analisi si passerà a quella di Progettazione che vede come protagonisti i 
\textit{Progettisti}, durante la quale questi devono trovare una soluzione soddisfacente al problema, 
definire un'{architettura}\ped{G} logica e i vari diagrammi che la rappresentano.
La Progettazione permette di: 
\begin{itemize}
\item[•] Ottimizzare l'uso delle risorse;
\item[•] Garantire la qualità del prodotto sviluppato;
\item[•] Suddividere il problema principale in tanti sotto problemi di complessità minore.
\end{itemize}

\paragraph{Architettura logica}\mbox{}\\
Bisogna definire un'architettura logica del prodotto che dovrà: 
\begin{itemize}
\item[•] Soddisfare i requisiti definiti nel documento di \AdR;
\item[•] Essere sicura in caso di malfunzionamenti o intrusioni;
\item[•] Essere {modulare}\ped{G} e formato da componenti riutilizzabili;
\item[•] Essere affidabile;
\item[•] Essere comprensibile per future manutenzioni.
\end{itemize}

\paragraph{Design Pattern}\mbox{}\\
I \textit{Progettisti} devono utilizzare il design pattern che ritengono più adatto al contesto per rendere l'applicazione più sicura ed efficiente possibile.
Ogni utilizzo di design pattern deve essere brevemente descritto ed accompagnato da un diagramma UML posto nella directory padre dei file sorgenti.

\subsubsection{Qualità}

\paragraph{Classificazione dei test}\mbox{}\\
I test implementati devono essere classificati secondo la seguente notazione:
\begin{center}
	T[Tipologia Test][Codice identificativo]
\end{center}
dove:
\begin{itemize}
	\item[•] \textbf{Tipologia test}: indica il tipo di test e può assumere i seguenti valori:
	\begin{list}{$\circ$}{}
		\item U: per i test di unità;
		\item I: per i test di integrazione;
		\item S: per i test di sistema;
	\end{list}
	\item[•] \textbf{Codice identificativo}: indica il codice numerico univoco del test.
\end{itemize}

\paragraph{Test di unità}\mbox{}\\
Devono essere definiti dei testi di unità necessari a garantire che tutte le componenti del sistema funzionino correttamente.

\paragraph{Test di integrazione}\mbox{}\\
Devono essere definite le classi di verifica necessarie a garantire che tutte le componenti del sistema funzionino correttamente.

\paragraph{Test di sistema}\mbox{}\\
Richiede che i vari componenti del sistema vengano integrati al fine di garantire che tutte le componenti del sistema funzionino correttamente.

\subsubsection{Codifica}
In questa sezione vengono descritte le norme che i \textit{Programmatori} devono seguire con l'obiettivo di scrivere codice leggibile, affidabile e manutenibile.

\paragraph{Linguaggi e Framework}\mbox{}\\
Lo sviluppo del progetto didattico richiede l'uso di diversi framwork e linguaggi di programmazione, ognuno mirato ad uno scopo preciso:
\begin{itemize}
	\item \textbf{JavaScript:} viene adottato Javascript alla specifica ES6 per lo sviluppo di gran parte della backend e della frontend;
	\item \textbf{React:} viene adottato il framework React 16.8 per lo sviluppo della frontend;
	\item \textbf{Java:} viene adottato Java con le specifiche del JDK 11 per lo sviluppo del modulo client per la connessione al server Freeling;
	\item \textbf{Spring:} viene adottato il framework Spring per facilitare lo sviluppo del modulo sopracitato, avvalendosi solo del modulo RESTful.
\end{itemize}

\paragraph{Stile di codifica}\mbox{}\\
Al fine di produrre codice uniforme, leggibile e manutenibile è richiesto che vengano rispettate
le seguenti convenzioni:
\begin{itemize}
\item[•] I nomi utilizzati devono essere chiari, descrittivi rispetto alla loro funzione e in inglese;
\item[•] Evitare nomi troppo simili tra loro che possano creare difficoltà nella comprensione del codice;
\item[•] Deve essere presente almeno un breve commento descrittivo per ogni classe e metodo;
\item[•] I commenti devono essere scritti in lingua inglese senza utilizzare abbreviazioni o altre ambiguità;
\item[•] Le modifiche al codice devono sempre riflettersi sui relativi commenti;
\item[•] Evitare commenti superflui, inappropriati o scurrili;
\item[•] Ogni file deve presentare un’intestazione con le seguenti informazioni:
\begin{list}{$\circ$}{}
\item Percorso e nome del file;
\item Nome e cognome dell’autore;
\item Data di creazione;
\item Breve descrizione del contenuto del file.
\end{list}
\end{itemize}
Il codice Java deve seguire le linee guida per lo stile proposte da Google reperibili all'indirizzo:
\begin{center}
\url{https://github.com/airbnb/javascript};
\end{center}
Il codice Javascript deve seguire le linee guida per lo stile proposte da Airbnb reperibili all'indirizzo:
\begin{center}
	\url{http://google.github.io/styleguide/};
\end{center}


\subsubsection{Strumenti di supporto}
\paragraph{RQConnect}\mbox{}\\ \label{sec:Trac}
Il tracciamento dei requisiti e dei casi d’uso avviene attraverso la piattaforma 
\textit{RQConnect}, installata su un server {Firebase}\ped{G} ad hoc per il gruppo \gruppo.
 È  un tool sviluppato da un componente del gruppo in occasione di questo progetto e 
 liberamente disponibile su GitHub.\\ È possibile aggiungere requisiti e casi d’uso 
 alla piattaforma e successivamente collegarli tra di loro, la schermata principale è 
 divisa in due colonne, in quella a sinistra c’è l’elenco dei requisiti e in quella a 
 destra i casi d’uso, cliccando su un elemento si apre una vista dettagliata nella quale 
 è possibile leggere i dettagli e collegare l’elemento, cliccando sul pulsante risolvi si 
 apre una finestra dalla quale è possibile selezionare gli elementi da collegare con l’aiuto 
 di un menu a tendina.\\Una volta completato l’inserimento ed il collegamento è possibile scaricare 
 l’intera lista in tabella \LaTeX.

\paragraph{Visual studio code}\mbox{}\\
L'{IDE}\ped{G} scelto per lo sviluppo è {Visual studio code}\ped{G}, versione: 1.31 disponibile al seguente link:\\
\begin{center}
	\url{https://code.visualstudio.com/Download};
\end{center}
Esso è stato scelto principalmente perchè gratis, include supporto per debugging e un controllo Git integrato oltre al {Syntax highlighting}\ped{G}, {IntelliSense}\ped{G}, {Snippet}\ped{G} e {code refactoring}\ped{G}.\\
Si farà riferimento ai seguenti linguaggi, piattaforme, librerie e framework:
\begin{itemize}
	\item[•]{JavaScript}\ped{G} 1.8.5;
	\item[•]{HTML5}\ped{G};
	\item[•]{Node.js}\ped{G} 10.15.1;
	\item[•]{React}\ped{G} 16.8.1;
	\item[•]{Bootstrap}\ped{G} 4.3.0.
\end{itemize}

I plugin utilizzati sono:
\begin{itemize}
	\item \textbf{Prettier - Code formatter:} Per la formattazione automatica del codice.
	Per settare la formattazione automatica al salvataggio del file si deve:
	\begin{itemize}
		\item Premere CMD + Shift + P;
		\item Andare su Open User Settings;
		\item Andare in Text Editor > Formatting;
		\item Spuntare la casella Format on save.
	\end{itemize}
	\item \textbf{Simple React Snippet:} Per facilitare la scrittura del codice con semplici comandi come:
	\begin{itemize}
		\item textbf{imrc:} Per l'import;
		\item textbf{cc:} Per creare le classi;
		
	\end{itemize}
\end{itemize}

\paragraph{Docker}\mbox{}\\
Lo strumento scelto per contenere la libreria di pos-tagging con il relativo server {TCP/IP}\ped{G} è Docker. Esso permette di inserire l'applicativo scritto in {C++}\ped{G} all'interno di contenitori software, fornendo un'astrazione aggiuntiva grazie alla virtualizzazione a livello di sistema operativo di Linux.
Per installare Docker nella versione 18.09, è necessario installare Docker Desktop in Windows 10 Pro o Enterprise e in MacOS:
\begin{center}
\url{https://www.docker.com/products/docker-desktop}
\end{center}
Mentre in Ubuntu Linux è consigliato seguire la seguente guida: 
\begin{flushleft}
	\url{https://www.digitalocean.com/community/tutorials/how-to-install-and-use-docker-on-ubuntu-18-04}
\end{flushleft}
Il Dockerfile contiene la configurazione per la compilazione e l'installazione dell'applicativo. 
Per generare l'immagine è necessario eseguire il seguente comando:
\begin{center}
	\texttt{\# docker build --tag=freeling:alpha .}
\end{center}
Successivamente quest'ultima potrà essere eseguita con il comando:	
\begin{center}
	\texttt{\# docker run -it --rm -p 50005:50005 freelingserver:alpha analyze -f es.cfg --server -p 50005}
\end{center}
L'istruzione crea un'istanza dell'immagine che è in attesa di richieste sulla porta 50005 che analizza frasi scritte in lingua spagnola.
Per permettere di analizzare frasi in più lingue creare istanze diverse con il comando precedente (docker run) sostituendo \texttt{es.cfg} con la lingua d'interesse, ad esempio: \texttt{it.cfg}. Alcune delle lingue disponibili sono: 
\begin{itemize}
	\item \textbf{en.cfg}: predispone il sistema ad analizzare frasi in inglese;
	\item \textbf{it.cfg}: predispone il sistema ad analizzare frasi in italiano;
	\item \textbf{es.cfg}: predispone il sistema ad analizzare frasi in spagnolo;
	\item \textbf{fr.cfg}: predispone il sistema ad analizzare frasi in francese;
	\item \textbf{de.cfg}: predispone il sistema ad
	analizzare frasi in tedesco.
\end{itemize}

\paragraph{Eclipse}\mbox{}\\
Per la parte di sviluppo in Java, il gruppo \gruppo \ ha deciso di adottare l'IDE Eclipse per le seguenti ragioni:
\begin{itemize}
	\item \'E gratuito;
	\item Gran parte dei membri del gruppo ne hanno già conoscenza;
	\item Integrazione con Maven;
	\item \'E multipiattaforma.
	\item Estensibilità tramite plugin per lo sviluppo con Spring.
\end{itemize}
Eclipse è reperibile al seguente link:\newline
\begin{center}
	\url{https://www.eclipse.org/downloads/}
\end{center}

\paragraph{Node.js}\mbox{}\\
Per lo sviluppo lato server in linguaggio Javascript ci si avvale dell'ultima versione Long Term Support (LTS) di Node.js, che, al momento della stesura di questo documento, è la 10.15.1 LTS.
Node.js è reperibile al seguente link:
\begin{center}
	\url{https://nodejs.org/it/}
\end{center} 
Il file package.json contiene tutte le configurazioni e dipendenze del progetto. Per installare tutti i moduli necessari è necessario eseguire il seguente comando nella cartella contenente il file package.json:
\begin{center}
	\texttt{npm install}
\end{center}
Per eseguire il progetto è necessario usare il comando:
\begin{center}
	\texttt{npm start}
\end{center}

\paragraph{Apache Tomcat}\mbox{}\\
Per l'esecuzione e il deploy sul server del modulo client Freeling scritto in Java, viene usato Apache Tomcat, che permette l'esecuzione di codice Java lato server. La configurazione di apache tomcat viene gestita dal framework Spring.
Maggiori informazioni sono reperibili al seguente link:\newline
\begin{center}
	\url{http://tomcat.apache.org/}
\end{center}

\paragraph{Firebase}\mbox{}\\
Lo storage dei dati raccolti dal prodotto finale avviene sul servizio Firebase di Google. La scelta è stata obbligata, in quanto quest'ultima tecnologia è espressamente richiesta da \proponente .\newline
Maggiori informazioni sono reperibili al seguente link:\newline
\begin{center}
	\url{https://firebase.google.com}
\end{center}

\paragraph{Checkstyle}\mbox{}\\
La conformità dello stile di codifica Java con le linee guida stabilite è garantita dal plugin Maven Checkstyle, che fa fallire la build in automatico in caso di errori nella forma del codice scritto. Il controllo sullo stile può essere eseguito anche manualmente con il seguente comando:
\begin{center}
	\texttt{mvn verify}
\end{center}


