\section{Lista di controllo}

Durante l’applicazione del {walkthrough}\ped{G} ai documenti, sono qui riportati gli errori più
frequenti. Per migliorare l'efficienza e l'efficacia da parte dei \textit{Verificatori} è opportuno che i controlli si basino sui seguenti punti:

\begin{itemize}
	\item[•] \textbf{Lingua italiana}:
	        \begin{list}{$\circ$}{}
	        	\item La prima parola di una voce dell’elenco puntato inizia con la lettera maiuscola;
	        	\item La voce finale dell’elenco puntato non termina con il punto;
	        	\item Una voce non finale dell’elenco puntato non termina con il punto e virgola.
	        \end{list}
	\item[•] \textbf{\LaTeX{}}:
	        \begin{list}{$\circ$}{}
	        	\item Date e ore scritte non rispettano il formato stabilito;
	        	\item Mancato utilizzo dei comandi personalizzati;
	        	\item Utilizzo scorretto delle parentesi graffe dopo i comandi \LaTeX{};
	        	\item Mancato aggiornamento dell’intestazione del documento dopo una modifica;
	        	\item Link e riferimenti non funzionanti o assenti.
	        \end{list}
	\item[•] \textbf{UML}:
	        \begin{list}{$\circ$}{}
	        	\item Casi d’uso non proporzionati correttamente tra loro;
	        \end{list}
	\item[•] \textbf{Glossario}:
			\begin{list}{$\circ$}{}
				\item Termini segnati con il comando \verb|{Parola}\ped{G}| ed  impropriamente non presenti nel Glossario;
				\item Mancata segnatura di termini presenti nel Glossario.
			\end{list}
	\item[•] \textbf{Nomi dei documenti}:
			\begin{list}{$\circ$}{}
				\item Mancata indicazione della versione di riferimento di un documento.
			\end{list}
\end{itemize}