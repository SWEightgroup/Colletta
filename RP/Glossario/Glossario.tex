\documentclass[a4paper, oneside, openany, dvipsnames, table]{article}
\usepackage{../template/SWEightStyle}
\usepackage{hyperref}
\newcommand{\Titolo}{Manuale Utente}

\newcommand{\Gruppo}{SWEight}

\newcommand{\Approvatore}{Damien Ciagola}
\newcommand{\Redattori}{Alberto Bacco \newline Sebastiano Caccaro \newline Gheorghe Isachi \newline Gionata Legrottaglie}
\newcommand{\Verificatori}{Francesco Corti \newline Francesco Magarotto}

\newcommand{\pathimg}{../template/img/logoSWEight.png}

\newcommand{\Versionedoc}{1.0.0}

\newcommand{\Distribuzione}{\proponente \newline Prof. Vardanega Tullio \newline Prof. Cardin Riccardo \newline Gruppo SWEight}

\newcommand{\Uso}{Esterno}

\newcommand{\NomeProgetto}{Colletta}

\newcommand{\Mail}{SWEightGroup@gmail.com}

\newcommand{\DescrizioneDoc}{Questo documento si occupa di fornire le modalità di utilizzo del software Colletta commissionato}

\renewcommand\thesection{}

\begin{document}
\copertina{}

\definecolor{greySWEight}{RGB}{255, 71, 87}
\definecolor{greyROwSWEight}{RGB}{234, 234, 234}

\section*{Registro delle modifiche}
{
	\rowcolors{2}{greyROwSWEight}{white}
	\renewcommand{\arraystretch}{1.5}
	\centering
	\begin{longtable}{ c c C{4cm}  c  c }
		
		\rowcolor{greySWEight}
		\textcolor{white}{\textbf{Versione}} & \textcolor{white}{\textbf{Data}} & \textcolor{white}{\textbf{Descrizione}} & \textcolor{white}{\textbf{Nominativo}} & \textcolor{white}{\textbf{Ruolo}}\\
		1.2.2 & 2019-02-25 & Ampliamento sezione 5.4 e 3.2.5.2 & Alberto Bacco & \reda{} \\
		
		1.2.1 & 2019-02-23 & Aggiunta sezione 3.2.5.8 Checkstyle & Sebastiano Caccaro & \reda{} \\		
		
		1.2.0 & 2019-02-20 & Aggiunta scelte tecnologiche 3.2.4.2, da 3.4.5.4 a 3.4.5.7, 4.3.1.4, 4.3.2.2, 4.4.6 e figlie & Sebastiano Caccaro & \reda{} \\	
		
		1.1.5 & 2019-02-20 & Modifica sezione 2 & Alberto Bacco & \reda{} \\
		
		1.1.4 & 2019-02-18 & Correzione errori grammatica, spostate sottosezioni di asana da 4.3 a 5.2, & Alberto Bacco & \reda{} \\
		
		1.1.3 & 2019-02-14 & Riorganizzazione e correzione errori sezione 5 & Enrico Muraro & \reda{} \\
		
		1.1.2 & 2019-02-03 & Modifica sottosezione 4.1.10, 4.3.1.4, 4.3.1.5, 4.3.1.6 & Alberto Bacco& \reda{} \\	
		
		1.1.1 & 2019-01-31 & Modifica struttura e contenuti sezione 3  & Damien Ciagola & \reda{} \\	
		
		1.1.0 & 2019-01-27 & Sezione Qualità 4.2 & Sebastiano Caccaro & \reda{} \\	
		
		1.0.1 & 2019-01-25 & Parziale ristrutturazione della struttura del documento & Sebastiano Caccaro & \reda{} \\		
		
		1.0.0 & 2019-01-11 & Approvazione per il rilascio & Sebastiano Caccaro & \Res{} \\
		
		0.9.0 & 2019-01-9 & Verifica finale & Francesco Corti & \ver{} \\
		
		0.9.0 & 2019-01-8 & Aggiunta lista di controllo & Gionata Legrottaglie & \reda{} \\
		
		0.8.0 & 2018-12-23 & Correzioni errori ortografici & Gionata Legrottaglie & \reda{} \\
		
		0.7.0 & 2018-12-20 & Verifica documento & Francesco Corti & \ver{}\\
		
		0.6.0 & 2018-12-18 & Aggiunta sottosezione 5.2.2.2, 5.2.2.3, 5.2.2.4 & Francesco Magarotto & \reda{} \\
		
		0.5.2 & 2018-12-16 & Modifica sezione 4.1.5.3 & Alberto Bacco & \reda{} \\
		
		0.5.2 & 2018-12-16 & Modifica sezione 4.1.5.3 & Alberto Bacco & \reda{} \\
		
		0.5.2 & 2018-12-16 & Aggiunte sottosezioni  & Alberto Bacco & \reda{} \\
		
		0.5.1 & 2018-12-15 & Aggiunte sottosezioni 5.3, 5.4, 5.5, 5.6, 5.7, 5.8 & Alberto Bacco & \reda{} \\
		
		0.5.0 & 2018-12-15 & Aggiunta sezione 5 e sottosezioni 5.1, 5.2 & Gionata Legrottaglie & \reda{} \\
		
		0.4.1 & 2018-12-11 & Aggiunta sezione 4.1.7.3.1 & Francesco Magarotto & \reda{} \\ 
		
		0.4.0 & 2018-12-10 & Aggiunte sottosezioni 4.1.5, 4.1.6, 4.1.7, 4.1.8 & Gionata Legrottaglie & \reda{} \\ 
		0.4.0 & 2018-12-09 & Aggiunta sezione 4 e sottosezioni 4.1.1, 4.1.2, 4.1.3, 4.1.4 & Gionata Legrottaglie & \reda{} \\ 
		
		0.3.1 & 2018-12-07 & Aggiunta sottosezione 3.2 & Gionata Legrottaglie & \reda{} \\ 
		
		0.3.0 & 2018-12-06 & Aggiunta sezione 3 e sottosezione 3.1 & Gionata Legrottaglie & \reda{} \\ 
		
		0.2.0 & 2018-12-05 & Aggiunti i riferimenti & Gionata Legrottaglie & \reda{} \\ 
		
		0.1.0 & 2018-11-30 & Aggiunta introduzione & Gionata Legrottaglie & \reda{} \\
		
		0.0.1 & 2018-11-28 & Creazione scheletro del documento & Gionata Legrottaglie & \reda{}\\
		
	\end{longtable}

}
\newpage
\tableofcontents
\newpage


\newpage
\section{A}  
\textbf{Ambiente cloud}:\\	 Indica un paradigma di erogazione di servizi offerti on demand da un fornitore ad un cliente finale attraverso la rete Internet come: l'archiviazione, l'elaborazione o la trasmissione dati, a partire da un insieme di risorse preesistenti, configurabili e disponibili in remoto.

\textbf{Apple MacOS}:\\ Nome del sistemo operativo sviluppato da Apple Inc.

\label{par:appr_auto}
\textbf{Apprendimento automatico}:\\	Rappresenta un insieme di metodi che vengono utilizzati statisticamente per migliorare progressivamente la performance di un algoritmo nell'identificare pattern nei dati. Questi metodi  permettono ad un elaboratore di apprendere nel tempo come cambiare il suo comportamento in base alla elaborazione di dati.

\textbf{Architettura}: \\ Nell'hardware si intende l'insieme dei criteri di progetto in base ai quali è progettato e realizzato un computer, oppure un dispositivo facente parte di esso. Nel software è l'organizzazione fondamentale di un sistema, definita dai suoi componenti, dalle relazioni reciproche tra i componenti e con l'ambiente, e i principi che ne governano la progettazione e l'evoluzione.

\textbf{Attività}:\\ Lavoro, esplicazione di lavoro, di energia (anche non materiale) da parte di singoli o di gruppi.


\newpage
\section{B}
\textbf{Bot}:\\	Programma che accede alla rete attraverso lo stesso tipo di canali utilizzati dagli utenti, per automatizzare i compiti che risultano gravosi o complessi per gli utenti umani, per esempio accedere alle pagine web, inviare messaggi in una chat e  muoversi nei videogiochi.
 
\textbf{Bootstrap}:\\	\'E una raccolta di strumenti liberi per la creazione di siti e applicazioni per il Web. Essa contiene modelli di progettazione basati su HTML e CSS,
 sia per la tipografia, che per le varie componenti dell'interfaccia, come moduli, pulsanti e navigazione, così come alcune estensioni opzionali di JavaScript.

\textbf{Branch}:\\ Comanda di Git utilizzato per l’implementazione di funzionalità tra loro isolate, cioè sviluppate in modo indipendente l’una dall’altra ma a partire dalla medesima radice.


\textbf{Broker}:\\ Organizza e gestisce i messaggi arrivati dai vari applicativi.

\textbf{Bug}:\\	Identifica un errore nella scrittura del codice sorgente, può essere prodotto dal compilatore di un software. Questo porta a comportamenti anomali o non previsti del programma. Meno comunemente, il termine bug può indicare un difetto di progettazione in un componente hardware, che ne causa un comportamento imprevisto o comunque diverso da quello specificato dal produttore.


\newpage
\section{C}
\textbf{C++}:\\	\'E un linguaggio di programmazione ad alto livello, orientato agli oggetti, con tipizzazione statica. È stato sviluppato (in origine col nome di "C con classi") da Bjarne Stroustrup ai Bell Labs nel 1983 come un miglioramento del linguaggio C tramite l'introduzione del paradigma di programmazione a oggetti, funzioni virtuali, overloading degli operatori, ereditarietà multipla, template e gestione delle eccezioni.Il linguaggio venne standardizzato nel 1998 (ISO/IEC 14882:1998 "Information Technology - Programming Languages - C++", aggiornato nel 2003). C++11, conosciuto anche come C++0x, è il nuovo standard per il linguaggio di programmazione C++ che sostituisce la revisione del 2003. Dopo una revisione minore nel 2014 (C++14), l'ultima versione dello standard (nota informalmente come C++17) è stata pubblicata nel 2017.

\textbf{Capabilty}:\\ \'E un concetto utilizzato nella sicurezza informatica ed è uno dei modelli di sicurezza esistenti. Una capability (conosciuta anche come chiave) è un token di autorità comunicabile e non falsificabile. Essa consiste in un valore che fa riferimento ad un oggetto insieme a una collezione di diritti di accesso. 

\textbf{Capitolato}:\\	Documento con lo scopo di facilitare e velocizzare la comprensione del contratto d'appalto. Esso descrive specifiche tecniche, caratteristiche generali e modalità di realizzazione degli intenti del committente.

\textbf{Caso d'uso}:\\	Tecnica usata nei processi di ingegneria del software per effettuare in maniera esaustiva e non ambigua, la raccolta dei requisiti al fine di produrre software di qualità. Essa consiste nel valutare ogni requisito focalizzandosi sugli attori che interagiscono col sistema, valutandone le varie interazioni.
 
\textbf{Card}:\\ Visualizazzione grafica della issue che può essere spostata tra le varie sezioni sulla bacheca del progetto per avere un quadro generale della situazione.

\textbf{Client}:\\	Applicativo che risiede sull'elaboratore dell'utente e che serve per richiedere dati, risorse o elaborazioni da parte di un computer centrale, che prende il nome di server.

\textbf{Code refactoring}:\\	"Tecnica strutturata per modificare la struttura interna di porzioni di codice senza modificarne il comportamento esterno", applicata per migliorare alcune caratteristiche non funzionali del software. I vantaggi che il refactoring persegue riguardano in genere un miglioramento della leggibilità, della manutenibilità, della riusabilità e dell'estendibilità del codice e la riduzione della sua complessità, eventualmente attraverso l'introduzione a posteriori di design pattern.
Il refactoring è un elemento importante delle principali metodologie emergenti di sviluppo del software (soprattutto object-oriented): per esempio delle metodologie agili, dell'extreme programming, e del test driven development.

\textbf{Codice numerico gerarchico}:\\  Indica la struttura organizzativa usata per avere tracciamento immediato delle varie sottocategorie che la compongono, ed  arrivare  facilmente al vertice della gerarchia.

\textbf{Commit}:\\ \'E un comando del software Git usato per salvare cambiamenti effettuati nella repository locale. Per includere i cambiamenti effettuati nel commit bisogna esplicitamente aggiungere i file aggiornati, modificati o creati, che si vogliono aggiungere alla repository.


\textbf{Committente}:\\	Chi ordina ad altri l’esecuzione di un lavoro, di una prestazione, o l’acquisto di una merce per conto proprio. \'E il soggetto titolare del potere decisionale e di spesa relativo alla gestione dell'appalto.

\textbf{Conflitto}:\\
In Git, si presenta quando due o più cambiamenti concorrenti sono stati applicati sulla stessa linea di codice, o quando una persona sta lavorando su un file che è stato eliminato.

\textbf{Connettori}:\\ Micro-funzioni che aiutano l'utente a personalizzare la sua routine.

\textbf{Consumer}:\\ L'utilizzatore di servizi prodotti dal sistema, in questo caso i messaggi creati dal producer.

\textbf{Convenzione}:\\	Accordo collettivo volto a determinare le caratteristiche che deve contenere il software finale.

\textbf{Crossplatform}:\\	 Software per computer implementato su più piattaforme di elaborazione. Il software Cross-platform può essere diviso in due tipi; uno richiede la costruzione o la compilazione individuale per ogni piattaforma che supporta e l'altro può essere eseguito direttamente su qualsiasi piattaforma senza preparazione speciale

\textbf{Cruscotto}:\\ Vedi Dashboard.

\textbf{Custom}:\\	Realizzato su misura in base alle necessità o alle funzioni specifiche che è destinato a soddisfare.	


\newpage
\section{D}
\textbf{Dashboard}:	\\ Una schermata di gestione e monitoraggio dei dati a disposizione.

\textbf{Diagramma di Gantt}:\\	Serve a pianificare un insieme di attività in un certo periodo di tempo. La struttura è organizzata in un piano cartesiano in cui nelle ascisse si dispone la scala temporale dall’inizio alla fine del progetto, e nelle ordinate le cose da fare per portare a termine il progetto. Il tempo necessario per svolgere un compito è rappresentato visivamente sul diagramma con una barra colorata che va dalla data di inizio alla data di fine dell’attività.

\textbf{Design Pattern}: 
Si tratta di una descrizione o modello logico da applicare per la risoluzione di un problema che può presentarsi in diverse situazioni durante le fasi di progettazione e sviluppo del software

\textbf{Diagramma UML}:\\ Schema grafico basato su UML (Unified Modeling Language) con lo scopo di rappresentare visivamente un sistema insieme ai suoi attori principali, ruoli, azioni, artefatti o classi, al fine di comprendere, alterare, mantenere o documentare meglio le informazioni riguardo al sistema.

\textbf{Distribuzione Linux}:\\	\'E una distribuzione software del sistema operativo Linux, realizzata a partire dal kernel Linux, un sistema di base GNU e solitamente anche diversi altri applicativi (talvolta anch'essi parte di GNU). Tali distribuzioni appartengono quindi alla sotto-famiglia dei sistemi operativi GNU e, più in generale, alla famiglia dei sistemi Unix-like, perché ispirati a Unix e in certa misura compatibili con esso.


\newpage
\section{E}
\textbf{Efficacia}:\\ In diritto, idoneità di un atto a produrre gli effetti che conseguono al suo compimento, e dunque capacità dell'atto di rendere effettive le finalità perseguite dalle parti con la sua valida conclusione. 

\textbf{Efficienza}:\\ Capacità di raggiungere gli obiettivi fornendo adeguate prestazioni e minimizzando le risorse.

\newpage
\section{F}
\textbf{Firebase}:\\
Piattaforma di sviluppo di applicazioni web e mobili sviluppata da Firebase Inc. nel 2011, poi acquisita da Google nel 2014. A partire da ottobre 2018, la piattaforma Firebase ha 18 prodotti, utilizzati da 1,5 milioni di app.

\textbf{Flusso}:\\
{[}Flusso dati{]} Dati generati in modo continuo da diverse origini sottoposti ad un monitoraggio continuo da parte di un elaboratore.

\textbf{Fonte}:\\ Provenienza da cui giungono i dati. Essa può essere fisica, come ad esempio un Insegnante o un Allievo, oppure un sistema automatico come la libreria di PoS-tagging.

\textbf{FreeLing}:\\ Libreria C++ che fornisce funzionalità di analisi del linguaggio (analisi morfologica, rilevamento di entità con nome, PoS-tagging, parsing, etichettatura di ruolo semantica, ecc.) per una varietà di lingue.

\newpage
\section{G}

\textbf{Gamification}: \\  \'E l'utilizzo di elementi mutuati dai giochi e delle tecniche di game design in contesti esterni ai giochi.

\textbf{Gerarchia}:\\ Sistema asimmetrico di graduazione e organizzazione delle cose di tipo piramidale.  
Come nel caso di "è il capo di...", "è parte di...", o "è meglio di...". Tali relazioni sono "asimmetriche" ovvero funzionano "in un senso" ma non funzionano "nell'altro".

\textbf{Git}:\\	 Software di controllo versione distribuito utilizzabile da interfaccia a riga di comando o GUI, creato da Linus Torvalds nel 2005.

\textbf{Gulpease}: \\  L'indice Gulpease è un indice di leggibilità di un testo tarato sulla lingua italiana. Rispetto ad altri ha il vantaggio di utilizzare la lunghezza delle parole in lettere anziché in sillabe, semplificandone il calcolo automatico.

\newpage
\section{H}
\textbf{HTML5}:\\	\'E linguaggio di markup nato per la formattazione e impaginazione di documenti ipertestuali disponibili nel web 1.0, oggi è utilizzato principalmente per il disaccoppiamento della struttura logica di una pagina web (definita appunto dal markup)
 e la sua rappresentazione, gestita tramite gli stili CSS per adattarsi alle nuove esigenze di comunicazione e pubblicazione all'interno di Internet. La versione 5 è stata rilasciata dal W3C Recommendation nell'ottobre 2014.

\textbf{Hardware}:\\	\'E la parte fisica di un computer, ovvero tutte quelle parti elettroniche, elettriche, meccaniche, magnetiche, ottiche che ne consentono il funzionamento. Più in generale il termine si riferisce a qualsiasi componente fisica di una periferica o di una apparecchiatura elettronica, ivi comprese le strutture di rete. L'insieme di tali componenti è anche detto componentistica.

\newpage
\section{I}
\textbf{IDE}:\\	Ambiente di sviluppo integrato (in lingua inglese Integrated Development Environment, anche Integrated Design Environment o Integrated Debugging Environment, rispettivamente ambiente integrato di progettazione e ambiente integrato di debugging), in informatica, è un software che, in fase di programmazione, aiuta i programmatori nello sviluppo del codice sorgente di un programma. Spesso l'IDE aiuta lo sviluppatore segnalando errori di sintassi del codice direttamente in fase di scrittura, oltre a tutta una serie di strumenti e funzionalità di supporto alla fase di sviluppo e debugging.

\textbf{IntelliSense}:\\	\'E una forma di completamento automatico resa popolare da Visual Studio Integrated Development Environment.
 Serve inoltre come documentazione per i nomi delle variabili, delle funzioni e dei metodi usando metadati e reflection. L'uso dell'Intellisense è un metodo conveniente per visualizzare la descrizione delle funzioni, in particolar modo la lista dei loro parametri. Questa tecnologia riesce a velocizzare lo sviluppo del software riducendo la quantità di input attraverso la tastiera richiesto. Riduce inoltre il bisogno di appoggiarsi a documentazione esterna dato che molte informazioni come firme dei metodi, lista dei parametri e altro appaiono in automatico sul focus dell'attenzione dello sviluppatore.

\textbf{Issue}:\\ Problema del quale si necessita di una soluzione. Una issue viene associata univocamente ad un task la cui soluzione deve essere assegnata tramite ticketing ad un membro del team.

\newpage
\section{J}
\textbf{JavaScript}:\\	Linguaggio di scripting orientato agli oggetti e agli eventi, comunemente utilizzato nella programmazione Web lato client per la creazione, in siti web e applicazioni web, di effetti dinamici interattivi tramite funzioni di script invocate da eventi innescati a loro volta in vari modi dall'utente sulla pagina web in uso.

\newpage
\section{K}
\textbf{Keyword}:\\	In un testo, le parole che vengono identificate come significative. Singolarmente una keyword è anche detta parola chiave.

\newpage
\section{L}
\textbf{Learning curve}:\\ Indica il rapporto tra tempo necessario per l'apprendimento e quantità di informazioni correttamente apprese. In informatica viene usata per descrivere la qualità di un programma. Infatti, tanto più un programma è intuitivo, ben progettato, organizzato e strutturato, minore sarà il tempo che un utente impiegherà per imparare a usarlo. 

\textbf{Librerie}:\\ Un insieme di funzioni o strutture dati predefinite e predisposte per essere collegate ad un software. Con lo scopo di fornire una collezione di entità di base pronte per l'uso.

\textbf{Linguaggio di markup}:\\  Insieme di regole che descrivono i meccanismi di rappresentazione (strutturali, semantici, presentazionali) di un testo, facendo uso di convenzioni rese standard, tali regole sono utilizzabili su più supporti.

\newpage
\section{M}
\textbf{Manutenibilità}:\\ Capacità di perfezionare o correggere un progetto, attraverso operazioni sistematiche.

\textbf{Maturità}:\\ \'E una caratteristica, basata su uno standard di best practice, che facilita la valutazione della qualità di un processo software. 

\textbf{Merge}:\\	Comando di Git che incorpora le modifiche dai commit nominati (dal momento in cui le loro storie si sono discostate dal ramo corrente) nel ramo corrente. Questo comando viene utilizzato per incorporare le modifiche da un altro repository e può essere utilizzato manualmente per unire le modifiche da un ramo all'altro.

\textbf{Microsoft Windows}:\\Nome del sistemo operativo prodotto da Microsoft Corporation dal 1985.

\textbf{Milestone}:\\Indica un traguardo con particolare valore strategico nel tempo che può derivare da un obbligo contrattuale oppure da un'opportunità decisa dal gruppo. Le milestone sono associate all'allineamento di un insieme di baseline; una milestone di qualità è delimitata per ampiezza, incrementale, misurabile, raggiungibile e dimostrabile agli stakeholder.

\textbf{ML}:\\Machine Learning. Sinonimo di \hyperref[par:appr_auto]{apprendimento automatico}.

\textbf{Modello incrementale}:\\Modello di sviluppo software basato sulla successione di una serie di passi strettamente incrementali: ciò significa che ogni modifica produce un avanzamento nello sviluppo, e non porta a regressione. Grazie a questa logica è quindi possibile dare un ordine ben preciso ai requisiti da sviluppare, dando priorità a quelli di maggior importanza per il committente.

\textbf{Modularità}:\\
Proprietà di un sistema informatico di essere composto da componenti indipendentemente modificali ma comunicanti.


\newpage
\section{N}

\textbf{Namespace}: \\ \'E una collezione di nomi di entità, definite dal programmatore, omogeneamente usate in uno o più file sorgente.

\textbf{NodeJS}:\\	\'E una piattaforma Open source event-driven per l'esecuzione di codice JavaScript Server-side, costruita sul motore JavaScript V8 di Google Chrome. Molti dei suoi moduli base sono scritti in JavaScript, e gli sviluppatori possono scrivere nuovi moduli in JavaScript.

\textbf{Nodi}:\\ 
{[}In una rete Bayesiana{]} rappresentano le variabili, e sono uniti tra loro da degli archi, che rappresentano invece le relazioni di dipendenza statistica fra le variabili stesse.

\textbf{Notazione a cammello}:\\	Pratica di scrivere parole composte o frasi unendo tutte le parole tra loro, ma lasciando le loro iniziali maiuscole.

\newpage
\section{O}
\textbf{Open Source}:\\	Viene utilizzato per riferirsi ad un software di cui i detentori dei diritti rendono pubblico il codice sorgente, favorendone il libero studio e permettendo a programmatori indipendenti di apportarvi modifiche ed estensioni. Questa possibilità è regolata tramite l'applicazione di apposite licenze d'uso. Il fenomeno ha tratto grande beneficio da Internet, perché esso permette a programmatori distanti di coordinarsi e lavorare allo stesso progetto.

\newpage
\section{P}
\label{par:parola}
\textbf{Parola}
Componente atomica di una frase.

\textbf{Pattern publisher/subscriber}:\\	 Si riferisce a un design pattern o stile architetturale utilizzato per la comunicazione asincrona fra diversi processi, oggetti o altri.

\textbf{PDF}:\\	Portable Document Format. Formato di file basato su un linguaggio di descrizione di pagina sviluppato da Adobe Systems nel 1993 per rappresentare documenti di testo e immagini in modo indipendente dall'hardware e dal software utilizzati per generarli o per visualizzarli.

\textbf{Periodo}:\\	Intervallo temporale del progetto, nel quale vengono svolte determinate attività. Non è consentita la sovrapposizione fra periodi.

\textbf{Periodo di investimento}:\\	Periodo non rendicontato, durante il quale vengono investite risorse (tempo e denaro) per presentare un'offerta al committente.

\textbf{Plugin}:\\	Programma non autonomo che interagisce con un software autonomo per ampliarne o estenderne le funzionalità originarie.

\textbf{Porting}:\\	Processo di trasposizione di un componente software, a volte anche con modifiche, volto a consentirne l'uso in un ambiente di esecuzione diverso da quello originale.

\textbf{PoS-tagging}:\\	 Processo di marcatura di una parola in un testo come corrispondente a una particolare parte di discorso , basato sia sulla sua definizione che sul suo contesto, cioè sulla sua relazione con parole adiacenti e correlate in una frase.

\textbf{Producer}:\\	Si occupa di creare e inoltrare le informazioni generate dal sistema.

\textbf{Project board}:\\	Strumento messo a disposizione da vari Issue Tracking System che permette di visualizzare lo stato di completamento di Issue e Pull Request.


\textbf{Proponente}:\\	Chi presenta una proposta, che presenta qualcosa affinché venga accettato, approvato. Nell'ambito del progetto didattico è l'azienda che propone il capitolato scelto, ovvero \proponente.

\textbf{Pull request}:\\	Richiesta di merge di un branch X derivato da Y nel branch Y. Permette agli sviluppatori nei rispettivi branch di lavorare indipendentemente, e di visualizzare le modifiche apportate prima di unire le linee di sviluppo.

\newpage
\section{Q}
\textbf{Qualità}:\\	Misura in cui un'entità soddisfa un determinato numero di aspettative rispetto a un certo numero di metriche oggettive precedentemente fissate. Nel caso del progetto didattico, terremo conto di:
\begin{itemize}
	\item Software;
	\item Processi;
	\item Documentazione.
\end{itemize}


\newpage
\section{R}
\textbf{React}:\\	Conosciuto anche con il nome ReactJS è una libreria JavaScript per la costruzione di interfacce utente. È gestito da Facebook e da una comunità di singoli sviluppatori e aziende. React può essere utilizzato come base per lo sviluppo di applicazioni singole o mobili. Le applicazioni Complex React richiedono in genere l'uso di librerie aggiuntive per la gestione dello stato, il routing e l'interazione con le API.
 
\textbf{Repository}:\\	Archivio o ambiente di un sistema informativo nel quale sono raccolti e conservati dati e informazioni corredati da descrizioni (metadati) in formato digitale, e direttamente accessibile dagli utenti. I repository rappresentano in qualche misura l’equivalente elettronico di una biblioteca.

\textbf{REST}:\\
Representational State Transfer. Sistema di trasmissione di dati su HTTP privo livelli e sessione. Viene impiegato come architettura software nei sistemi distribuiti.

\textbf{Requisito}:\\	Particolare proprietà richiesta necessaria per conseguire uno scopo. Nello specifico, si categorizzano in:
\begin{itemize}
	\item \textbf{Requisito di fidatezza:} requisito di sistema che è incluso per aiutare a raggiungere un livello di fidatezza richiesto per il sistema;
	\item \textbf{Requisito funzionale:} definizione di una funzione o caratteristica che deve essere implementata in un sistema;
	\item \textbf{Requisito non funzionale:} vincolo o comportamento atteso che si applica a un sistema. Può riferirsi sia alle proprietà principali del software che sto sviluppando sia al processo di sviluppo.
\end{itemize}

\textbf{Risorsa}:\\	Persona, equipaggiamento, impianto, o qualsiasi altra cosa necessaria al compimento di una attività di progetto. Due dimensioni che caratterizzano una risorsa sono:
\begin{itemize}
\item Disponibilità;
\item Costo.
\end{itemize} 

\textbf{RSS}:\\
RDF Site Summary. Tecnologia basata su XML tramite la quale un utente può creare un proprio feed di notizie personalizzato.
\newpage
\section{S}
\textbf{Skill}:\\	Sinonimo di abilità. Ovvero capacità di una persona di fare una determinata cosa. Nel caso di Amazon Alexa, funzione aggiuntiva programmabile

\textbf{Slack}:
\begin{enumerate}
	\item Software di messaggistica specializzato per la collaborazione all'interno di un gruppo di lavoro;
	\item Tempo aggiunto alla pianificazione di un'attività per bilanciare eventuali imprevisti o rallentamenti.
\end{enumerate} 

\textbf{Snippet}:\\	Frammenti ed esempi di codice sorgente, di solito distribuiti nel pubblico dominio o come freeware. Lo scopo fondamentale di questi esempi di codifica è didattico e di supporto. Di norma, con uno snippet viene mostrata un'intera unità funzionale di codice corrispondente a un piccolo programma, oppure una singola funzione, una classe, un template o una raccolta di funzioni correlate. Di solito uno snippet nasce con gli stessi scopi di un esempio applicativo su un testo di programmazione: per mostrare il codice di una soluzione "standard" e già sperimentata a un dato problema, per illustrare "trucchi" di implementazione non banali e degni di nota, per evidenziare peculiarità di un determinato compilatore, come esempio di portabilità del codice, oppure anche per creare simpatici divertissement in poche righe.

\textbf{Software-as-service}:\\	Modello di distribuzione del software applicativo dove un produttore di software sviluppa, opera e gestisce un'applicazione web che mette a disposizione dei propri clienti via Internet.

\textbf{Software di controllo di versione}:\\ Software che si occupa della  gestione di versioni multiple di un insieme di file o informazioni.

\textbf{Sotto-caso d'uso}:\\
Un "sotto-caso d'uso" deve fornire lo stesso servizio generale del "super-caso d'uso", eventualmente producendo valore aggiuntivo, o fornendolo a qualche tipologia di attore aggiuntiva, o seguendo un procedimento parzialmente diverso per ottenere il risultato.

\textbf{Specifiche UML}:\\	
Insieme di regole che devono essere rispettate durante l'utilizzo di UML.

\textbf{SPICE}\\
Acronimo di Software Process Improvement and Capability Determination, corrisponde allo standard ISO/IEC 15504. \'E utile per garantire la qualità di processo.

\textbf{Standard}:\\	Sono documenti tecnici utilizzati in svariati ambiti. Stabiliscono specifiche tecniche ovvero materiali, criteri di progettazione, processi e metodi di realizzazione e produzione, cui si perviene dopo opportuni processi di normazione o standardizzazione da parte di opportuni enti, aventi tutti caratteristiche proprie e peculiari. Questi enti aderiscono volontariamente e contribuiscono i soggetti nazionali o internazionali dell'industria o delle relative associazioni.

\textbf{Standard ISO}:\\	International Organization for Standardization. Organizzazione internazionale che emette standard in svariati settori.

\textbf{Standard ISO$\backslash$IEC}:\\	International Electrotechnical Commission. Organizzazione internazionale per la definizione di standard in materia di elettricità, elettronica e tecnologie correlate. Molti dei suoi standard sono definiti in collaborazione con l'ISO.

\textbf{Strategia}:\\Insieme di criteri di scelta atti ad affrontare determinate situazioni.

\textbf{Syntax highlighting}:\\	Colorazione della sintassi si intende la caratteristica di un software, solitamente editor di testo, di visualizzare un testo con differenti colori e font in base a particolari regole sintattiche. Questa caratteristica, utilizzata soprattutto per il codice sorgente, facilita la scrittura in un linguaggio strutturato come un linguaggio di programmazione o un linguaggio di markup che dispone di una sintassi e di una grammatica precise.
 
\newpage
\section{T}
\textbf{Tag}:\\	Etichetta, tipicamente un comando inserito tra parentesi angolari.

\textbf{TCP/IP}:\\	Transmission Control Protocol, anche chiamato Transfer Control Protocol è un protocollo di rete a pacchetto di livello di trasporto, appartenente alla suite di protocolli Internet, che si occupa di controllo di trasmissione, ovvero rendere affidabile la comunicazione dati in rete tra mittente e destinatario. È definito nella RFC 793 e su di esso si appoggia gran parte delle applicazioni della rete Internet. \'E presente solo sui terminali di rete (host) e non sui nodi interni di commutazione della rete di trasporto, implementato come strato software di  rete all'interno del rispettivo sistema operativo ed il sistema terminale in trasmissione vi accede attraverso l'uso di opportune chiamate di sistema definite nelle API di sistema. Internet Protocol (IP) è un protocollo di rete appartenente alla suite di protocolli Internet TCP/IP su cui è basato il funzionamento della rete Internet. 
IP è un protocollo di interconnessione di reti (Inter-Networking Protocol), classificato al livello di rete del modello ISO/OSI, nato per interconnettere reti eterogenee per tecnologia, prestazioni, gestione, pertanto implementato sopra altri protocolli di livello collegamento, come Ethernet o ATM. \'E un protocollo a pacchetti  senza connessione e di tipo best effort, che non garantisce cioè alcuna forma di affidabilità della comunicazione in termini di controllo di errore, controllo di flusso e controllo di congestione, che può essere invece realizzata dai protocolli di trasporto di livello superiore (livello 4), come TCP.
 
\textbf{Telematica}:\\ Quando l'informatica ha incontrato le tecnologie delle telecomunicazioni, è nata la telematica. Questo termine, dunque, designa l'insieme delle applicazioni che consentono il trasferimento di dati a distanza, per esempio per mezzo di una rete. 

\textbf{Template}:\\ \'E un documento creato come riferimento per la struttura esterna o interna di processi o file, chiamato anche "modello" o "struttura base / scheletro", o più correntemente "modulo".

\textbf{Termine}\\
Vedi \hyperref[par:parola]{parola}.

\textbf{Tool}:\\	Strumento software che esegue una specifica funzione.

\textbf{Troubleshooting}:\\	Consiste nell'identificazione del malfunzionamento e in una ricerca della sua causa attraverso un processo di eliminazione progressiva delle possibili cause conosciute.

\newpage 
\section{U}

\textbf{UML}:\\ \'E un linguaggio di modellazione e specifica basato sul paradigma orientato agli oggetti, permette di descrivere l'architettura di un sistema in dettaglio. La sigla sta per Unified Modeling Language.

\newpage
\section{V}
\textbf{Validazione}:\\	Processo mediante il quale ci si accerta che i requisiti siano stati rispettati.

\textbf{Verifica}:\\	Operazione di controllo per mezzo della quale si procede all'accertamento della correttezza dei risultati nella forma e nei loro contenuti.

\textbf{Versione}:\\ Indica lo stato di avanzamento di un documento o di un pacchetto software.

\textbf{Versionamento}:\\	Gestione di versioni multiple di un insieme di informazioni.

\textbf{Visual Studio Code}:\\	Editor di codice sorgente sviluppato da Microsoft per Windows, Linux e macOS. Esso include supporto per il debugging, un controllo per Git integrato, Syntax highlighting, IntelliSense, Snippet e refactoring del codice. È anche personalizzabile: gli utenti possono cambiare il tema dell'editor,  le scorciatoie da tastiera, e le preferenze. È un software libero, anche se la versione ufficiale è sotto una licenza proprietaria. Visual Studio Code è basato su Electron, un framework con cui è possibile sviluppare applicazioni NodeJS.


\newpage
\section{W}
\textbf{Walkthrough}\\
Analisi del testo o del sorgente eseguita in modalità white-box. Richiede l'impiego di uno o più membri del team.

\end{document}