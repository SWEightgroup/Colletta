\documentclass[a4paper, oneside, openany, dvipsnames, table]{article}
\usepackage{../../template/SWEightStyle}
\newcommand{\Titolo}{Verbale Riunione 2018-12-12}

\newcommand{\Gruppo}{SWEight}

\newcommand{\ACapoRedazione}{Francesco Magarotto}

\newcommand{\Verifica}{Francesco Corti}

\newcommand{\Approvazione}{Sebastiano Caccaro}

\newcommand{\Distribuzione}{Vardanega Tullio \newline Cardin Riccardo \newline Gruppo SWEight}

\newcommand{\Uso}{Interno}

\newcommand{\NomeProgetto}{Colletta}

\newcommand{\Mail}{SWEightGroup@gmail.com}

\newcommand{\DescrizioneDoc}{Questo documento si occupa di riportare quanto discusso nella riunione del 12-12-2018}


\begin{document}
\copertina{}

\definecolor{greySWEight}{RGB}{255, 71, 87}
\definecolor{greyROwSWEight}{RGB}{234, 234, 234}

\section*{Registro delle modifiche}
{
	\rowcolors{2}{greyROwSWEight}{white}
	\renewcommand{\arraystretch}{1.5}
	\centering
	\begin{longtable}{ c c  C{4cm}  c  c }
		
		\rowcolor{greySWEight}
		\textcolor{white}{\textbf{Versione}} & \textcolor{white}{\textbf{Data}} & \textcolor{white}{\textbf{Descrizione}} & \textcolor{white}{\textbf{Nominativo}} & \textcolor{white}{\textbf{Ruolo}}\\
		
		1.0.2 & 2019-03-02 & Aggiunti nuovi termini del documento Piano di Progetto & Isachi Gheorghe &\reda{}\\
		
		1.0.1 & 2019-02-23 & Verifica del documento &  Francesco Corti & \ver{}\\
		
		1.0.1 & 2019-02-20 & Aggiunti nuovi termini del documento Norme di Progetto & Isachi Gheorghe &\reda{}\\
		
		1.0.0 & 2019-01-09 & Approvazione & Sebastiano Caccaro & \Res{}\\
						
		0.1.1 & 2019-01-08 & Verifica del documento & Bacco Alberto & \ver{}\\
		
		0.1.1 & 2019-01-04 & Aggiunti termini del documento Norme di Progetto & Isachi Gheorghe &\reda{}\\
		
		0.1.0 & 2019-01-01 & Aggiunti termini del documento Analisi dei Requisiti & Isachi Gheorghe &\reda{}\\
		
		0.0.4 & 2018-12-29 & Verifica del documento & Bacco Alberto & \ver{}\\
				
		0.0.4 & 2018-12-27 & Aggiunti termini del documento Piano di Qualifica & Isachi Gheorghe &\reda{}\\
				
		0.0.3 & 2018-12-26 &Aggiunti termini del documento Piano di Progetto & Isachi Gheorghe & \reda{}\\
				
		0.0.2 & 2018-12-17 & Aggiunti termini del documento Studio di Fattibilità & Isachi Gheorghe &\reda{}\\
		
		0.0.1 & 2018-12-15 & Scheletro del glossario & Damien Ciagola & \reda{}\\
		
	\end{longtable}

}
\newpage
\tableofcontents
\newpage
\section{Informazioni Generali}
\begin{itemize}
\item \textbf{Motivazione:} Discussione con la proponente per feedback e risoluzione problemi;
\item \textbf{Luogo:} Skype;
\item \textbf{Data:} 2019-03-02;
\item \textbf{Partecipanti del gruppo:} \hfill
	\begin{itemize}
	\item Bacco Alberto;
	\item Caccaro Sebastiano;
	\item Ciagola Damien;
	\item Corti Francesco;
	\item Isachi Gheorghe;
	\item Magarotto Francesco;
	\item Muraro Enrico.
	\end{itemize} 
\item \textbf{Partecipanti esterni:} Giulio Paci - Mivoq;
\item \textbf{Ora:} 11:00 - 12:00;
\item \textbf{Segretario:} Bacco Alberto.
\end{itemize}

\section{Ordine del Giorno}
\begin{itemize}
	\item \textbf{Funzionalità insegnanti}: permesso di creare classi di studenti, gestione degli insegnanti preferiti;
	\item \textbf{Analisi grammaticale}: gestione della punteggiatura;
	\item \textbf{Lingue}: come gestire più lingue;
	\item \textbf{Filtri}: richiesta di filtri in particolare;
	\item \textbf{Correzione}: visualizzazione e modalità di correzione;
	\item \textbf{Salvataggio nel database}: quali dati sono ritenuti interessanti.
\end{itemize}

\subsection{Resoconto}
\begin{itemize}
	\item \textbf{Funzionalità insegnanti - VER-1-2019-03-02.1}: 
	\begin{itemize}
		\item \textbf{Creazione classi di alunni - VER-1-2019-03-02.1.1}: per facilitare l'assegnazione degli esercizi da parte dell'insegnante,
		 si è deciso che quest'ultimo può raggruppare gli alunni in classi. Questo di velocizzare
		il processo di assegnazione;
		\item \textbf{Compilazione automatica creazione esercizio - VER-2-2019-03-02.1.2}:
		quando l'insegnante crea l'esercizio gli viene fornita la soluzione di automatica di FreeLing,
		e in seguito può modificare solamente i campi che ritiene non idonei;
		\item \textbf{Insegnante preferito - VER-1-2019-03-02.1.3}: 
		gli allievi possono avere più insegnanti preferiti, e vedere gli esercizi di tutti gli insegnanti;
	\end{itemize}
	\item \textbf{Analisi grammaticale - VER-1-2019-03-02.2}:
	\begin{itemize}
		\item \textbf{Periodi - VER-1-2019-03-02.2.1}: possibilità di inserire più periodi e poi suddividerli in frasi più semplici è preferibile;
		\item \textbf{Gestione della punteggiatura - VER-1-2019-03-02.2.2}: la punteggiatura verrà mandata a FreeLing perché 
		può essere d'interesse al fine della raccolta dati, non sarà però visualizzabile nella correzione dell'esercizio;
	\end{itemize}
	\item \textbf{Gestione delle lingue - VER-1-2019-03-02.3}: 
	\begin{itemize}
		\item \textbf{Interfaccia utente - VER-1-2019-03-02.1}: dare la possibilità agli utenti di visualizzare l'interfaccia dell'applicativo in una lingua diversa da quella dell'esercizio,
		per permettere a studenti stranieri principianti di eseguire gli esercizi con facilità;
		\item \textbf{Lingue - VER-1-2019-03-02.2}: lingue disponibili per gli esercizi: 
		Inglese, Italiano, Spagnolo, Francese. Possibilità di aggiungere ulteriori lingue in futuro rimane aperta;
		\item \textbf{Etichette - VER-1-2019-03-02.3.3}: le etichette della soluzione automatica
		vengono visualizzate con la stessa lingua dell'interfaccia grafica dell'applicativo;
	\end{itemize}
	\item \textbf{Correzione - VER-1-2019-03-02.4}:
	assegnare punteggi diversi in base alla gravità dell'errore;
	
	\item \textbf{Dati interessanti da salvare nel database - VER-1-2019-03-02.5}:
	\begin{itemize}
		\item \textbf{Affidabilità - VER-1-2019-03-02.5.1}: dare delle statistiche che indichino quale correzione è la più affidabile
		(es. un insegnanti con più allievi, oppure un insegnante che dà la correzione uguale alla maggioranza degli 
		altri insegnanti acquista maggiore credibilità, quindi avrà statistiche più elevate); 
		\item \textbf{Versioni - VER-1-2019-03-02.5.2}: le frasi uguali vengono salvate una volta sola, se hanno correzioni diverse vengono salvate le varie versioni;
		%\item \textbf{Formato - VER-1-2019-03-02.5.3}: salvare i dati come flusso per risparmiare memoria e migliorare l'efficienza nell'utilizzarli.
	\end{itemize}

\end{itemize}
\end{document}