\documentclass[a4paper, oneside, openany, dvipsnames, table]{article}
\usepackage{../../template/SWEightStyle}
\newcommand{\Titolo}{Manuale Utente}

\newcommand{\Gruppo}{SWEight}

\newcommand{\Approvatore}{Damien Ciagola}
\newcommand{\Redattori}{Alberto Bacco \newline Sebastiano Caccaro \newline Gheorghe Isachi \newline Gionata Legrottaglie}
\newcommand{\Verificatori}{Francesco Corti \newline Francesco Magarotto}

\newcommand{\pathimg}{../template/img/logoSWEight.png}

\newcommand{\Versionedoc}{1.0.0}

\newcommand{\Distribuzione}{\proponente \newline Prof. Vardanega Tullio \newline Prof. Cardin Riccardo \newline Gruppo SWEight}

\newcommand{\Uso}{Esterno}

\newcommand{\NomeProgetto}{Colletta}

\newcommand{\Mail}{SWEightGroup@gmail.com}

\newcommand{\DescrizioneDoc}{Questo documento si occupa di fornire le modalità di utilizzo del software Colletta commissionato}


\begin{document}
\copertina{}

\definecolor{greySWEight}{RGB}{255, 71, 87}
\definecolor{greyROwSWEight}{RGB}{234, 234, 234}

\section*{Registro delle modifiche}
{
	\rowcolors{2}{greyROwSWEight}{white}
	\renewcommand{\arraystretch}{1.5}
	\centering
	\begin{longtable}{ c c C{4cm}  c  c }
		
		\rowcolor{greySWEight}
		\textcolor{white}{\textbf{Versione}} & \textcolor{white}{\textbf{Data}} & \textcolor{white}{\textbf{Descrizione}} & \textcolor{white}{\textbf{Nominativo}} & \textcolor{white}{\textbf{Ruolo}}\\
		1.2.2 & 2019-02-25 & Ampliamento sezione 5.4 e 3.2.5.2 & Alberto Bacco & \reda{} \\
		
		1.2.1 & 2019-02-23 & Aggiunta sezione 3.2.5.8 Checkstyle & Sebastiano Caccaro & \reda{} \\		
		
		1.2.0 & 2019-02-20 & Aggiunta scelte tecnologiche 3.2.4.2, da 3.4.5.4 a 3.4.5.7, 4.3.1.4, 4.3.2.2, 4.4.6 e figlie & Sebastiano Caccaro & \reda{} \\	
		
		1.1.5 & 2019-02-20 & Modifica sezione 2 & Alberto Bacco & \reda{} \\
		
		1.1.4 & 2019-02-18 & Correzione errori grammatica, spostate sottosezioni di asana da 4.3 a 5.2, & Alberto Bacco & \reda{} \\
		
		1.1.3 & 2019-02-14 & Riorganizzazione e correzione errori sezione 5 & Enrico Muraro & \reda{} \\
		
		1.1.2 & 2019-02-03 & Modifica sottosezione 4.1.10, 4.3.1.4, 4.3.1.5, 4.3.1.6 & Alberto Bacco& \reda{} \\	
		
		1.1.1 & 2019-01-31 & Modifica struttura e contenuti sezione 3  & Damien Ciagola & \reda{} \\	
		
		1.1.0 & 2019-01-27 & Sezione Qualità 4.2 & Sebastiano Caccaro & \reda{} \\	
		
		1.0.1 & 2019-01-25 & Parziale ristrutturazione della struttura del documento & Sebastiano Caccaro & \reda{} \\		
		
		1.0.0 & 2019-01-11 & Approvazione per il rilascio & Sebastiano Caccaro & \Res{} \\
		
		0.9.0 & 2019-01-9 & Verifica finale & Francesco Corti & \ver{} \\
		
		0.9.0 & 2019-01-8 & Aggiunta lista di controllo & Gionata Legrottaglie & \reda{} \\
		
		0.8.0 & 2018-12-23 & Correzioni errori ortografici & Gionata Legrottaglie & \reda{} \\
		
		0.7.0 & 2018-12-20 & Verifica documento & Francesco Corti & \ver{}\\
		
		0.6.0 & 2018-12-18 & Aggiunta sottosezione 5.2.2.2, 5.2.2.3, 5.2.2.4 & Francesco Magarotto & \reda{} \\
		
		0.5.2 & 2018-12-16 & Modifica sezione 4.1.5.3 & Alberto Bacco & \reda{} \\
		
		0.5.2 & 2018-12-16 & Modifica sezione 4.1.5.3 & Alberto Bacco & \reda{} \\
		
		0.5.2 & 2018-12-16 & Aggiunte sottosezioni  & Alberto Bacco & \reda{} \\
		
		0.5.1 & 2018-12-15 & Aggiunte sottosezioni 5.3, 5.4, 5.5, 5.6, 5.7, 5.8 & Alberto Bacco & \reda{} \\
		
		0.5.0 & 2018-12-15 & Aggiunta sezione 5 e sottosezioni 5.1, 5.2 & Gionata Legrottaglie & \reda{} \\
		
		0.4.1 & 2018-12-11 & Aggiunta sezione 4.1.7.3.1 & Francesco Magarotto & \reda{} \\ 
		
		0.4.0 & 2018-12-10 & Aggiunte sottosezioni 4.1.5, 4.1.6, 4.1.7, 4.1.8 & Gionata Legrottaglie & \reda{} \\ 
		0.4.0 & 2018-12-09 & Aggiunta sezione 4 e sottosezioni 4.1.1, 4.1.2, 4.1.3, 4.1.4 & Gionata Legrottaglie & \reda{} \\ 
		
		0.3.1 & 2018-12-07 & Aggiunta sottosezione 3.2 & Gionata Legrottaglie & \reda{} \\ 
		
		0.3.0 & 2018-12-06 & Aggiunta sezione 3 e sottosezione 3.1 & Gionata Legrottaglie & \reda{} \\ 
		
		0.2.0 & 2018-12-05 & Aggiunti i riferimenti & Gionata Legrottaglie & \reda{} \\ 
		
		0.1.0 & 2018-11-30 & Aggiunta introduzione & Gionata Legrottaglie & \reda{} \\
		
		0.0.1 & 2018-11-28 & Creazione scheletro del documento & Gionata Legrottaglie & \reda{}\\
		
	\end{longtable}

}
\newpage
\tableofcontents
\newpage
\section{Informazioni Generali}
\begin{itemize}
\item \textbf{Motivazione:} Discussione con la proponente per feedback e risoluzione problemi;
\item \textbf{Luogo:} Skype;
\item \textbf{Data:} 2019-03-02;
\item \textbf{Partecipanti del gruppo:} \hfill
	\begin{itemize}
	\item Bacco Alberto;
	\item Caccaro Sebastiano;
	\item Ciagola Damien;
	\item Corti Francesco;
	\item Isachi Gheorghe;
	\item Magarotto Francesco;
	\item Muraro Enrico.
	\end{itemize} 
\item \textbf{Partecipanti esterni:} Giulio Paci - Mivoq;
\item \textbf{Ora:} 11:00 - 12:00;
\item \textbf{Segretario:} Bacco Alberto.
\end{itemize}

\section{Ordine del Giorno}
\begin{itemize}
	\item \textbf{Funzionalità insegnanti}: permesso di creare classi di studenti, gestione degli insegnanti preferiti;
	\item \textbf{Analisi grammaticale}: gestione della punteggiatura;
	\item \textbf{Lingue}: come gestire più lingue;
	\item \textbf{Filtri}: richiesta di filtri in particolare;
	\item \textbf{Correzione}: visualizzazione e modalità di correzione;
	\item \textbf{Salvataggio nel database}: quali dati sono ritenuti interessanti.
\end{itemize}

\subsection{Resoconto}
\begin{itemize}
	\item \textbf{Funzionalità insegnanti - VER-1-2019-03-02.1}: 
	\begin{itemize}
		\item \textbf{Creazione classi di alunni - VER-1-2019-03-02.1.1}: per facilitare l'assegnazione degli esercizi da parte dell'insegnante,
		 si è deciso che quest'ultimo può raggruppare gli alunni in classi. Questo di velocizzare
		il processo di assegnazione;
		\item \textbf{Compilazione automatica creazione esercizio - VER-2-2019-03-02.1.2}:
		quando l'insegnante crea l'esercizio gli viene fornita la soluzione di automatica di FreeLing,
		e in seguito può modificare solamente i campi che ritiene non idonei;
		\item \textbf{Insegnante preferito - VER-1-2019-03-02.1.3}: 
		gli allievi possono avere più insegnanti preferiti, e vedere gli esercizi di tutti gli insegnanti;
	\end{itemize}
	\item \textbf{Analisi grammaticale - VER-1-2019-03-02.2}:
	\begin{itemize}
		\item \textbf{Periodi - VER-1-2019-03-02.2.1}: possibilità di inserire più periodi e poi suddividerli in frasi più semplici è preferibile;
		\item \textbf{Gestione della punteggiatura - VER-1-2019-03-02.2.2}: la punteggiatura verrà mandata a FreeLing perché 
		può essere d'interesse al fine della raccolta dati, non sarà però visualizzabile nella correzione dell'esercizio;
	\end{itemize}
	\item \textbf{Gestione delle lingue - VER-1-2019-03-02.3}: 
	\begin{itemize}
		\item \textbf{Interfaccia utente - VER-1-2019-03-02.1}: dare la possibilità agli utenti di visualizzare l'interfaccia dell'applicativo in una lingua diversa da quella dell'esercizio,
		per permettere a studenti stranieri principianti di eseguire gli esercizi con facilità;
		\item \textbf{Lingue - VER-1-2019-03-02.2}: lingue disponibili per gli esercizi: 
		Inglese, Italiano, Spagnolo, Francese. Possibilità di aggiungere ulteriori lingue in futuro rimane aperta;
		\item \textbf{Etichette - VER-1-2019-03-02.3.3}: le etichette della soluzione automatica
		vengono visualizzate con la stessa lingua dell'interfaccia grafica dell'applicativo;
	\end{itemize}
	\item \textbf{Correzione - VER-1-2019-03-02.4}:
	assegnare punteggi diversi in base alla gravità dell'errore;
	
	\item \textbf{Dati interessanti da salvare nel database - VER-1-2019-03-02.5}:
	\begin{itemize}
		\item \textbf{Affidabilità - VER-1-2019-03-02.5.1}: dare delle statistiche che indichino quale correzione è la più affidabile
		(es. un insegnanti con più allievi, oppure un insegnante che dà la correzione uguale alla maggioranza degli 
		altri insegnanti acquista maggiore credibilità, quindi avrà statistiche più elevate); 
		\item \textbf{Versioni - VER-1-2019-03-02.5.2}: le frasi uguali vengono salvate una volta sola, se hanno correzioni diverse vengono salvate le varie versioni;
		%\item \textbf{Formato - VER-1-2019-03-02.5.3}: salvare i dati come flusso per risparmiare memoria e migliorare l'efficienza nell'utilizzarli.
	\end{itemize}

\end{itemize}
\end{document}