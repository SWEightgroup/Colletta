\documentclass[a4paper, oneside, openany, dvipsnames, table]{article}
\usepackage{../../template/SWEightStyle}
\newcommand{\Titolo}{Verbale Riunione 2018-12-12}

\newcommand{\Gruppo}{SWEight}

\newcommand{\ACapoRedazione}{Francesco Magarotto}

\newcommand{\Verifica}{Francesco Corti}

\newcommand{\Approvazione}{Sebastiano Caccaro}

\newcommand{\Distribuzione}{Vardanega Tullio \newline Cardin Riccardo \newline Gruppo SWEight}

\newcommand{\Uso}{Interno}

\newcommand{\NomeProgetto}{Colletta}

\newcommand{\Mail}{SWEightGroup@gmail.com}

\newcommand{\DescrizioneDoc}{Questo documento si occupa di riportare quanto discusso nella riunione del 12-12-2018}


\begin{document}
\copertina{}

\definecolor{greySWEight}{RGB}{255, 71, 87}
\definecolor{greyROwSWEight}{RGB}{234, 234, 234}

\section*{Registro delle modifiche}
{
	\rowcolors{2}{greyROwSWEight}{white}
	\renewcommand{\arraystretch}{1.5}
	\centering
	\begin{longtable}{ c c  C{4cm}  c  c }
		
		\rowcolor{greySWEight}
		\textcolor{white}{\textbf{Versione}} & \textcolor{white}{\textbf{Data}} & \textcolor{white}{\textbf{Descrizione}} & \textcolor{white}{\textbf{Nominativo}} & \textcolor{white}{\textbf{Ruolo}}\\
		
		1.0.2 & 2019-03-02 & Aggiunti nuovi termini del documento Piano di Progetto & Isachi Gheorghe &\reda{}\\
		
		1.0.1 & 2019-02-23 & Verifica del documento &  Francesco Corti & \ver{}\\
		
		1.0.1 & 2019-02-20 & Aggiunti nuovi termini del documento Norme di Progetto & Isachi Gheorghe &\reda{}\\
		
		1.0.0 & 2019-01-09 & Approvazione & Sebastiano Caccaro & \Res{}\\
						
		0.1.1 & 2019-01-08 & Verifica del documento & Bacco Alberto & \ver{}\\
		
		0.1.1 & 2019-01-04 & Aggiunti termini del documento Norme di Progetto & Isachi Gheorghe &\reda{}\\
		
		0.1.0 & 2019-01-01 & Aggiunti termini del documento Analisi dei Requisiti & Isachi Gheorghe &\reda{}\\
		
		0.0.4 & 2018-12-29 & Verifica del documento & Bacco Alberto & \ver{}\\
				
		0.0.4 & 2018-12-27 & Aggiunti termini del documento Piano di Qualifica & Isachi Gheorghe &\reda{}\\
				
		0.0.3 & 2018-12-26 &Aggiunti termini del documento Piano di Progetto & Isachi Gheorghe & \reda{}\\
				
		0.0.2 & 2018-12-17 & Aggiunti termini del documento Studio di Fattibilità & Isachi Gheorghe &\reda{}\\
		
		0.0.1 & 2018-12-15 & Scheletro del glossario & Damien Ciagola & \reda{}\\
		
	\end{longtable}

}
\newpage
\tableofcontents
\newpage
\section{Informazioni Generali}
\begin{itemize}
\item \textbf{Motivazione:} Discussione con la proponente per feedback e risoluzione problemi;
\item \textbf{Luogo:} Skype;
\item \textbf{Data:} 2019-03-02.
\item \textbf{Partecipanti del gruppo:} \hfill
	\begin{itemize}
	\item Bacco Alberto;
	\item Caccaro Sebastiano;
	\item Ciagola Damien;
	\item Corti Francesco;
	\item Isachi Gheorghe;
	\item Magarotto Francesco;
	\item Muraro Enrico.
	\end{itemize} 
\item \textbf{Ora:} 10:00 - 11:00;
\item \textbf{Segretario:} Bacco Alberto.
\end{itemize}

\section{Ordine del Giorno}
\begin{itemize}
	\item \textbf{funzionalità insegnanti:} permesso di creare classi di studenti, gestione degli insegnanti preferiti;
	\item \textbf{analisi grammaticale:} gestione della punteggiatura;
	\item \textbf{lingue:} come gestire più lingue;
	\item \textbf{filtri:} richiesta di filtri in particolare;
	\item \textbf{correzione:} visualizzazione e modalità di correzione;
	\item \textbf{salvataggio su DB:} quali dati sono ritenuti interessanti.
\end{itemize}

\subsection{Resoconto}
\begin{itemize}
	\item \textbf{Funzionalità insegnanti - VER-2-2019-02-19.1 :} 
	\begin{itemize}
		\item \textbf{Creazione classi di alunni - VER-2-2019-02-19.1.1:}Per facilitare l'assegnazione degli esercizi si è deciso che l'insegnante
		può raggruppare gli alunni in classi. Questo permette all'insegnante di velocizzare
		il processo di assegnazione.
		\item \textbf{Compilazione automatica creazione esercizio - VER-2-2019-02-19.1.2:}
		Quando l'insegnante crea l'esercizio gli viene fornita la soluzione di automatica di freeling,
		e in seguito può modificare solamente i campi che ritiene non idonei.
		\item \textbf{Insegnante preferito - VER-2-2019-02-19.1.3:} alievo può avere piu insegnanti preferiti e vede gli esercizi di tutti gli insegnanti.
	\end{itemize}
	\item \textbf{Analisi grammaticale - VER-2-2019-02-19.2:}
	\begin{itemize}
		\item \textbf{Periodi - VER-2-2019-02-19.2.1}Possibilità di inserire piu periodi e poi suddividerli in frasi piu semplici è preferibile, decisione non ancora definitiva.
		\item \textbf{Gestione della punteggiatura - VER-2-2019-02-19.2.2:}La punteggiatura verrà mandata a freeling perchè 
		dal punto di vista della registrazione dei dati può essere interessante, non sarà però visualizzabile nella correzione dell'esercizio.  
	\end{itemize}
	\item \textbf{Gestione delle lingue - VER-2-2019-02-19.3:} 
	\begin{itemize}
		\item \textbf{dashboard - VER-2-2019-02-19.3.1:}Dare la possibilità agli utenti di visualizzare la dashboard in una lingua diversa dalla lingue dell'esercizio,
		per permettere a studenti stranieri alle prime armi di eseguire gli esercizi con facilità.
		\item \textbf{Gestione della punteggiatura - VER-2-2019-02-19.3.2:}Lingue esercizio: 
		Inglese, Italiano, Spagnolo, Francese. Possibilità di aggiungere ulteriori lingue in fututro rimane aperta.
		\item \textbf{Etichette - VER-2-2019-02-19.3.3:} Le etichette della soluzione automatica
		vanno visualizzate con la lingua della dashboard.
	\end{itemize}
	\item \textbf{Correzione - VER-2-2019-02-19.4 :} 
	Assegnare punteggi diversi in base alla gravirà dell'errore.
	
	\item \textbf{Dati interessanti da salvare su DB - VER-2-2019-02-19.5:}
	\begin{itemize}
		\item \textbf{Affidabilità - VER-2-2019-02-19.5.1:} Dare delle statistiche che indichino quale correzione è la piu affidabile.
		(es. insegnanti con piu allievi, oppure insegnante che da la correzione uguale alla maggioranza degli 
		altri insegnanti. Questi devono avere punteggi di affidabilità più alti.) 
		\item \textbf{Versioni - VER-2-2019-02-19.5.2}Frasi uguali salvate una volta sola, se hanno correzioni diverse vengono salvate tutte.
		\item \textbf{Formato - VER-2-2019-02-19.5.3}Salvare i dati come flusso per risparmiare memoria.
	\end{itemize}

\end{itemize}

\subsection{numeroVerbale:} 2

\subsection{Data:} 2019-03-02.

\end{document}