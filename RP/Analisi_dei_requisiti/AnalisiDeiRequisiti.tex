%generare il pdf con il comando: pdflatex main.tex
\documentclass[a4paper, oneside, openany, dvipsnames, table]{article}
\usepackage{../template/SWEightStyle}
\usepackage{tabularx}
\usepackage{ltablex}
\usepackage{verbatim}
\newcommand{\Titolo}{Verbale Riunione 2018-12-12}

\newcommand{\Gruppo}{SWEight}

\newcommand{\ACapoRedazione}{Francesco Magarotto}

\newcommand{\Verifica}{Francesco Corti}

\newcommand{\Approvazione}{Sebastiano Caccaro}

\newcommand{\Distribuzione}{Vardanega Tullio \newline Cardin Riccardo \newline Gruppo SWEight}

\newcommand{\Uso}{Interno}

\newcommand{\NomeProgetto}{Colletta}

\newcommand{\Mail}{SWEightGroup@gmail.com}

\newcommand{\DescrizioneDoc}{Questo documento si occupa di riportare quanto discusso nella riunione del 12-12-2018}


\begin{document}
\copertina{} 
\definecolor{greySWEight}{RGB}{255, 71, 87}
\definecolor{greyROwSWEight}{RGB}{234, 234, 234}

\section*{Registro delle modifiche}
{
	\rowcolors{2}{greyROwSWEight}{white}
	\renewcommand{\arraystretch}{1.5}
	\centering
	\begin{longtable}{ c c  C{4cm}  c  c }
		
		\rowcolor{greySWEight}
		\textcolor{white}{\textbf{Versione}} & \textcolor{white}{\textbf{Data}} & \textcolor{white}{\textbf{Descrizione}} & \textcolor{white}{\textbf{Nominativo}} & \textcolor{white}{\textbf{Ruolo}}\\
		
		1.0.2 & 2019-03-02 & Aggiunti nuovi termini del documento Piano di Progetto & Isachi Gheorghe &\reda{}\\
		
		1.0.1 & 2019-02-23 & Verifica del documento &  Francesco Corti & \ver{}\\
		
		1.0.1 & 2019-02-20 & Aggiunti nuovi termini del documento Norme di Progetto & Isachi Gheorghe &\reda{}\\
		
		1.0.0 & 2019-01-09 & Approvazione & Sebastiano Caccaro & \Res{}\\
						
		0.1.1 & 2019-01-08 & Verifica del documento & Bacco Alberto & \ver{}\\
		
		0.1.1 & 2019-01-04 & Aggiunti termini del documento Norme di Progetto & Isachi Gheorghe &\reda{}\\
		
		0.1.0 & 2019-01-01 & Aggiunti termini del documento Analisi dei Requisiti & Isachi Gheorghe &\reda{}\\
		
		0.0.4 & 2018-12-29 & Verifica del documento & Bacco Alberto & \ver{}\\
				
		0.0.4 & 2018-12-27 & Aggiunti termini del documento Piano di Qualifica & Isachi Gheorghe &\reda{}\\
				
		0.0.3 & 2018-12-26 &Aggiunti termini del documento Piano di Progetto & Isachi Gheorghe & \reda{}\\
				
		0.0.2 & 2018-12-17 & Aggiunti termini del documento Studio di Fattibilità & Isachi Gheorghe &\reda{}\\
		
		0.0.1 & 2018-12-15 & Scheletro del glossario & Damien Ciagola & \reda{}\\
		
	\end{longtable}

}
\newpage
\tableofcontents
\newpage
\listoffigures
\newpage
\listoftables
\newpage

\section{Introduzione}
\subsection{Scopo del documento}
Il presente documento ha lo scopo di fornire agli sviluppatori uno specchietto informativo sul design strutturale e logico della piattaforma Colletta. Il documento sarà inoltre
corredato da diagrammi UML 2.X delle principali scelte prese dal gruppo SWEight e descriverà le tecnologie utilizzate nella realizzazione dell’applicazione.
\subsection{Scopo del prodotto}
Il prodotto da realizzare consta in un’applicazione web che fornisca uno strumento per creare e svolgere esercizi di analisi grammaticale, e al contempo né raccolga i risultati. I dati raccolti verranno impiegati dagli sviluppatori dell’azienda proponente come strumento per il miglioramento di algoritmi di {apprendimento automatico}\ped{G}. Nello specifico il prodotto verrà utilizzato da tre tipologie di utenti:
le/gli insegnanti che si occuperanno della creazione degli esercizi,
gli allievi che potranno svolgere gli esercizi e ottenere delle valutazioni e gli sviluppatori che filtreranno i dati secondo alcuni criteri, e infine li scaricheranno.\\Il prodotto si interfaccerà con un’applicazione di {PoS-tagging}\ped{G}, come {FreeLing}\ped{G}, a cui verrà delegata l’esecuzione dell’analisi grammaticale delle frasi.
\subsection{Glossario}
Al fine di rendere il documento il più comprensibile possibile e permetterne una rapida fruizione, viene allegato il \G{} in cui sono presenti i termini contraddistinti dal pedice G. Tali termini includono abbreviazioni, acronimi, termini di natura tecnica, oppure sono fonte di ambiguità e pertanto necessitano di una definizione che renda il loro significato inequivocabile. 
Ogni termine, solo alla prima occorrenza per documento, verrà contrassegnato con la dicitura sopra indicata e rimanderà alla medesima definizione nel \G{}.
\newpage
\newpage
\subsection{Gerarchia Utenti}
\begin{figure}[H]
\centering
\includegraphics[width=17cm]{img/Gerarchia_utenti.png} 
\caption{Panoramica utenti di tutta la piattaforma}\label{fig:1}
\end{figure}
\newpage
\section{Casi d'uso}
\section{Utente generico}
\subsection{Panoramica utente generico}
\begin{figure}[H]
\centering
\includegraphics[width=17cm]{img/PanoramicaUtenteGenerico.png} 
\caption{Panoramica utente generico}\label{fig:1}
\end{figure}


\subsection{UC 1.1 Autenticazione}
\begin{figure}[H]
\centering
\includegraphics[width=17cm]{img/UC11.png} 
\caption{Caso d'uso 1.1}\label{fig:11}
\end{figure}
\begin{itemize}
\item[•]\textbf{Attori}: Utente non autenticato;
\item[•]\textbf{Descrizione}:  L’utente non identificato inserisce username e password e si autentica accedendo alla dashboard;
\item[•]\textbf{Precondizione}: L’utente non è autenticato;
\item[•]\textbf{Postcondizione}: L’utente viene autenticato all’interno del sistema;
\item[•]\textbf{Flusso degli eventi principale}:
\begin{enumerate}
\item UC 1.1.1 - Inserimento nome utente;
\item UC 1.1.2 - Inserimento password.
\end{enumerate}
\item[•]\textbf{Estensioni}:
\begin{enumerate}
\item Errore inserimento nome utente con relativo messaggio d’errore;
\item Errore inserimento password con relativo messaggio d’errore.
\end{enumerate}
\end{itemize}

\subsubsection{UC 1.1.1 - Inserimento nome utente}
\begin{itemize}
\item[•]\textbf{Attori}: Utente non autenticato;
\item[•]\textbf{Descrizione}: L’utente inserisce un nome utente durante la registrazione;
\item[•]\textbf{Precondizione}: L’utente non è autenticato;
\item[•]\textbf{Postcondizione}: L’utente ha inserito un nome utente;
\end{itemize}

\subsubsection{UC 1.1.2 - Inserimento password}
\begin{itemize}
\item[•]\textbf{Attori}: Utente non autenticato;
\item[•]\textbf{Descrizione}: L’utente inserisce una password durante la registrazione;
\item[•]\textbf{Precondizione}: L’utente non è autenticato;
\item[•]\textbf{Postcondizione}: L’utente ha inserito una password;
\end{itemize}

\subsection{UC 1.2 - Inserimento frase libera}
\begin{figure}[H]
\centering
\includegraphics[width=17cm]{img/UC12.png} 
\caption{Caso d'uso 1.2}\label{fig:12}
\end{figure}
\begin{itemize}
\item[•]\textbf{Attori}: Utente non autenticato;
\item[•]\textbf{Descrizione}: L’allievo inserisce la propria frase in modo da ricevere l’analisi grammaticale di essa;
\item[•]\textbf{Precondizione}: Il sistema offre la possibilità di inserire una frase;
\item[•]\textbf{Postcondizione}: Una frase è stata correttamente inserita;
\item[•]\textbf{Flusso degli eventi principale}:
\begin{enumerate}
\item UC 3.4.3 - Visualizzazione soluzione.
\end{enumerate}
\item[•]\textbf{Estensioni}:
\begin{enumerate}
\item UC 1.2.1 - Visualizzazione errore inserimento frase.
\end{enumerate}
\end{itemize}

\subsubsection{UC 1.2.1 - Visualizzazione errore inserimento frase}
\begin{itemize}
\item[•]\textbf{Attori}: Utente non autenticato, Libreria di pos-tagging;
\item[•]\textbf{Descrizione}: La frase inserita non è accettata dalla libreria di pos-tagging. L'allievo visualizza un errore e può inserire una nuova frase;
\item[•]\textbf{Precondizione}: L’allievo ha provato ad inserire una frase;
\item[•]\textbf{Postcondizione}: L’allievo ha visualizzato il messaggio di errore e può inserire una nuova frase;
\end{itemize}

\newpage

\section{Insegnante}

\subsection{UC2.1 Registrazione insegnanti}

\begin{figure}[H]
\centering
\includegraphics[width=17cm]{img/UC21.png} 
\caption{Caso d'uso UC2.1}
\end{figure}

\begin{itemize}
	\item[•] \textbf{Attori}: Utente non registrato;
	\item[•] \textbf{Descrizione}:  L’utente non registrato compila il form di registrazione relativo all’insegnante completando la registrazione;
	\item[•] \textbf{Precondizione}: L’utente non è registrato;
	\item[•] \textbf{Postcondizione}: L’utente si è registrato come insegnante;
	\item[•] \textbf{Flusso degli eventi}:
		\begin{enumerate}
			\item UC2.1.1 Inserimento nome utente;
			\item UC2.1.2 Inserimento email;
			\item UC2.1.3 Inserimento password;
			\item UC2.1.4 Inserimento conferma password;
			\item UC2.1.5 Inserimento data di nascita;
			\item UC2.1.6 Inserimento scuola di appartenenza;
			\item UC2.1.7 Inserimento materia insegnata.
		\end{enumerate}
	\item[•] \textbf{Estensioni}:
		\begin{enumerate}
			\item UC2.1.8 Errore utente omonimo.
		\end{enumerate}
\end{itemize}

\subsubsection{UC2.1.1 Inserimento nome utente}
\begin{itemize}
	\item[•] \textbf{Attori}: Utente non registrato;
	\item[•] \textbf{Descrizione}: L'utente non registrato inserisce il proprio nome utente nelle apposite caselle di testo;
	\item[•] \textbf{Precondizione}: L’utente non è registrato;
	\item[•] \textbf{Postcondizione}: L'utente ha inserito il proprio nome utente;
	\item[•] \textbf{Estensioni}:
		\begin{enumerate}
			\item UC2.1.8 Errore presenza di un utente omonimo.
		\end{enumerate}
\end{itemize}

\subsubsection{UC2.1.2 Inserimento email istituzionale}
\begin{itemize}
	\item[•] \textbf{Attori}: Utente non registrato;
	\item[•] \textbf{Descrizione}: L'utente non registrato inserisce la propria email istituzionale nell'apposita casella di testo;
	\item[•] \textbf{Precondizione}: L’utente non è registrato;
	\item[•] \textbf{Postcondizione}: L'utente ha inserito la propria email istituzionale;	
	\item[•] \textbf{Estensioni}:
		\begin{enumerate}			
			\item UC2.1.8 Errore presenza utente omonimo.
		\end{enumerate}
\end{itemize}

\subsubsection{UC2.1.3 Inserimento password}
\begin{itemize}
	\item[•] \textbf{Attori}: Utente non registrato;
	\item[•] \textbf{Descrizione}: L'utente non registrato inserisce la propria password nell'apposita casella di testo;
	\item[•] \textbf{Precondizione}: L’utente non è registrato;
	\item[•] \textbf{Postcondizione}: L'utente ha inserito la propria password nell'apposita casella;
\end{itemize}

\subsubsection{UC2.1.5 Inserimento data di nascita}
\begin{itemize}
	\item[•] \textbf{Attori}: Utente non registrato;
	\item[•] \textbf{Descrizione}: L'utente non registrato inserisce la propria data di nascita nell'apposita casella di testo;
	\item[•] \textbf{Precondizione}: L’utente non è registrato;
	\item[•] \textbf{Postcondizione}: L'utente ha inserito la propria data di nascita;
	\item[•] \textbf{Estensioni}:
		\begin{enumerate}
			\item UC2.1.9 Data di nascita errata.
		\end{enumerate}
\end{itemize}

\subsubsection{UC2.1.6 Inserimento scuola di appartenenza}
\begin{itemize}
	\item[•] \textbf{Attori}: Utente non registrato;
	\item[•] \textbf{Descrizione}: L'utente non registrato inserisce la propria scuola di appartenenza nell'apposita casella di testo;
	\item[•] \textbf{Precondizione}: L’utente non è registrato;
	\item[•] \textbf{Postcondizione}: L'utente ha inserito la scuola di appartenenza;
\end{itemize}

\subsubsection{UC2.1.7 Inserimento materia insegnata}
\begin{itemize}
	\item[•] \textbf{Attori}: Insegnante;
	\item[•] \textbf{Descrizione}: L'utente non registrato inserisce la materia insegnata nell'apposita casella di testo;
	\item[•] \textbf{Precondizione}: L’utente non è registrato;
	\item[•] \textbf{Postcondizione}: L'utente ha inserito la materia insegnata.
\end{itemize}

\subsubsection{UC2.1.8 Errore presenza utente omonimo}
\begin{itemize}
	\item[•] \textbf{Attori}: Utente non registrato;
	\item[•] \textbf{Descrizione}:  L'utente non registrato sta provando a registrarsi come insegnante con un nome utente già presente all'interno del sistema;
	\item[•] \textbf{Precondizione}: L’utente non è registrato;
	\item[•] \textbf{Postcondizione}: Viene visualizzato un messaggio di errore.
\end{itemize}

\subsubsection{UC2.1.9 Data di nascita non valida}
\begin{itemize}
	\item[•] \textbf{Attori}: Utente non registrato;
	\item[•] \textbf{Descrizione}: L'utente non registrato sta provando a registrarsi come insegnante con una data di nascita errato o non valida;
	\item[•] \textbf{Precondizione}: L’utente non è registrato;
	\item[•] \textbf{Postcondizione}: Viene visualizzato un messaggio di errore.
\end{itemize}


\subsection{UC2.2 Visualizza profilo personale}

\begin{figure}[H]
\centering
\includegraphics[width=14cm]{img/UC22.png} 
\caption{Caso d'uso UC2.2}
\end{figure}

\begin{itemize}
	\item[•] \textbf{Attori}: Insegnante;
	\item[•] \textbf{Descrizione}: L’insegnante visualizza il suo profilo personale;

	\item[•] \textbf{Precondizione}: Il sistema offre l’accesso a una dashboard con tutti i dati;

	\item[•] \textbf{Postcondizione}:  Il profilo personale dell’insegnante è aperto;
	\item[•] \textbf{Flusso degli eventi}:
		\begin{enumerate}
			\item UC2.2.1 Visualizza storico frasi inserite;
			\item UC2.2.2 Visualizza esercizi svolti dagli allievi;
			\item UC2.2.3 Visualizza i propri allievi;
			\item UC2.2.4 Modifica esercizio;
			\item UC2.2.5 Elimina esercizio.
		\end{enumerate}
\end{itemize}

\subsubsection{UC2.2.1  Visualizza storico frasi inserite}

\begin{figure}[H]
\centering
\includegraphics[width=10cm]{img/UC221.png} 
\caption{Caso d'uso UC2.2.1}
\end{figure}

\begin{itemize}
	\item[•] \textbf{Attori}:  Insegnante	\item[•] \textbf{Descrizione}: L’insegnante visualizza nel suo profilo personale lo storico delle frasi inserite; 
	\item[•] \textbf{Precondizione}: Il sistema offre la possibilità di visualizzare lo storico delle frasi inserite dall’insegnante;
	\item[•] \textbf{Postcondizione}:  L’insegnante può navigare all’interno della lista di frasi che ha inserito.
	\item[•] \textbf{Flusso degli eventi}:
		\begin{enumerate}
			\item UC2.2.1.1  Seleziona esercizio per avere più dettagli.
		\end{enumerate}
\end{itemize}

\subsubsection{UC2.2.1.1 Seleziona esercizio per avere più dettagli}
\begin{itemize}
	\item[•] \textbf{Attori}: Insegnante;
	\item[•] \textbf{Descrizione}:  L’insegnante seleziona dalla lista degli esercizi assegnati un esercizio e ne visualizza i dettagli;
	\item[•] \textbf{Precondizione}: Il sistema offre la possibilità di visualizzare i dettagli relativi ad un esercizio assegnato;
	\item[•] \textbf{Postcondizione}: L’insegnante legge le specifiche dell’esercizio selezionato.
\end{itemize}

\subsubsection{UC2.2.2  Visualizza esercizi svolti dagli allievi}
\begin{figure}[H]
\centering
\includegraphics[width=10cm]{img/UC222.png} 
\caption{Caso d'uso UC2.2.2}
\end{figure}
\begin{itemize}
	\item[•] \textbf{Attori}: Insegnante;
	\item[•] \textbf{Descrizione}:  L’insegnante è nella sezione profilo personale ed entra
		nel registro che conserva gli esercizi svolti dagli allievi;
	\item[•] \textbf{Precondizione}:  L’insegnante ha degli allievi e ha assegnato a loro degli esercizi che sono stati svolti e consegnati;

	\item[•] \textbf{Postcondizione}: L’insegnante può navigare all’interno degli esercizi svolti 
                       svolti dagli allievi; 

	\item[•] \textbf{Flusso degli eventi}:
		\begin{enumerate}
			\item UC2.2.1.1  Seleziona esercizio specifico per avere più dettagli.	
		\end{enumerate}
\end{itemize}

\subsubsection{UC2.2.3 Visualizza i propri allievi}

\begin{figure}[H]
\centering
\includegraphics[width=10cm]{img/UC223.png} 
\caption{Caso d'uso UC2.2.3}
\end{figure}

\begin{itemize}
	\item[•] \textbf{Attori}: Insegnante;
	\item[•] \textbf{Descrizione}: L’insegnante visualizza la lista dei suoi allievi;
	\item[•] \textbf{Precondizione}: Il sistema offre all’insegnante di visualizzare la lista dei propri allievi;
	\item[•] \textbf{Postcondizione}: L’insegnante visualizza la lista dei propri allievi;
	\item[•] \textbf{Flusso degli eventi}:
		\begin{enumerate}
			\item UC2.2.3.1 Visualizza preferenze allievi.
		\end{enumerate}
\end{itemize}

\subsubsection{UC2.2.3.1 Visualizza preferenze allievi}
\begin{itemize}
	\item[•] \textbf{Attori}: Insegnante;
	\item[•] \textbf{Descrizione}: L’insegnante visualizza nel suo profilo personale le preferenze da parte degli allievi;
	\item[•] \textbf{Precondizione}: Il sistema offre la possibilità di poter segnare un insegnante come preferito;
	\item[•] \textbf{Postcondizione}: L’insegnante può vedere le preferenze degli allievi.
\end{itemize}


\subsubsection{UC2.2.4 Modifica esercizio}

\begin{figure}[H]
\centering
\includegraphics[width=10cm]{img/UC224.png} 
\caption{Caso d'uso UC2.2.4}
\end{figure}

\begin{itemize}
	\item[•] \textbf{Attori}: Insegnante;
	\item[•] \textbf{Descrizione}: L’insegnante può modificare un esercizio precedentemente inserito;
	\item[•] \textbf{Precondizione}: Il sistema offre la possibilità di modificare il testo, la
				lingua la visibilità i tag e le soluzioni alternative 
				dell’esercizio;
	\item[•] \textbf{Postcondizione}: L’esercizio è stato modificato;
	\item[•] \textbf{Flusso degli eventi}:
		\begin{enumerate}
			\item UC2.3.2 Scrittura testo esercizio;
			\item UC2.3.5 Modifica tag automatici;
			\item UC2.3.8 Modificare visibilità esercizio.
			\item UC2.3.9 Salva modifiche
		\end{enumerate}
		    
\end{itemize}   	
	
\subsubsection{UC2.2.5 Elimina esercizio}
\begin{itemize}
	\item[•] \textbf{Attori}: Insegnante;
	\item[•] \textbf{Descrizione}: L’insegnante elimina un esercizio che è stato assegnato;
	\item[•] \textbf{Precondizione}: Il sistema offre la possibilità di eliminare un esercizio che è stato assegnato;
	\item[•] \textbf{Postcondizione}: L’esercizio assegnato è stato cancellato.
\end{itemize}

%%%%%%%%%%%%%%%%%%%%%%%%%%%%%%%%%%%%%%%%%%%%%%%%%%%%%%%%%%%%%%%%%%%%%%%%%%%%%%%%%%%%%%%%%%%%%%%%%%%%%%%%%%%%%%%%%%

  %%%%%%%%%%%%		PARTE STEFANO		%%%%%%%%%%%%%

%%%%%%%%%%%%%%%%%%%%%%%%%%%%%%%%%%%%%%%%%%%%%%%%%%%%%%%%%%%%%%%%%%%%%%%%%%%%%%%%%%%%%%%%%%%%%%%%%%%%%%%%%%%%%%%%%%

\begin{figure}[H]
	\centering
	\includegraphics[width=10cm]{img/UC23.png} 
	\caption{Caso d'uso UC2.3}
\end{figure}


\subsection{UC2.3 Inserimento di un nuovo esercizio}
\begin{itemize}
	\item[•] \textbf{Attori}: Insegnante;
	\item[•] \textbf{Descrizione}: Insegnante attraverso una form aggiunge il testo di un nuovo esercizio;
	\item[•] \textbf{Precondizione}: Il sistema offre la possibilità di aggiungere un esercizio nel sistema;
	\item[•] \textbf{Postcondizione}: Un nuovo esercizio è stato aggiunto;
	\item[•] \textbf{Flusso degli eventi}:
	\begin{enumerate}
		\item UC2.3.1 Predisporre lingua esercizio;
		\item UC2.3.2 Scrittura testo esercizio;
		\item UC2.3.3 Correzione automatica esercizio;
		\item UC2.3.4 Modifica tag automatici;
		\item UC2.3.5 Inserire più soluzioni;
		\item UC2.3.6 Modificare visibilità esercizio;
		\item UC2.3.7 Salva modifiche;
		\item UC2.3.9 Conferma aggiunta esercizio;
	\end{enumerate}
	\item[•] \textbf{Estensioni}:	
	\begin{enumerate}
		\item UC2.3.8 Visualizzazione errori.
	\end{enumerate}
\end{itemize}

\subsubsection{UC2.3.1 Predisporre lingua esercizio}
\begin{itemize}
	\item[•] \textbf{Attori}: Insegnante;
	\item[•] \textbf{Descrizione}: L'insegnante sceglie la lingua dell’esercizio che vuole scrivere;
	\item[•] \textbf{Precondizione}: Il sistema offre la possibilità di scegliere tra più lingue il testo 
			dell’esercizio;
	\item[•] \textbf{Postcondizione}: La lingua è stata scelta;
\end{itemize}

\subsubsection{UC2.3.2 Scrittura testo esercizio}
\begin{itemize}
	\item[•] \textbf{Attori}: Insegnante;
	\item[•] \textbf{Descrizione}: L'insegnante scrive il testo dell’esercizio;
	\item[•] \textbf{Precondizione}: Il sistema offre la possibilità di scrivere un testo;
	\item[•] \textbf{Postcondizione}: Il testo è stato scritto;
	\item[•] \textbf{Flusso degli eventi}:
	\begin{enumerate}
		\item UC2.3.1.1	Cancellazione ultimo carattere;
		\item UC2.3.1.2	Testo esercizio completato;
	\end{enumerate}
	\item[•] \textbf{Estensioni}:	
	\begin{enumerate}
		\item UC2.3.3 Interruzione scrittura testo esercizio.
	\end{enumerate}
\end{itemize}


\subsubsection{UC2.3.3	Correzione automatica esercizio}
\begin{itemize}
	\item[•] \textbf{Attori}: Insegnante e libreria di PoS-tagging;
	\item[•] \textbf{Descrizione}: L’insegnante genera automaticamente i tag dal testo dell’esercizio;
	\item[•] \textbf{Precondizione}: Il sistema offre la possibilità di scrivere un testo e generare dei tag grammaticali per ogni parola del testo;
	\item[•] \textbf{Postcondizione}: I tag relativi al testo scritto sono stati generati.
\end{itemize}

\subsubsection{UC2.3.4	Modifica tag automatici}
\begin{itemize}
	\item[•] \textbf{Attori}: Insegnante e libreria di pos-tagging;
	\item[•] \textbf{Descrizione}: L’insegnante modifica/corregge i tag generati automaticamente;
	\item[•] \textbf{Precondizione}: Il sistema offre la possibilità di modificare i tag generati con la correzione automatica;
	\item[•] \textbf{Postcondizione}: I tag relativi al testo scritto sono stati modificati o corretti;
	\item[•] \textbf{Estensioni}:
	\begin{enumerate}
		\item UC002.6 Interruzione modifica/correzione dei tag.
	
	\end{enumerate}
\end{itemize}

\subsubsection{UC2.3.5	Inserire più soluzioni}
\begin{itemize}
	\item[•] \textbf{Attori}: Insegnante e libreria di pos-tagging;
	\item[•] \textbf{Descrizione}: L'insegnante inserisce tag alternativi all’esercizio consultando la libreria di PoS-tagging;
	\item[•] \textbf{Precondizione}: Il sistema offre la possibilità di inserire più tag per parola ad 
			esercizio;
	\item[•] \textbf{Postcondizione}: Una soluzione alternativa è stata aggiunta.
\end{itemize}

\subsubsection{UC2.3.6	Modificare visibilità esercizio}
\begin{itemize}
	\item[•] \textbf{Attori}: Insegnante;
	\item[•] \textbf{Descrizione}: L'insegnante decide chi può visualizzare e svolgere esercizio;
	\item[•] \textbf{Precondizione}: Il sistema offre la possibilità di monitorare visibilità esercizio;
	\item[•] \textbf{Postcondizione}: L’esercizio è stato reso visibile o meno ad un gruppo di utenti.
\end{itemize}

\subsubsection{UC2.3.7	Salva modifiche}
\begin{itemize}
	\item[•] \textbf{Attori}: Insegnante;
	\item[•] \textbf{Descrizione}: L'insegnante salva esercizio e possibili modifiche;
	\item[•] \textbf{Precondizione}: Il sistema offre la possibilità di salvare gli esercizi creati;
	\item[•] \textbf{Postcondizione}: Un esercizio è stato salvato.
\end{itemize}

\subsubsection{UC2.3.8 Visualizzazione errori}
\begin{itemize}
	\item[•] \textbf{Attori}: Insegnante;
	\item[•] \textbf{Descrizione}: L'insegnante visualizza gli errori relativi alla sintassi e la forma dell’esercizio che sta creando;
	\item[•] \textbf{Precondizione}: Il sistema offre la possibilità di visualizzare gli errori effettuati nella scrittura del testo ;
	\item[•] \textbf{Postcondizione}: L’esercizio è stato visionato dal sistema, e gli errori sono stati visualizzati.
\end{itemize}

\subsubsection{UC2.3.1.1 Conferma aggiunta esercizio}
\begin{itemize}
	\item[•] \textbf{Attori}: Insegnante;
	\item[•] \textbf{Descrizione}: L'insegnante aggiunge un esercizio al sistema;
	\item[•] \textbf{Precondizione}: Il sistema offre la possibilità di aggiungere un esercizio;
	\item[•] \textbf{Postcondizione}: L’esercizio è stato aggiunto correttamente al sistema.
\end{itemize}

%\subsection{•}
%\begin{itemize}
%	\item[•] \textbf{Attori}:;
%	\item[•] \textbf{Descrizione}:;
%	\item[•] \textbf{Precondizione}:;
%	\item[•] \textbf{Postcondizione}:;
%	\item[•] \textbf{Flusso degli eventi}:;
%	\begin{enumerate}
%		\item
%		\item
%	\end{enumerate}
%	\item[•] \textbf{Estensioni}:;	
%	\begin{enumerate}
%		\item
%		\item
%	\end{enumerate}
%\end{itemize}






\newpage
\subsection{Allievi}
\subsubsection{Panoramica allievi}
DA FARE

\begin{figure}[H]
\centering
\includegraphics[width=17cm]{img/Panoramica Allievi.png} 
\caption{Panoramica allievi}\label{fig:31}
\end{figure}


\subsubsection{UC 3.2 - Visualizzazione della dashboard\ped{G}}
\begin{figure}[H]
\centering
\includegraphics[width=17cm]{img/UC32.png} 
\caption{Caso d'uso 3.2}\label{fig:32}
\end{figure}
\begin{itemize}
\item[•]\textbf{Attori}: Allievo;
\item[•]\textbf{Descrizione}: L’allievo apre la propria dashboard\ped{G};
\item[•]\textbf{Precondizione}: Il sistema offre la possibilità di visualizzare la propria dashboard\ped{G};
\item[•]\textbf{Postcondizione}: L’allievo ha aperto la dashboard\ped{G} e può vederne tutte le componenti;
\item[•]\textbf{Flusso degli eventi principale}:
\begin{enumerate}
\item UC 3.2.1 - Visualizzazione progressi;
\item UC 3.2.2 - Visualizzazione traguardi;
\item UC 3.2.3 - Visualizzazione valutazioni;
\item UC 3.2.4 - Inserimento insegnante preferito.
\end{enumerate}
\end{itemize}

\subsubsection{UC 3.2.1 - Visualizzazione progressi}
\begin{itemize}
\item[•]\textbf{Attori}: Allievo;
\item[•]\textbf{Descrizione}: L’allievo visualizza i suoi progressi: numero di esercizi svolti, corretti ed errati;
\item[•]\textbf{Precondizione}: Il sistema offre la possibilità di visualizzare i progressi raggiunti;
\item[•]\textbf{Postcondizione}: L’allievo può visualizzare i propri progressi;
\end{itemize}

\subsubsection{UC 3.2.2 - Visualizzazione traguardi}
\begin{itemize}
\item[•]\textbf{Attori}: Allievo;
\item[•]\textbf{Descrizione}: L’allievo visualizza i traguardi raggiunti;
\item[•]\textbf{Precondizione}: Il sistema offre la possibilità di visualizzare i traguardi;
\item[•]\textbf{Postcondizione}: L’allievo può visualizzare tutti i traguardi raggiunti;
\end{itemize}

\subsubsection{UC 3.2.3 - Visualizzazione valutazioni}
\begin{itemize}
\item[•]\textbf{Attori}: Allievo;
\item[•]\textbf{Descrizione}: L’allievo visualizza le valutazioni di tutti gli esercizi svolti, sia esercizi assegnati che svolti indipendentemente;
\item[•]\textbf{Precondizione}: Il sistema offre la possibilità di visualizzare le valutazioni;
\item[•]\textbf{Postcondizione}: L’allievo può visualizzare tutte le valutazioni ricevute;
\end{itemize}

\subsubsection{UC 3.2.4 - Inserimento insegnante preferito}
\begin{itemize}
\item[•]\textbf{Attori}: Allievo;
\item[•]\textbf{Descrizione}: L’allievo inserisce il nome utente dell’insegnante da prediligere quando riceve la correzione di un esercizio. L’inserimento di un nuovo insegnante sovrascrive il precedente;
\item[•]\textbf{Precondizione}: Il sistema offre la possibilità di inserire un insegnante preferito;
\item[•]\textbf{Postcondizione}: L’allievo ha inserito un insegnante preferito;
\end{itemize}

\subsubsection{UC 3.3 - Ricerca esercizio}
\begin{figure}[H]
\centering
\includegraphics[width=17cm]{img/UC33.png} 
\caption{Caso d'uso 3.3}\label{fig:33}
\end{figure}
\begin{itemize}
\item[•]\textbf{Attori}: Allievo;
\item[•]\textbf{Descrizione}: L’allievo ricerca un esercizio attraverso frasi suggerite dal sistema o attraverso l’inserimento di una frase libera;
\item[•]\textbf{Precondizione}: Il sistema offre la possibilità di ricercare un esercizio;
\item[•]\textbf{Postcondizione}: L’allievo può navigare una lista di esercizi consigliati o inserire una frase;
\item[•]\textbf{Flusso degli eventi principale}:
\begin{enumerate}
\item UC 3.3.1 - Selezione esercizio;
\item UC 3.3.2 - Inserimento frase libera.
\end{enumerate}
\end{itemize}

\subsubsection{UC 3.3.1 - Selezione esercizio}
\begin{itemize}
\item[•]\textbf{Attori}: Allievo;
\item[•]\textbf{Descrizione}: L’allievo seleziona un esercizio nella lista delle frasi consigliate;
\item[•]\textbf{Precondizione}: Il sistema offre la possibilità di visualizzare una lista di esercizi;
\item[•]\textbf{Postcondizione}: Un esercizio è stato correttamente selezionato;
\item[•]\textbf{Flusso degli eventi principale}:
\begin{enumerate}
\item UC 3.3.1.1 - Selezione insegnante;
\item UC 3.4 - Svolgimento esercizio.
\end{enumerate}
\end{itemize}

\subsubsection{UC 3.3.1.1 - Selezione insegnante}\begin{figure}[H]
\centering
\includegraphics[width=17cm]{img/UC331.png} 
\caption{Caso d'uso 3.3.1}\label{fig:331}
\end{figure}
\begin{itemize}
\item[•]\textbf{Attori}: Allievo;
\item[•]\textbf{Descrizione}: L’allievo seleziona l’insegnante da cui vuole ricevere la correzione dell’esercizio, se nessun insegnante ha predisposto quella frase verrà utilizzato il sistema di correzione automatico. Quando è possibile l’insegnante preferito viene selezionato automaticamente;
\item[•]\textbf{Precondizione}: L’allievo ha selezionato un esercizio;
\item[•]\textbf{Postcondizione}: L’allievo ha selezionato da chi vuole ricevere la correzione;
\end{itemize}

\subsubsection{UC 3.3.2 - Inserimento frase libera}
\begin{itemize}
\item[•]\textbf{Attori}: Allievo;
\item[•]\textbf{Descrizione}: L’allievo inserisce la propria frase in modo da ricevere l’analisi grammaticale di essa;
\item[•]\textbf{Precondizione}: Il sistema offre la possibilità di inserire una frase;
\item[•]\textbf{Postcondizione}: Una frase è stata correttamente inserita;
\item[•]\textbf{Flusso degli eventi principale}:
\begin{enumerate}
\item UC 3.4.3 - Visualizzazione soluzione.
\end{enumerate}
\item[•]\textbf{Estensioni}:
\begin{enumerate}
\item UC 3.3.2.1 - Visualizzazione errore inserimento frase.
\end{enumerate}
\end{itemize}

\subsubsection{UC 3.3.2.1 - Visualizzazione errore inserimento frase}
\begin{itemize}
\item[•]\textbf{Attori}: Allievo, Libreria di pos-tagging;
\item[•]\textbf{Descrizione}: La frase inserita non è accettata dalla libreria di pos-tagging. L'allievo visualizza un errore e può inserire una nuova frase;
\item[•]\textbf{Precondizione}: L’allievo ha provato ad inserire una frase;
\item[•]\textbf{Postcondizione}: L’allievo ha visualizzato il messaggio di errore e può inserire una nuova frase;
\end{itemize}

\subsubsection{UC 3.4 - Svolgimento esercizio}
\begin{figure}[H]
\centering
\includegraphics[width=17cm]{img/UC34.png} 
\caption{Caso d'uso 3.4}\label{fig:34}
\end{figure}
\begin{itemize}
\item[•]\textbf{Attori}: Allievo;
\item[•]\textbf{Descrizione}: L’allievo può svolgere l’esercizio scegliendo le classi grammaticali per ciascuna parola da una apposita lista;
\item[•]\textbf{Precondizione}: L’allievo ha selezionato un esercizio;
\item[•]\textbf{Postcondizione}: L’allievo ha svolto un esercizio;
\item[•]\textbf{Flusso degli eventi principale}:
\begin{enumerate}
\item UC 3.4.1 - Classificazione parola;
\item UC 3.4.3 - Visualizzazione soluzione.
\end{enumerate}
\item[•]\textbf{Estensioni}:
\begin{enumerate}
\item UC 3.4.2 - Interruzione svolgimento esercizio.
\end{enumerate}
\end{itemize}

\subsubsection{UC 3.4.1 - Classificazione parola}
\begin{itemize}
\item[•]\textbf{Attori}: Allievo;
\item[•]\textbf{Descrizione}: L’allievo seleziona la classe grammaticale di una parola da una lista predefinita;
\item[•]\textbf{Precondizione}: Il sistema dà la possibilità di selezionare la classe grammaticale di una parola;
\item[•]\textbf{Postcondizione}: L’allievo ha selezionato la classe grammaticale di una parola;
\end{itemize}

\subsubsection{UC 3.4.2 - Interruzione svolgimento esercizio}
\begin{itemize}
\item[•]\textbf{Attori}: Allievo;
\item[•]\textbf{Descrizione}: L’allievo interrompe l’esercizio, scartando i dati inseriti fino a quel momento, e ritorna nella sezione di ricerca di un esercizio;
\item[•]\textbf{Precondizione}: L’allievo ha iniziato a svolgere un esercizio;
\item[•]\textbf{Postcondizione}: L’allievo ha interrotto l’esercizio, torna nella sezione di ricerca di un esercizio;
\end{itemize}

\subsubsection{UC 3.4.3 - Visualizzazione soluzione}
\begin{itemize}
\item[•]\textbf{Attori}: Allievo, Libreria di pos-tagging;
\item[•]\textbf{Descrizione}: L’allievo visualizza la correzione secondo l’insegnante selezionato in precedenza. Se era stata inserita una frase libera la correzione viene eseguita dal sistema di correzione automatico;
\item[•]\textbf{Precondizione}: L’allievo ha svolto un esercizio oppure ha inserito una frase libera;
\item[•]\textbf{Postcondizione}: L’allievo ha visualizzato la soluzione dell’esercizio;
\item[•]\textbf{Flusso degli eventi principale}:
\begin{enumerate}
\item UC 3.4.3.1 - Selezione soluzione alternativa.
\end{enumerate}
\end{itemize}

\subsubsection{UC 3.4.3.1 - Selezione soluzione alternativa}
\begin{figure}[H]
\centering
\includegraphics[width=17cm]{img/UC343.png} 
\caption{Caso d'uso 3.4.3}\label{fig:343}
\end{figure}
\begin{itemize}
\item[•]\textbf{Attori}: Allievo;
\item[•]\textbf{Descrizione}: L’allievo può selezionare da una lista una soluzione alternativa proposta dallo stesso insegnante;
\item[•]\textbf{Precondizione}: L’allievo ha visualizzato la correzione dell’esercizio svolto;
\item[•]\textbf{Postcondizione}: L’allievo ha visualizzato una soluzione alternativa dell’esercizio svolto;
\end{itemize}



\newpage
\subsection{Sviluppatori}
\subsubsection{Panoramica sviluppatore}
\begin{figure}[H]
	\centering
	\includegraphics[width=17cm, keepaspectratio]{img/UC4x.png} 
	\caption{Panoramica sviluppatore}\label{fig:4x}
\end{figure}
\subsubsection{UC 4.1 - Visualizzazione dashboard}
\begin{itemize}
	\item[•]\textbf{Attori}: Sviluppatore;
	\item[•]\textbf{Descrizione}: lo sviluppatore ha effettuato l'autenticazione e visualizza la propria dashboard;
	\item[•]\textbf{Precondizione}: lo sviluppatore ha effettuato l'autenticazione;
	\item[•]\textbf{Postcondizione}: lo sviluppatore visualizza la propria dashboard; 
	\item[•]\textbf{Flusso degli eventi}: lo sviluppatore ha effettuato l'autenticazione e viene automaticamente reindirizzato alla pagina che contiene la dashboard.
\end{itemize}
\subsubsection{UC 4.2 - Visualizzazione elenco frasi}
\begin{itemize}
	\item[•]\textbf{Attori}: Sviluppatore;
	\item[•]\textbf{Descrizione}: lo sviluppatore visualizza un elenco di frasi accompagnate 
	dalla data di creazione e dall’autore ordinate cronologicamente;
	\item[•]\textbf{Precondizione}:  lo sviluppatore si è autenticato nel sistema e visualizza la propria dashboard.
	\item[•]\textbf{Postcondizione}: lo sviluppatore visualizza l'elenco delle preposizioni ordinate cronologicamente;
	\item[•]\textbf{Flusso degli eventi}: lo sviluppatore ha selezionato la voce relativa alla pagina contenente l’elenco delle frasi e ne visualizza il contenuto. 
\end{itemize}
\subsubsection{UC 4.3 - Filtraggio dei dati}
\begin{figure}[H]
	\centering
	\includegraphics[width=14cm, keepaspectratio]{img/UC430.png} 
	\caption{Caso d'uso UC 4.3}\label{fig:430}
\end{figure}
\begin{itemize}
	\item[•]\textbf{Attori}: Sviluppatore;
	\item[•]\textbf{Descrizione}: lo sviluppatore applica dei filtri ai dati, ottenendo solo quelli d'interesse;
	\item[•]\textbf{Precondizione}: lo sviluppatore visualizza le proposizioni in ordine cronologico (UC 4.2);
	\item[•]\textbf{Postcondizione}: lo sviluppatore visualizza i dati che rispettano i filtri scelti;
	\item[•]\textbf{Flusso degli eventi}:
	\begin{enumerate}
		\item UC 4.3.1 - Inserimento filtro periodo temporale;
		\item UC 4.3.2 - Inserimento filtro {fonte}\ped{G};
		\item UC 4.3.3 - Selezione filtro lingue;
		\item UC 4.3.4 - Ricerca parole chiave;
		\item UC 4.3.5 - Inserimento filtro livello attendibilità.
	\end{enumerate}
	\item[•]\textbf{Estensioni}:
	\begin{enumerate}
		\item UC 4.3.7 - Rimozione filtro;
		\item UC 4.3.1.1 - Visualizzazione messaggio errore data errata;
		\item UC 4.3.4.1 - Visualizzazione messaggio errore input errato.
	\end{enumerate}
	\item[•]\textbf{Inclusioni}:
	\begin{enumerate}
		\item UC 4.3.6 - Visualizzazione dati filtrati.
	\end{enumerate}
\end{itemize}
\subsubsection{UC 4.3.1 - Inserimento filtro periodo temporale}
\begin{itemize}
	\item[•]\textbf{Attori}: Sviluppatore;
	\item[•]\textbf{Descrizione}: lo sviluppatore seleziona un periodo temporale, ottenendo i dati salvati in un determinato arco di tempo;
	\item[•]\textbf{Precondizione}: lo sviluppatore visualizza le proposizioni in base ai filtri stabiliti;
	\item[•]\textbf{Postcondizione}: lo sviluppatore visualizza le frasi in base al filtraggio inserito;
	\item[•]\textbf{Flusso degli eventi}: 
	\begin{enumerate}
		\item Scelta data inizio periodo;
		\item Scelta data fine periodo.
	\end{enumerate}
	\item[•]\textbf{Estensioni}: 
	\begin{enumerate}
		\item UC 4.3.1.1 - Visualizzazione messaggio errore data errata;
		\item UC 4.3.7 - Rimozione filtro.
	\end{enumerate}
	\item[•]\textbf{Inclusioni}:
	\begin{enumerate}
		\item UC 4.3.6 - Visualizzazione dati filtrati.
	\end{enumerate}
\end{itemize}

\subsubsection{UC 4.3.1.1 - Visualizzazione messaggio errore data errata}
\begin{itemize}
	\item[•]\textbf{Attori}: Sviluppatore;
	\item[•]\textbf{Descrizione}: lo sviluppatore visualizza un messaggio di errore sul periodo selezionato;
	\item[•]\textbf{Precondizione}: lo sviluppatore ha inserito una data non conforme all'interno del filtro relativo al periodo temporale;
	\item[•]\textbf{Postcondizione}: il filtro è stato erroneamente inserito pertanto viene visualizzato un messaggio di errore;
	\item[•]\textbf{Flusso degli eventi}: lo sviluppatore ha inserito una data non conforme, cioè che non rispetta il formato stabilito o che indica un periodo temporale non idoneo, pertanto visualizza un messaggio di errore.
\end{itemize}

\subsubsection{UC 4.3.2 - Inserimento filtro fonte}
\begin{itemize}
	\item[•]\textbf{Attori}: Sviluppatore;
	\item[•]\textbf{Descrizione}: lo sviluppatore seleziona una o più fonti degli esercizi, come ad esempio le correzione provenienti da determinati docenti o dal sistema di correzione automatico;
	\item[•]\textbf{Precondizione}: lo sviluppatore visualizza le proposizioni in base ai filtri stabiliti;
	\item[•]\textbf{Postcondizione}: lo sviluppatore ha impostato i valori del filtro fonte;
	\item[•]\textbf{Flusso degli eventi}: lo sviluppatore seleziona da un elenco le fonti di cui è interessato visualizzare i dati;
	\item[•]\textbf{Estensioni}: 
	\begin{enumerate}
		\item UC 4.3.7 - Rimozione filtro.
	\end{enumerate}
	\item[•]\textbf{Inclusioni}:
	\begin{enumerate}
		\item UC 4.3.6 - Visualizzazione dati filtrati.
	\end{enumerate}
\end{itemize}

\subsubsection{UC 4.3.3 -  Selezione filtro lingue}
\begin{itemize}
	\item[•]\textbf{Attori}: Sviluppatore;
	\item[•]\textbf{Descrizione}: lo sviluppatore seleziona una o più lingue da un elenco di lingue predefinito al fine di ottenere solamente le proposizioni scritte in tali lingue;
	\item[•]\textbf{Precondizione}: lo sviluppatore visualizza le proposizioni in base ai filtri stabiliti;
	\item[•]\textbf{Postcondizione}: lo sviluppatore ha stabilito i valori di filtraggio relativi alle lingue;
	\item[•]\textbf{Flusso degli eventi}: lo sviluppatore seleziona da un elenco le lingue di cui è interessato vedere le proposizioni memorizzate;
	\item[•]\textbf{Estensioni}: 
	\begin{enumerate}
		\item UC 4.3.7 - Rimozione filtro.
	\end{enumerate}
	\item[•]\textbf{Inclusioni}:
	\begin{enumerate}
		\item UC 4.3.6 - Visualizzazione dati filtrati.
	\end{enumerate}
\end{itemize}

\subsubsection{UC 4.3.4 - Ricerca frase}
\begin{itemize}
	\item[•]\textbf{Attori}: Sviluppatore;
	\item[•]\textbf{Descrizione}: lo sviluppatore scrive una o più parole chiave al fine di cernere le frasi contenenti tali parole;
	\item[•]\textbf{Precondizione}: lo sviluppatore visualizza le frasi in base al filtraggio scelto;
	\item[•]\textbf{Postcondizione}: lo sviluppatore ha impostato le parole chiave;
	\item[•]\textbf{Flusso degli eventi}: lo sviluppatore seleziona da un elenco le lingue di cui è interessato vedere le proposizioni memorizzate;
	\item[•]\textbf{Estensioni}: 
	\begin{enumerate}
		\item UC 4.3.4.1 - Visualizzazione messaggio di errore input errato;
		\item UC 4.3.7 - Rimozione filtro.
	\end{enumerate}
	\item[•]\textbf{Inclusioni}:
	\begin{enumerate}
		\item UC 4.3.6 - Visualizzazione dati filtrati.
	\end{enumerate}
\end{itemize}

\subsubsection{UC 4.3.4.1 - Visualizzazione messaggio errore input errato}
\begin{itemize}
	\item[•]\textbf{Attori}: Sviluppatore;
	\item[•]\textbf{Descrizione}: lo sviluppatore visualizza un messaggio di errore sulla ricerca.
	\item[•]\textbf{Precondizione}: lo sviluppatore ha inserito una frase non conforme;
	\item[•]\textbf{Postcondizione}: lo sviluppatore visualizza un messaggio di errore relativo 
	all'inserimento di un input errato, cioè non conforme a ciò che il sistema si attendeva.
	\item[•]\textbf{Flusso degli eventi}: lo sviluppatore inserisce un input non conforme, cioè contenente caratteri non ammessi, e visualizza un messaggio di input errato.
\end{itemize}


\subsubsection{UC 4.3.5 - Inserimento filtro livello attendibilità}
\begin{itemize}
	\item[•]\textbf{Attori}: Sviluppatore;
	\item[•]\textbf{Descrizione}: lo sviluppatore seleziona un valore numerico rappresentante il livello di attendibilità delle frasi che vuole visualizzare, ovvero frasi le cui correzioni sono uguali a quelle di altre insegnanti acquisiscono un livello di attendibilità maggiore rispetto ad altre;
	\item[•]\textbf{Precondizione}: i valori di uno o più filtri sono stati inseriti;
	\item[•]\textbf{Postcondizione}: l'insegnante ha selezionato il livello di attendibilità.
	\item[•]\textbf{Flusso degli eventi}: lo sviluppatore seleziona da un elenco predefinito il livello di attendibilità desiderato.
\item[•]\textbf{Estensioni}: 
	\begin{enumerate}
		\item UC 4.3.7 - Rimozione filtro.
	\end{enumerate}
	\item[•]\textbf{Inclusioni}:
	\begin{enumerate}
		\item UC 4.3.6 - Visualizzazione dati filtrati.
	\end{enumerate}
\end{itemize}



\subsubsection{UC 4.3.6 - Visualizzazione dati filtrati}
\begin{itemize}
	\item[•]\textbf{Attori}: Sviluppatore;
	\item[•]\textbf{Descrizione}: selezionato uno o più filtri, i dati vengono mostrati secondo i vincoli inseriti;
	\item[•]\textbf{Precondizione}: i valori di uno o più filtri sono stati inseriti;
	\item[•]\textbf{Postcondizione}: i dati mostrati rispettano tali vincoli.
	\item[•]\textbf{Flusso degli eventi}: lo sviluppatore inserisce un filtro e visualizza tutti i dati che rispettano tale filtro.
\end{itemize}

\subsubsection{UC 4.3.7 - Rimozione filtro}
\begin{itemize}
	\begin{figure}[H]
		\centering
		\includegraphics[width=14cm, keepaspectratio]{img/UC437.png} 
		\caption{Caso d'uso UC 4.3.7 relativo alla rimozione dei filtri.}\label{fig:437}
	\end{figure}
	\item[•]\textbf{Attori}: Sviluppatore;
	\item[•]\textbf{Descrizione}: lo sviluppatore cancella il valore del filtro selezionato.
	\item[•]\textbf{Precondizione}: il filtro contiene un valore ed è applicato un filtraggio ai dati rispettando le condizioni di tale filtro;
	\item[•]\textbf{Postcondizione}: il filtro è stato rimosso pertanto i dati visualizzati non rispetteranno più tale vincolo;
	\item[•]\textbf{Flusso degli eventi}: lo sviluppatore seleziona il filtro e lo rimuove.
\end{itemize}

\subsubsection{UC 4.4 - Visualizzazione storico frase}
\begin{itemize}
	\item[•]\textbf{Attori}: Sviluppatore;
	\item[•]\textbf{Descrizione}: lo sviluppatore seleziona una frase dall'elenco e ne visualizza lo storico contenente la correzione fornita dal sistema automatico, ed eventualmente quelle redatte anche da insegnanti diversi;
	\item[•]\textbf{Precondizione}: lo sviluppatore visualizza le frasi in base ai filtri stabiliti;
	\item[•]\textbf{Postcondizione}: lo sviluppatore visualizza tutte le correzioni della frase;
	\item[•]\textbf{Flusso degli eventi}: lo sviluppatore seleziona la frase e appare lo storico della frase.
\end{itemize}

\subsubsection{UC 4.5 - Scaricamento dati}
\begin{figure}[H]
	\centering
	\includegraphics[width=10cm, keepaspectratio]{img/UC450.png} 
	\caption{Caso d'uso UC 4.5}\label{fig:450}
\end{figure}
\begin{itemize}
	\item[•]\textbf{Attori}: Sviluppatore;
	\item[•]\textbf{Descrizione}: lo sviluppatore esegue il download dei dati d'interesse;
	\item[•]\textbf{Precondizione}: lo sviluppatore visualizza i dati in ordine cronologico, e nel caso siano applicati filtri, vede solo i dati che rispettano tali filtri;
	\item[•]\textbf{Postcondizione}: lo sviluppatore scarica un file contenente le analisi grammaticali in formato anonimo;
	\item[•]\textbf{Flusso degli eventi}:
	\begin{enumerate}
		\item UC 4.5.1 - Scelta contenuti download;
		\item UC 4.5.2 - Scelta formato file;
		\item UC 4.5.3 - Scelta formato archivio.
	\end{enumerate}
\end{itemize}

\subsubsection{UC 4.5.1 - Scelta contenuti download}
\begin{itemize}
	\item[•]\textbf{Attori}: Sviluppatore;
	\item[•]\textbf{Descrizione}: lo sviluppatore sceglie gli elementi di interesse che vuole siano scaricati una volta completata la procedura di download;
	\item[•]\textbf{Precondizione}: lo sviluppatore visualizza le preposizioni in base ai filtri stabiliti;
	\item[•]\textbf{Postcondizione}: lo sviluppatore visualizza i contenuti che intende scaricare, come ad esempio lo storico di una frase;
	\item[•]\textbf{Flusso degli eventi}: lo sviluppatore seleziona dall'elenco, eventualmente filtrato, le proposizioni d'interesse, come ad esempio quelle provenienti da i soli docenti selezionati.
\end{itemize}

\subsubsection{UC 4.5.2 - Scelta formato file}
\begin{itemize}
	\item[•]\textbf{Attori}: Sviluppatore;
	\item[•]\textbf{Descrizione}:  lo sviluppatore seleziona da un elenco prestabilito il formato dei file che intende scaricare;
	\item[•]\textbf{Precondizione}: lo sviluppatore non ha stabilito il formato dei file che intende scaricare;
	\item[•]\textbf{Postcondizione}: lo sviluppatore ha stabilito il formato dei file che intende scaricare;
	\item[•]\textbf{Flusso degli eventi principale}:  lo sviluppatore seleziona dall'elenco prestabilito un'opzione che determina il formato dei file che contengono i dati precedentemente scelti.
\end{itemize}

\subsubsection{UC 4.5.3 - Scelta formato archivio}
\begin{itemize}
	\item[•]\textbf{Attori}: Sviluppatore;
	\item[•]\textbf{Descrizione}: lo sviluppatore seleziona da un elenco prestabilito il formato dell'archivio compresso con cui vuole che i dati d'interesse siano compressi;
	\item[•]\textbf{Precondizione}: il formato dell'archivio contenente i dati d'interesse non è definito;
	\item[•]\textbf{Postcondizione}: il formato dell'archivio contenente i dati d'interesse è stato definito;
	\item[•]\textbf{Flusso degli eventi}: lo sviluppatore seleziona dall'elenco prestabilito un'opzione che determina il formato dell'archivio che contiene i file relativi ai dati precedentemente scelti.
\end{itemize}


\subsubsection{UC 4.6 - Creazione modello}
\begin{figure}[H]
	\centering
	\includegraphics[width=17cm, keepaspectratio]{img/UC460.png} 
	\caption{Caso d'uso UC 4.6}\label{fig:460}
\end{figure}
\begin{itemize}
	\item[•]\textbf{Attori}: Sviluppatore;
	\item[•]\textbf{Descrizione}: lo sviluppatore crea un nuovo modello;
	\item[•]\textbf{Precondizione}: lo sviluppatore non ha creato il modello;
	\item[•]\textbf{Postcondizione}: lo sviluppatore ha creato un nuovo modello;
	\item[•]\textbf{Flusso degli eventi}:  
	\begin{enumerate}
		\item UC 4.6.1 - Scelta parametri funzione;
		\item UC 4.6.2 - Inserimento nome modello;
		\item UC 4.6.3 - Salvataggio modello.
	\end{enumerate}
	\item[•]\textbf{Estensioni}:  
	\begin{enumerate}
		\item UC 4.6.4 - Valore parametro non valido;
		\item UC 4.6.5 - Visualizzazione messaggio errore nome duplicato.
	\end{enumerate}
\end{itemize}

\subsubsection{UC 4.6.1 - Scelta parametri funzione}
\begin{itemize}
	\item[•]\textbf{Attori}: Sviluppatore;
	\item[•]\textbf{Descrizione}: lo sviluppatore sceglie i parametri della funzione per la creazione di un nuovo modello con i relativi valori associati;
	\item[•]\textbf{Precondizione}: lo sviluppatore non ha scelto i parametri da inserire;
	\item[•]\textbf{Postcondizione}: lo sviluppatore ha inserito i parametri per la creazione di un nuovo modello;
	\item[•]\textbf{Flusso degli eventi}:  lo sviluppatore inserisce tutti i parametri necessari che verranno impiegati nella realizzazione del modello.
	\item[•] \textbf{Estensioni}: 
	\begin{enumerate}
		\item UC 4.6.4 - Valore parametro non valido.
	\end{enumerate}
\end{itemize}

\subsubsection{UC 4.6.2 - Inserimento nome modello}
\begin{itemize}
	\item[•]\textbf{Attori}: Sviluppatore;
	\item[•]\textbf{Descrizione}: lo sviluppatore inserisce il nome per il modello che intende salvare;
	\item[•]\textbf{Precondizione}: lo sviluppatore non ha inserito il nome del modello;
	\item[•]\textbf{Postcondizione}: lo sviluppatore ha inserito il nome del modello;
	\item[•]\textbf{Flusso degli eventi}: lo sviluppatore inserisce una stringa che definisce il nome del modello che verrà memorizzato.
	\item[•] \textbf{Estensioni}: 
	\begin{enumerate}
		\item UC 4.5.5 - Visualizzazione messaggio errore nome duplicato.
	\end{enumerate}
\end{itemize}

\subsubsection{UC 4.6.3 - Salvataggio modello}
\begin{itemize}
	\item[•]\textbf{Attori}: Sviluppatore;
	\item[•]\textbf{Descrizione}: lo sviluppatore salva il modello di cui ha definito il nome;
	\item[•]\textbf{Precondizione}: lo sviluppatore ha inserito il nome del modello;
	\item[•]\textbf{Postcondizione}: lo sviluppatore ha salvato il modello;
	\item[•]\textbf{Flusso degli eventi}: lo sviluppatore seleziona l'opzione relativa al salvataggio che comporta il salvataggio del modello.
\end{itemize}

\subsubsection{UC 4.6.4 - Valore parametro non valido}
\begin{itemize}
	\item[•]\textbf{Attori}: Sviluppatore;
	\item[•]\textbf{Descrizione}: il sistema segnala un messaggio di errore relativo al valore del parametro non valido;
	\item[•]\textbf{Precondizione}: lo sviluppatore ha inserito il valore del parametro;
	\item[•]\textbf{Postcondizione}: lo sviluppatore visualizza un messaggio di errore relativo al valore del parametro non valido;
	\item[•]\textbf{Flusso degli eventi}:  lo sviluppatore ha inserito un valore per il parametro non valido pertanto visualizza un messaggio d'errore.
\end{itemize}
\subsubsection{UC 4.6.5 - Visualizzazione messaggio errore nome duplicato}
\begin{itemize}
	\item[•]\textbf{Attori}: Sviluppatore;
	\item[•]\textbf{Descrizione}: lo sviluppatore visualizza un messaggio di errore relativo all'inserimento di un modello avente lo stesso nome di uno già inserito;
	\item[•]\textbf{Precondizione}: lo sviluppatore ha inserito il nome del modello;
	\item[•]\textbf{Postcondizione}: lo sviluppatore visualizza il messaggio di errore e inserisce un nuovo nome.
	\item[•]\textbf{Flusso degli eventi}:  lo sviluppatore visualizza il messaggio di errore e può inserire una stringa che definisce il nome del modello che verrà memorizzato.
\end{itemize}

\subsubsection{UC 4.6.6 - Interruzione inserimento modello}
\begin{itemize}
	\item[•]\textbf{Attori}: Sviluppatore;
	\item[•]\textbf{Descrizione}: lo sviluppatore interrompe l'inserimento del modello;
	\item[•]\textbf{Precondizione}: lo sviluppatore sta inserendo il modello;
	\item[•]\textbf{Postcondizione}: nessun nuovo modello è stato inserito.
	\item[•]\textbf{Flusso degli eventi}: lo sviluppatore interrompe l'inserimento del modello abbandonando la pagina.
\end{itemize}
\subsubsection{UC 4.7 - Visualizzazione elenco modelli}
\begin{itemize}
	\item[•]\textbf{Attori}: Sviluppatore;
	\item[•]\textbf{Descrizione}: lo sviluppatore visualizza l'elenco dei propri modelli realizzati;
	\item[•]\textbf{Precondizione}: lo sviluppatore si è autenticato nel sistema;
	\item[•]\textbf{Postcondizione}: lo sviluppatore visualizza l'elenco dei propri modelli;
	\item[•]\textbf{Flusso degli eventi}:  lo sviluppatore seleziona la voce del menù relativa ai modelli e ne visualizza l'elenco.
\end{itemize}


\subsubsection{UC 4.8 - Visualizzazione contenuto modello} 
\begin{itemize}
	\item[•]\textbf{Attori}: Sviluppatore;
	\item[•]\textbf{Descrizione}: lo sviluppatore visualizza il contenuto del modello selezionato.
	\item[•]\textbf{Precondizione}: lo sviluppatore visualizza l'elenco dei modelli (UC 4.6);
	\item[•]\textbf{Postcondizione}: lo sviluppatore visiona il contenuto del modello selezionato;
	\item[•]\textbf{Flusso degli eventi}: lo sviluppatore seleziona dall'elenco dei modelli quello d'interesse e ne visualizza il contenuto.
\end{itemize}

\subsubsection{UC 4.9 - Scaricamento modello}
\begin{itemize}
	\item[•]\textbf{Attori}: Sviluppatore;
	\item[•]\textbf{Descrizione}: lo sviluppatore ottiene un file contenente il modello precedentemente creato;
	\item[•]\textbf{Precondizione}: lo sviluppatore visualizza il contenuto del modello che intende scaricare.
	\item[•]\textbf{Postcondizione}: lo sviluppatore ottiene un file contenente il modello selezionato;
	\item[•]\textbf{Flusso degli eventi}: lo sviluppatore seleziona l'opzione di scaricamento e ottiene un file contenente il modello.
\end{itemize}

\subsubsection{UC 4.10 - Eliminazione modello}
\begin{itemize}
	\item[•]\textbf{Attori}: Sviluppatore;
	\item[•]\textbf{Descrizione}: lo sviluppatore elimina il modello;
	\item[•]\textbf{Precondizione}: lo sviluppatore visualizza l'elenco dei propri modelli;
	\item[•]\textbf{Postcondizione}: lo sviluppatore ha eliminato il modello; 
	\item[•]\textbf{Flusso degli eventi}: 
	\begin{enumerate}
		\item Selezione modello;
		\item Selezione procedura eliminazione.
	\end{enumerate}   
\end{itemize}


\subsubsection{UC 4.11 - Modifica dati utente}
\begin{figure}[H]
	\centering
	\includegraphics[width=12cm, keepaspectratio]{img/UC411.png} 
	\caption{Caso d'uso UC 4.11:  Modifica dati utente per Sviluppatore}\label{fig:411}
\end{figure}
\begin{itemize}
	\item[•]\textbf{Attori}: Sviluppatore;
	\item[•]\textbf{Descrizione}: lo sviluppatore modifica i propri dati personali, cioè tutti i dati personali inseriti in fase di registrazione;
	\item[•]\textbf{Precondizione}: lo sviluppatore visualizza la propria dashboard;
	\item[•]\textbf{Postcondizione}: lo sviluppatore ha modificato uno o più dati personali; 
	\item[•]\textbf{Flusso degli eventi}: 
	\begin{enumerate}
		\item Selezione procedura modifica dati personali;
		\item UC 4.11.1 - Modifica email; 
		\item UC 4.11.2 - Modifica nome;
		\item UC 4.11.3 - Modifica cognome;
		\item UC 4.11.4 - Modifica password;
		\item UC 4.11.5 - Modifica data di nascita;
		\item UC 4.11.6 - Modifica lingua interfaccia applicativo;
		\item UC 4.11.7 - Conferma modifica dati.
	\end{enumerate}
\end{itemize}
\subsubsection{UC 4.11.1 - Modifica email}
\begin{itemize}
	\item[•]\textbf{Attori}: Sviluppatore;
	\item[•]\textbf{Descrizione}: lo sviluppatore modifica la propria email;
	\item[•]\textbf{Precondizione}: lo sviluppatore sta modificando i propri dati personali;
	\item[•]\textbf{Postcondizione}: lo sviluppatore ha modificato la propria email; 
	\item[•]\textbf{Flusso degli eventi}: 
	\begin{enumerate}
		\item Selezione campo email;
		\item Modifica la stringa che rappresenta la propria email.
	\end{enumerate}
\end{itemize}
\subsubsection{UC 4.11.2 - Modifica nome}
\begin{itemize}
	\item[•]\textbf{Attori}: Sviluppatore;
	\item[•]\textbf{Descrizione}: lo sviluppatore modifica il proprio nome;
	\item[•]\textbf{Precondizione}: lo sviluppatore sta modificando i propri dati personali;
	\item[•]\textbf{Postcondizione}: lo sviluppatore ha modificato il proprio nome; 
	\item[•]\textbf{Flusso degli eventi}: 
	\begin{enumerate}
		\item Selezione campo nome;
		\item Modifica la stringa che rappresenta il nome.
	\end{enumerate}
\end{itemize}
\subsubsection{UC 4.11.3 - Modifica cognome}
\begin{itemize}
	\item[•]\textbf{Attori}: Sviluppatore;
	\item[•]\textbf{Descrizione}: lo sviluppatore modifica il proprio cognome;
	\item[•]\textbf{Precondizione}: lo sviluppatore sta modificando i propri dati personali;
	\item[•]\textbf{Postcondizione}: lo sviluppatore ha modificato il proprio cognome; 
	\item[•]\textbf{Flusso degli eventi}: 
	\begin{enumerate}
		\item Selezione campo cognome;
		\item Modifica la stringa che rappresenta il cognome.
	\end{enumerate}
\end{itemize}
\subsubsection{UC 4.11.4 - Modifica password}
\begin{figure}[H]
	\centering
	\includegraphics[width=15cm, keepaspectratio]{img/UC4114.png} 
	\caption{Caso d'uso UC 4.11.4, modifica password per Sviluppatore}\label{fig:4114}
\end{figure}
\begin{itemize}
	\item[•]\textbf{Attori}: Sviluppatore;
	\item[•]\textbf{Descrizione}: lo sviluppatore modifica il proprio cognome;
	\item[•]\textbf{Precondizione}: lo sviluppatore sta modificando i propri dati personali;
	\item[•]\textbf{Postcondizione}: lo sviluppatore ha modificato il proprio cognome; 
	\item[•]\textbf{Flusso degli eventi}: 
	\begin{enumerate}
		\item UC 4.11.4.1 - Inserimento nuova password;
		\item UC 4.11.4.2 - Inserimento conferma nuova password.
	\end{enumerate}
\end{itemize}

\subsubsection{UC 4.11.4.1 - Inserimento nuova password}
\begin{itemize}
	\item[•]\textbf{Attori}: Sviluppatore;
	\item[•]\textbf{Descrizione}: lo sviluppatore inserisce la nuova password;
	\item[•]\textbf{Precondizione}: lo sviluppatore sta modificando i propri dati personali;
	\item[•]\textbf{Postcondizione}: lo sviluppatore ha inserito il valore della nuova password; 
	\item[•]\textbf{Flusso degli eventi}: 
	\begin{enumerate}
		\item Selezione campo dati relativo alla nuova password;
		\item Inserimento stringa rappresentante la password.
	\end{enumerate}
	\item[•]\textbf{Estensioni}:
	\begin{enumerate}
		\item UC 4.11.4.3 - Visualizzazione messaggio formato password non valido.
	\end{enumerate}
\end{itemize}

\subsubsection{UC 4.11.4.2 - Inserimento conferma nuova password}
\begin{itemize}
	\item[•]\textbf{Attori}: Sviluppatore;
	\item[•]\textbf{Descrizione}: lo sviluppatore inserisce conferma la nuova password, reinserendola nell'apposito campo;
	\item[•]\textbf{Precondizione}: lo sviluppatore sta modificando i propri dati personali;
	\item[•]\textbf{Postcondizione}: lo sviluppatore ha inserito il valore del campo conferma nuova password; 
	\item[•]\textbf{Flusso degli eventi}: 
	\begin{enumerate}
		\item Selezione campo dati relativo alla conferma nuova password;
		\item Inserimento stringa rappresentante la password.
	\end{enumerate}
	\item[•]\textbf{Estensioni}:
	\begin{enumerate}
		\item UC 4.11.4.4 - Visualizzazione messaggio password diverse.
	\end{enumerate}
\end{itemize}

\subsubsection{UC 4.11.4.3 - Visualizzazione messaggio formato password non valido}
\begin{itemize}
	\item[•]\textbf{Attori}: Sviluppatore;
	\item[•]\textbf{Descrizione}: lo sviluppatore ha inserito una password con un formato non valido;
	\item[•]\textbf{Precondizione}: lo sviluppatore sta modificando i propri dati personali;
	\item[•]\textbf{Postcondizione}: lo sviluppatore visualizza un messaggio di errore relativo all'inserimento di una password che non rispetta un formato valido; 
	\item[•]\textbf{Flusso degli eventi}: lo sviluppatore ha inserito una password che non rispetta i criteri accettati dal sistema, pertanto riceve un messaggio che indica la presenza di un formato non adatto.
\end{itemize}

\subsubsection{UC 4.11.4.4 - Visualizzazione messaggio password diverse}
\begin{itemize}
	\item[•]\textbf{Attori}: Sviluppatore;
	\item[•]\textbf{Descrizione}: lo sviluppatore ha inserito un valore di conferma password che non corrisponde al valore della nuova password inserita precedentemente, pertanto visualizza un messaggio che indica che le due password non corrispondono;
	\item[•]\textbf{Precondizione}: lo sviluppatore ha inserito il valore del campo conferma nuova password;
	\item[•]\textbf{Postcondizione}: lo sviluppatore visualizza un messaggio di errore relativo all'inserimento di una password che non combacia con quella inserita nel campo nuova password; 
	\item[•]\textbf{Flusso degli eventi}: lo sviluppatore ha inserito una password che non combacia con quella inserita nel campo nuova password, pertanto riceve un messaggio che indica la presenza di tale difformità.
\end{itemize}

\subsubsection{UC 4.11.5 - Modifica data di nascita}
\begin{itemize}
	\item[•]\textbf{Attori}: Sviluppatore;
	\item[•]\textbf{Descrizione}: lo sviluppatore modifica il proprio cognome;
	\item[•]\textbf{Precondizione}: lo sviluppatore sta modificando i propri dati personali;
	\item[•]\textbf{Postcondizione}: lo sviluppatore ha modificato il proprio cognome; 
	\item[•]\textbf{Flusso degli eventi}: 
	\begin{enumerate}
		\item Selezione campo data di nascita;
		\item Modifica la stringa che rappresenta la data di nascita, inserendo il valore corretto.
	\end{enumerate}
\end{itemize}
\subsubsection{UC 4.11.6 - Modifica lingua interfaccia applicativo}
\begin{itemize}
	\item[•]\textbf{Attori}: Sviluppatore;
	\item[•]\textbf{Descrizione}: lo sviluppatore modifica la lingua dell'applicativo;
	\item[•]\textbf{Precondizione}: lo sviluppatore sta modificando i propri dati personali;
	\item[•]\textbf{Postcondizione}: lo sviluppatore ha modificato la lingua dell'applicativo; 
	\item[•]\textbf{Flusso degli eventi}: 
	\begin{enumerate}
		\item Selezione campo dati lingua applicativo;
		\item Selezione da un elenco predefinito la lingua dell'applicativo desiderata.
	\end{enumerate}
\end{itemize}

\subsubsection{UC 4.11.7 - Conferma modifica dati}
\begin{itemize}
	\item[•]\textbf{Attori}: Sviluppatore;
	\item[•]\textbf{Descrizione}: lo sviluppatore conferma la modifica dei dati inseriti;
	\item[•]\textbf{Precondizione}: lo sviluppatore ha modificato i propri dati;
	\item[•]\textbf{Postcondizione}: lo sviluppatore conferma la modifica e i dati vengono registrati nel sistema; 
	\item[•]\textbf{Flusso degli eventi}: lo sviluppatore seleziona l'opzione di conferma di modifica dati.
\end{itemize}

\subsubsection{UC 4.12 - Logout}
\begin{itemize}
	\item[•]\textbf{Attori}: Sviluppatore;
	\item[•]\textbf{Descrizione}: lo sviluppatore effettua il logout dal sistema e viene reindirizzato alla pagina di login;
	\item[•]\textbf{Precondizione}: lo sviluppatore si è autenticato;
	\item[•]\textbf{Postcondizione}: lo sviluppatore ha chiuso la sessione e viene reindirizzato alla pagina di autenticazione; 
	\item[•]\textbf{Flusso degli eventi}: lo sviluppatore seleziona il collegamento al logout e termina la sessione.
\end{itemize}
\newpage
\section{Amministratore}
\subsection{UC 5.1 - Approvazione registrazione}
\begin{itemize}
\item[•] \textbf{Attore}: Amministratore;

\item[•] \textbf{Descrizione}: L’amministratore approva la richiesta di registrazione da parte dello sviluppatore;

\item[•] \textbf{Precondizione}: Lo sviluppatore ha effettuato richiesta di registrazione, l'amministratore visualizza le richieste di registrazione da parte degli sviluppatori che desiderano accedere ai dati della piattaforma;

\item[•] \textbf{Postcondizione}: L’amministratore approva lo sviluppatore;

\item[•] \textbf{Flusso degli eventi}:

\begin{enumerate}

\item Visualizzazione elenco richieste;

\item Selezione richiesta sviluppatore;

\item Approvazione utenza sviluppatore.

\end{enumerate}

\end{itemize}


\subsection{UC 5.2 - Rifiuto registrazione}
\begin{itemize}
\item[•] \textbf{Attore}: Amministratore;

\item[•] \textbf{Descrizione}: L’amministratore rifiuta la richiesta di registrazione da parte dello sviluppatore;

\item[•] \textbf{Precondizione}: Lo sviluppatore ha richiesto l’approvazione, l’amministratore visualizza la richiesta di registrazione da parte dello sviluppatore che desidera accedere ai dati della piattaforma;

\item[•] \textbf{Postcondizione}: L’amministratore rifiuta la richiesta di registrazione dello sviluppatore;

\item[•] \textbf{Flusso degli eventi}:

\begin{enumerate}

\item Visualizzazione elenco richieste;

\item Selezione richiesta sviluppatore;

\item Rifiuto utenza sviluppatore.

\end{enumerate}
\end{itemize}
\subsection{UC 5.3 - Eliminazione utenza}
\begin{itemize}
\item[•] \textbf{Attore}: Amministratore;

\item[•] \textbf{Descrizione}: L’amministratore elimina un’utenza dal sistema;

\item[•] \textbf{Precondizione}: L’amministratore \`{e} autenticato e visualizza l’elenco degli utenti nel sistema, e l’utenza da eliminare \`{e} registrata nel sistema;

\item[•] \textbf{Postcondizione}: L’amministratore ha eliminato l’utenza dal sistema; 

\item[•] \textbf{Flusso degli eventi}:

\begin{enumerate}

\item Visualizzazione elenco utenti;

\item Selezione utente;

\item Eliminazione utente.

\end{enumerate}

\end{itemize}


\section{Requisiti}

DESCRIZIONE DA FARE!
\subsection{Requisiti funzionali} 
\renewcommand{\arraystretch}{1.5}
\def\tabularxcolumn#1{m{#1}}
\begin{tabularx}{\textwidth}{cXccc}
 
\rowcolor{greySWEight}
   \textcolor{white}{\textbf{Identificativo}} &
   \textcolor{white}{\textbf{Descrizione}}&
   \textcolor{white}{\textbf{Importanza}}&
   \textcolor{white}{\textbf{Tipo}}&
   \textcolor{white}{\textbf{Fonte}}\endhead
 
%%################## INIZIO ENTRY ################################    
%Identificativo
R-1FU001 &
 
%Descrizione
Un utente non autenticato può registrare alla piattaforme come allievo, insegnante, sviluppatore o amministratore &
 
%Importanza
Obbligatorio &
 
%Tipo
Funzionale &
 
%Fonte
Capitolato \\
%%##################### FINE ENTRY ################################

%%################## INIZIO ENTRY ################################    
%Identificativo
R-1FU002 &

%Descrizione
Un utente non autenticato può effettuare il login &

%Importanza
Obbligatorio &

%Tipo
Funzionale &

%Fonte
Capitolato \\
%%##################### FINE ENTRY ################################
%%################## INIZIO ENTRY ################################    
%Identificativo
R-1FA001 &

%Descrizione
Un allievo può visualizzare la propria dashboard &

%Importanza
Obbligatorio &

%Tipo
Funzionale &

%Fonte
Capitolato \\
%%##################### FINE ENTRY ################################
%%################## INIZIO ENTRY ################################    
%Identificativo
R-3FA002 &

%Descrizione

Un allievo può visualizzare i propri progressi &

%Importanza
Opzionale &

%Tipo
Funzionale &

%Fonte
Capitolato \\
%%##################### FINE ENTRY ################################
%%################## INIZIO ENTRY ################################    
%Identificativo
R-1FA003 &

%Descrizione


Un allievo può visualizzare i propri traguardi &

%Importanza
Desiderabile &

%Tipo
Funzionale &

%Fonte
Capitolato \\
%%##################### FINE ENTRY ################################
%%################## INIZIO ENTRY ################################    
%Identificativo
R-1FA004 &

%Descrizione


Un allievo può visualizzare le correzioni dell’insegnante e del correttore automatico &

%Importanza
Obbligatorio &

%Tipo
Funzionale &

%Fonte
Capitolato \\
%%##################### FINE ENTRY ################################
%%################## INIZIO ENTRY ################################    
%Identificativo
R-1FA005 &

%Descrizione



Un allievo può cercare un esercizio &

%Importanza
Obbligatorio &

%Tipo
Funzionale &

%Fonte
Capitolato \\
%%##################### FINE ENTRY ################################
%%################## INIZIO ENTRY ################################    
%Identificativo
R-1FA006 &

%Descrizione
Un allievo può scegliere la lingua dell’esercizio da svolgere &

%Importanza
Obbligatorio &

%Tipo
Funzionale &

%Fonte
Capitolato \\
%%##################### FINE ENTRY ################################
%%################## INIZIO ENTRY ################################    
%Identificativo
R-1FA007 &

%Descrizione

Un allievo può svolgere un esercizio &

%Importanza
Obbligatorio &

%Tipo
Funzionale &

%Fonte
Capitolato \\
%%##################### FINE ENTRY ###############################
%%################## INIZIO ENTRY ################################    
%Identificativo
R-1FA008 &

%Descrizione

Un allievo può inserire un esercizio, utilizzando una frase libera &

%Importanza
Obbligatorio &

%Tipo
Funzionale &

%Fonte
Capitolato \\
%%##################### FINE ENTRY ###############################
%%################## INIZIO ENTRY ################################    
%Identificativo
R-3FI001 &

%Descrizione
Un insegnante può visualizzare il proprio profilo &

%Importanza
Opzionale &

%Tipo
Funzionale &

%Fonte
Interno \\
%%##################### FINE ENTRY ###############################
%%################## INIZIO ENTRY ################################    
%Identificativo
R-1FI002 &

%Descrizione
Un insegnante può visualizzare lo storico delle frasi inserite &

%Importanza
Obbligatorio &

%Tipo
Funzionale &

%Fonte
Interno \\
%%##################### FINE ENTRY ###############################
%%################## INIZIO ENTRY ################################    
%Identificativo
R-1FI003 &

%Descrizione
Un insegnante può visualizzare gli esercizi svolti dagli allievi &

%Importanza
Obbligatorio &

%Tipo
Funzionale &

%Fonte
Capitolato \\
%%##################### FINE ENTRY ###############################
%%################## INIZIO ENTRY ################################    
%Identificativo
R-3FI004 &

%Descrizione
Un insegnante può visualizzare i propri allievi &

%Importanza
Opzionale &

%Tipo
Funzionale &

%Fonte
Interno \\
%%##################### FINE ENTRY ###############################
%%################## INIZIO ENTRY ################################    
%Identificativo
R-3FI005 &

%Descrizione
Un insegnante può modificare una sua soluzione precedentemente inserita &

%Importanza
Opzionale &

%Tipo
Funzionale &

%Fonte
Interno \\
%%##################### FINE ENTRY ###############################
%%################## INIZIO ENTRY ################################    
%Identificativo
R-3FI006 &

%Descrizione
Un insegnante può eliminare una sua soluzione  dell’esercizio &

%Importanza
Opzionale &

%Tipo
Funzionale &

%Fonte
Interno \\
%%##################### FINE ENTRY ###############################
%%################## INIZIO ENTRY ################################    
%Identificativo
R-1FI007 &

%Descrizione
Un insegnante può inserire un nuovo esercizio &

%Importanza
Obbligatorio &

%Tipo
Funzionale &

%Fonte
Capitolato \\
%%##################### FINE ENTRY ###############################
%%################## INIZIO ENTRY ################################    
%Identificativo
R-1FI008 &

%Descrizione
Un insegnante può scegliere la lingua dell’esercizio da inserire &

%Importanza
Obbligatorio &

%Tipo
Funzionale &

%Fonte
Capitolato \\
%%##################### FINE ENTRY ###############################
%%################## INIZIO ENTRY ################################    
%Identificativo
R-1FI009 &

%Descrizione
Un insegnante può inserire una soluzione personale all’esercizio inserito &

%Importanza
Obbligatorio &

%Tipo
Funzionale &

%Fonte
Capitolato \\
%%##################### FINE ENTRY ###############################
%%################## INIZIO ENTRY ################################    
%Identificativo
R-1FI010 &

%Descrizione
Un insegnante può utilizzare una soluzione della libreria di post-tagging &

%Importanza
Obbligatorio &

%Tipo
Funzionale &

%Fonte
Capitolato \\
%%##################### FINE ENTRY ###############################
%%################## INIZIO ENTRY ################################    
%Identificativo
R-3FI011 &

%Descrizione

Un insegnante può inserire più soluzioni per un esercizio &

%Importanza
Opzionale &

%Tipo
Funzionale &

%Fonte
Interno \\
%%##################### FINE ENTRY ###############################
%%################## INIZIO ENTRY ################################    
%Identificativo
R-1FI012 &

%Descrizione
Un insegnante può visualizzare la conferma dell'avvenuto inserimento di un esercizio &

%Importanza
Obbligatorio &

%Tipo
Funzionale &

%Fonte
Interno \\
%%##################### FINE ENTRY ###############################
%%################## INIZIO ENTRY ################################    
%Identificativo
R-1FI013 &

%Descrizione
Un insegnante può visualizzare la soluzione dell’esercizio di un singolo alunno &

%Importanza
Obbligatorio &

%Tipo
Funzionale &

%Fonte
Interno \\
%%##################### FINE ENTRY ###############################
%%################## INIZIO ENTRY ################################    
%Identificativo
R-1FI014 &

%Descrizione
Un insegnante può valutare la soluzione dell’esercizio di un singolo alunno &

%Importanza
Desiderabile &

%Tipo
Funzionale &

%Fonte
Interno \\
%%##################### FINE ENTRY ###############################
%%################## INIZIO ENTRY ################################    
%Identificativo
R-1FS001 &

%Descrizione
Uno sviluppatore può visualizzare l’elenco delle frasi &

%Importanza
Obbligatorio &

%Tipo
Funzionale &

%Fonte
Capitolato \\
%%##################### FINE ENTRY ###############################
%%################## INIZIO ENTRY ################################    
%Identificativo
R-2FS002 &

%Descrizione
Uno sviluppatore può filtrare i dati per data di inserimento &

%Importanza
Desiderabile &

%Tipo
Funzionale &

%Fonte
Capitolato \\
%%##################### FINE ENTRY ###############################
%%################## INIZIO ENTRY ################################    
%Identificativo
R-1FS003 &

%Descrizione
Uno sviluppatore può filtrare i dati per fonte &

%Importanza
Obbligatorio &

%Tipo
Funzionale &

%Fonte
Capitolato \\
%%##################### FINE ENTRY ###############################
%%################## INIZIO ENTRY ################################    
%Identificativo
R-2FS004 &

%Descrizione
Uno sviluppatore può filtrare i dati per fonte &

%Importanza
Desiderabile &

%Tipo
Funzionale &

%Fonte
Interno \\
%%##################### FINE ENTRY ###############################
%%################## INIZIO ENTRY ################################    
%Identificativo
R-2FS005 &

%Descrizione
Uno sviluppatore può filtrare i dati per lingua &

%Importanza
Desiderabile &

%Tipo
Funzionale &

%Fonte
Interno \\
%%##################### FINE ENTRY ###############################
%%################## INIZIO ENTRY ################################    
%Identificativo
R-2FS006 &

%Descrizione
Uno sviluppatore può visualizzare i dati filtrati &

%Importanza
Desiderabile &

%Tipo
Funzionale &

%Fonte
Capitolato \\
%%##################### FINE ENTRY ###############################
%%################## INIZIO ENTRY ################################    
%Identificativo
R-1FS007 &

%Descrizione
Uno sviluppatore può scaricare i dati filtrati &

%Importanza
Obbligatorio &

%Tipo
Funzionale &

%Fonte
Capitolato \\
%%##################### FINE ENTRY ###############################
%%################## INIZIO ENTRY ################################    
%Identificativo
R-1FS008 &

%Descrizione
Uno sviluppatore può visualizzare lo storico di una frase &

%Importanza
Obbligatorio &

%Tipo
Funzionale &

%Fonte
Capitolato \\
%%##################### FINE ENTRY ###############################
%%################## INIZIO ENTRY ################################    
%Identificativo
R-1FS009 &

%Descrizione
Uno sviluppatore può aver accesso ai modelli &

%Importanza
Obbligatorio &

%Tipo
Funzionale &

%Fonte
Capitolato \\
%%##################### FINE ENTRY ###############################
%%################## INIZIO ENTRY ################################    
%Identificativo
R-1FS010 &

%Descrizione
Uno sviluppatore può rimuovere un filtro di ricerca &

%Importanza
Obbligatorio &

%Tipo
Funzionale &

%Fonte
Interno \\
%%##################### FINE ENTRY ###############################
%%################## INIZIO ENTRY ################################    
%Identificativo
R-1FS011 &

%Descrizione
Uno sviluppatore può scegliere i contenuti da scaricare &

%Importanza
Obbligatorio &

%Tipo
Funzionale &

%Fonte
Capitolato \\
%%##################### FINE ENTRY ###############################
%%################## INIZIO ENTRY ################################    
%Identificativo
R-3FS012 &

%Descrizione
Uno sviluppatore può scegliere il formato del file da scaricare &

%Importanza
Opzionale &

%Tipo
Funzionale &

%Fonte
Interno \\
%%##################### FINE ENTRY ###############################
%%################## INIZIO ENTRY ################################    
%Identificativo
R-3FS013 &

%Descrizione
Uno sviluppatore può scegliere il formato dell’archivio da scaricare &

%Importanza
Opzionale &

%Tipo
Funzionale &

%Fonte
Interno \\
%%##################### FINE ENTRY ###############################
%%################## INIZIO ENTRY ################################    
%Identificativo
R-2FAM001 &

%Descrizione
Un amministratore può approvare una registrazione di uno sviluppatore &

%Importanza
Desiderabile &

%Tipo
Funzionale &

%Fonte
Interno \\
%%##################### FINE ENTRY ###############################
%%################## INIZIO ENTRY ################################    
%Identificativo
R-2FAM002 &

%Descrizione
Un amministratore può rifiutare una registrazione di uno sviluppatore &

%Importanza
Desiderabile &

%Tipo
Funzionale &

%Fonte
Interno \\
%%##################### FINE ENTRY ###############################
%%################## INIZIO ENTRY ################################    
%Identificativo
R-2FAM003 &

%Descrizione
Un amministratore può eliminare un utente &

%Importanza
Desiderabile &

%Tipo
Funzionale &

%Fonte
Interno \\
%%##################### FINE ENTRY ###############################
%%################## INIZIO ENTRY ################################    
%Identificativo
R-2FAM004 &

%Descrizione
Un amministratore può visualizzare l’intera lista degli utenti &

%Importanza
Desiderabile &

%Tipo
Funzionale &

%Fonte
Interno \\
%%##################### FINE ENTRY ###############################
%%################## INIZIO ENTRY ################################    
%Identificativo
R-2FAM005 &

%Descrizione
Un amministratore può visualizzare l’intera lista delle richieste in sospeso &

%Importanza
Desiderabile &

%Tipo
Funzionale &

%Fonte
Interno \\
%%##################### FINE ENTRY ###############################

\caption{Tabella requisiti funzionali} \label{tab:tabellarequisitifunzionali}
\end{tabularx}




%%##################### FINE requisiti funzionali ############################

\subsection{Requisiti di vincolo} 


\begin{tabularx}{\textwidth}{cXccc}
	
	\rowcolor{greySWEight}
	\textcolor{white}{\textbf{Identificativo}} &
	\textcolor{white}{\textbf{Descrizione}}&
	\textcolor{white}{\textbf{Importanza}}&
	\textcolor{white}{\textbf{Tipo}}&
	\textcolor{white}{\textbf{Fonte}}\endhead
	
	%%################## INIZIO ENTRY #############################    
	%Identificativo
	R-1V001 &
	
	%Descrizione
	L’applicazione è web o mobile &
	
	%Importanza
	Obbligatorio &
	
	%Tipo
	Vincolo &
	
	%Fonte
	Capitolato \\
	%%##################### FINE ENTRY ############################
	%%################## INIZIO ENTRY #############################    
	%Identificativo
	R-1V002 &
	
	%Descrizione
	L’applicazione utilizza una libreria di post-tagging &
	
	%Importanza
	Obbligatorio &
	
	%Tipo
	Vincolo &
	
	%Fonte
	Capitolato \\
	%%##################### FINE ENTRY #############################
	%%################## INIZIO ENTRY #############################    
	%Identificativo
	R-1V003 &
	
	%Descrizione
	L’applicazione utilizza firebase &
	
	%Importanza
	Obbligatorio &
	
	%Tipo
	Vincolo &
	
	%Fonte
	Capitolato \\
	%%##################### FINE ENTRY #############################
	%%################## INIZIO ENTRY #############################    
	%Identificativo
	R-1V004 &
	
	%Descrizione
	Il funzionamento del prodotto richiede una connessione a internet &
	
	%Importanza
	Obbligatorio &
	
	%Tipo
	Vincolo &
	
	%Fonte
	Capitolato \\
	%%##################### FINE ENTRY #############################
	
	
	\caption{Tabella requisiti di vincolo} \label{tab:tabellarequisitivincolo}
\end{tabularx}

%%##################### FINE requisiti di vincolo ############################


\subsection{Requisiti di qualità} 


\begin{tabularx}{\textwidth}{cXccc}
	
	\rowcolor{greySWEight}
	\textcolor{white}{\textbf{Identificativo}} &
	\textcolor{white}{\textbf{Descrizione}}&
	\textcolor{white}{\textbf{Importanza}}&
	\textcolor{white}{\textbf{Tipo}}&
	\textcolor{white}{\textbf{Fonte}}\endhead
	
	%%################## INIZIO ENTRY #############################    
	%Identificativo
	R-1Q1001 &
	
	%Descrizione
	Sono rispettate le norme e le metriche sulla stesura dei documenti indicate nel documento “Norme di progetto v1.0.0” &
	
	%Importanza
	Obbligatorio &
	
	%Tipo
	Qualità &
	
	%Fonte
	Interno \\
	%%##################### FINE ENTRY ############################
	%%################## INIZIO ENTRY #############################    
	%Identificativo
	R-1Q1002 &
	
	%Descrizione
	Deve essere realizzato un manuale manutentore che descriva l’utilizzo del codice &
	
	%Importanza
	Obbligatorio &
	
	%Tipo
	Qualità &
	
	%Fonte
	Interno \\
	%%##################### FINE ENTRY ############################
	%%################## INIZIO ENTRY #############################    
	%Identificativo
	R-1Q1003 &
	
	%Descrizione
	Il manuale manutentore deve essere disponibile in lingua italiana &
	
	%Importanza
	Obbligatorio &
	
	%Tipo
	Qualità &
	
	%Fonte
	Interno \\
	%%##################### FINE ENTRY ############################
	%%################## INIZIO ENTRY #############################    
	%Identificativo
	R-1Q1004 &
	
	%Descrizione
	Deve essere fornito un manuale utente  &
	
	%Importanza
	Obbligatorio &
	
	%Tipo
	Qualità &
	
	%Fonte
	Interno \\
	%%##################### FINE ENTRY ############################
	%%################## INIZIO ENTRY #############################    
	%Identificativo
	R-1Q1005 &
	
	%Descrizione
	Il manuale utente deve essere disponibile in lingua italiana   &
	
	%Importanza
	Obbligatorio &
	
	%Tipo
	Qualità &
	
	%Fonte
	Interno \\
	%%##################### FINE ENTRY ############################
	
	
	\caption{Tabella requisiti di qualità} \label{tab:tabellarequisitiqualità}
\end{tabularx}

%%##################### FINE requisiti di qualità ############################

\section{Tracciamento dei requisiti}
\subsection{Tracciamento fonti-requisiti} 

\begin{tabularx}{\textwidth}{c|r}
	
	\rowcolor{greySWEight}
	\textcolor{white}{\textbf{Fonte}} &
	\textcolor{white}{\textbf{Requisito}}\endhead
	
	Capitolato & R-1FU001\\
	& R-1FU002\\
	& R-1FA001\\
	& R-3FA002\\
	& R-1FA003\\
	& R-1FA004\\
	& R-1FA005\\
	& R-1FA006\\
	& R-1FA007\\
	& R-1FA008\\
	& R-1FI003\\
	& R-1FI007\\
	& R-1FI008\\
	& R-1FI009\\
	& R-1FI010\\
	& R-1FS001\\
	& R-2FS002\\
	& R-1FS003\\
	& R-2FS006\\
	& R-2FS007\\
	& R-2FS008\\
	& R-2FS009\\
	& R-1FS011\\
	& R-1V001\\
	& R-1V002\\
	& R-1V003\\
	& R-1V004\\
	 
	
	
	\caption{Tabella tracciamento capitolato-requisiti} \label{tab:tabellafonterequisiti}
\end{tabularx}

\begin{tabularx}{\textwidth}{c|r}
	
	\rowcolor{greySWEight}
	\textcolor{white}{\textbf{Fonte}} &
	\textcolor{white}{\textbf{Requisito}}\endhead
	
	Interno & R-3FI001\\
	& R-1FI002\\
	& R-3FI004\\
	& R-3FI005\\
	& R-3FI006\\
	& R-3FI011\\
	& R-3FI012\\
	& R-3FI013\\
	& R-3FI014\\
	& R-2FS004\\
	& R-2FS005\\
	& R-1FS010\\
	& R-3FS012\\
	& R-3FS013\\
	& R-2FAM001\\
	& R-2FAM002\\
	& R-2FAM003\\
	& R-2FAM004\\
	& R-2FAM005\\
	& R-1Q1001\\
	& R-1Q1002\\
	& R-1Q1003\\
	& R-1Q1005\\
	
	
	\caption{Tabella tracciamento Interno-requisiti} \label{tab:tabellafonterequisiti}
\end{tabularx}
\end{document}


