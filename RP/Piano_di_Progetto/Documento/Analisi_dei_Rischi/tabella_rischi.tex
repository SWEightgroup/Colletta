\newcommand{\barra}{\hline}
\renewcommand{\arraystretch}{1.5}
\def\tabularxcolumn#1{m{#1}}
\begin{tabularx}{\textwidth}{C{2.5cm} C{3cm} X C{3cm}}
\rowcolor{greySWEight}
	\textcolor{white}{\textbf{Codice}} &
    \textcolor{white}{\textbf{Nome}} & \textcolor{white}{\textbf{Descrizione}}&
    \textcolor{white}{\textbf{Livello di Rischio}}\endhead
    A-001 &
 \textbf  	
 	{Conflitti fra i membri del gruppo}&
    Per molti membri del gruppo questa è la prima esperienza di lavoro in gruppo con un certo
    numero di persone: ciò potrebbe causare inconvenienti di natura interpersonale.&
    
    Probabilità: \newline \textbf{Bassa}\newline
    Gravità: \newline \textbf{Alta}\\
    
    Contromisure &
    \multicolumn{3}{p{13.2cm}}{
    Ogni problema andrà tempestivamente riportato al responsabile. Ove non sia possibile
    trovare una soluzione, il responsabile cercherà di assegnare ruoli e attività che 
    minimizzino l'interazione fra i membri in causa.
    }\\
    \barra
    
    
 A-002 &
 \textbf
 	{Assenza prolungata di un membro del gruppo}&
    È possibile che, a causa di problemi di salute o familiari, un membro del gruppo possa
    non poter svolgere le sue mansioni per un certo periodo di tempo. &
    Probabilità: \newline \textbf{Bassa}\newline
    Gravità: \newline \textbf{Alta}\\
    
    Contromisure&
    \multicolumn{3}{p{13.2cm}}{
    A seconda della natura del problema e delle attività lasciate in sospeso, il responsabile
    può ridistribuire il carico di lavoro del membro assente o posticiparle e rivedere
    la pianificazione
    }\\
    \barra

    B-001 &
\textbf
    {Incompatibilità orari dei membri del gruppo}&
   A causa di ubicazione geografica e diversi impegni universitari e lavorativi
   dei vari membri del gruppo, può essere complicato incontrarsi di persona per
   discutere del progetto.&
   Probabilità: \newline \textbf{Alta}\newline
   Gravità: \newline \textbf{Bassa}\\
   
   Contromisure&
   \multicolumn{3}{p{13.2cm}}{    
   Creazione di una tabella oraria con gli impegni di ogni membro del gruppo. \newline
   Ogni riunione avrà uno scopo ben preciso, e ogni membro è tenuto a prepararsi
   attentamente per sfruttare al meglio il tempo disponibile. \newline
   In caso non sia possibile organizzare un incontro fisico, è sempre possibile
   discutere in videoconferenza.
   Una riunione può essere svolta anche in assenza 2 membri.
   }\\
   \barra
   
  M-001 &
\textbf
    {Inesperienza tecnologica}&
   Alcuni membri del gruppo potrebbero non conoscere alcune delle tecnologie utilizzate nel
   progetto&
   Probabilità: \newline \textbf{Alta}\newline
   Gravità: \newline \textbf{Media}\\
   
   Contromisure&
   \multicolumn{3}{p{13.2cm}}{    
   Studio individuale delle tecnologie sconosciute, eventualmente coadiuvato da un membro più
   esperto in una determinata tecnologia. L'assegnazione delle attività deve tenere conto delle
   conoscenze tecnologiche degli assegnatari.
   }\\
   \barra
   
M-002 &   
\textbf   
   	{Danni hardware e software}&
   Possibili malfunzionamenti agli strumenti di lavoro possono rallentare lo svolgimento delle attività o
   causare la perdita di lavoro già svolto.
   &
   Probabilità: \newline \textbf{Bassa}\newline
   Gravità: \newline \textbf{Media}\\
   
   Contromisure&
   \multicolumn{3}{p{13.2cm}}{    
   Ogni incremento significativo nello svolgimento di un'attività va tempestivamente versionato nel cloud.
 
   }\\
   \barra
   
A-003 &
\textbf
   {Inesperienza organizzativa}&
   Nessuno dei membri del team ha mai lavorato ad un progetto con questo livello di organizzazione.
   Pertanto, può risultare difficile stimare il costo temporale delle attività e organizzare quest'ultime
   nel tempo.
   &
   Probabilità: \newline \textbf{Alta}\newline
   Gravità: \newline \textbf{Alta}\\
   
   Contromisure&
   \multicolumn{3}{p{13.2cm}}{    
   Ogni attività è comprensiva di un periodo di slack. Pianificare {milestone}\ped{G} più frequenti durante le prime
   fasi del progetto, in modo da limitare il margine di errore e da correggere il tiro con i consuntivi.
 
   }\\
   \barra
   
A-004 &
\textbf
   {Cambiamento dei requisiti da parte della proponente}&
   Per vari motivi, la proponente MIVOQ potrebbe decidere di aggiungere o rimuovere dei 
   requisiti.
   &
   Probabilità: \newline \textbf{Bassa}\newline
   Gravità: \newline \textbf{Alta}\\
   
   Contromisure&
   \multicolumn{3}{p{13.2cm}}{    	
	Tempestiva revisione dell'analisi dei requisiti. In caso siano già iniziate le fasi di progettazione o codifica, interruzione delle attività in corso in attesa della revisione dell'analisi dei requisiti, e dirottamento delle risorse su altre attività. In ogni caso, lavoro
	a stretto contatto con la proponente, per averne ben chiari tutti i requisiti.
   }\\
   \barra
   
A-005 &   
\textbf
   {Analisi dei requisiti incompleta}&
\'E possibile che, a causa di inesperienza o incompetenza, l'Analisi dei requisiti possa non essere completa, causando problemi nei periodi successivi.
   &
   Probabilità: \newline \textbf{Media}\newline
   Gravità: \newline \textbf{Alta}\\
   
   Contromisure&
   \multicolumn{3}{p{13.2cm}}{    	
Particolare attenzione va posta alla verifica dell'analisi dei requisiti, in ogni periodo dello sviluppo, da parte di ciascuno dei membri del team. Dialogo continuo con la {proponente}\ped{G}.
   }\\
   \barra
   
    M-003 &
\textbf
   {Pianificazione fallacea}&
È possibile che, a causa di inesperienza o incompetenza, la pianificazione possa non essere corretta per alcuni obbiettivi e/o attività. Ciò potrebbe portare ad uno spreco di tempo di  risorse o causare ritardi, oltre a far differire il costo finale da quello preventivato.
   &
   Probabilità: \newline \textbf{Media}\newline
   Gravità: \newline \textbf{Media}\\
   
   Contromisure&
   \multicolumn{3}{p{13.2cm}}{    	
Creazione di più milestone, in modo da poter più facilmente ripianificare in caso di errori. Nel caso si verifichi il rischio di un ritardo, verrà valutato se ridistribuire alcuni compiti o riorganizzare la pianificazione.
   }\\
   \barra        
  
\caption{Analisi dei rischi} \label{tab:tabellarischi} 
\end{tabularx}



