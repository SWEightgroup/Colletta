\paragraph{Diagrammi UML}\mbox{}\\
Con l'obiettivo di rendere chiare le soluzioni progettuali utilizzate, è necessario l'utilizzo di {diagrammi UML}\ped{G}. Quest'ultimi devono essere realizzati utilizzando lo {standard}\ped{G} 2.0.

È richiesta la realizzazione di:
\begin{itemize}
	\item[•] \textbf{Diagrammi delle attività}: descrivono un processo o un algoritmo;
	\item[•] \textbf{Diagrammi dei package}: per raggruppare elementi e fornire un {namespace}\ped{G} per gli elementi raggruppati;
	\item[•] \textbf{Diagrammi di sequenza}: rappresentano una sequenza di processi o funzioni;
	\item[•] \textbf{Diagrammi di classi}: rappresentano le classi utilizzate e le loro relazioni.
\end{itemize}
In caso vengano utilizzati dei {Design Pattern}\ped{G} sarà necessario accompagnarli con una descrizione ed un diagramma UML.

\subsubsection{Progettazione}\mbox{}\\
Dopo aver terminato la fase di Analisi si passerà a quella di Progettazione che vede come protagonisti i \textit{Progettisti}, durante la quale questi devono trovare una soluzione soddisfacente al problema, definire un'architettura logica e i vari diagrammi che la rappresentano.
La Progettazione permette di: 
\begin{itemize}
\item[•] Ottimizzare l'uso delle risorse;
\item[•] Garantire la qualità del prodotto sviluppato;
\item[•] Suddividere il problema principale in tanti sotto problemi di complessità minore.
\end{itemize}

\paragraph{Architettura logica}\mbox{}\\
Bisogna definire un'{architettura}\ped{G} logica del prodotto che dovrà: 
\begin{itemize}
\item[•] Soddisfare i requisiti definiti nel documento di \AdR;
\item[•] Essere sicura in caso di malfunzionamenti o intrusioni;
\item[•] Essere {modulare}\ped{G} e formato da componenti riutilizzabili;
\item[•] Essere affidabile;
\item[•] Essere comprensibile per future manutenzioni.
\end{itemize}

\paragraph{Design Pattern}\mbox{}\\
I \textit{Progettisti} devono utilizzare il design pattern che ritengono più adatto al contesto per rendere l'applicazione più sicura ed efficiente possibile.
Ogni utilizzo di design pattern deve essere brevemente descritto ed accompagnato da un diagramma UML che ne esemplifica il funzionamento.

\subsubsection{Qualità}

\paragraph{Classificazione dei test}\mbox{}\\
I test implementati devono essere classificati secondo la seguente notazione:
\begin{center}
	T[Tipologia Test][Codice identificativo]
\end{center}
dove:
\begin{itemize}
	\item[•] Tipologia test: indica il tipo di test e può assumere i seguenti valori:
	\begin{list}{$\circ$}{}
		\item U: per i test di unità;
		\item I: per i test di integrazione;
		\item S: per i test di sistema;
	\end{list}
	\item[•] Codice identificativo: indica il codice numerico del test.
\end{itemize}

\paragraph{Test di unità}\mbox{}\\
Devono essere definiti dei testi di unità necessari a garantire che tutte le componenti del sistema funzionino correttamente.

\paragraph{Test di integrazione}\mbox{}\\
Devono essere definite le classi di verifica necessarie a garantire che tutte le componenti del sistema funzionino correttamente.

\paragraph{Test di sistema}\mbox{}\\
Richiede che i vari componenti del sistema vengano integrati al fine di garantire che tutte le componenti del sistema funzionino correttamente.

\subsubsection{Codifica}
In questa sezione vengono descritte le norme che i \textit{Programmatori} devono seguire con l'obiettivo di scrivere codice leggibile, affidabile e mantenibile.

\paragraph{Stile di codifica}\mbox{}\\
Al fine di produrre codice uniforme, leggibile e manutenibile è richiesto che vengano rispettate
le seguenti convenzioni:
\begin{itemize}
\item[•] I nomi utilizzati devono essere chiari, descrittivi rispetto alla loro funzione e in inglese;
\item[•] Evitare nomi troppo simili tra loro che possano creare difficoltà nella comprensione del codice;
\item[•] Deve essere presente almeno un breve commento descrittivo per ogni classe e metodo;
\item[•] I commenti devono essere scritti in lingua inglese senza utilizzare abbreviazioni o altre ambiguità;
\item[•] Le modifiche al codice devono sempre riflettersi sui relativi commenti;
\item[•] Evitare commenti superflui, inappropriati o scurrili;
\item[•] Oogni file deve presentare un’intestazione con le seguenti informazioni:
\begin{list}{$\circ$}{}
\item Percorso e nome del file;
\item Nome e cognome dell’autore;
\item Data di creazione;
\item Breve descrizione del contenuto del file.
\end{list}
\end{itemize}
Il codice deve seguire le linee guida reperibili all’indirizzo:
\begin{center}
\url{http://google.github.io/styleguide/};
\end{center}
\paragraph{Documentazione}\mbox{}\\
fare

\subsubsection{Strumenti di supporto}
\paragraph{Tracciamento}\mbox{}\\ \label{sec:Trac}
Il tracciamento dei requisiti e dei casi d’uso avviene attraverso la piattaforma \textit{RQConnect}, installata su un server {Firebase}\ped{G} ad hoc per il gruppo \gruppo. È  un tool sviluppato da un componente del gruppo in occasione di questo progetto e liberamente disponibile su GitHub.\\ E' possibile aggiungere requisiti e casi d’uso alla piattaforma e successivamente collegarli tra di loro, la schermata principale è divisa in due colonne, in quella a sinistra c’è l’elenco dei requisiti e in quella a destra i casi d’uso, cliccando su un elemento si apre una vista dettagliata nella quale è possibile leggere i dettagli e collegare l’elemento, cliccando sul pulsante risolvi si apre una finestra dalla quale è possibile selezionare gli elementi da collegare con l’aiuto di un menu a tendina.\\Una volta completato l’inserimento ed il collegamento è possibile scaricare l’intera lista sottoforma di tabella LaTex.


