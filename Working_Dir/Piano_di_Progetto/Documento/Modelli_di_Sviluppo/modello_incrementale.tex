L'analisi dei rischi ha stimato un basso rischio di cambio dei requisiti da parte della proponente.
Ciò significa che è possibile adottare un modello di sviluppo che prevede un'approfondita fase di 
analisi e progettazione iniziale.
La scelta è quindi ricaduta sul modello di sviluppo incrementale, che, inoltre, presenta altre
caratteristiche congeniali alle intenzioni del gruppo \gruppo \space :
\begin{itemize}
    \item Lo sviluppo del prodotto software è diviso in varie attività singole, comportando i seguenti vantaggi:
    \begin{itemize}
    	\item Maggior semplicità dello sviluppo di un'attività;
    	\item Maggior semplicità in fase di test e {troubleshooting}\ped{G};
    	\item Migliore parallelizzazione del lavoro;
    	\item Minore probabilità di incorrere in ritardi;
    	\item Maggiore facilità nel rilascio;
    \end{itemize}
    \item Le attività sono ordinate in ordine di importanza, e vengono sviluppati prima i requisiti di maggiore rilevanza per il committente;
    \item Creazione di milestone per suddividere meglio il lavoro e per ottenere riscontri dal committente sul soddisfacimento dei requisiti.
    \item Possibilità di pianificare le attività atte alla realizzazione del {Proof of Concept}\ped{G} e della baseline architetturale per adempire ai rispettivi obblighi imposti da Technology Baseline e Product Baseline. In particolare, il rendere il Proof of Concept parte integrante del prodotto comporterà un risparmio in termini di tempo e denaro.
\end{itemize}

\subsection{Numero incrementi massimi}
L'adozione del modello incremetale comporta anche il fissare il numero massimo di incrementi, pertanto il gruppo ha deciso di fissare tale limite a 10. Questo permette di realizzare {milestone}\ped{G} di riferimento durante il progetto, garantendo allo stesso tempo una analisi preventiva dei problemi che potrebbero insorgere. Ad ogni milestone corrisponde una baseline, alla quale viene associato un incremento significativo del prodotto. Il primo incremento significativo è il {Proof of Concept}\ped{G} che contiene le funzionalità basilari dell'applicativo per agevolare le attività di codifica successive. In seguito alla \RP{} gli incrementi interesseranno:
\begin{enumerate}
	\item Sviluppo requisiti obbligatori, che essendo sviluppati per primi saranno soggetti ad un'attività di verifica più accurata;
	\item Integrazione delle componenti sviluppate, onde evitare il problema dell'{integration big-bang}\ped{G};
	\item L'uso di {feature branch}\ped{G} permetterà lo sviluppo delle funzionalità del prodotto in maniera parallela;
	\item Viene minimizzato il rischio di non soddisfacimento dei requisiti fondamentali grazie alla possibilità di feedback anticipato da parte della Proponente.
\end{enumerate}