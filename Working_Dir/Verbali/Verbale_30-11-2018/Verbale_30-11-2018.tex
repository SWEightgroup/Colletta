\documentclass[a4paper, oneside, openany, dvipsnames, table]{article}
\usepackage{../../template/SWEightStyle}
\newcommand{\Titolo}{Verbale Riunione 2018-12-12}

\newcommand{\Gruppo}{SWEight}

\newcommand{\ACapoRedazione}{Francesco Magarotto}

\newcommand{\Verifica}{Francesco Corti}

\newcommand{\Approvazione}{Sebastiano Caccaro}

\newcommand{\Distribuzione}{Vardanega Tullio \newline Cardin Riccardo \newline Gruppo SWEight}

\newcommand{\Uso}{Interno}

\newcommand{\NomeProgetto}{Colletta}

\newcommand{\Mail}{SWEightGroup@gmail.com}

\newcommand{\DescrizioneDoc}{Questo documento si occupa di riportare quanto discusso nella riunione del 12-12-2018}


\begin{document}
\copertina{}

\definecolor{greySWEight}{RGB}{255, 71, 87}
\definecolor{greyROwSWEight}{RGB}{234, 234, 234}

\section*{Registro delle modifiche}
{
	\rowcolors{2}{greyROwSWEight}{white}
	\renewcommand{\arraystretch}{1.5}
	\centering
	\begin{longtable}{ c c  C{4cm}  c  c }
		
		\rowcolor{greySWEight}
		\textcolor{white}{\textbf{Versione}} & \textcolor{white}{\textbf{Data}} & \textcolor{white}{\textbf{Descrizione}} & \textcolor{white}{\textbf{Nominativo}} & \textcolor{white}{\textbf{Ruolo}}\\
		
		1.0.2 & 2019-03-02 & Aggiunti nuovi termini del documento Piano di Progetto & Isachi Gheorghe &\reda{}\\
		
		1.0.1 & 2019-02-23 & Verifica del documento &  Francesco Corti & \ver{}\\
		
		1.0.1 & 2019-02-20 & Aggiunti nuovi termini del documento Norme di Progetto & Isachi Gheorghe &\reda{}\\
		
		1.0.0 & 2019-01-09 & Approvazione & Sebastiano Caccaro & \Res{}\\
						
		0.1.1 & 2019-01-08 & Verifica del documento & Bacco Alberto & \ver{}\\
		
		0.1.1 & 2019-01-04 & Aggiunti termini del documento Norme di Progetto & Isachi Gheorghe &\reda{}\\
		
		0.1.0 & 2019-01-01 & Aggiunti termini del documento Analisi dei Requisiti & Isachi Gheorghe &\reda{}\\
		
		0.0.4 & 2018-12-29 & Verifica del documento & Bacco Alberto & \ver{}\\
				
		0.0.4 & 2018-12-27 & Aggiunti termini del documento Piano di Qualifica & Isachi Gheorghe &\reda{}\\
				
		0.0.3 & 2018-12-26 &Aggiunti termini del documento Piano di Progetto & Isachi Gheorghe & \reda{}\\
				
		0.0.2 & 2018-12-17 & Aggiunti termini del documento Studio di Fattibilità & Isachi Gheorghe &\reda{}\\
		
		0.0.1 & 2018-12-15 & Scheletro del glossario & Damien Ciagola & \reda{}\\
		
	\end{longtable}

}
\newpage
\tableofcontents
\newpage
\section{Informazioni Generali}
\begin{itemize}
\item \textbf{Motivazione:} Discussione sul way of working;
\item \textbf{Luogo:} LabP036, Plesso Paolotti, Padova;
\item \textbf{Data:} 2018-11-30.
\item \textbf{Partecipanti del gruppo:} \hfill
	\begin{itemize}
	\item Bacco Alberto;
	\item Caccaro Sebastiano;
	\item Ciagola Damien;
	\item Corti Francesco;
	\item Isachi Gheorghe;
	\item Magarotto Francesco;
	\item Muraro Enrico.
	\end{itemize} 
\item \textbf{Ora:} 10:30 - 12:00;
\item \textbf{Segretario:} Caccaro Sebastiano.
\end{itemize}

\section{Ordine del Giorno}
\begin{description}
\item [Way of working: ] Stabilita gestione attività.
\end{description}

\section{Resoconto}


\subsection{Way of working}
\begin{itemize}
\item Stabilito monitoraggio dei progressi, verranno usati i seguenti strumenti:
\begin{itemize}
	\item Gestione delle milestone di GitHub;
	\item Project di GitHub, in configurazione Automated Kanban with reviews, per le macroattivià.
\end{itemize}


\item Stabilito processo di gestione delle attività in GitHub:
	\begin{enumerate}
	\item Un' attività viene creata dal responsabile o da un membro del gruppo. Ogni attività:
		\begin{itemize}
			\item Appartiene a un progetto;
			\item Appartiene a una milestone;
			\item Ha un'etichetta "enhancement".
		\end{itemize}
	\item L'attività viene assegnata ad un membro appropriato del gruppo;
	\item Quando il suddetto membro comincia a svolgere l'attività, la sposta su "In Progess"
	\item Una volta finito lo svolgimento dell'attività, lo stesso membro che l'ha svolta la sposta
		  nella colonna "Pending Approval" e aggiunge l'etichetta "Da verificare";
	\item Un verificatore prende in carico l'attività. Dopo aver svolto la verifica può:
	\begin{itemize}
		\item Togliere l'etichetta "Da verificare", e spostare l'attività in "Pending Approval" se la verifica non ha trovato
			  problemi;
		\item Togliere l'etichetta "Da verificare", e far tornare l'attività a "In Progress" se ha trovato particolari problemi,
		      segnalati tramite commenti alla issue.
	\end{itemize}
	\item Il responsabile decide se validare lo svolgimento dell'attività:
	\begin{itemize}
		\item In caso positivo, l'attività viene chiusa e segnata come completata;
		\item In caso negativo, torna allo step precedente.
	\end{itemize}
	\end{enumerate}	
\item Valulato l'utilizzo di GNU Aspell per la correzione ortografica dei documenti.
	  Necessari approfondimenti.
\end{itemize}




\newpage
\end{document}