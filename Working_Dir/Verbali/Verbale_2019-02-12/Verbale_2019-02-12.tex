\documentclass[a4paper, oneside, openany, dvipsnames, table]{article}
\usepackage{../../template/SWEightStyle}
\newcommand{\Titolo}{Verbale Riunione 2018-12-12}

\newcommand{\Gruppo}{SWEight}

\newcommand{\ACapoRedazione}{Francesco Magarotto}

\newcommand{\Verifica}{Francesco Corti}

\newcommand{\Approvazione}{Sebastiano Caccaro}

\newcommand{\Distribuzione}{Vardanega Tullio \newline Cardin Riccardo \newline Gruppo SWEight}

\newcommand{\Uso}{Interno}

\newcommand{\NomeProgetto}{Colletta}

\newcommand{\Mail}{SWEightGroup@gmail.com}

\newcommand{\DescrizioneDoc}{Questo documento si occupa di riportare quanto discusso nella riunione del 12-12-2018}


\begin{document}
\copertina{}

\definecolor{greySWEight}{RGB}{255, 71, 87}
\definecolor{greyROwSWEight}{RGB}{234, 234, 234}

\section*{Registro delle modifiche}
{
	\rowcolors{2}{greyROwSWEight}{white}
	\renewcommand{\arraystretch}{1.5}
	\centering
	\begin{longtable}{ c c  C{4cm}  c  c }
		
		\rowcolor{greySWEight}
		\textcolor{white}{\textbf{Versione}} & \textcolor{white}{\textbf{Data}} & \textcolor{white}{\textbf{Descrizione}} & \textcolor{white}{\textbf{Nominativo}} & \textcolor{white}{\textbf{Ruolo}}\\
		
		1.0.2 & 2019-03-02 & Aggiunti nuovi termini del documento Piano di Progetto & Isachi Gheorghe &\reda{}\\
		
		1.0.1 & 2019-02-23 & Verifica del documento &  Francesco Corti & \ver{}\\
		
		1.0.1 & 2019-02-20 & Aggiunti nuovi termini del documento Norme di Progetto & Isachi Gheorghe &\reda{}\\
		
		1.0.0 & 2019-01-09 & Approvazione & Sebastiano Caccaro & \Res{}\\
						
		0.1.1 & 2019-01-08 & Verifica del documento & Bacco Alberto & \ver{}\\
		
		0.1.1 & 2019-01-04 & Aggiunti termini del documento Norme di Progetto & Isachi Gheorghe &\reda{}\\
		
		0.1.0 & 2019-01-01 & Aggiunti termini del documento Analisi dei Requisiti & Isachi Gheorghe &\reda{}\\
		
		0.0.4 & 2018-12-29 & Verifica del documento & Bacco Alberto & \ver{}\\
				
		0.0.4 & 2018-12-27 & Aggiunti termini del documento Piano di Qualifica & Isachi Gheorghe &\reda{}\\
				
		0.0.3 & 2018-12-26 &Aggiunti termini del documento Piano di Progetto & Isachi Gheorghe & \reda{}\\
				
		0.0.2 & 2018-12-17 & Aggiunti termini del documento Studio di Fattibilità & Isachi Gheorghe &\reda{}\\
		
		0.0.1 & 2018-12-15 & Scheletro del glossario & Damien Ciagola & \reda{}\\
		
	\end{longtable}

}
\newpage
\tableofcontents
\newpage
\section{Informazioni Generali}
\begin{itemize}
\item \textbf{Motivazione:} Discussione sulle tecnologie e strumenti da utilizzare;
\item \textbf{Luogo:} Google Hangouts;
\item \textbf{Data:} 2019-02-12.
\item \textbf{Partecipanti del gruppo:} \hfill
	\begin{itemize}
	\item Bacco Alberto;
	\item Caccaro Sebastiano;
	\item Ciagola Damien;
	\item Corti Francesco;
	\item Isachi Gheorghe;
	\item Legrottaglie Gionata;
	\item Magarotto Francesco;
	\item Muraro Enrico.
	\end{itemize} 
\item \textbf{Ora:} 19:00 - 20:30;
\item \textbf{Segretario:} Caccaro Sebastiano.
\end{itemize}

\section{Ordine del Giorno}
\begin{itemize}
	\item \textbf{Discussione tecnologie da utilizzare:} Sono state scelte le tecnologie da utilizzare per la PoC e per il lo sviluppo del progetto;
	\item \textbf{Poc:} Sono state discusse la forma e le modalità della PoC;
	\item \textbf{Documentazione:} Sono stati discussi incrementi da apportare alla documentazione.
\end{itemize}

\section{Resoconto}
\subsection{Discussione sulle tecnologie da utilizzare}
Prima della riunione, ogni membro era stato chiamato a documentarsi sulle possibili tecnologie da utilizzare per la parte di realizzazione vera e propria del progetto didattico. Dopo averne discusso, la scelta è ricaduta su:
\begin{itemize}
	\item \textbf{Docker 18.09.1:} Per ridurre il rischio di incorrere in problematiche riguardanti la configurazione del software, quando possibile, andranno usate immagini Docker per uniformare l'ambiente di esecuzione fra i membri del gruppo;
	\item \textbf{Python 3.6.3}: Verrà usato Python per sviluppare le rest API per l'analisi grammaticale con FreeLing. Ci si avvalerà, inoltre, dei test di unità e della documentazione integrati nel linguaggio;
	\item \textbf{Node Js 10.15.1}: Node Js è la piattaforma sulla quale verrà sviluppata la parte dinamica del progetto in Javascript;
	\item \textbf{React:} il gruppo \gruppo \ userà il framework REACT per la programmazione dell'interfaccia utente. Ciò comporta l'adozione del design pattern FLUX;
	\item \textbf{Bootstrap:} Bootstrap verrà usato per curare l'estetica del sito.
\end{itemize}

Si è inoltre discussa l'adozione, da parte di tutti i membri del gruppo, di \textit{Visual Studio Code} come ambiente di sviluppo principale, e del plugin \textit{Prettier} per far aderire lo stile del codice a quanto fissato nelle \NdP .

\subsection{Proof of Concept}
Si è discusso sugli obiettivi della PoC, e si è arrivati alle seguenti conclusioni:
\begin{itemize}
	\item La PoC deve essere il primo incremento per la realizzazione del progetto. Pertanto essa non può essere un pacchetto usa e getta, ma al contrario, deve essere la base per lo sviluppo nei successivi periodi;
	\item La PoC dovrà includere l'analisi e la correzione di frasi, almeno in lingua italiana. Ciò richiede lo sviluppo delle Rest API. La raccolta dati e le divisioni per utenti possono essere trattate nei successivi periodi.
\end{itemize}
	
\subsection{Documentazione}
Sono state discussi i prossimi incrementi da apportare alla documentazione, e si è fatto il punto sulle correzioni post RR da apportare:
\begin{itemize}
	\item \textbf{Pdp: }Leggera ripianificazione per dare maggior rilievo alla Proof of Concept.
	\item \textbf{Pdq: }Rimozione di alcune sezioni già spostate nelle Norme. Aggiungere specifiche dei test come previsto dalle \NdP .
\end{itemize}



\end{document}