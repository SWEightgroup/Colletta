\documentclass[a4paper, oneside, openany, dvipsnames, table]{article}
\usepackage{../../template/SWEightStyle}
\newcommand{\Titolo}{Verbale Riunione 2018-12-12}

\newcommand{\Gruppo}{SWEight}

\newcommand{\ACapoRedazione}{Francesco Magarotto}

\newcommand{\Verifica}{Francesco Corti}

\newcommand{\Approvazione}{Sebastiano Caccaro}

\newcommand{\Distribuzione}{Vardanega Tullio \newline Cardin Riccardo \newline Gruppo SWEight}

\newcommand{\Uso}{Interno}

\newcommand{\NomeProgetto}{Colletta}

\newcommand{\Mail}{SWEightGroup@gmail.com}

\newcommand{\DescrizioneDoc}{Questo documento si occupa di riportare quanto discusso nella riunione del 12-12-2018}


\begin{document}
\copertina{}

\definecolor{greySWEight}{RGB}{255, 71, 87}
\definecolor{greyROwSWEight}{RGB}{234, 234, 234}

\section*{Registro delle modifiche}
{
	\rowcolors{2}{greyROwSWEight}{white}
	\renewcommand{\arraystretch}{1.5}
	\centering
	\begin{longtable}{ c c  C{4cm}  c  c }
		
		\rowcolor{greySWEight}
		\textcolor{white}{\textbf{Versione}} & \textcolor{white}{\textbf{Data}} & \textcolor{white}{\textbf{Descrizione}} & \textcolor{white}{\textbf{Nominativo}} & \textcolor{white}{\textbf{Ruolo}}\\
		
		1.0.2 & 2019-03-02 & Aggiunti nuovi termini del documento Piano di Progetto & Isachi Gheorghe &\reda{}\\
		
		1.0.1 & 2019-02-23 & Verifica del documento &  Francesco Corti & \ver{}\\
		
		1.0.1 & 2019-02-20 & Aggiunti nuovi termini del documento Norme di Progetto & Isachi Gheorghe &\reda{}\\
		
		1.0.0 & 2019-01-09 & Approvazione & Sebastiano Caccaro & \Res{}\\
						
		0.1.1 & 2019-01-08 & Verifica del documento & Bacco Alberto & \ver{}\\
		
		0.1.1 & 2019-01-04 & Aggiunti termini del documento Norme di Progetto & Isachi Gheorghe &\reda{}\\
		
		0.1.0 & 2019-01-01 & Aggiunti termini del documento Analisi dei Requisiti & Isachi Gheorghe &\reda{}\\
		
		0.0.4 & 2018-12-29 & Verifica del documento & Bacco Alberto & \ver{}\\
				
		0.0.4 & 2018-12-27 & Aggiunti termini del documento Piano di Qualifica & Isachi Gheorghe &\reda{}\\
				
		0.0.3 & 2018-12-26 &Aggiunti termini del documento Piano di Progetto & Isachi Gheorghe & \reda{}\\
				
		0.0.2 & 2018-12-17 & Aggiunti termini del documento Studio di Fattibilità & Isachi Gheorghe &\reda{}\\
		
		0.0.1 & 2018-12-15 & Scheletro del glossario & Damien Ciagola & \reda{}\\
		
	\end{longtable}

}
\newpage
\tableofcontents
\newpage
\section{Informazioni Generali}
\begin{itemize}
\item \textbf{Motivazione:} fissare un incontro con la proponente per mostrare il prodotto soggetto a validazione;
\item \textbf{Luogo:} Google Hangouts;
\item \textbf{Data:} 2019-05-02;
\item \textbf{Partecipanti del gruppo:} \hfill
	\begin{itemize}
		\item Damien Ciagola;
		\item Francesco Corti;
		\item Francesco Magarotto;
		\item Enrico Muraro;
		\item Alberto Bacco.
		\item Sebastiano Caccaro;
		\item Gionanta Legrottaglie.
	\end{itemize} 
\item \textbf{Ora:} 19:00;
\item \textbf{Segretario:} Francesco Magarotto.
\end{itemize}

\section{Ordine del Giorno}
\begin{itemize}
	\item \textbf{Contatto con la proponente}: contattare la proponente per mostrare il prodotto realizzato;
	\item \textbf{Verifica dei requisiti svolti}: si sono verificati i requisiti da completare in vista della conclusione del progetto.
\end{itemize}

\subsection{Resoconto}
\begin{itemize}
	\item \textbf{Email - VER-5-2019-05-02.1}: si è scritta una mail alla proponente per fissare un incontro al fine di mostrare le funzionalità offerte dal prodotto commissionato;
	\item \textbf{Modelli AI - VER-5-2019-05-02.2}: i requisiti opzionali relativi ai modelli per l'allenamento dell'algoritmo di apprendimento automatico sono troppo onerosi rispetto alle risorse a disposizione pertanto non verranno realizzate.
	
\end{itemize}
\end{document}