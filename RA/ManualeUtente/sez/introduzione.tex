\section{Introduzione}
\subsection{Scopo del documento}
Il documento \`e una guida rivolta agli utenti utilizzatori della piattaforma di analisi grammaticale \textit{Colletta}, siano essi allievi, insegnanti o sviluppatori. Lo scopo è di illustrare brevemente gli aspetti di base del prodotto e le possibili interazioni che un'utente può avere con l'applicazione.
\subsection{Scopo del prodotto}
Il prodotto è una piattaforma collaborativa di raccolta dati in cui gli utenti possono predisporre e/o svolgere piccoli esercizi di analisi grammaticale. Lo scopo è raccogliere dati relativi sia  agli esercizi predisposti, che al loro svolgimento da parte degli utenti. Sviluppatori e ricercatori utilizzeranno queste informazione per insegnare ad un elaboratore a svolgere i medesimi esercizi, mediante tecniche di apprendimento automatico.
\subsection{Glossario}
In appendice al documento \`e stato inserito un glossario contenente tutti i termini necessari alla piena comprensione del testo. Essendo \textit{Colletta} destinato a una utenza con basse conoscenze nel dominio informatico, cercheremo di essere più semplici possibile per rendere comprensibile questa guida. Al fine di evitare incomprensioni, si precisa che ogni parola inserita a glossario verrà seguita da una G a pedice.

