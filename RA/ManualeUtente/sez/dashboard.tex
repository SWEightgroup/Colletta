\section{Istruzione per l'utilizzo}
  L'header della {dashboard}\ped{G} è uguale come struttura per ogni tipo di utente. Su di esso troviamo un link che porta alla modifica dei dati personali, chiamato \textit{Profilo}, e un link, \textit{Esci}, che, se cliccato, effettua il logout dal sistema.



\subsection{Interfaccia}
    \begin{figure}[H]
        \centering
   \includegraphics[width=17cm]{sez/img/istruzioni/panoramica.png} 
        \caption{Panoramica dell'interfaccia}\label{fig:1}
    \end{figure}
  La struttura generale di una pagina è la stessa per ogni utente loggato. Sono presenti i seguenti elementi di base:
    \begin{itemize}
        \item Barra del menu;
        \item {Sidebar}\ped{G};
        \item Contenuto della pagina.
    \end{itemize}
 A seconda del tipo di utente (Allievo, Insegnante, Sviluppatore, Amministratore) e della pagina selezionata, verranno visualizzati a schermo contenuti diversi. Ogni utente ha comunque a disposizione lo stesso header e un link al suo pannello utente.


\subsection{Utente non autenticato}
    \subsubsection{Registrazione}
    	\begin{figure}[H]
        	\centering
        	\includegraphics[width=1\linewidth]{sez/img/autenticazione/formRegistrazione.PNG} 
        	\caption{Form per la registrazione}\label{fig:registrazione}
    	\end{figure}
	  Se non si è ancora registrati è possibile farlo cliccando sul bottone \textit{Registrati} presente nella barra del menu. Una volta compilato il form sarà possibile accedere alla piattaforma come utente autenticato. Se si sceglie come ruolo \textit{sviluppatore} sarà necessario attendere una conferma da parte dell'amministratore prima di poter accedere. I dati sono tutti obbligatori e il form visualizzato sarà quello in \autoref{fig:registrazione}.\\
	  Dopo la registrazione, si verrà automaticamente loggati nel profilo appena creato, eccetto nel caso in cui l'utente si sia registrato come sviluppatore.
    \subsubsection{Login}
    	\begin{figure}[H]
        	\centering
        	\includegraphics[width=1\linewidth]{sez/img/autenticazione/formAccedi.PNG} 
        	\caption{Dati per effettuare l'accesso}\label{fig:1}
    	\end{figure}
 	  Dopo aver effettuato la registrazione si accede cliccando su \textit{Accedi} nella barra del menu. Si accede inserendo email e password.


% UTENTE AUTENTICATO
\subsection{Utente autenticato generico}

    \subsubsection{Logout}
    Per effettuare il {logout}\ped{G} si deve cliccare sulla voce \textit{Esci} dalla barra del menu. Facendo ciò, l'utente termina la propria sessione.
    \subsubsection{Modifica dati}
    Cliccando su \textit{Profilo} l'utente ha la possibilità di visualizzare e modificare i propri dati tramite un form simile a quello di registrazione. L'utente non può modificare il proprio ruolo e la propria e-mail dopo l'iscrizione.

    \subsection{Allievo}
      L'allievo si iscrive nel portale Colletta per svolgere esercizi di analisi grammaticale. Può svolgere esercizi assegnati da un'insegnante o svolgere esercizi liberi tramite il sistema di correzione automatica del sistema. L'allievo ha poi la possibilità di confrontare la sua soluzione con quella presentata dal sistema ricevendo anche un voto.
        \subsubsection{Sidebar}
          La sidebar dell'allievo presenta le seguenti voci:
            \begin{itemize}
                \item Pannello utente;
                \item Esercitazione libera;
                \item Compiti per casa;
                \item Esercizi svolti;
                \item Esercizi pubblici;
                \item Insegnanti preferiti.
            \end{itemize}
            

            
        \subsubsection{Pannello utente}
			\begin{figure}[H]
        		\centering
        		\includegraphics[width=1\linewidth]{sez/img/studente/panelloUtente.PNG} 
        		\caption{Panello utente: Allievo}\label{fig:1}
    		\end{figure}        
        
          Il pannello utente è un riassunto di tutti i progressi e le attività svolte dall'allievo. Il pannello utente mostra un messaggio di benvenuto, e delle informazioni utili come se li sono state assegnanti esercizi da fare, i suoi progressi, quanti esercizi ha svolto e quanti le ha resi pubblici. % l'allievo è quindi libero di selezionare una delle voci di menu ed esercitarsi.
%        	\begin{itemize}
%        		\item Progressi;
%        		\item Traguardi;
%        		\item Traguardo corrente;
%        		\item Prossimo traguardo;
%        		\item Valutazioni;
%        		\item Esercizi recenti;
%        		\item Insegnanti preferiti.
%        	\end{itemize}
        
        
               
	\newpage
        \subsubsection{Esercitazione libera}      
        	\begin{figure}[H]
                \centering
                \includegraphics[width=17cm]{sez/img/studente/esercitazioneLiberaEsegui.PNG} 
                \caption{Svolgimento esercizio libero}\label{fig:1}
        	\end{figure}
          In questa pagina è possibile svolgere un esercizio inserendo nella casella di testo una frase da analizzare. Prima di inserire la frase abbiamo la possibilità di selezionare la lingua per analizzare la frase nella lingua desiderata, attualmente possiamo scegliere tra \textit{italiano}  e \textit{inglese}.
          Se la frase non è stata inserita da nessun insegnante la correzione sarà generata automaticamente.\\
        \\ Svolgimento:
        	\begin{enumerate}        
            	\item Scrivere la frase da analizzare dentro al form;
            	\item Cliccare su \textit{Svolgi l'esercizio};
            	\item Svolgere l'esercizio e cliccare \textit{Completa}.
        	\end{enumerate}
        	\label{sec:esLib}
        	Lo svolgimento dell'esercizio è guidato da dei bottoni che aiutano l'allievo a eseguire un'analisi precisa. Ogni parola presenta dei bottoni sottostanti che rappresentano le scelte disponibili: cliccandoci sopra, il testo del bottone verrà aggiunto alla soluzione, e verrà mostrato un nuovo set di bottoni con ulteriori opzioni per l'analisi. In caso non comparissero più pulsanti, significa che l'analisi per quella parola è terminata. In ogni momento, l'allievo può decidere di resettare la soluzione per una determinata parola (icona cestino), o annullare l'ultima scelta (freccia indietro).

      
        \newpage
  		\subsubsection{Compiti per casa}
 		  In questa sezione l'allievo ha la possibilità di visualizzare gli esercizi a lui assegnati, scegliendo di svolgere uno qualsiasi fra gli esercizi elencati.
        	\begin{figure}[H]
            	\centering
            	\includegraphics[width=17cm]{sez/img/studente/compitopercasa.PNG} 
            	\caption{Scelta esercizio da risolvere}\label{fig:1}
        	\end{figure}

		  L'allievo sceglie l'esercizio da svolgere tra quelli che sono stati assegnati cliccando su \textit{Svolgi l'esercizio}.  
		  
		  
		   
       \subsubsection{Svolgimento esercizio scelto}
        	\begin{figure}[H]
            	\centering
            	\includegraphics[width=17cm]{sez/img/studente/svolgimentoesercizio.PNG} 
            	\caption{Svolgimento esercizio}\label{fig:1}
        	\end{figure}      
	L'allievo esegue quindi l'analisi della frase allo stesso modo descritto in \S\ref{sec:esLib}. La soluzione visualizzata alla fine sarà in questo caso quella dell'insegnante che ha assegnato l'esercizio.
        
        
        
        \subsubsection{Esercizi svolti}
        	\begin{figure}[H]
            	\centering
            	\includegraphics[width=17cm]{sez/img/studente/esercizisvolti.PNG} 
            	\caption{Storico esercizi svolti}\label{fig:1}
        	\end{figure}
          In questa pagina è possibile visualizzare lo storico degli esercizi che sono stati svolti. Per ogni esercizio svolto sono visualizzabili la data di aggiunta, la frase analizzata e il nome dell'insegnante che ha assegnato l'esercizio.
          
          
           \subsubsection{Esercizi pubblici}
        	\begin{figure}[H]
            	\centering
            	\includegraphics[width=17cm]{sez/img/studente/esercizipubblici.PNG} 
            	\caption{Elenco esercizi pubblici}\label{fig:1}
        	\end{figure}
        	In questa pagina è possibile visualizzare tutti gli esercizi pubblicati sia da insegnanti che altri allievi, abbiamo la possibilità  di selezionare tra tutti gli esercizi o solo quelli pubblicati dai nostri insegnanti preferiti e ovviamente possiamo svolgerli come un normale esercizio. 
        
        
        
\subsubsection{Insegnanti preferiti}
        	\begin{figure}[H]
            	\centering
            	\includegraphics[width=17cm]{sez/img/studente/insegnantepreferito.png} 
            	\caption{Elenco insegnanti preferiti}\label{fig:1}
        	\end{figure}        
        In questa pagina troviamo l'elenco di tutti gli insegnanti presenti nella lista di sinistra clicccando sul pulsante \textit{"+"} possiamo aggiungerli tra i nostri insegnanti preferiti, se invece si vuole rimuovere un insegnante dalla lista basta cliccare sul pulsante \textit{"-"} che compare se il cursore viene avvicinato all'insegnante  che si vuole rimuovere.
        
        
        
\newpage
    \subsection{Insegnante}
      L'insegnante è l'utente che può inserire esercizi privati o decidere di assegnarli ai propri allievi. 
        \subsubsection{Sidebar}
          La sidebar dell'insegnante presenta le seguenti voci:
        	\begin{itemize}
            	\item Pannello utente;
            	\item Inserisci esercizio;
            	\item Esercizi inseriti;
            	\item Gestione classi.
        	\end{itemize}
        
        
        
        \subsubsection{Pannello utente}
        \begin{figure}[H]
        		\centering
        		\includegraphics[width=1\linewidth]{sez/img/insegnante/panelloUtente.PNG} 
        		\caption{Panello utente: Allievo}\label{fig:1}
    		\end{figure}
          Il pannello utente mostra un messaggio di benvenuto all'insegnante, che poi è libero di spostarsi sulla sidebar e assegnare esercizi ai propri allievi.
        
        
        \subsubsection{Inserisci esercizio}
          Questa sezione da la possibilità all'insegnante di inserire un esercizio nel sistema.
        	\begin{figure}[H]
            	\centering
        		\includegraphics[width=17cm]{sez/img/insegnante/inserisciEsercizio.PNG} 
            	\caption{Inserimento e assegnazione esercizio}\label{fig:1}
        	\end{figure}
        
          Dopo aver inserito la frase, verrà visualizzata la correzione automatica. Se ritenuta errata c'è la possibilità di modificare la soluzione cliccando su \textit{Modifica}. Durante lo svolgimento dell'esercizio l'insegnante ha la possibilità di tornare indietro di un passo, o di resettare completamente la soluzione per ogni parola. Questo processo è analogo a quello presentato in \S\ref{sec:esLib}. Finita la correzione, l'insegnate ha la possibilità di assegnare l'esercizio ad uno o più allievi o ad una o più classi, e può decidere se renderlo pubblico o meno l'esercizio. Cliccando \textit{Completa} esso verrà aggiunto nel database.
        
        
        
        \subsubsection{Esercizi inseriti}
        
        Funzionalità al momento non disponibile. Questa sezione permette di visualizzare tutti gli esercizi inseriti, si possono vedere la frase inserita e la data di aggiunta.
        
        
        
        
        \subsubsection{Esercizi allievi}        
         Funzionalità al momento non disponibile. In questa sezione sono visibili i risultati dagli allievi.
        
        
        
        
	\newpage
    \subsection{Sviluppatore}
    Lo sviluppatore si iscrive al sito perché interessato a scaricare i dati prodotti dagli utenti durante l'esecuzione di esercizi di analisi grammaticale.
    	\subsubsection{Sidebar} 
    	  La sidebar dello sviluppatore presenta le seguenti voci:
    		\begin{itemize}
    			\item Pannello utente;
    			\item Pannello sviluppatore.
    		\end{itemize}
    
    
    
    
    	\subsubsection{Pannello utente}
    	  Il pannello utente mostra un messaggio di benvenuto allo sviluppatore, che è libero di procedere allo scaricamento dei dati prodotti dagli utenti.



    	\subsubsection{Pannello sviluppatore}
    		\begin{figure}[H]
				\centering
				\includegraphics[width=17cm]{sez/img/sviluppatore/datipronti.png}
				\caption{Scaricamento dati disponibili}\label{fig:1}
			\end{figure}
		  Cliccando sul bottone di \textit{Scarica}, lo sviluppatore può scaricare i dati prodotti dagli utenti, avrà successivamente la possibilità di filtrare questi dati.




	\newpage
	\subsection{Amministratore}
	L'amministratore ha la possibilità di gestire tutti gli utenti che non siano amministratori, inoltre può approvare o declinare le richieste di iscrizione degli sviluppatori.
		\subsubsection{Sidebar}
		  Voci nella sidebar:
			\begin{itemize}
				\item Pannello utente;
				\item Sviluppatori;
				\item Utenti.
			\end{itemize}



		\subsubsection{Pannello utente}
			\begin{figure}[H]
				\centering
				\includegraphics[width=17cm]{sez/img/amministratore/panelloadmin.PNG}
				\caption{Panoramica dei dati}\label{fig:1}
			\end{figure}
		
		 Il pannello dà il benvenuto all' amministratore, che è libero di procedere alla gestione degli utenti.
		 L'amministratore ha a disposizione un breve riassunto della situazione che ha de gestire, il panello si presenta diviso in 4 macrosezioni dove può più facilmente capire cosa deve fare, per esempio la sezione \textit{In Attesa} dice che ci sono richieste da parte degli sviluppatori.




		\subsubsection{Sviluppatori}
			\begin{figure}[H]
				\centering
				\includegraphics[width=17cm]{sez/img/amministratore/conf_ric_svil.PNG}
				\caption{Richieste di iscrizione degli sviluppatori}\label{fig:1}
			\end{figure}
		  In questa pagina l'amministratore può approvare o rifiutare le richieste di iscrizione degli sviluppatori. Viene presentata una lista contente gli sviluppatori che hanno richiesto di iscriversi al sistema. Ogni sviluppatore ha un nome e una mail. L'amministratore, premendo su \textit{Accetta}, consente allo sviluppatore di fare il login, se invece preme su \textit{Rifiuta}, l'amministratore cancella lo sviluppatore dal sistema. Premendo su \textit{Aggiorna}, l'amministratore aggiorna la lista di sviluppatori.


		\subsubsection{Utenti}
			\begin{figure}[H]
				\centering
				\includegraphics[width=17cm]{sez/img/amministratore/gestisciutenti.PNG}
				\caption{Gestione utenti}\label{fig:1}
			\end{figure}
		  In questa pagina l'amministratore può visualizzare gli utenti iscritti ed eventualmente eliminarli dal sito. Per ogni utente sono presenti nome, email e ruolo (Insegnante, Sviluppatore o Allievi). Gli amministratori non figurano in questo elenco. L'amministratore, cliccando su \textit{Elimina}, compare un alert dove li si chiede se è sicuro di voler eliminare l'utente selezionato, avendo la possibilità di ripensare prima di effettuare un'azione irreversibile, se invece è sicuro dell'azione, elimina l'utente dal sistema. Premendo su \textit{Aggiorna}, l'amministratore aggiorna la lista di utenti.
