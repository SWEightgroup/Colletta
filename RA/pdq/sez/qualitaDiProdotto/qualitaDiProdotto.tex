\section{Qualità di prodotto}
Per garantire la qualità dei prodotti, viene adottato lo standard 
ISO/IEC 9126 (vedi \NdP \space §4.2). Quest'ultimo permette di monitorare 
la qualità del software, fornendo delle metriche per misurarla.\newline
Sono fissati i seguenti obiettivi:
\begin{itemize}
	\item La \textbf{documentazione} deve essere:
		\begin{itemize}
			\item Facilmente leggibile;
			\item Scritta in modo corretto, secondo le regole della lingua italiana.
		\end{itemize}
	\item Il \textbf{software} deve:
		\begin{itemize}
			\item Soddisfare i requisiti stabiliti nell'\AdR ;
			\item Garantire semplicità di utilizzo;			
			\item Garantire semplicità di manutenzione;
			\item Garantire affidabilità.
		\end{itemize}
		
\end{itemize}
\subsection{Qualità dei documenti}


\subsubsection{MD001 - Indice Gulpease}\mbox{}
L'Indice Gulpease è un indice di leggibilità di un testo tarato sulla lingua italiana. 
\paragraph{Parametri adottati:}
\begin{itemize}
	\item Range accettabile: $[40 , 100]$
	\item Range ottimale: $[55 , 100]$
\end{itemize}

\subsection{Qualità software}
Alcune metriche per il software sono più adatte per alcuni linguaggi di programmazione e meno per altri. Come indicato nel \PdP , il gruppo \gruppo \space sceglierà quali linguaggi usare nel periodo di Progettazione Architetturale; pertanto, le metriche presenti in questa sezione non sono da considerarsi complete o definitive.\newline
Per alcune metriche, può mancare un'indicazione di valori accettabili e ottimali: ciò significa che il team si riserva di definirli in futuri incrementi.

\subsubsection{MS001 - Numero di Metodi}\mbox{}
Numero medio di metodi contenuti nelle classi di un package. 
\paragraph{Parametri adottati:}
\begin{itemize}
	\item Range accettabile: $ ]7,9]$;
	\item Range ottimale: $[0,7]$.
\end{itemize}

\subsubsection{MS002 - Numero di Parametri}\mbox{}
Numero di parametri passati a un metodo. 
\paragraph{Parametri adottati:}
\begin{itemize}
	\item Range accettabile: $ ]5,8]$;
	\item Range ottimale: $[0,5]$.
\end{itemize}

\subsubsection{MS003 - Funzioni di interfaccia per package}\mbox{}
Numero di funzione che un package espone. 
\paragraph{Parametri adottati:}
\begin{itemize}
	\item Range accettabile: $ ]10,20]$;
	\item Range ottimale: $[0,10]$.
\end{itemize}

\subsubsection{MS004 - Complessità Ciclomatica}\mbox{}
Numero di cammini linearmente indipendenti attraverso il grafo di controllo di flusso. 
\paragraph{Parametri adottati:}
\begin{itemize}
	\item Range accettabile: $ ]10,17]$;
	\item Range ottimale: $[0,10]$.
\end{itemize}

\subsubsection{MS005 - Campi dati per classe}\mbox{}
Numero di campi dati contenuti da una classe. 
\paragraph{Parametri adottati:}
\begin{itemize}
	\item Range accettabile: $ ]10,15]$;
	\item Range ottimale: $[0,10]$.
\end{itemize}

\subsubsection{MS006 - Commenti per linee di codice}\mbox{}
percentuale di righe di commento sul totale delle righe (righe vuote escluse)
\paragraph{Parametri adottati:}
\begin{itemize}
	\item Range accettabile: $[10\%,15\%]$;
	\item Range ottimale: $[15\%,100\%]$.
\end{itemize}

\subsubsection{MS007 - Code Coverage}\mbox{}
Percentuale delle linee di codice coperte dai test.
\paragraph{Parametri adottati:}
\begin{itemize}
	\item Range accettabile: $[80\%,90\%[$;
	\item Range ottimale: $[90\%,100\%]$.
\end{itemize}
\subsubsection{MS008 - Superamento test}\mbox{}
Percentuale di test superati. 
\paragraph{Parametri adottati:}
\begin{itemize}
	\item Range accettabile: $[100\%,100\%]$;
	\item Range ottimale: $[100\%,100\%]$.
\end{itemize}

\subsubsection{MS009 - Requisiti obbligatori soddisfatti}\mbox{}
Percentuale di requisiti obbligatori stabiliti dalla proponente soddisfatti.
\paragraph{Parametri adottati:}
\begin{itemize}
	\item Range accettabile: $[100\%,100\%]$;
	\item Range ottimale: $[100\%,100\%]$.
\end{itemize}
	
\subsubsection{MS010 - Media di build Travis settimanali}\mbox{}
Media delle build effettuate su Travis CI settimanalmente.
\paragraph{Parametri adottati:}
\begin{itemize}
	\item Range accettabile: $[15,25[$;
	\item Range ottimale: $[25,\infty)$.
\end{itemize}

\subsubsection{MS011 - Percentuale build Travis superate}\mbox{}
Percentuale delle build Travis superate con successo.
\paragraph{Parametri adottati:}
\begin{itemize}
	\item Range accettabile: $[75\%,85\%[$;
	\item Range ottimale: $[85\%,100\%]$.
\end{itemize}