Al fine di garantire un prodotto di qualità, è fondamentale avere un elevato 
standard qualitativo anche per i processi. Per garantire tutto ciò, 
è stato deciso di aderire allo standard {SPICE}\ped{G} (ISO/IEC 15504, vedi \NdP \space §4.2): quest'ultimo permette 
di valutare il livello di {maturità}\ped{G} e capacità ({capability}\ped{G}) 
dei processi, al fine di apportare modifiche migliorative.\newline
Sono fissati i seguenti obiettivi:
\begin{itemize}
	\item Rispetto di tempi e costi descritti nel \PdP ;
	\item Continuo miglioramento dei processi;
	\item Misurabilità dello stato dei processi.
\end{itemize}

    Le metriche presentate in questa sezione monitorano lo stato dei processi del progetto analizzando l'uso che essi fanno di tempo e risorse finanziarie. Sono particolarmente utili per il \Res , che può quindi decidere di apportare modifiche alla pianificazione quando necessario.
    \subsection{Processi}
    \subsubsection{MP001 - Schedule Variance}\mbox{}
        La Schedule Variance indica se una certa attività o processo è in anticipo, in pari, o in ritardo rispetto alla data di scadenza prevista
        \paragraph{Parametri adottati:}
        \begin{itemize}
            \item Range accettabile: $ ]0 , 3]$;
            \item Range ottimale: $(-\infty , 0]$.
        \end{itemize}

    \subsubsection{MP002 - Budget Variance}\mbox{}
        La Budget Variance misura, ad una determinata data, lo scostamento fra quanto speso e quanto preventivato. 
        \paragraph{Parametri adottati:}
        \begin{itemize}
            \item Range accettabile: $]1\% , 9\%]$;
            \item Range ottimale: $(-\infty , 1\%]$.
        \end{itemize}