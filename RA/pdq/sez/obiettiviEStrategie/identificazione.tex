Data la difficoltà nel misurare direttamente la qualità, sono state scelte delle specifiche metriche descritte in §4.2.2 di \NdP. 
Ognuna di queste è identificata univocamente da un codice alfanumerico in modo da renderle facilmente tracciabili e quindi controllabili costantemente.

Ogni metrica elencata conterrà le seguenti voci:
\begin{itemize}
	\item \textbf{Id - Nome};
	\item \textbf{Descrizione};
	\item \textbf{Parametri adottati: }range di valori su cui confrontare le misure ottenute. Sono definiti i seguenti intervalli:
	\begin{itemize}
		\item Accettabile;
		\item Ottimale.
	\end{itemize}		
	
\end{itemize}