
\section*{F}
\textbf{Freeling}: the library for pos-tagging developed by TALP Research Center written in C++;
\section*{R}
\textbf{REST}: stands for Representational State Transfer. REST is an architectural style for providing standards between computer systems on the web, making it easier for systems to communicate with each other. RESTful systems, are characterized by how they are stateless and separate the concerns of client and server in fact the implementation of the client and the implementation of the server can be done independently without each knowing about the other.
\section*{P}
\textbf{Pos-tagging}: part-of-speech tagging, also called grammatical tagging or word-category disambiguation, is the process of marking up a word in a text (corpus) as corresponding to a particular part of speech; \\ 
\textbf{POJO}: Plain Old Java Object, is an ordinary Java object, not bound by any special restriction and not requiring any class path. In Spring it refers to a Java object (instance of definition) that isn't bogged down by framework extensions;
\section*{J}
\textbf{JSON}: JavaScript Object Notation, is a lightweight data-interchange format.  It is easy for humans to read and write. It is easy for machines to parse and generate.\\
\textbf{JWT}: JSON Web Token, a JSON-based open standard (RFC 7519) for creating access tokens that assert some number of claims;
