The purpose of this chapter is to describe the tools used by \gruppo{} to develop the application.
If you are not interested in contributing to this project but you just want to run it, you can use any editor.
\subsection{IntelliJ IDEA}
The default IDE for development is IntelliJ IDEA Community created by Jet Brains, you can use it for Java and JSX development. The community edition is free and multi-platform, it runs on Windows, MacOs and Linux.

\subsection{Visual Studio Code}
An alternative IDE is Visual Studio Code developed by Microsoft, it's free and open-source and you can use it to write Java and JSX code. There are some plugins created by Pivotal and RedHat which allow you to have an environment similar to IntelliJ IDEA.

\subsection{Maven}
To manage the project you need Maven. It downloads all the dependencies including Spring Boot, compiles the source code and finally runs the application. Maven is written in Java, so you just need the {Oracle JDK}\ or the OpenJDK at least version 8.
You can download it at the link \href{https://maven.apache.org/}{Maven page}

\subsection{React}
To better debug React components it is recommended to use the following plugins:
\begin{itemize}
\item \textbf{Mozilla Firefox}(current version 66): \\
\url{https://addons.mozilla.org/it/firefox/addon/react-devtools/};
\item \textbf{Google Chrome}(current version 73): \\
\href{https://chrome.google.com/webstore/detail/react-developer-tools/fmkadmapgofadopljbjfkapdkoienihi}{https://chrome.google.com/webstore/detail/react-developer-tools/}
\end{itemize}
Remember to disable cache in the browser during the development.

\subsection{Redux}
To better debug in Redux it is recommended to use the plugin available at\\ \url{https://extension.remotedev.io/} which can be used as extension in Google Chrome 73 and Mozilla Firefox 66.

 
