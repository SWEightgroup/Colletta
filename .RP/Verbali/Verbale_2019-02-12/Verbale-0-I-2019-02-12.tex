\documentclass[a4paper, oneside, openany, dvipsnames, table]{article}
\usepackage{../../template/SWEightStyle}
\newcommand{\Titolo}{Manuale Utente}

\newcommand{\Gruppo}{SWEight}

\newcommand{\Approvatore}{Damien Ciagola}
\newcommand{\Redattori}{Alberto Bacco \newline Sebastiano Caccaro \newline Gheorghe Isachi \newline Gionata Legrottaglie}
\newcommand{\Verificatori}{Francesco Corti \newline Francesco Magarotto}

\newcommand{\pathimg}{../template/img/logoSWEight.png}

\newcommand{\Versionedoc}{1.0.0}

\newcommand{\Distribuzione}{\proponente \newline Prof. Vardanega Tullio \newline Prof. Cardin Riccardo \newline Gruppo SWEight}

\newcommand{\Uso}{Esterno}

\newcommand{\NomeProgetto}{Colletta}

\newcommand{\Mail}{SWEightGroup@gmail.com}

\newcommand{\DescrizioneDoc}{Questo documento si occupa di fornire le modalità di utilizzo del software Colletta commissionato}


\begin{document}
\copertina{}

\definecolor{greySWEight}{RGB}{255, 71, 87}
\definecolor{greyROwSWEight}{RGB}{234, 234, 234}

\section*{Registro delle modifiche}
{
	\rowcolors{2}{greyROwSWEight}{white}
	\renewcommand{\arraystretch}{1.5}
	\centering
	\begin{longtable}{ c c C{4cm}  c  c }
		
		\rowcolor{greySWEight}
		\textcolor{white}{\textbf{Versione}} & \textcolor{white}{\textbf{Data}} & \textcolor{white}{\textbf{Descrizione}} & \textcolor{white}{\textbf{Nominativo}} & \textcolor{white}{\textbf{Ruolo}}\\
		1.2.2 & 2019-02-25 & Ampliamento sezione 5.4 e 3.2.5.2 & Alberto Bacco & \reda{} \\
		
		1.2.1 & 2019-02-23 & Aggiunta sezione 3.2.5.8 Checkstyle & Sebastiano Caccaro & \reda{} \\		
		
		1.2.0 & 2019-02-20 & Aggiunta scelte tecnologiche 3.2.4.2, da 3.4.5.4 a 3.4.5.7, 4.3.1.4, 4.3.2.2, 4.4.6 e figlie & Sebastiano Caccaro & \reda{} \\	
		
		1.1.5 & 2019-02-20 & Modifica sezione 2 & Alberto Bacco & \reda{} \\
		
		1.1.4 & 2019-02-18 & Correzione errori grammatica, spostate sottosezioni di asana da 4.3 a 5.2, & Alberto Bacco & \reda{} \\
		
		1.1.3 & 2019-02-14 & Riorganizzazione e correzione errori sezione 5 & Enrico Muraro & \reda{} \\
		
		1.1.2 & 2019-02-03 & Modifica sottosezione 4.1.10, 4.3.1.4, 4.3.1.5, 4.3.1.6 & Alberto Bacco& \reda{} \\	
		
		1.1.1 & 2019-01-31 & Modifica struttura e contenuti sezione 3  & Damien Ciagola & \reda{} \\	
		
		1.1.0 & 2019-01-27 & Sezione Qualità 4.2 & Sebastiano Caccaro & \reda{} \\	
		
		1.0.1 & 2019-01-25 & Parziale ristrutturazione della struttura del documento & Sebastiano Caccaro & \reda{} \\		
		
		1.0.0 & 2019-01-11 & Approvazione per il rilascio & Sebastiano Caccaro & \Res{} \\
		
		0.9.0 & 2019-01-9 & Verifica finale & Francesco Corti & \ver{} \\
		
		0.9.0 & 2019-01-8 & Aggiunta lista di controllo & Gionata Legrottaglie & \reda{} \\
		
		0.8.0 & 2018-12-23 & Correzioni errori ortografici & Gionata Legrottaglie & \reda{} \\
		
		0.7.0 & 2018-12-20 & Verifica documento & Francesco Corti & \ver{}\\
		
		0.6.0 & 2018-12-18 & Aggiunta sottosezione 5.2.2.2, 5.2.2.3, 5.2.2.4 & Francesco Magarotto & \reda{} \\
		
		0.5.2 & 2018-12-16 & Modifica sezione 4.1.5.3 & Alberto Bacco & \reda{} \\
		
		0.5.2 & 2018-12-16 & Modifica sezione 4.1.5.3 & Alberto Bacco & \reda{} \\
		
		0.5.2 & 2018-12-16 & Aggiunte sottosezioni  & Alberto Bacco & \reda{} \\
		
		0.5.1 & 2018-12-15 & Aggiunte sottosezioni 5.3, 5.4, 5.5, 5.6, 5.7, 5.8 & Alberto Bacco & \reda{} \\
		
		0.5.0 & 2018-12-15 & Aggiunta sezione 5 e sottosezioni 5.1, 5.2 & Gionata Legrottaglie & \reda{} \\
		
		0.4.1 & 2018-12-11 & Aggiunta sezione 4.1.7.3.1 & Francesco Magarotto & \reda{} \\ 
		
		0.4.0 & 2018-12-10 & Aggiunte sottosezioni 4.1.5, 4.1.6, 4.1.7, 4.1.8 & Gionata Legrottaglie & \reda{} \\ 
		0.4.0 & 2018-12-09 & Aggiunta sezione 4 e sottosezioni 4.1.1, 4.1.2, 4.1.3, 4.1.4 & Gionata Legrottaglie & \reda{} \\ 
		
		0.3.1 & 2018-12-07 & Aggiunta sottosezione 3.2 & Gionata Legrottaglie & \reda{} \\ 
		
		0.3.0 & 2018-12-06 & Aggiunta sezione 3 e sottosezione 3.1 & Gionata Legrottaglie & \reda{} \\ 
		
		0.2.0 & 2018-12-05 & Aggiunti i riferimenti & Gionata Legrottaglie & \reda{} \\ 
		
		0.1.0 & 2018-11-30 & Aggiunta introduzione & Gionata Legrottaglie & \reda{} \\
		
		0.0.1 & 2018-11-28 & Creazione scheletro del documento & Gionata Legrottaglie & \reda{}\\
		
	\end{longtable}

}
\newpage
\tableofcontents
\newpage
\section{Informazioni Generali}
\begin{itemize}
\item \textbf{Motivazione:} Discussione sulle tecnologie e strumenti da utilizzare;
\item \textbf{Luogo:} Google Hangouts;
\item \textbf{Data:} 2019-02-12.
\item \textbf{Partecipanti del gruppo:} \hfill
	\begin{itemize}
	\item Bacco Alberto;
	\item Caccaro Sebastiano;
	\item Ciagola Damien;
	\item Corti Francesco;
	\item Isachi Gheorghe;
	\item Legrottaglie Gionata;
	\item Magarotto Francesco;
	\item Muraro Enrico.
	\end{itemize} 
\item \textbf{Ora:} 19:00 - 20:30;
\item \textbf{Segretario:} Caccaro Sebastiano.
\end{itemize}

\section{Ordine del Giorno}
\begin{itemize}
	\item \textbf{Discussione tecnologie da utilizzare:} Sono state scelte le tecnologie da utilizzare per la PoC e per lo sviluppo del progetto;
	\item \textbf{Poc:} Sono state discusse la forma e le modalità della PoC;
	\item \textbf{Documentazione:} Sono stati discussi incrementi da apportare alla documentazione.
\end{itemize}

\section{Resoconto}
\subsection{Discussione sulle tecnologie da utilizzare}
Prima della riunione, ogni membro era stato chiamato a documentarsi sulle possibili tecnologie da utilizzare per la parte di realizzazione vera e propria del progetto didattico. Dopo averne discusso, la scelta è ricaduta su:
\begin{itemize}
	\item \textbf{Docker 18.09.1 - VER-0-2019-02-12.1:} Per ridurre il rischio di incorrere in problematiche riguardanti la configurazione del software, quando possibile, andranno usate immagini Docker per uniformare l'ambiente di esecuzione fra i membri del gruppo;
	\item \textbf{Python 3.6.3 - VER-0-2019-02-12.2}: Verrà usato Python per sviluppare le rest API per l'analisi grammaticale con FreeLing. Ci si avvarrà, inoltre, dei test di unità e della documentazione integrati nel linguaggio;
	\item \textbf{Node Js 10.15.1 - VER-0-2019-02-12.3 :} Node.Js è la piattaforma sulla quale verrà sviluppata la parte dinamica del progetto in Javascript;
	\item \textbf{React - VER-0-2019-02-12.4 :} il gruppo \gruppo \ userà il framework REACT per la programmazione dell'interfaccia utente. Ciò comporta l'adozione del design pattern FLUX;
	\item \textbf{Bootstrap:} Bootstrap verrà usato per curare l'estetica del sito.
\end{itemize}

Si è inoltre discussa l'adozione, da parte di tutti i membri del gruppo, di \textit{Visual Studio Code} come ambiente di sviluppo principale, e del plugin \textit{Prettier} per far aderire lo stile del codice a quanto fissato nelle \NdP .

\subsection{Proof of Concept}
Si è discusso sugli obiettivi della PoC, e si è arrivati alle seguenti conclusioni:
\begin{itemize}
	\item  \textbf{VER-0-2019-02-12.5 :} La PoC deve essere il primo incremento per la realizzazione del progetto. Pertanto essa non può essere un pacchetto usa e getta, ma al contrario, deve essere la base per lo sviluppo nei successivi periodi;
	\item  \textbf{VER-0-2019-02-12.6 :}La PoC dovrà includere l'analisi e la correzione di frasi, almeno in lingua italiana. Ciò richiede lo sviluppo delle Rest API. La raccolta dati e le divisioni per utenti possono essere trattate nei successivi periodi.
\end{itemize}
	
\subsection{Documentazione}
Sono stati discussi i prossimi incrementi da apportare alla documentazione, e si è fatto il punto sulle correzioni post RR da apportare:
\begin{itemize}
	\item \textbf{\PdP  - VER-0-2019-02-12.7 :}Leggera ripianificazione per dare maggior rilievo alla Proof of Concept.
	\item \textbf{\PdQ  - VER-0-2019-02-12.8 : }Rimozione di alcune sezioni già spostate nelle \NdP. Aggiungere specifiche dei test come previsto dalle \NdP .
\end{itemize}

\end{document}