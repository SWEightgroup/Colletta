\subsection{System requirements}
\subsubsection{Windows}
\begin{itemize}
\item [•]\textbf{CPU}: Intel X86 family;
\item [•]\textbf{RAM}: at least 2GB of RAM;
\item [•]\textbf{Disk's space}: at least 1GB;
\item [•]\textbf{Operating system}: Windows 7 or superior, 32-bit or 64-bit versions;
\item [•]\textbf{Java}: Java SE Development Kit 8;
\item [•]\textbf{Node.js}: Node.js 10.15.1;
\item [•]\textbf{Maven}: Maven 3.6.0;
\item [•]\textbf{Browser}: Any browser which supports Javascript, HTML5 and CSS3.

\end{itemize}

\subsubsection{Ubuntu}
\begin{itemize}
\item [•]\textbf{CPU}: Intel X86 family;
\item [•]\textbf{RAM}: at least 2GB of RAM;
\item [•]\textbf{Disk's space}: at least 1GB;
\item [•]\textbf{Java}: OpenJDK 8 / Oracle JDK 8;
\item [•]\textbf{Node.js}: Node.js 10.15.1;
\item [•]\textbf{Maven}: Maven 3.6.0;
\item [•]\textbf{Browser}: Any browser which supports Javascript, HTML5 and CSS3.
\end{itemize}

\subsubsection{MacOS}
\begin{itemize}
\item [•]\textbf{Mac Model}: all the models sold from 2011 onwards;
\item [•]\textbf{RAM}: at least 2GB of RAM;
\item [•]\textbf{Disk’s space}: at least 1GB;
\item [•]\textbf{Operating system}: OS X 10.10 Yosemite.
\item [•]\textbf{Java}: OpenJDK 8 / Oracle JDK 8;
\item [•]\textbf{Node.js}: Node.js 10.15.1;
\item [•]\textbf{Maven}: Maven 3.6.0;
\item [•]\textbf{Browser}: Any browser which supports Javascript, HTML5 and CSS3.
\end{itemize}

\subsection{Configuration}
The webserver Tomcat is integrated in the \texttt{pom.xml} so you don't need any particular configuration if you are using MacOs or any Linux distribution.
In Windows you need the set the environment variables check on the setting and add to the "PATH" list the absolute path to the JDK and the Maven bins folders.
Usually in Windows, Node.js automatically adds its path to the environment variable.
\subsection{Execution}
To run the backend, open a terminal or cmd (not PowerShell) in the \texttt{Backend} folder, be sure the \texttt{pom.xml} is present in the folder, then run the command: 
\begin{center}
\texttt{mvn clean install}
\end{center} 
The command automatically performs the following actions:
\begin{enumerate}
\item Compile the code;
\item Execute test (unit test and static test);
\item Create the executable jar file in the \texttt{target} folder.
\end{enumerate}
Once you have completed the build, run the command from the terminal:\\
\begin{center}
\texttt{java -jar target/colletta-*.jar}
\end{center}
Now Spring Boot is running, to run the frontend just open a terminal window in the \texttt{Frontend} folder and run the command: 
\begin{center}
\texttt{npm install}
\texttt{npm start}
\end{center}
\begin{enumerate}
\item \texttt{npm install} will download all the dependencies needed to execute the code;
\item \texttt{npm start} will run the development server; 
\end{enumerate}
At the end a new browser window will be opened and the frontend will be loaded.

\textit{We are planning to introduce Webpack dependency to integrate the frontend and backend build life-cycle inside Maven.}