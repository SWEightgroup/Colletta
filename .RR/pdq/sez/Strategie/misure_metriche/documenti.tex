Le metriche presentate in questa sezione hanno come scopo fornire dei parametri per garantire un buon livello di leggibilità dei documenti.

\paragraph{Indice Gulpease}\mbox{}\\[0,3cm]
L'Indice Gulpease è un indice di leggibilità di un testo tarato sulla lingua italiana. Rispetto ad altri ha il vantaggio di utilizzare la lunghezza delle parole in lettere anziché in sillabe, semplificandone il calcolo automatico.
L'indice di Gulpease considera due variabili linguistiche: la lunghezza della parola e la lunghezza della frase rispetto al numero delle lettere.\\[0,2cm]
L'indice di Gulpease può assumere valori fra 0 e 100 e si calcola come segue:
\[
IG = 89 + \frac{300 \cdot \textit{(numero delle frasi)} - 10 \cdot \textit{(numero delle lettere)}}{\textit{numero delle parole}}
\]

\textbf{Parametri adottati:}
\begin{itemize}
	\item Range accettabile: $[40 , 100]$
	\item Range ottimale: $[55 , 100]$
\end{itemize}
