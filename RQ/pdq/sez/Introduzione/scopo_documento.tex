% vecchia introduzione
% In questo documento è illustrata la {strategia}\ped{G} di {verifica}\ped{G} e 
% {validazione}\ped{G} del gruppo \gruppo . Tale strategia è fondamentale per dare una 
% misurazione oggettiva e quantificabile del livello di {qualità}\ped{G} di quanto viene 
% prodotto. \newline
% Ciò è vantaggioso sia per il gruppo \gruppo , che può facilmente individuare difetti 
% durante lo svolgimento del progetto, sia per il {committente}\ped{G}, che può costantemente 
% monitorare la qualità del prodotto in base a criteri oggettivi e prestabiliti.

% alternativa alla prima introduzione
In questo documento sono illustrate le {strategie}\ped{G} di {verifica}\ped{G}e 
{validazione}\ped{G} del gruppo \gruppo. 
Tale strategia ci si assicura la qualità dei processi, dei documenti e delle procedure 
utilizzate per gestire e sviluppare i risultati finali.
Lo scopo di questo documento è descrivere le informazioni necessarie per gestire efficacemente 
la qualità del progetto, dalla pianificazione alla consegna, comprendendo obiettivi 
di qualità, responsabilità, e l'approccio di gestione della qualità per 
garantire che gli obiettivi siano raggiunti.\newline
Ciò è vantaggioso sia per il gruppo \gruppo , che può facilmente individuare difetti 
durante lo svolgimento del progetto, sia per il {committente}\ped{G}, che può costantemente 
monitorare la qualità del prodotto in base a criteri oggettivi e prestabiliti.

