\begin{itemize}
	\item \textbf{Piano di Progetto:} \PdP;
	\item \textbf{Slide del corso di Ingegneria del Software:}
	\begin{itemize}
				   \item \url{https://www.math.unipd.it/~tullio/IS-1/2018/Dispense/L10.pdf}, Progettazione software;
				   \item \url{https://www.math.unipd.it/~tullio/IS-1/2018/Dispense/L13.pdf}, Qualità del software;
				   \item \url{https://www.math.unipd.it/~tullio/IS-1/2018/Dispense/L14.pdf}, Qualità di processo.
	\end{itemize}
	\item \textbf{Ian Sommerville, Software Engineering, Nona edizione:}
		\begin{itemize}
		  	\item Capitolo 24: Quality management, tratta di qualità, metriche e standard sul software;
		  	\item Capitolo 26: Process improvement, illustra analisi, misure e metriche per il miglioramento di processo.
		\end{itemize}
	\item \textbf{Standard ISO/IEC 9126:}
		\begin{itemize}
			\item \url{https://it.wikipedia.org/wiki/ISO/IEC_9126}, modello di qualità del software;
		\end{itemize}
	\item \textbf{Standard ISO/IEC 15504:}
		 \begin{itemize}
				 \item \url{https://en.wikipedia.org/wiki/ISO/IEC_15504}, standard SPICE per i processi software;
				  \item \url{https://www.researchgate.net/publication/29453909_The_ISOIEC_15504_Measurement_Framework_for_Process_Capability_and_CMMI}, metriche per "capability" e "maturity" dei processi software.
		\end{itemize}
	\item \textbf{Metriche software:}\newline
				  \url{https://en.wikipedia.org/wiki/Software_metric}, elenco delle possibili metriche da utilizzare;
	\item \textbf{Metriche sui processi:}\newline
				  \url{https://it.wikipedia.org/wiki/Metriche_di_progetto}, metriche di progetto;
	\item \textbf{Indice Gulpease:}\newline
				 \url{https://it.wikipedia.org/wiki/Indice_Gulpease};
	\item \textbf{NOM:} Number of methods\newline
				  \url{http://support.objecteering.com/objecteering6.1/help/us/metrics/metrics_in_detail/number_of_methods.htm}, metrica NOM;
	\item \textbf{Complessità ciclomatica: }\newline
				 \url{https://blogs.msdn.microsoft.com/zainnab/2011/05/17/code-metrics-cyclomatic-complexity/};
		
\end{itemize}