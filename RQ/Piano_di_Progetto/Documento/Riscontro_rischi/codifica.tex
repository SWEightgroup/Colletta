Nella seguente tabella è descritto il riscontro dei rischi nel periodo di Progettazione Architetturale.

\renewcommand{\arraystretch}{1.5}
\def\tabularxcolumn#1{m{#1}}
\begin{tabularx}{\textwidth}{C{0.2\textwidth}XX}
\rowcolor{greySWEight}
    \textcolor{white}{\textbf{Nome}} &
    \textcolor{white}{\textbf{Descrizione}}&
    \textcolor{white}{\textbf{Soluzione}}%&
    %\textcolor{white}{\textbf{Correzione}}
    \endhead
    
\textbf{B-001\newline Incompatibilità orari dei membri del gruppo}&

%Descrizione
Alcuni componenti essendo studenti lavoratori full-time non hanno possibilità di confrontarsi e svolgere le attività di codifica con gli altri membri durante la giornata.
&
%Soluzione
Le attività sono state svolte in costante contatto con tutti i componenti. In particolare, la stesura del codice è stata collettiva e collaborativa, si è fatto uso di Hangouts per comunicare durante le sessioni codifica.\\
\hline

%%%%%%%%%%%%%%%%%%%%%%%%%%%%%%%%%%%%%%%%%%%%%%%%%%%%%%%%%%%%%%%%%%%%%
%Nome
\textbf{M-001\newline Inesperienza tecnologica}&

%Descrizione
Un componente lavora full-time in un settore non pertinente a quello informatico, pertanto non riesce ad acquisire le medesime conoscenze rispetto agli altri membri  &

%Soluzione
I membri più esperti con tali tecnologie hanno fornito spiegazioni per l'apprendimento cercando di coinvolgere il più possibile il componente, tale azione però non è sempre possibile, perché aumenta il rischio d'iterazione rispetto ad un incremento. Inoltre, non è possibile per questioni d'orario e di disponibilità. \\
\hline
   %%%%%%%%%%%%%%%%%%%%%%%%%%
\rowcolor{white}
\caption{Riscontro dei rischi nel periodo di Progettazione in dettaglio e codifica}
\end{tabularx}