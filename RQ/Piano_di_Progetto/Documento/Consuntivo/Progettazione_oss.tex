La mancata comprensione delle attività richieste nel periodo di Progettazione Architetturale, in particolare riguardanti la Technology Baseline, ha portato ad alcuni cambiamenti rispetto al prospetto orario che era stato preventivato. In particolare, le ore destinate al ruolo di \textit{Progettista} non sono state svolte in toto, poiché durante questa fase non è richiesto progettare l'applicativo in tutte le sue parti, tale attività compete al periodo successivo alla di Progettazione in Dettaglio e Codifica. Infine, come si può notare dal consuntivo, sono state necessarie più ore di quelle preventivate per i seguenti ruoli:
\begin{itemize}
	\item \RdP{}: il \Res{} ha dovuto fare da mediatore all'interno dei componenti del gruppo soprattutto durante la fase di scelta degli strumenti tecnologici;
	\item \ana{}: per la correzione e l'incremento del documenti presentati in ingresso alla Revisione dei Requisiti e in seguito al colloquio avvenuto con la Proponente;
	\item \ver{}: per la verifica dei documenti che avevano necessità di un'ulteriore verifica conseguentemente al colloquio avvenuto con la Proponente;
	\item \progr{}: le ore per la realizzazione del PoC non erano state preventivate  in quanto non si erano comprese le richieste del Committente per la Revisione di Progettazione. 
\end{itemize} 
Questo ha portato ad un incremento delle ore preventivate per persona (29 ore) rendicontando un totale di 30 ore per componente, portando il monte ore nel periodo rendicontato a 104 ore per persona.
Malgrado l'incremento orario, le ore da \textit{Progettista} non svolte ha portato un risparmio di \euro 136,00, tale valore non scende sotto la soglia minima richiesta dal Committente ma è da tenere sotto stretto controllo, in quanto sinonimo di una scarsa comprensione delle richieste avanzate dal Committente.
\paragraph{Ritardi}\mbox{}\\
Il contatto con la proponente avvenuto il 2019-03-02, a causa di impegni di lavoro della stessa, ha chiarito alcuni requisiti espressi nel capitolato. Gli analisti hanno quindi provveduto a modificare l'Analisi dei Requisiti, tale attività ha portato a dei ritardi rispetto a quanto pianificato.

\paragraph{Attività future}\mbox{}\\
In considerazione degli errori commessi, si ritiene opportuno ripianificare le attività per i periodi futuri 
relativi alla Progettazione in Dettaglio e Codifica e alla Verifica e Validazione. 


