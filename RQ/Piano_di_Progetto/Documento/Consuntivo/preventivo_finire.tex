Nella seguente tabella è riportato il preventivo a finire, che tiene conto dei periodi fin'ora consuntivati. I periodi di investimento sono riportati solo a scopo informativo, e non sono inclusi nel totale rendicontato.

\begin{table}[H]
	\centering
	\begin{tabular}{ccc}
	\rowcolor{greySWEight}
	\textcolor{white}{\textbf{Periodo}} &
	\textcolor{white}{\textbf{Preventivo in Euro}} & 
	\textcolor{white}{\textbf{Costo in Euro}} \\
	Analisi & 3.500,00 & 3.700,00 \\
	Consolidamento & 775,00 & 795,00 \\
	\rowcolor{greySWEight}
	\multicolumn{3}{c}{ \textcolor{white}{\textbf{Rendicontato}} } \\
	Progettazione Architetturale & 4.681,00 & 4.545,00 \\
	Progettazione di Dettaglio e Codifica & 7.415,00 & 7.463,00 \\
	Verifica e Validazione & 2.957,00 & 2.957,00 \\
	\textbf{Totale} & \textbf{19.328,00} & \textbf{19.460,00} \\
	\textbf{Rendicontato} & \textbf{15.053,00} & \textbf{14.965,00} \\
	\multicolumn{2}{c}{\textbf{Totale preventivato in sede di \RR{}} } & \textbf{15.061,00} \\
	\end{tabular}
	\caption{Preventivo a finire}
\end{table}
\paragraph{Considerazioni}\mbox{}\\
Il preventivo a finire semifinale comprende una rivisitazione del periodo di Verifica e Validazione, tenendo in considerazione i rischi riscontrati nel periodo di Progettazione in dettaglio e codifica. Lo scostamento tra il preventivo presentato alla \RR{} rispetto a quello semifinale ammonta ad un totale di 100 \euro, tale differenza potrebbe subire lievi scostamenti che verranno riportati nel consuntivo finale