\documentclass[a4paper, oneside, openany, dvipsnames, table]{article}
\usepackage{../../template/SWEightStyle}
\newcommand{\Titolo}{Verbale Riunione 2018-12-12}

\newcommand{\Gruppo}{SWEight}

\newcommand{\ACapoRedazione}{Francesco Magarotto}

\newcommand{\Verifica}{Francesco Corti}

\newcommand{\Approvazione}{Sebastiano Caccaro}

\newcommand{\Distribuzione}{Vardanega Tullio \newline Cardin Riccardo \newline Gruppo SWEight}

\newcommand{\Uso}{Interno}

\newcommand{\NomeProgetto}{Colletta}

\newcommand{\Mail}{SWEightGroup@gmail.com}

\newcommand{\DescrizioneDoc}{Questo documento si occupa di riportare quanto discusso nella riunione del 12-12-2018}


\begin{document}
\copertina{}

\definecolor{greySWEight}{RGB}{255, 71, 87}
\definecolor{greyROwSWEight}{RGB}{234, 234, 234}

\section*{Registro delle modifiche}
{
	\rowcolors{2}{greyROwSWEight}{white}
	\renewcommand{\arraystretch}{1.5}
	\centering
	\begin{longtable}{ c c  C{4cm}  c  c }
		
		\rowcolor{greySWEight}
		\textcolor{white}{\textbf{Versione}} & \textcolor{white}{\textbf{Data}} & \textcolor{white}{\textbf{Descrizione}} & \textcolor{white}{\textbf{Nominativo}} & \textcolor{white}{\textbf{Ruolo}}\\
		
		1.0.2 & 2019-03-02 & Aggiunti nuovi termini del documento Piano di Progetto & Isachi Gheorghe &\reda{}\\
		
		1.0.1 & 2019-02-23 & Verifica del documento &  Francesco Corti & \ver{}\\
		
		1.0.1 & 2019-02-20 & Aggiunti nuovi termini del documento Norme di Progetto & Isachi Gheorghe &\reda{}\\
		
		1.0.0 & 2019-01-09 & Approvazione & Sebastiano Caccaro & \Res{}\\
						
		0.1.1 & 2019-01-08 & Verifica del documento & Bacco Alberto & \ver{}\\
		
		0.1.1 & 2019-01-04 & Aggiunti termini del documento Norme di Progetto & Isachi Gheorghe &\reda{}\\
		
		0.1.0 & 2019-01-01 & Aggiunti termini del documento Analisi dei Requisiti & Isachi Gheorghe &\reda{}\\
		
		0.0.4 & 2018-12-29 & Verifica del documento & Bacco Alberto & \ver{}\\
				
		0.0.4 & 2018-12-27 & Aggiunti termini del documento Piano di Qualifica & Isachi Gheorghe &\reda{}\\
				
		0.0.3 & 2018-12-26 &Aggiunti termini del documento Piano di Progetto & Isachi Gheorghe & \reda{}\\
				
		0.0.2 & 2018-12-17 & Aggiunti termini del documento Studio di Fattibilità & Isachi Gheorghe &\reda{}\\
		
		0.0.1 & 2018-12-15 & Scheletro del glossario & Damien Ciagola & \reda{}\\
		
	\end{longtable}

}
\newpage
\tableofcontents
\newpage
\section{Informazioni Generali}
\begin{itemize}
\item \textbf{Motivazione:} verbalizzare le scelte prese in seguito alle correzioni dopo la Technology Baseline;
\item \textbf{Luogo:} Google Hangouts;
\item \textbf{Data:} 2019-03-25;
\item \textbf{Partecipanti del gruppo:} \hfill
	\begin{itemize}
		\item Damien Ciagola;
		\item Francesco Corti;
		\item Francesco Magarotto;
		\item Enrico Muraro;
		\item Alberto Bacco.
	\end{itemize} 
\item \textbf{Ora:} 17:00;
\item \textbf{Segretario:} Alberto Bacco.
\end{itemize}

\section{Ordine del Giorno}
\begin{itemize}
	\item \textbf{MongoDb}: 
	dopo le indicazioni in sede di Technology Baseline è stato 
	collegato il database alla backend (Firebase) tramite Spring, questo ha evidenziato delle problematiche riguardanti la stesura
	del codice per le query. 
	Abbiamo provato ad implementare MongoDB con l'apposito framework Spring Data MongoDB, riscontrando
	una maggiore semplicità nella gestione delle Repository, e
	l'interfaccia MongoRepository ci ha agevolato nella 
	formazione delle query;
	\item \textbf{MongoDB Compass}: il membro del gruppo Francesco Magarotto ha proposto l'utilizzo dell'interfaccia grafica MongoDB Compass che permette di analizzare e visualizzare  i dati presenti nel database in maniera agevole senza l'utilizzo delle query di MongoDB. 
\end{itemize}

\subsection{Resoconto}
\begin{itemize}
	\item \textbf{MongoDB - VER-4-2019-03-25.1}: 
	eseguito il passaggio a MongoDB dopo averne discusso con la 
	proponente. Utilizzo di Spring Data MongoDB su suggerimento del prof. Cardin;
	\item \textbf{MongoDB Compass - VER-4-2019-03-25.2}:
	utilizzo del programma MongoDB Compass per la visualizzazione grafica dei dati presenti all'interno del database.
	
	
\end{itemize}
\end{document}