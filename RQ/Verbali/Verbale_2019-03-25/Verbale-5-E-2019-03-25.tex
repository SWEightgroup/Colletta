\documentclass[a4paper, oneside, openany, dvipsnames, table]{article}
\usepackage{../../template/SWEightStyle}
\newcommand{\Titolo}{Verbale Riunione 2018-12-12}

\newcommand{\Gruppo}{SWEight}

\newcommand{\ACapoRedazione}{Francesco Magarotto}

\newcommand{\Verifica}{Francesco Corti}

\newcommand{\Approvazione}{Sebastiano Caccaro}

\newcommand{\Distribuzione}{Vardanega Tullio \newline Cardin Riccardo \newline Gruppo SWEight}

\newcommand{\Uso}{Interno}

\newcommand{\NomeProgetto}{Colletta}

\newcommand{\Mail}{SWEightGroup@gmail.com}

\newcommand{\DescrizioneDoc}{Questo documento si occupa di riportare quanto discusso nella riunione del 12-12-2018}


\begin{document}
\copertina{}

\definecolor{greySWEight}{RGB}{255, 71, 87}
\definecolor{greyROwSWEight}{RGB}{234, 234, 234}

\section*{Registro delle modifiche}
{
	\rowcolors{2}{greyROwSWEight}{white}
	\renewcommand{\arraystretch}{1.5}
	\centering
	\begin{longtable}{ c c  C{4cm}  c  c }
		
		\rowcolor{greySWEight}
		\textcolor{white}{\textbf{Versione}} & \textcolor{white}{\textbf{Data}} & \textcolor{white}{\textbf{Descrizione}} & \textcolor{white}{\textbf{Nominativo}} & \textcolor{white}{\textbf{Ruolo}}\\
		
		1.0.2 & 2019-03-02 & Aggiunti nuovi termini del documento Piano di Progetto & Isachi Gheorghe &\reda{}\\
		
		1.0.1 & 2019-02-23 & Verifica del documento &  Francesco Corti & \ver{}\\
		
		1.0.1 & 2019-02-20 & Aggiunti nuovi termini del documento Norme di Progetto & Isachi Gheorghe &\reda{}\\
		
		1.0.0 & 2019-01-09 & Approvazione & Sebastiano Caccaro & \Res{}\\
						
		0.1.1 & 2019-01-08 & Verifica del documento & Bacco Alberto & \ver{}\\
		
		0.1.1 & 2019-01-04 & Aggiunti termini del documento Norme di Progetto & Isachi Gheorghe &\reda{}\\
		
		0.1.0 & 2019-01-01 & Aggiunti termini del documento Analisi dei Requisiti & Isachi Gheorghe &\reda{}\\
		
		0.0.4 & 2018-12-29 & Verifica del documento & Bacco Alberto & \ver{}\\
				
		0.0.4 & 2018-12-27 & Aggiunti termini del documento Piano di Qualifica & Isachi Gheorghe &\reda{}\\
				
		0.0.3 & 2018-12-26 &Aggiunti termini del documento Piano di Progetto & Isachi Gheorghe & \reda{}\\
				
		0.0.2 & 2018-12-17 & Aggiunti termini del documento Studio di Fattibilità & Isachi Gheorghe &\reda{}\\
		
		0.0.1 & 2018-12-15 & Scheletro del glossario & Damien Ciagola & \reda{}\\
		
	\end{longtable}

}
\newpage
\tableofcontents
\newpage
\section{Informazioni Generali}
\begin{itemize}
\item \textbf{Motivazione:} Verbalizzare le scelte prese in seguito alle correzioni post technology baseline;
\item \textbf{Luogo:} hangout;
\item \textbf{Data:} 2019-03-25;
\item \textbf{Partecipanti del gruppo:} \hfill
	\begin{itemize}
		\item Ciagola Damien;
		\item Corti Francesco;
		\item Magarotto Francesco;
		\item Muraro Enrico;
		\item Bacco Alberto.
	\end{itemize} 
\item \textbf{Ora:} 17:00;
\item \textbf{Segretario:} Bacco Alberto.
\end{itemize}

\section{Ordine del Giorno}
\begin{itemize}
	\item \textbf{MongoDb}: 
	Dopo le indicazioni in sede di Technology Baseline è stato 
	collegato il databse alla backend in Spring, e ciò ha evidenziato
	alcune problematiche per l'autenticazione e per la stesura
	del codice delle query. 
	Abbiamo provato ad implementare Spring Data MongoDB, riscontrando
	una maggiore semplicità nella gestione delle Repository, e
	l'interfaccia MongoRepository ci ha agevolato nella 
	formazione delle query.
\end{itemize}

\subsection{Resoconto}
\begin{itemize}
	\item \textbf{MongoDB - VER-5-2019-03-25.1}: 
	Eseguito il passaggio a mongoDB dopo averne discusso con la 
	proponente. Utilizzo di Spring Data MongoDB su suggerimento del 
	prof. Cardin.
	
	
\end{itemize}
\end{document}