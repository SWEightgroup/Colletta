%generare il pdf con il comando: pdflatex main.tex
\documentclass[a4paper, oneside, openany, dvipsnames, table]{article}
\usepackage{./template/SWEightStyle}
\usepackage{tabularx}
\newcommand{\mc}{\multicolumn} % handy shortcut macro
\newcolumntype{Z}{>{\centering\arraybackslash}X}
\newcommand{\Titolo}{Verbale Riunione 2018-12-12}

\newcommand{\Gruppo}{SWEight}

\newcommand{\ACapoRedazione}{Francesco Magarotto}

\newcommand{\Verifica}{Francesco Corti}

\newcommand{\Approvazione}{Sebastiano Caccaro}

\newcommand{\Distribuzione}{Vardanega Tullio \newline Cardin Riccardo \newline Gruppo SWEight}

\newcommand{\Uso}{Interno}

\newcommand{\NomeProgetto}{Colletta}

\newcommand{\Mail}{SWEightGroup@gmail.com}

\newcommand{\DescrizioneDoc}{Questo documento si occupa di riportare quanto discusso nella riunione del 12-12-2018}

\usepackage{xcolor}
\usepackage{hyperref}




\begin{document}
\copertina{}
\newpage
\rowcolors{2}{greyROwSWEight}{white}
\section*{Registro delle modifiche}
{
	\renewcommand{\arraystretch}{1.5}
	\centering
	\begin{longtable}{ c c C{6cm} c c }
		\rowcolor{greySWEight}
		\textcolor{white}{\textbf{Versione}} & \textcolor{white}{\textbf{Data}} & \textcolor{white}{\textbf{Descrizione}} & \textcolor{white}{\textbf{Autore}} & \textcolor{white}{\textbf{Ruolo}}\\
			
		0.1.0 & 2019-04-xx & XXX & XXX & \ver{} \\ 
		
		0.1.0 & 2019-03-31 & Aggiunto \S1, \S2, \S3 & Gheorghe Isachi & \reda{} \\
		
		0.0.1 & 2019-03-28 & Creazione scheletro del documento & Sebastiano Caccaro & \reda{}
		
	\end{longtable}

}
\newpage
\tableofcontents
\newpage
\listoffigures
\newpage
\listoftables

\newpage
\newpage
\section{Introduction}
\subsection{Scopo del documento}
Il presente documento ha lo scopo di fornire agli sviluppatori uno specchietto informativo sul design strutturale e logico della piattaforma Colletta. Il documento sarà inoltre
corredato da diagrammi UML 2.X delle principali scelte prese dal gruppo SWEight e descriverà le tecnologie utilizzate nella realizzazione dell’applicazione.
\subsection{Scopo del prodotto}
Il prodotto da realizzare consta in un’applicazione web che fornisca uno strumento per creare e svolgere esercizi di analisi grammaticale, e al contempo né raccolga i risultati. I dati raccolti verranno impiegati dagli sviluppatori dell’azienda proponente come strumento per il miglioramento di algoritmi di {apprendimento automatico}\ped{G}. Nello specifico il prodotto verrà utilizzato da tre tipologie di utenti:
le/gli insegnanti che si occuperanno della creazione degli esercizi,
gli allievi che potranno svolgere gli esercizi e ottenere delle valutazioni e gli sviluppatori che filtreranno i dati secondo alcuni criteri, e infine li scaricheranno.\\Il prodotto si interfaccerà con un’applicazione di {PoS-tagging}\ped{G}, come {FreeLing}\ped{G}, a cui verrà delegata l’esecuzione dell’analisi grammaticale delle frasi.
\subsection{Glossario}
Al fine di rendere il documento il più comprensibile possibile e permetterne una rapida fruizione, viene allegato il \G{} in cui sono presenti i termini contraddistinti dal pedice G. Tali termini includono abbreviazioni, acronimi, termini di natura tecnica, oppure sono fonte di ambiguità e pertanto necessitano di una definizione che renda il loro significato inequivocabile. 
Ogni termine, solo alla prima occorrenza per documento, verrà contrassegnato con la dicitura sopra indicata e rimanderà alla medesima definizione nel \G{}.
\newpage	

\newpage
\section{Development Requirements}
\subsection{System requirements}
\subsubsection{Windows}
\begin{itemize}
\item [•]\textbf{CPU}: Intel X86 family;
\item [•]\textbf{RAM}: at least 2GB of RAM;
\item [•]\textbf{Disk's space}: at least 1GB;
\item [•]\textbf{Operating system}: Windows 7 or superior, 32-bit or 64-bit versions;
\item [•]\textbf{Java}: Java SE Development Kit 8;
\item [•]\textbf{Node.js}: Node.js 10.15.1;
\item [•]\textbf{Maven}: Maven 3.6.0;
\item [•]\textbf{Browser}: Any browser which support Javascript, HTML5 and CSS3.

\end{itemize}

\subsubsection{Ubuntu}
\begin{itemize}
\item [•]\textbf{CPU}: Intel X86 family;
\item [•]\textbf{RAM}: at least 2GB of RAM;
\item [•]\textbf{Disk's space}: at least 1GB;
\item [•]\textbf{Java}: OpenJDK 8 / Oracle JDK 8;
\item [•]\textbf{Node.js}: Node.js 10.15.1;
\item [•]\textbf{Maven}: Maven 3.6.0;
\item [•]\textbf{Browser}: Any browser which support Javascript, HTML5 and CSS3.
\end{itemize}

\subsubsection{MacOS}
\begin{itemize}
\item [•]\textbf{Mac Model}: all the models sold from 2011 onwards;
\item [•]\textbf{RAM}: at least 2GB of RAM;
\item [•]\textbf{Disk’s space}: at least 1GB;
\item [•]\textbf{Operating system}: OS X 10.10 Yosemite.
\item [•]\textbf{Java}: OpenJDK 8 / Oracle JDK 8;
\item [•]\textbf{Node.js}: Node.js 10.15.1;
\item [•]\textbf{Maven}: Maven 3.6.0;
\item [•]\textbf{Browser}: Any browser which support Javascript, HTML5 and CSS3.
\end{itemize}

\subsection{Configuration}
The webserver Tomcat is integrated in the \texttt{pom.xml} so you don't need any particular configuration if you are using MacOs or any Linux distro.
In Windows you need the set the environment variables check on the setting and add to the "PATH" list the absolute path to the JDK and the Maven bins folders.
Usually in Windows, Node.js automatically adds its path to the environment variable.
\subsection{Execution}
To run the backend part, open a terminal or cmd (not PowerShell) in the \texttt{Backend} folder, be sure the \texttt{pom.xml} is present in the folder, than run the command: 
\begin{center}
\texttt{mvn clean install}
\end{center} 
The command automatically performs the following actions:
\begin{enumerate}
\item Compile the code;
\item Execute test (unit test and static test);
\item Create the executable jar file in the \texttt{target} folder.
\end{enumerate}
Once you have completed the build, run the command from the terminal:\\
\begin{center}
\texttt{java -jar target/colletta-*.jar}
\end{center}
Now Spring Boot is running, to run the frontend just open a terminal window in the "Frontend" folder and run the command: 
\begin{center}
\texttt{npm start}
\end{center}
This will automatically open a new browser window with the application frontend.

\textit{We are planning to introduce Webpack dependency to integrate the frontend and backend build life-cycle inside Maven.}

\newpage
\section{Workspace Configuration}
The purpose of this chapter is to describe the tools used by \gruppo{} to develop the application.
Obviously, if you are not interested in contributing to this project but you just want to run it, you can use any editor.
\subsection{IntelliJ IDEA}
The default IDE for development is IntelliJ IDEA Community created by Jet Brains, you can use it for Java and JSX development. The commmunity edition used is free and multi-platform, it runs on Windows, MacOs and Linux.

\subsection{Visual Studio Code}
An alternative IDE is Visual Studio Code developed by Microsoft, it's free and open-source and you can use it to write Java and JSX code. There are some plugins created from Pivotal and RedHat which allow you to have an environment similar to IntelliJ IDEA.

\subsection{Maven}
To manage the project you need Maven. It downloads all the dependencies including Spring Boot, compiles the source code and finally runs the application. Maven is written in Java, so you just need the {Oracle JDK}\ or the OpenJDK at least version 8.
You can download it at the link \href{https://maven.apache.org/}{Maven page}

\subsection{React}
To better debug React components it is recommended to use the following plugins:
\begin{itemize}
\item \textbf{Mozilla Firefox}(current version 66): \\
\url{https://addons.mozilla.org/it/firefox/addon/react-devtools/};
\item \textbf{Google Chrome}(current version 73): \\
\href{https://chrome.google.com/webstore/detail/react-developer-tools/fmkadmapgofadopljbjfkapdkoienihi}{https://chrome.google.com/webstore/detail/react-developer-tools/}
\end{itemize}
Remember to disable cache in the browser during the development.

\subsection{Redux}
To better debug in Redux it is recommended to use the plugin available at\\ \url{https://extension.remotedev.io/} which can be used as extension in Google Chrome 73 and Mozilla Firefox 66.

 


\newpage
\section{FrontEnd}
This section is intended to make the developer understand the working of the Colletta frontend, and to allow him or her to add functionalities  to the software package.
In order fully understand the contents below, the developer must have a certain degree of familiarity with React and Redux. If that's not the case, we strongly recommend the reader to at least acquire some basic knowledge on the topics.

\subsection{Directory tree}
\begin{figure}[H]
\centering
\begin{forest}
  for tree={
    font=\ttfamily,
    grow'=0,
    child anchor=west,
    parent anchor=south,
    anchor=west,
    calign=first,
    inner xsep=7pt,
    edge path={
      \noexpand\path [draw, \forestoption{edge}]
      (!u.south west) +(7.5pt,0) |- (.child anchor) pic {folder} \forestoption{edge label};
    },
    before typesetting nodes={
      if n=1
        {insert before={[,phantom]}}
        {}
    },
    fit=band,
    before computing xy={l=15pt},
  }  
[Frontend
	[src
		[actions
		]
		[assets
		]
		[constants
		]
		[helpers
		]
		[reducer
		]
		[store
		]
		[view
			[containers]
			[components]
		]
	]
]
\end{forest}
\caption{Frontend directory tree}
\label{fig:FrontDir}
\end{figure}

Each folder contains a specific set of files:
\begin{itemize}
	\item \textbf{actions:} the modules in this folder are responsible for creating and dispatching the actions to the reducers;
	\item \textbf{assets:} static files like font and images;
	\item \textbf{constants:} data collections and constants used in various part of the code, i.e. the label used for the translation;
	\item \textbf{helpers:} standard js functions or classes which have some use in the code, i.e. the label translator;
	\item \textbf{reducers:} all the reducers responsible for the creation of a new state;
	\item \textbf{store:} a single file creating and giving access to the centralized state;
	\item \textbf{view:} classes rendering the information in the store. They are divided in \textit{components} and \textit{containers}.	 The key point to bare in mind when talking about components and containers is the following: containers are "smart", they observe the store and can call actions; components, on the other hand, are basically just static functions.
\end{itemize}

\subsection{Modify or add features}
\subsubsection{Components}
Components extend the React \texttt{component} abstract class and implement the \texttt{render()} method. They can be viewed as a pure functions of the props passed by their father component or container. They do not have access to the store. When adding or modifying a component the following rules should be followed:
\begin{itemize}
	\item Since the global state of the application is managed by Redux, do not use or create the local state of the component. Instead, rely solely on the props;
	\item Helper functions may be defined in the component class, but none of them should call action creators or external resources such as API calls;
	\item All components must be placed in the \texttt{src/component} folder;
	\item Every component which needs to render some text must have a language prop to call to the translator module;
\end{itemize}
When adding a new component, one can start from the following snippet:
\begin{lstlisting}
import React, { Component } from 'react';
import _translator from '../../helpers/Translator';
class SampleComponent extends Component {

  render() {
    const { prop1,prop2,prop3 } = this.props;
    //Do stuff here
    return (
      <React.Fragment>
        {/* Stuff to render */}
      </React.Fragment>
    );
  }
}
export default SampleComponent;
\end{lstlisting}


\subsubsection{Containers}
Containers extend the React \texttt{component} abstract class and implement the \texttt{render()} method. They can read the store and alters its state via actions. Containers can also have props passed to them, just like a container. When adding or modifying a component the following rules should be followed:
\begin{itemize}
	\item Since the global state of the application is managed by Redux, do not use or create the local state of the container. Instead, rely solely on the props and on the store;
	\item The store and the actions should not be accessed directly for performance and readability reasons. Instead, the should be mapped to the props;
	\item All containers must be placed in the most appropriate \texttt{src/view/containers} sub-directory. Creation of new sub-directories is allowed if necessary.
\end{itemize}

When adding a new container, one can start from the following snippet:
\begin{lstlisting}
import React, { Component } from 'react';
import { connect } from 'react-redux';
import _translator from '../../../helpers/Translator';

class SampleContainer extends Component {

  render() {
    const { prop1, prop2, prop3, action1Prop, action2Prop } = this.props;
    // Do stuff
    return (
      // Stuff to render
    );
  }
}
const mapStateToProps = store => {
  return {
    prop1: store.object1,
    prop2: store.object2,
    prop3: store.object3
  };
};

const mapDispatchToProps = dispatch => {
  return {
    action1Prop: () => dispatch(action1()),
    action2Prop: () => dispatch(action2())
  };
};
//both action1 and action2 must be imported

export default connect(
  mapStateToProps,
  mapDispatchToProps
)(SampleContainer);
\end{lstlisting}


\subsubsection{Rest API calls}
When implementing a new Rest API call (RAP from now on), the developer must stick to the following guidelines:
\begin{itemize}
	\item RAPs must be implemented inside an action creator and should not be put inside components or containers;
	\item RAPs must use the Axios module;
	\item RAPs must pass an authorization token (exception made for Login and SignUp) which is kept in the store. The snippet below will show clarify how;
	\item If a RAP does not need additional data, the data field should be replaced by \texttt{\{\}}, otherwise the response will display a 403 error.
\end{itemize}
When adding a new RAP, one can start from the following snippet:
\begin{lstlisting}
export const sampleActionCreator = objectToSend => {
  return dispatch => {
    axios
      .post(
        'http://localhost:8081/sample-call',
        {
          ...objectToSend
        },
        {
          headers: {
            'content-type': 'application/json',
            Authorization: store.getState().auth.token
          }
        }
      )
      .then(response => {
        // Maybe do something
        dispatch({ type: 'SAMPLE-ACTION', dataToDispatch });
      })
      .catch(() => {/* Handle Error Here*/});
  };
};
\end{lstlisting}


\subsubsection{Interface Language}
\subsubsection{Analysis Languages}






%%sia front end che backend devono avere una descrzione dell'archietettura, e poi descrivere come 
%%estendere il tutto. Vedi marvin (con correzioni)

\newpage
\section{BackEnd}
This section is intended to make the developer understand the working of the Colletta backend, and to allow him or her to add functionalities  to the software package.
In order to fully understand the contents below, the developer must have a certain degree of familiarity with Java, the framework Spring Boot with his subframework Spring Data MongoDB, MongoDB and Maven. If that's not the case, we strongly recommend the reader to at least acquire some basic knowledge on the topics.

\subsection{Directory tree}

\begin{figure}[H]
\centering
\begin{forest}
  for tree={
    font=\ttfamily,
    grow'=0,
    child anchor=west,
    parent anchor=south,
    anchor=west,
    calign=first,
    inner xsep=7pt,
    edge path={
      \noexpand\path [draw, \forestoption{edge}]
      (!u.south west) +(7.5pt,0) |- (.child anchor) pic {folder} \forestoption{edge label};
    },
    before typesetting nodes={
      if n=1
        {insert before={[,phantom]}}
        {}
    },
    fit=band,
    before computing xy={l=15pt},
  }  
[Backend
	[src
		[main 
			[java
				[it
					[colletta
						[controller]
						[error]
						[exceptions]
						[library]
						[model
							[helper]
						]
						[repository]
						[security]
						[service]						
					]
				]
			]
			[resources]
		]	
		[test
			%[it
			%	[colletta
			%		[repository
			%			[config]
			%			[exercise]
			%			[phrase]
			%			[user]
			%		]
			%		[service
			%			[user]
			%		]
			%	]
			%]
		]				
	]
]
\end{forest}
\caption{Backend directory tree}
\label{fig:FrontDir}
\end{figure}

Each folder contains a specific set of files:
\begin{itemize}
\item  \textbf{controller:} classes that handle HTTP Request, marked with \textit{@RestController} Spring annotation;
\item  \textbf{error:} custom class for error handling;
\item  \textbf{exceptions:} custom class for exception handling;
\item  \textbf{library:} classes that connect the application to the PoS-tagging library;
\item  \textbf{model:} has the folder \textit{helper} which contains the transfer object (DTO), the other Model files are standard POJO objects that represent JSON objects store inside the database;
\item  \textbf{repository:} classes that encapsulates the set of objects persisted in a data store, in our case MongoDB, and the operations performed over them, providing a more object-oriented view of the persistence layer;
\item  \textbf{security:} classes that manages the security of the application through encryption and token management;
\item  \textbf{service:} classes that make up the logical business part of the application;
\item  \textbf{test:}  the tests that the application must pass to enter in a safe running state;
\item \textbf{resources:} contains the configuration file for connection to the database.
\end{itemize}

\subsubsection{Data and logic separation}
\begin{figure}[H]
\centering 
\includegraphics[scale=0.3]{uml/backendArchitecture.png} 
\caption{Data and logic separation}
\end{figure}

The design pattern used for the backend is: \textbf{Spring MVC}.
Respectively with the Controller, Service and Repository classes.
\\ 
To get more information about the architecture chosen for the backend, follow the link below: \href{https://docs.spring.io/spring/docs/current/spring-framework-reference/web.html}{Spring Web MVC}.

\subsubsection{Security}
The system was designed to secure client server communication so that only a user registered in the system can make calls which are then resolved by interacting with the database. 
In authentication, when the user successfully logs in using their credentials, a JSON Web Token will be returned, instead if the user has registered the token will be generated. Whenever the user wants to access a protected route or resource he/her must necessarily send his token that identifies him/her as a user correctly registered in the system. If this is not done the system will decline the request. 
The security system has been implemented so that the only requests that are authorized without token recognition are the login (\textit{/login}) and the registration (\textit{/sign-up}).\\
The system uses \textit{JWT tokens} for token authentication, more information is available at the link: \href{https://jwt.io/introduction/}{Introduction to JSON Web Tokens}. 


\appendix
\addcontentsline{toc}{part}{Appendices}

\newpage
\section{Glossary}
Nel documento è possibile incontrare termini tecnici, i quali potrebbero non essere immediatamente chiari al lettore. Per disambiguarne il significato, essi sono stati marcati con una \ped{G} a pedice e la loro definizione è reperibile nel glossario fornito separatamente.

\end{document}
