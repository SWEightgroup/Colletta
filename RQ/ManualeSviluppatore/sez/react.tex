This section is intended to make the developer understand the working of the Colletta frontend, and to allow him or her to add functionalities  to the software package.
In order fully understand the contents below, the developer must have a certain degree of familiarity with React and Redux. If that's not the case, we strongly recommend the reader to at least acquire some basic knowledge on the topics.

\subsection{Directory tree}
\begin{figure}[H]
\centering
\begin{forest}
  for tree={
    font=\ttfamily,
    grow'=0,
    child anchor=west,
    parent anchor=south,
    anchor=west,
    calign=first,
    inner xsep=7pt,
    edge path={
      \noexpand\path [draw, \forestoption{edge}]
      (!u.south west) +(7.5pt,0) |- (.child anchor) pic {folder} \forestoption{edge label};
    },
    before typesetting nodes={
      if n=1
        {insert before={[,phantom]}}
        {}
    },
    fit=band,
    before computing xy={l=15pt},
  }  
[Frontend
	[src
		[actions
		]
		[assets
		]
		[constants
		]
		[helpers
		]
		[reducer
		]
		[store
		]
		[view
			[containers]
			[components]
		]
	]
]
\end{forest}
\caption{Frontend directory tree}
\label{fig:FrontDir}
\end{figure}

Each folder contains a specific set of files:
\begin{itemize}
	\item \textbf{actions:} the modules in this folder are responsible for creating and dispatching the actions to the reducers;
	\item \textbf{assets:} static files like font and images;
	\item \textbf{constants:} data collections and constants used in various part of the code, i.e. the label used for the translation;
	\item \textbf{helpers:} standard js functions or classes which have some use in the code, i.e. the label translator;
	\item \textbf{reducers:} all the reducers responsible for the creation of a new state;
	\item \textbf{store:} a single file creating and giving access to the centralized state;
	\item \textbf{view:} classes rendering the information in the store. They are divided in \textit{components} and \textit{containers}.
	
\end{itemize}