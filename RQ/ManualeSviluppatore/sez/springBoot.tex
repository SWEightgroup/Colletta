This section is intended to make the developer understand the working of the \texttt{Colletta} backend, and to allow him or her to add functionalities  to the software package.
In order fully understand the contents below, the developer must have a certain degree of familiarity with \textbf{Java}, the framework \textbf{Spring Boot}, \textbf{MongoDB} and \textbf{Maven}. If that's not the case, we strongly recommend the reader to at least acquire some basic knowledge on the topics.

\subsection{Directory tree}

\begin{figure}[H]
\centering
\begin{forest}
  for tree={
    font=\ttfamily,
    grow'=0,
    child anchor=west,
    parent anchor=south,
    anchor=west,
    calign=first,
    inner xsep=7pt,
    edge path={
      \noexpand\path [draw, \forestoption{edge}]
      (!u.south west) +(7.5pt,0) |- (.child anchor) pic {folder} \forestoption{edge label};
    },
    before typesetting nodes={
      if n=1
        {insert before={[,phantom]}}
        {}
    },
    fit=band,
    before computing xy={l=15pt},
  }  
[Backend
	[src
		[main 
			[java
				[it
					[colletta
						[controller]
						[error]
						[exceptions]
						[library]
						[model
							[helper]
						]
						[repository]
						[security]
						[service]						
					]
				]
			]
			[resources]
		]	
%		[test
%			[it
%				[colletta
%					[repository
%						[config]
%						[exercise]
%						[phrase]
%						[user]
%					]
%					[service
%						[user]
%					]
%				]
%			]
%		]				
	]
]
\end{forest}
\caption{Backend directory tree}
\label{fig:FrontDir}
\end{figure}


Each folder contains a specific set of files:
\begin{itemize}
\item 
\end{itemize}