\section{Introduzione}
\subsection{Scopo del documento}
Il documento è una giuda per gli utenti utilizzatori della piattaforma di analisi grammaticale Colletta sia che siano allievi, insegnanti o sviluppatori. Lo scopo è di illustrare brevemente gli aspetti di base del prodotto e le possibili interazioni che un'utente può avere con l'applicazione.
\subsection{Scopo del prodotto}
Lo scopo del prodotto è la realizzazione di una piattaforma interattiva per la raccolta dati relativi ad esercizi di analisi grammaticale, che verranno impiegati come dati sorgente in algoritmi di apprendimento automatico. In particolare per ogni tipologia di utente è stata sviluppata una dashboard opportuna per essere intuibile e facile da utilizzare.
\subsection{Glossario}
In appendice al documento \`e stato inserito un glossario contenente tutti i termini necessari alla piena comprensione del testo.  Essendo Colletta destinato a una utenza con basse conoscenze nel dominio informatico cercheremo di essere più semplici possibile per rendere comprensibile questa guida. Al fine di evitare incomprensioni, si precisa che ogni parola inserita a glossario verrà seguita da una G a pedice.

\section{Istruzioni per l'utilizzo}
\subsection{Requisiti software}
Di seguito vengo riportate le versioni minime garantite per il funzionamento del prodotto Colletta:
\begin{itemize}
\item {Sistema operativo:} qualsiasi;
\item {Browser web:} ogni browser web aggiornato all'ultima versione, tranne Internet Explorer e Edge.
\end{itemize}
Si specifica inoltre che JavaScript deve essere abilitato sul browser per il corretto funzionamento dell’applicazione.
\subsection{Requisiti hardware}
Connessione ad Internet.
\subsection{Registrazione}
Senza una registrazione non è possibile accedere alla piattaforma.
\subsection{Login}
Dopo aver effettuato la registrazione si accede tramite username e password.