Il codice relativo alla Product Baseline lo si può trovare al seguente link:
\begin{center}
\href{https://github.com/SWEightgroup/Development}{Colletta}. 
\end{center}

\subsection{Maven project}
Una volta fatto il clone della repository o dopo aver scaricato lo zip, posizionarsi nella directory "Mockup-V2/Backend" della repo e utilizzare i seguenti comandi:

\begin{center}
\textsf{mvn clean install}
\end{center}

\begin{center}
\textsf{mvn exec:java}
\end{center}

\subsection{Node.js}
Per lo sviluppo del front-end ci si avvale dell’ultima versione Long Term Support (LTS) di Node.js, che, al momento della stesura di questo documento, è la 10.15.1 LTS. Node.js è reperibile al
seguente link:

\begin{center}
\href{https://nodejs.org/it/}{Node.js}. 
\end{center}

Il file package.json contiene tutte le configurazioni e dipendenze del progetto. Per installare tutti i moduli
necessari è necessario eseguire il seguente comando nella cartella contenente il file package.json:
\begin{center}
\textsf{npm install}
\end{center}
Per eseguire il progetto è necessario usare il comando:
\begin{center}
\textsf{npm start}
\end{center}
