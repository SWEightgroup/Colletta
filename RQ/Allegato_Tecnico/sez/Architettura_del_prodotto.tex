\subsection{Panoramica}
\begin{figure}[H]
\centering
\includegraphics[width=17cm, keepaspectratio]{img/photo_2019-04-08_19-26-02.jpg} 
\caption{Exercise insert}
\end{figure}

\subsection{MongoDB Database}
\begin{figure}[H]
\centering
\includegraphics[width=17cm, keepaspectratio]{img/mongodb.png} 
\caption{MongoDB database}
\end{figure}
\newpage

\subsection{Design pattern utilizzati}
\subsubsection{Backend}
\paragraph{Adapter}
L'utilizzo della libreria di Freeling, per il pos-tagging, ha richiesto la creazione di un Adapter nella variante Object Adapter per poter adattare le funzionalità strettamente necessarie.
\begin{figure}[H]
\includegraphics[width=17cm]{img/Adapter.png} 
\caption{Diagramma delle classi Adapter}
\end{figure}
La classe FreelingSocketClient viene fornita dai creatori della libreria e si occupa di realizzare la connessione con il server Freeling scritto in C++.
\paragraph{Controller - Service - Repository - Model}
L'architettura realizzata all'interno di Spring Web consta nella presenza di:
\begin{itemize}
\item \textbf{Controller}: un a cui è delegato il compito di gestire le richieste provenienti dalla parte frontend e alla cattura delle eccezioni;
\item \textbf{Service}: realizzano la business-logic;
\item \textbf{Repository}: realizzano il layer di persistenza gestendo la base di dati;
\item \textbf{Model}: rappresentano oggetti  Plain Old Java Object, un'istanza di un model rappresenta un documento di una collezione.
\end{itemize}
\begin{figure}[H]
\centering
\includegraphics[width=14cm]{img/springArch.png}
\caption{Scherma generale architettura in Spring}
\end{figure}

\subsubsection{Frontend}
\paragraph{Redux}\mbox{}\\

\begin{figure}[H]
    \centering
	\includegraphics[width=0.7\linewidth]{img/Flux.pdf}
	\caption{Schema del design pattern Flux}
\end{figure}

Per la frontend viene utilizzato React con design pattern Redux, un'evoluzione del design pattern Flux.
In Redux, tutti i dati scorrono in modo unidirezionale attraverso i seguenti componenti:
\begin{itemize}
    \item \textbf{Store: }oggetto immutabile che contiene l'intero stato dell'applicazione in maniera centralizzata;
    \item \textbf{Reducers: }sono funzioni pure. Ogni volta che i Reducer ricevono una nuova azione, 
    processano l'azione ricevuta e, in caso sia necessario apportare delle modifiche allo stato, restituiscono un nuovo oggetto 
    \item \textbf{Action creators: }funzioni che facilitano la gestione del dispatch (creazione di azioni da mandare ai reducers); 
    \item \textbf{View: }componenti grafiche, il loro contenuto dipende dallo store. Sono implementate attraverso un \textit{Observer Pattern} sullo store stesso.
    Ad ogni cambiamento di stato prodotto da un reducers vengono renderizzate le componenti collegate ad esso.
\end{itemize} 


\subsection{Diagrammi dei package}
\subsubsection{Data Transfer Object}
Le classi Helper rappresentano Data Transfer Object (DTO), vengono utilizzate dalla classe \texttt{Controller.java} per fornire un oggetto per il trasferimento dati dalla frontend senza ricorrere a JSON troppo complessi.

\subsubsection{Builder}
I model sono dotati ognuno di un builder, tale classe interna non è codificata ma realizzata tramite un plugin denominato lambok. A seguire viene mostrato un esempio di un builder creato automaticamente.
\begin{figure}[H]
\centering
\includegraphics[width=15cm]{img/builder.png}
\caption{Esempio di builder presente nella classe UserModel}
\end{figure}

\newpage
\subsubsection{Model}
Viene di seguito riportato il diagramma delle classi del package model.
\begin{figure}[H]
\centering
\includegraphics[width=17cm, keepaspectratio]{img/model.png} 
\caption{Model}
\end{figure}
\newpage

\subsubsection{Controller e service}
Viene di seguito riportato il diagramma delle classi di package controller e service.
\begin{figure}[H]
\centering
\includegraphics[width=17cm, keepaspectratio]{img/Controller-service.png} 
\caption{Controller e Service}
\end{figure}

\subsubsection{View}
Viene di seguito riportato il diagramma delle classi di package della view.
\begin{figure}[H]
\centering
\includegraphics[width=17cm, keepaspectratio]{img/view.jpg} 
\caption{View}
\end{figure}
\newpage
\subsubsection{Service e repository}
Viene di seguito riportato il diagramma delle classi di package service e repository.
\begin{figure}[H]
\centering
\includegraphics[width=17cm, keepaspectratio]{img/Service-repository.png} 
\caption{Service e Repository}
\end{figure}


\newpage
\subsection{Diagrammi di sequenza}
\subsubsection{Inserimento di un esercizio}
Il diagramma di sequenza rappresenta l'azione di inserimento di un esercizio nel sistema.
\begin{figure}[H]
\centering
\includegraphics[width=17cm, keepaspectratio]{img/Exercise-insert.png} 
\caption{Exercise insert}
\end{figure}
\subsubsection{Descrizione Diagramma attività figura 11}
\begin{enumerate}
        \item L'insegnante inserisce il testo della frase;
        \item L'insegnante riceve la correzione automatica della frase fornita dalla backend;
        \item L'insegnante modifica la soluzione automatica;
        \item L'insegnante può inserire una soluzione alternativa;
        \item L'insegnante stabilisce se l'esercizio è pubblico o privato;
        \item L'insegnante assegna l'esercizio ad uno o più studenti;
        \item L'insegnante può assegnare un esercizio ad una classe di studenti;
        \item L'insegnante pubblica l'esercizio.
    \end{enumerate}

\subsubsection{Valutazione esercizio con voto}
\begin{figure}[H]
\centering
\includegraphics[width=17cm, keepaspectratio]{img/Student-exercise-mark.png} 
\caption{Student's exercise mark}
\end{figure}

\paragraph{Descrizione diagramma attività figura 12}
\begin{enumerate}
\item Lo studente si è autenticato;
\item Lo studente è nella sezione "esercizi"; 
\item Lo studente è nella sezione "esercizi per casa";
\item Lo studente visualizza la lista di esercizi per casa; 
\item Lo studente scrive la soluzione di un esercizio;
\item Lo studente manda la sua soluzione di un esercizio; 
\item Lo studente riceve la valutazione della sua soluzione precedentemente inviata; 
\item Lo studente visualizza la valutazione.
\end{enumerate}


\subsubsection{Abilitazione di uno sviluppatore}
\begin{figure}[H]
\centering
\includegraphics[width=17cm, keepaspectratio]{img/Accept-or-decline-developer.png} 
\caption{Accept or decline developer}
\end{figure}

\paragraph{Descrizione diagramma attività figura 13}
\begin{enumerate}
\item L'amministratore si è autenticato;
\item L'amministratore riceve la lista degli sviluppatori non ancora approvati; 
\item L'amministratore decide se approvare o declinare la richiesta di uno sviluppatore; 
\item Node decisione:
\begin{enumerate}
	\item L'amministratore approva la richiesta di uno sviluppatore, lo sviluppatore viene quindi abilitato;
	\item L'amministratore declina la richiesta di uno sviluppatore, lo sviluppatore viene cancellato dal sistema. 
\end{enumerate}
\end{enumerate}



\subsubsection{Login}
Il diagramma di sequenza riportato qui di seguito raffigura il processo di login, durante il quale l'utente che vuole accedere può essere autenticato dal sistema.
\begin{figure}[H]
\centering
\includegraphics[width=17cm, keepaspectratio]{img/Authorization.png} 
\caption{Authorization}
\end{figure}




