\subsection{Scopo del documento}
Il presente documento ha lo scopo di fornire agli sviluppatori uno specchietto informativo sul design strutturale e logico della piattaforma Colletta. Il documento sarà inoltre
corredato da diagrammi UML 2.X delle principali scelte prese dal gruppo SWEight e descriverà le tecnologie utilizzate nella realizzazione dell’applicazione.
\subsection{Scopo del prodotto}
Il prodotto da realizzare consta in un’applicazione web che fornisca uno strumento per creare e svolgere esercizi di analisi grammaticale, e al contempo né raccolga i risultati. I dati raccolti verranno impiegati dagli sviluppatori dell’azienda proponente come strumento per il miglioramento di algoritmi di {apprendimento automatico}\ped{G}. Nello specifico il prodotto verrà utilizzato da tre tipologie di utenti:
le/gli insegnanti che si occuperanno della creazione degli esercizi,
gli allievi che potranno svolgere gli esercizi e ottenere delle valutazioni e gli sviluppatori che filtreranno i dati secondo alcuni criteri, e infine li scaricheranno.\\Il prodotto si interfaccerà con un’applicazione di {PoS-tagging}\ped{G}, come {FreeLing}\ped{G}, a cui verrà delegata l’esecuzione dell’analisi grammaticale delle frasi.
\subsection{Glossario}
Al fine di rendere il documento il più comprensibile possibile e permetterne una rapida fruizione, viene allegato il \G{} in cui sono presenti i termini contraddistinti dal pedice G. Tali termini includono abbreviazioni, acronimi, termini di natura tecnica, oppure sono fonte di ambiguità e pertanto necessitano di una definizione che renda il loro significato inequivocabile. 
Ogni termine, solo alla prima occorrenza per documento, verrà contrassegnato con la dicitura sopra indicata e rimanderà alla medesima definizione nel \G{}.
\newpage