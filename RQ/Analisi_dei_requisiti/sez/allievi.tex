\subsection{Allievi}
\subsubsection{Panoramica allievo}

\begin{figure}[H]
\centering
\includegraphics[width=17cm]{img/PanoramicaAllievi.png} 
\caption{Panoramica allievo}\label{fig:31}
\end{figure}


\subsubsection{UC 3.1 - Visualizzazione della dashboard}
\begin{itemize}
\item[•]\textbf{Attori}: Allievo;
\item[•]\textbf{Descrizione}: l'allievo visualizza la propria dashboard;
\item[•]\textbf{Precondizione}: l'allievo si è autenticato;
\item[•]\textbf{Postcondizione}: l'allievo visualizza la dashboard;
\item[•]\textbf{Flusso degli eventi}: l'allievo ha effettuato il login e viene reinderizzato alla propria dashboard.
\end{itemize}

\subsubsection{UC 3.2 - Visualizzazione progressi}
\begin{itemize}
\item[•]\textbf{Attori}: Allievo;
\item[•]\textbf{Descrizione}: l'allievo visualizza i suoi progressi;
\item[•]\textbf{Precondizione}: l'allievo visualizza la propria dashboard;
\item[•]\textbf{Postcondizione}: l'allievo visualizza i propri progressi;
\item[•]\textbf{Flusso degli eventi}:
\begin{enumerate}
	\item Visualizzazione numero esercizi corretti;
	\item Visualizzazione punteggio medio.
\end{enumerate}
\end{itemize}

\subsubsection{UC 3.3 - Visualizzazione traguardi}

\begin{figure}[H]
\centering
\includegraphics[scale=0.7]{img/UC33.png} 
\caption{Caso d'uso UC 3.3}
\end{figure}
\begin{itemize}
\item[•]\textbf{Attori}: Allievo;
\item[•]\textbf{Descrizione}: l'allievo visualizza i traguardi da lui raggiunti;
\item[•]\textbf{Precondizione}: l'allievo visualizza la propria dashboard;
\item[•]\textbf{Postcondizione}: l'allievo visualizza tutti i traguardi raggiunti;
\item[•]\textbf{Flusso degli eventi}: 
	\begin{itemize}
		\item[•] UC 3.3.1 - Visualizzazione prossimo traguardo;
		\item[•] UC 3.3.2 - Visualizzazione traguardo corrente.
	\end{itemize}
\end{itemize}

\subsubsection{UC 3.3.1 - Visualizzazione prossimo traguardo}
\begin{itemize}
\item[•]\textbf{Attori}: Allievo;
\item[•] \textbf{Descrizione}: l'allievo visualizza, nella propria dashboard, l'ammontare di punti mancante al raggiungimento del prossimo traguardo;
\item[•] \textbf{Precondizione}: l'allievo visualizza i propri traguardi;
\item[•] \textbf{Postcondizione}: l'allievo visualizza il punteggio mancante per raggiungere il traguardo successivo;
\item[•] \textbf{Flusso degli eventi}: l'allievo il punteggio mancante per il prossimo traguardo.
\end{itemize}

\subsubsection{UC 3.3.2 - Visualizzazione traguardo corrente}
\begin{itemize}
\item[•]\textbf{Attori}: Allievo;
\item[•] \textbf{Descrizione}: l'allievo visualizza, nella propria dashboard, il proprio traguardo personale raggiunto;
\item[•] \textbf{Precondizione}: l'allievo visualizza i propri traguardi;
\item[•] \textbf{Postcondizione}: l'allievo visualizza il traguardo corrente;
\item[•] \textbf{Flusso degli eventi}: l'allievo seleziona il traguardo corrente.
\end{itemize}


\subsubsection{UC 3.4 - Visualizzazione valutazioni}
\begin{itemize}
\item[•]\textbf{Attori}: Allievo;
\item[•]\textbf{Descrizione}: l'allievo visualizza le proprie valutazioni sugli esercizi;
\item[•]\textbf{Precondizione}: l'allievo visualizza la propria dashboard;
\item[•]\textbf{Postcondizione}: l'allievo visualizza tutte le valutazioni ricevute;
\item[•]\textbf{Flusso degli eventi}: l'allievo visualizza le valutazioni di tutti gli esercizi svolti, sia esercizi assegnati che svolti indipendentemente;
\item[•] \textbf{Flusso degli eventi alternativo}: 
	\begin{itemize}
		\item UC 3.4.1 - Visualizzazione valutazione esercizio		.
	\end{itemize}
\end{itemize}

\subsubsection{UC 3.4.1 - Visualizzazione valutazione esercizio}
\begin{itemize}
\item[•]\textbf{Attori}: Allievo;
\item[•]\textbf{Descrizione}: l'allievo visualizza la valutazione di un esercizio da lui precedentemente svolto;
\item[•]\textbf{Precondizione}: l'allievo sta visualizzando le sue valutazioni;
\item[•]\textbf{Postcondizione}: l'allievo visualizza il la valutazione dell'esercizio;
\item[•]\textbf{Flusso degli eventi}: l'allievo seleziona un esercizio specifico.
\end{itemize}

\subsubsection{UC 3.5 - Inserimento insegnante preferito}
\begin{itemize}
\item[•]\textbf{Attori}: Allievo;
\item[•]\textbf{Descrizione}: l'allievo inserisce un nuovo insegnante preferito;
\item[•]\textbf{Precondizione}: l'allievo visualizza la propria dashboard;
\item[•]\textbf{Postcondizione}: l'allievo ha inserito un insegnante preferito;
\item[•]\textbf{Flusso degli eventi}: l'allievo inserisce il nominativo di un insegnante da prediligere quando riceve la correzione di un esercizio.
\end{itemize}

\subsubsection{UC 3.6 - Visualizzazione insegnanti preferiti}
\begin{itemize}
	\item[•]\textbf{Attori}: Allievo;
	\item[•]\textbf{Descrizione}: l'allievo visualizza la lista degli insegnanti preferiti;
	\item[•]\textbf{Precondizione}: l'allievo visualizza la propria dashboard;
	\item[•]\textbf{Postcondizione}: l'allievo visualizza gli insegnanti preferiti.
	\item[•]\textbf{Flusso degli eventi}: l'allievo visualizza la lista degli insegnanti preferiti che ha precedentemente inserito.
\end{itemize}

\subsubsection{UC 3.7 - Rimozione insegnante preferito}
\begin{itemize}
	\item[•]\textbf{Attori}: Allievo;
	\item[•]\textbf{Descrizione}: l'allievo rimuove un insegnante preferito;
	\item[•]\textbf{Precondizione}: l'allievo visualizza la lista di insegnanti preferiti;
	\item[•]\textbf{Postcondizione}: l'allievo ha rimosso un insegnante preferito dalla lista;
	\item[•]\textbf{Flusso degli eventi}: 
	\begin{enumerate}
	 \item Seleziona insegnante dalla lista;
	 \item Elimina insegnante selezionato.
	\end{enumerate}
\end{itemize}

\subsubsection{UC 3.8 - Selezione lingua esercizi}
\begin{itemize}
	\item[•]\textbf{Attori}: Allievo;
	\item[•]\textbf{Descrizione}: l'allievo seleziona la lingua in cui vuole svolgere gli esercizi;
	\item[•]\textbf{Precondizione}: l'allievo visualizza la propria dashboard;
	\item[•]\textbf{Postcondizione}: l'allievo ha selezionato una lingua per gli esercizi;
	\item[•]\textbf{Flusso degli eventi}: l'allievo seleziona da un elenco predefinito la lingua degli esercizi.
\end{itemize}

\subsubsection{UC 3.9 - Visualizzazione elenco esercizi}
\begin{itemize}
\item[•]\textbf{Attori}: Allievo;
\item[•]\textbf{Descrizione}:  l'allievo visualizza un elenco di frasi consigliate dal sistema come esercizi oppure gli esercizi assegnati dall'insegnante;
\item[•]\textbf{Precondizione}: l'allievo visualizza la propria dashboard;
\item[•]\textbf{Postcondizione}: l'allievo visualizza un elenco di esercizi;
\item[•]\textbf{Flusso degli eventi}: l'allievo seleziona la sezione riguardante gli esercizi che possono essere da lui svolti.
\end{itemize}

\subsubsection{UC 3.10 - Esercitazione libera}
\begin{figure}[H]
	\centering
	\includegraphics[width=17cm]{img/UC310.png} 
	\caption{Caso d'uso 3.10}\label{fig:310}
\end{figure}
\begin{itemize}
\item[•]\textbf{Attori}: Allievo;
\item[•]\textbf{Descrizione}: l'allievo inserisce liberamente una frase in modo da riceverne l'analisi grammaticale;
\item[•]\textbf{Precondizione}: l'allievo visualizza la propria dashboard;
\item[•]\textbf{Postcondizione}: l'allievo ha visualizzato l'analisi grammaticale della frase inserita.
\item[•]\textbf{Flusso degli eventi}:
\begin{enumerate}
	\item UC 3.10.1 - Inserimento frase libera;
	\item UC 3.10.3 - Visualizzazione analisi automatica.
\end{enumerate}
\end{itemize}

\subsubsection{UC 3.10.1 - Inserimento frase libera}
\begin{itemize}
	\item[•]\textbf{Attori}: Allievo;
	\item[•]\textbf{Descrizione}: l'allievo può inserire una frase libera da svolgere;
	\item[•]\textbf{Precondizione}: l'allievo si è autenticato;
	\item[•]\textbf{Postcondizione}: l'allievo ha inserito una frase;
	\item[•]\textbf{Flusso degli eventi}: l'allievo inserisce una frase libera da svolgere autonomamente per poi ricevere una valutazione automatica;
	\item[•]\textbf{Estensioni}:
	\begin{enumerate}
		\item UC 3.10.2 - Visualizzazione errore inserimento frase.
	\end{enumerate}
\end{itemize}

\subsubsection{UC 3.10.2 - Visualizzazione errore inserimento frase}
\begin{itemize}
	\item[•]\textbf{Attori}: Allievo;
	\item[•]\textbf{Descrizione}: l'allievo visualizza un messaggio di errore durante l'inserimento di una frase;
	\item[•]\textbf{Precondizione}: l'allievo ha inserito una frase;
	\item[•]\textbf{Postcondizione}: l'allievo ha visualizzato il messaggio di errore e può inserire una nuova frase;
	\item[•]\textbf{Flusso degli eventi}: durante l'inserimento di una frase libera nel sistema tale frase non viene accettata, viene visualizzato un errore.
\end{itemize}

\subsubsection{UC 3.10.3 - Visualizzazione analisi automatica}
\begin{itemize}
	\item[•]\textbf{Attori}: Allievo, Libreria di pos-tagging;
	\item[•]\textbf{Descrizione}: l'allievo visualizza l'analisi grammaticale della frase inserita;
	\item[•]\textbf{Precondizione}: l'allievo ha inserito una frase libera;
	\item[•]\textbf{Postcondizione}: l'allievo visualizza l'analisi grammaticale della frase inserita;
	\item[•]\textbf{Flusso degli eventi}: l'allievo ha richiesto l'analisi automatica della frase da lui inserita.
\end{itemize}

\subsubsection{UC 3.11 - Svolgimento esercizio}
\begin{figure}[H]
	\centering
	\includegraphics[width=17cm]{img/UC311.png} 
	\caption{Caso d'uso 3.11}\label{fig:311}
\end{figure}
\begin{itemize}
	\item[•]\textbf{Attori}: Allievo;
	\item[•]\textbf{Descrizione}: l'allievo può svolgere l'esercizio scegliendo le classi grammaticali per ciascuna parola da un' apposita lista;
	\item[•]\textbf{Precondizione}: l'allievo ha selezionato un esercizio;
	\item[•]\textbf{Postcondizione}: l'allievo ha svolto un esercizio;
	\item[•]\textbf{Flusso degli eventi}:
	\begin{enumerate}
		\item Selezione esercizio;
		\item UC 3.11.3 - Classificazione parola;
		\item UC 3.12 - Visualizzazione soluzione.
	\end{enumerate}
	\item[•]\textbf{Estensioni}:
	\begin{enumerate}
		\item UC 3.11.1 - Interruzione svolgimento esercizio;
	\end{enumerate}
	\item[•] \textbf{Flusso alternativo}:
	\begin{itemize}
				\item UC 3.11.2 - Selezione insegnante.
	\end{itemize}
\end{itemize}

\subsubsection{UC 3.11.1 - Interruzione svolgimento esercizio}
\begin{itemize}
	\item[•]\textbf{Attori}: Allievo;
	\item[•]\textbf{Descrizione}: l'allievo interrompe lo svolgimento di un esercizio;
	\item[•]\textbf{Precondizione}: l'allievo inizia a svolgere un esercizio;
	\item[•]\textbf{Postcondizione}: l'allievo interrompe l'esercizio, torna nella sezione di visualizzazione elenco esercizi;
	\item[•]\textbf{Flusso degli eventi}: durante lo svolgimento di un esercizio l'allievo lo interrompe, scartando i dati inseriti fino a quel momento, e ritorna nella sezione di visualizzazione elenco esercizi.
\end{itemize}

\subsubsection{UC 3.11.2 - Selezione insegnante}
\begin{itemize}
	\item[•]\textbf{Attori}: Allievo;
	\item[•]\textbf{Descrizione}: l'allievo seleziona l'insegnante da cui vuole ricevere la correzione, se nessun insegnante ha predisposto quella frase verrà utilizzato il sistema di correzione automatico;
	\item[•]\textbf{Precondizione}: l'allievo ha selezionato un esercizio;
	\item[•]\textbf{Postcondizione}: l'allievo seleziona l'insegnante dal quale vuole ricevere la correzione;
	\item[•]\textbf{Flusso degli eventi}: durante lo svolgimento di un esercizio  l'allievo seleziona l'insegnante da cui vuole ricevere la correzione dell'esercizio.
\end{itemize}

\subsubsection{UC 3.11.3 - Classificazione parola}
\begin{itemize}
	\item[•]\textbf{Attori}: Allievo;
	\item[•]\textbf{Descrizione}: l'allievo assegna una classe grammaticale ad una parola;
		\item[•]\textbf{Precondizione}: l'allievo sta svolgendo un esercizio.
	\item[•]\textbf{Postcondizione}: l'allievo seleziona la classe grammaticale di una parola.
	\item[•]\textbf{Flusso degli eventi}: durante lo svolgimento di un esercizio  l'allievo seleziona i tag da una lista predefinita: nome, pronome, articolo, aggettivo, verbo, preposizione, congiunzione, avverbio ed esclamazione.
\end{itemize}


\subsubsection{UC 3.12 - Visualizzazione soluzione}
\begin{figure}[H]
	\centering
	\includegraphics[width=17cm]{img/UC312.png} 
	\caption{Caso d'uso UC 3.12}\label{fig:312}
\end{figure}
\begin{itemize}
	\item[•]\textbf{Attori}: Allievo;
	\item[•]\textbf{Descrizione}: l'allievo visualizza la correzione secondo l'insegnante selezionato in precedenza;
	\item[•]\textbf{Precondizione}: l'allievo ha svolto un esercizio;
	\item[•]\textbf{Postcondizione}: l'allievo visualizza la soluzione dell'esercizio;
	\item[•]\textbf{Flusso degli eventi}:
	\begin{enumerate}
		\item UC 3.12.1 - Visualizzazione valutazione esercizio.  
	\end{enumerate}
\end{itemize}


\subsubsection{UC 3.12.1 - Visualizzazione valutazione esercizio}   

\begin{itemize}
\item[•]\textbf{Attori}: Allievo;
\item[•]\textbf{Descrizione}:  l'allievo riceve una valutazione di un esercizio svolto;
\item[•]\textbf{Precondizione}: l'allievo visualizza la soluzione dell'esercizio svolto;
\item[•]\textbf{Postcondizione}: l'allievo visualizza una valutazione sull'esercizio svolto;
\item[•]\textbf{Flusso degli eventi}: l'allievo ha ri
\end{itemize}

\subsubsection{UC 3.13 - Modifica dati utente}
\begin{figure}[H]
	\centering
	\includegraphics[width=13cm]{img/modificadatiutenteallievo.png} 
	\caption{Caso d'uso UC 3.13}\label{fig:312}
\end{figure}
\begin{itemize}
\item[•] \textbf{Attori}: Allievo;
\item[•] \textbf{Descrizione}: l'allievo modifica i propri dati personali, cioè tutti i dati personali inseriti in fase di registrazione;
\item[•] \textbf{Precondizione}: l'allievo visualizza la propria dashboard;
\item[•] \textbf{Postcondizione}: l'allievo ha modificato uno o più dati personali;
\item[•] \textbf{Flusso degli eventi}:
\begin{enumerate}
	\item[•] Selezione procedura modifica dati personali;
	\item[•] UC 3.13.1 - Modifica email;
	\item[•] UC 3.13.2 - Modifica nome;
	\item[•] UC 3.13.3 - Modifica cognome;
	\item[•] UC 3.13.4 - Modifica password;
	\item[•] UC 3.13.5 - Modifica data di nascita;
	\item[•] UC 3.13.6 - Modifica lingua
	 interfaccia applicativo;
\end{enumerate}
\end{itemize}


\subsubsection{UC 3.13.1 - Modifica email}
\begin{itemize}
	\item[•]\textbf{Attori}: Allievo;
	\item[•]\textbf{Descrizione}: l'allievo modifica la propria email;
	\item[•]\textbf{Precondizione}: l'allievo sta modificando i propri dati personali;
	\item[•]\textbf{Postcondizione}: l'allievo ha modificato la propria email; 
	\item[•]\textbf{Flusso degli eventi}: 
	\begin{enumerate}
		\item Selezione campo email;
		\item Modifica la stringa che rappresenta la propria email.
	\end{enumerate}
\end{itemize}
\subsubsection{UC 3.13.2 - Modifica nome}
\begin{itemize}
	\item[•]\textbf{Attori}: Allievo;
	\item[•]\textbf{Descrizione}: l'allievo modifica il proprio nome;
	\item[•]\textbf{Precondizione}: l'allievo sta modificando i propri dati personali;
	\item[•]\textbf{Postcondizione}: l'allievo ha modificato il proprio nome; 
	\item[•]\textbf{Flusso degli eventi}: 
	\begin{enumerate}
		\item Selezione campo nome;
		\item Modifica la stringa che rappresenta il nome.
	\end{enumerate}
\end{itemize}
\subsubsection{UC 3.13.3 - Modifica cognome}
\begin{itemize}
	\item[•]\textbf{Attori}: Allievo;
	\item[•]\textbf{Descrizione}: l'allievo modifica il proprio cognome;
	\item[•]\textbf{Precondizione}: l'allievo sta modificando i propri dati personali;
	\item[•]\textbf{Postcondizione}: l'allievo ha modificato il proprio cognome; 
	\item[•]\textbf{Flusso degli eventi}: 
	\begin{enumerate}
		\item Selezione campo cognome;
		\item Modifica la stringa che rappresenta il cognome.
	\end{enumerate}
\end{itemize}
\subsubsection{UC 3.13.4 - Modifica password}
\begin{figure}[H]
	\centering
	\includegraphics[width=15cm, keepaspectratio]{img/UC3134.png} 
	\caption{Caso d'uso UC 3.13.4}\label{fig:3134}
\end{figure}
\begin{itemize}
	\item[•]\textbf{Attori}: Allievo;
	\item[•]\textbf{Descrizione}: l'allievo modifica la propria password;
	\item[•]\textbf{Precondizione}: l'allievo sta modificando i la propria password personale;
	\item[•]\textbf{Postcondizione}: l'allievo ha modificato la propria password personale; 
	\item[•]\textbf{Flusso degli eventi}: 
	\begin{enumerate}
		\item UC 3.13.4.1 - Inserimento nuova password;
		\item UC 3.13.4.2 - Inserimento conferma nuova password.
	\end{enumerate}
\end{itemize}
\subsubsection{UC 3.13.5 - Modifica data di nascita}
\begin{itemize}
	\item[•]\textbf{Attori}: Allievo;
	\item[•]\textbf{Descrizione}: l'allievo modifica il proprio cognome;
	\item[•]\textbf{Precondizione}: l'allievo sta modificando i propri dati personali;
	\item[•]\textbf{Postcondizione}: l'allievo ha modificato il proprio cognome; 
	\item[•]\textbf{Flusso degli eventi}: 
	\begin{enumerate}
		\item Selezione campo data di nascita;
		\item Modifica la stringa che rappresenta la data di nascita, inserendo il valore corretto.
	\end{enumerate}
\end{itemize}
\subsubsection{UC 3.13.6 - Modifica lingua interfaccia applicativo}
\begin{itemize}
	\item[•]\textbf{Attori}: Allievo;
	\item[•]\textbf{Descrizione}: lo sviluppatore modifica la lingua dell'applicativo;
	\item[•]\textbf{Precondizione}: lo sviluppatore sta modificando i propri dati personali;
	\item[•]\textbf{Postcondizione}: lo sviluppatore ha modificato la lingua dell'applicativo; 
	\item[•]\textbf{Flusso degli eventi}: 
	\begin{enumerate}
		\item Selezione campo dati lingua applicativo;
		\item Selezione da un elenco predefinito la lingua dell'applicativo desiderata.
	\end{enumerate}
\end{itemize}

\subsubsection{UC 3.13.4.1 - Inserimento nuova password}
\begin{itemize}
	\item[•]\textbf{Attori}: Allievo;
	\item[•]\textbf{Descrizione}: l'allievo inserisce la nuova password;
	\item[•]\textbf{Precondizione}: l'allievo sta modificando i propri dati personali;
	\item[•]\textbf{Postcondizione}: l'allievo ha inserito il valore della nuova password; 
	\item[•]\textbf{Flusso degli eventi}: 
	\begin{enumerate}
		\item Selezione campo dati relativo alla nuova password;
		\item Inserimento stringa rappresentante la password.
	\end{enumerate}
	\item[•]\textbf{Estensioni}:
	\begin{enumerate}
		\item UC 3.13.4.3 - Visualizzazione messaggio formato password non valido.
	\end{enumerate}
\end{itemize}

\subsubsection{UC 3.13.4.2 - Inserimento conferma nuova password}
\begin{itemize}
	\item[•]\textbf{Attori}: Allievo;
	\item[•]\textbf{Descrizione}: l'allievo inserisce conferma la nuova password, reinserendola nell'apposito campo;
	\item[•]\textbf{Precondizione}: l'allievo sta modificando i propri dati personali;
	\item[•]\textbf{Postcondizione}: l'allievo ha inserito il valore del campo conferma nuova password; 
	\item[•]\textbf{Flusso degli eventi}: 
	\begin{enumerate}
		\item Selezione campo dati relativo alla conferma nuova password;
		\item Inserimento stringa rappresentante la password.
	\end{enumerate}
	\item[•]\textbf{Estensioni}:
	\begin{enumerate}
		\item UC 3.13.4.4 - Visualizzazione messaggio password diverse.
	\end{enumerate}
\end{itemize}

\subsubsection{UC 3.13.4.3 - Visualizzazione messaggio formato password non valido}
\begin{itemize}
	\item[•]\textbf{Attori}: Allievo;
	\item[•]\textbf{Descrizione}: l'allievo ha inserito una password con un formato non valido;
	\item[•]\textbf{Precondizione}: l'allievo sta modificando i propri dati personali;
	\item[•]\textbf{Postcondizione}: l'allievo visualizza un messaggio di errore relativo all'inserimento di una password che non rispetta un formato valido; 
	\item[•]\textbf{Flusso degli eventi}: l'allievo ha inserito una password che non rispetta i criteri accettati dal sistema, pertanto riceve un messaggio che indica la presenza di un formato non adatto.
\end{itemize}

\subsubsection{UC 3.13.4.4 - Visualizzazione messaggio password diverse}
\begin{itemize}
	\item[•]\textbf{Attori}: Allievo;
	\item[•]\textbf{Descrizione}: l' allievo ha inserito un valore di conferma password che non corrisponde al valore della nuova password inserita precedentemente, pertanto visualizza un messaggio che indica che le due password non corrispondono;
	\item[•]\textbf{Precondizione}: l'allievo ha inserito il valore del campo conferma nuova password;
	\item[•]\textbf{Postcondizione}: l'allievo visualizza un messaggio di errore relativo all'inserimento di una password che non combacia con quella inserita nel campo nuova password; 
	\item[•]\textbf{Flusso degli eventi}: l'allievo ha inserito una password che non combacia con quella inserita nel campo nuova password, pertanto riceve un messaggio che indica la presenza di tale difformità.
\end{itemize}

\subsubsection{UC 3.14 - Logout}
\begin{itemize}
    \item[•] \textbf{Attori}: Allievo;
    \item[•] \textbf{Descrizione}: l'allievo effettua il logout dal sistema;
    \item[•] \textbf{Precondizione}: l'allievo si è autenticato;
    \item[•] \textbf{Postcondizione}: l'allievo effettua il logout dal sistema e viene reindirizzato alla pagina di login;
    \item[•] \textbf{Flusso degli eventi}: l'allievo seleziona il bottone di logout e esce dalla sessione.
\end{itemize}