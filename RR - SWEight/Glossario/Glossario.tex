\documentclass[a4paper, oneside, openany, dvipsnames, table]{article}
\usepackage{../template/SWEightStyle}
\newcommand{\Titolo}{Manuale Utente}

\newcommand{\Gruppo}{SWEight}

\newcommand{\Approvatore}{Damien Ciagola}
\newcommand{\Redattori}{Alberto Bacco \newline Sebastiano Caccaro \newline Gheorghe Isachi \newline Gionata Legrottaglie}
\newcommand{\Verificatori}{Francesco Corti \newline Francesco Magarotto}

\newcommand{\pathimg}{../template/img/logoSWEight.png}

\newcommand{\Versionedoc}{1.0.0}

\newcommand{\Distribuzione}{\proponente \newline Prof. Vardanega Tullio \newline Prof. Cardin Riccardo \newline Gruppo SWEight}

\newcommand{\Uso}{Esterno}

\newcommand{\NomeProgetto}{Colletta}

\newcommand{\Mail}{SWEightGroup@gmail.com}

\newcommand{\DescrizioneDoc}{Questo documento si occupa di fornire le modalità di utilizzo del software Colletta commissionato}

\renewcommand\thesection{}

\begin{document}
\copertina{}

\definecolor{greySWEight}{RGB}{255, 71, 87}
\definecolor{greyROwSWEight}{RGB}{234, 234, 234}

\section*{Registro delle modifiche}
{
	\rowcolors{2}{greyROwSWEight}{white}
	\renewcommand{\arraystretch}{1.5}
	\centering
	\begin{longtable}{ c c C{4cm}  c  c }
		
		\rowcolor{greySWEight}
		\textcolor{white}{\textbf{Versione}} & \textcolor{white}{\textbf{Data}} & \textcolor{white}{\textbf{Descrizione}} & \textcolor{white}{\textbf{Nominativo}} & \textcolor{white}{\textbf{Ruolo}}\\
		1.2.2 & 2019-02-25 & Ampliamento sezione 5.4 e 3.2.5.2 & Alberto Bacco & \reda{} \\
		
		1.2.1 & 2019-02-23 & Aggiunta sezione 3.2.5.8 Checkstyle & Sebastiano Caccaro & \reda{} \\		
		
		1.2.0 & 2019-02-20 & Aggiunta scelte tecnologiche 3.2.4.2, da 3.4.5.4 a 3.4.5.7, 4.3.1.4, 4.3.2.2, 4.4.6 e figlie & Sebastiano Caccaro & \reda{} \\	
		
		1.1.5 & 2019-02-20 & Modifica sezione 2 & Alberto Bacco & \reda{} \\
		
		1.1.4 & 2019-02-18 & Correzione errori grammatica, spostate sottosezioni di asana da 4.3 a 5.2, & Alberto Bacco & \reda{} \\
		
		1.1.3 & 2019-02-14 & Riorganizzazione e correzione errori sezione 5 & Enrico Muraro & \reda{} \\
		
		1.1.2 & 2019-02-03 & Modifica sottosezione 4.1.10, 4.3.1.4, 4.3.1.5, 4.3.1.6 & Alberto Bacco& \reda{} \\	
		
		1.1.1 & 2019-01-31 & Modifica struttura e contenuti sezione 3  & Damien Ciagola & \reda{} \\	
		
		1.1.0 & 2019-01-27 & Sezione Qualità 4.2 & Sebastiano Caccaro & \reda{} \\	
		
		1.0.1 & 2019-01-25 & Parziale ristrutturazione della struttura del documento & Sebastiano Caccaro & \reda{} \\		
		
		1.0.0 & 2019-01-11 & Approvazione per il rilascio & Sebastiano Caccaro & \Res{} \\
		
		0.9.0 & 2019-01-9 & Verifica finale & Francesco Corti & \ver{} \\
		
		0.9.0 & 2019-01-8 & Aggiunta lista di controllo & Gionata Legrottaglie & \reda{} \\
		
		0.8.0 & 2018-12-23 & Correzioni errori ortografici & Gionata Legrottaglie & \reda{} \\
		
		0.7.0 & 2018-12-20 & Verifica documento & Francesco Corti & \ver{}\\
		
		0.6.0 & 2018-12-18 & Aggiunta sottosezione 5.2.2.2, 5.2.2.3, 5.2.2.4 & Francesco Magarotto & \reda{} \\
		
		0.5.2 & 2018-12-16 & Modifica sezione 4.1.5.3 & Alberto Bacco & \reda{} \\
		
		0.5.2 & 2018-12-16 & Modifica sezione 4.1.5.3 & Alberto Bacco & \reda{} \\
		
		0.5.2 & 2018-12-16 & Aggiunte sottosezioni  & Alberto Bacco & \reda{} \\
		
		0.5.1 & 2018-12-15 & Aggiunte sottosezioni 5.3, 5.4, 5.5, 5.6, 5.7, 5.8 & Alberto Bacco & \reda{} \\
		
		0.5.0 & 2018-12-15 & Aggiunta sezione 5 e sottosezioni 5.1, 5.2 & Gionata Legrottaglie & \reda{} \\
		
		0.4.1 & 2018-12-11 & Aggiunta sezione 4.1.7.3.1 & Francesco Magarotto & \reda{} \\ 
		
		0.4.0 & 2018-12-10 & Aggiunte sottosezioni 4.1.5, 4.1.6, 4.1.7, 4.1.8 & Gionata Legrottaglie & \reda{} \\ 
		0.4.0 & 2018-12-09 & Aggiunta sezione 4 e sottosezioni 4.1.1, 4.1.2, 4.1.3, 4.1.4 & Gionata Legrottaglie & \reda{} \\ 
		
		0.3.1 & 2018-12-07 & Aggiunta sottosezione 3.2 & Gionata Legrottaglie & \reda{} \\ 
		
		0.3.0 & 2018-12-06 & Aggiunta sezione 3 e sottosezione 3.1 & Gionata Legrottaglie & \reda{} \\ 
		
		0.2.0 & 2018-12-05 & Aggiunti i riferimenti & Gionata Legrottaglie & \reda{} \\ 
		
		0.1.0 & 2018-11-30 & Aggiunta introduzione & Gionata Legrottaglie & \reda{} \\
		
		0.0.1 & 2018-11-28 & Creazione scheletro del documento & Gionata Legrottaglie & \reda{}\\
		
	\end{longtable}

}
\newpage
\tableofcontents
\newpage


\newpage
\section{A}  
\textbf{Ambiente cloud}:\\	 Indica un paradigma di erogazione di servizi offerti on demand da un fornitore ad un cliente finale attraverso la rete Internet come: l'archiviazione, l'elaborazione o la trasmissione dati, a partire da un insieme di risorse preesistenti, configurabili e disponibili in remoto.

\textbf{Apple MacOS}:\\ Nome del sistemo operativo sviluppato da Apple Inc.

\textbf{Apprendimento automatico}:\\	Rappresenta un insieme di metodi che vengono utilizzati statisticamente per migliorare progressivamente la performance di un algoritmo nell'identificare pattern nei dati. Questi metodi  permettono ad un elaboratore di apprendere nel tempo come cambiare il suo comportamento in base alla eleborazione di dati.

\textbf{Attività}:\\ Lavoro, esplicazione di lavoro, di energia (anche non materiale) da parte di singoli o di gruppi.


\newpage
\section{B}
\textbf{Bot}:\\	Programma che accede alla rete attraverso lo stesso tipo di canali utilizzati dagli utenti, per automatizzare i compiti che risultano gravosi o complessi per gli utenti umani, per esempio accedere alle pagine web, inviare messaggi in una chat e  muoversi nei videogiochi. 

\textbf{Branch}:\\ Comanda di Git utilizzato per l’implementazione di funzionalità tra loro isolate, cioè sviluppate in modo indipendente l’una dall’altra ma a partire dalla medesima radice.

%% Da Sist 
\textbf{Broker}:\\ Organizza e gestisce i messaggi arrivati dai vari applicativi.

%% Controllare se c'è Hardware nel glossario !!!!!!!!
\textbf{Bug}:\\	Identifica un errore nella scrittura del codice sorgente, può essere prodotto dal compilatore di un software. Questo porta a comportamenti anomali o non previsti del programma. Meno comunemente, il termine bug può indicare un difetto di progettazione in un componente hardware, che ne causa un comportamento imprevisto o comunque diverso da quello specificato dal produttore.


\newpage
\section{C}
\textbf{Capitolato}:\\	Documento con lo scopo di facilitare e velocizzare la comprensione del contratto d'appalto. Esso descrive specifiche tecniche, caratteristiche generali e modalità di realizzazione degli intenti del committente.

\textbf{Caso d'uso}:\\	Tecnica usata nei processi di ingegneria del software per effettuare in maniera esaustiva e non ambigua, la raccolta dei requisiti al fine di produrre software di qualità. Essa consiste nel valutare ogni requisito focalizzandosi sugli attori che interagiscono col sistema, valutandone le varie interazioni.

%% da sistemare 
\textbf{Card}:\\ Visualizazzione grafica della issue che può essere spostata tra le varie sezioni sulla bacheca del progetto per avere un quadro generale della situazione.

\textbf{Client}:\\	Applicativo che risiede sull'elaboratore dell'utente e che serve per richiedere dati, risorse o elaborazioni da parte di un computer centrale, che prende il nome di server.

\textbf{Codice numerico gerarchico}:\\  Indica la struttura organizzativa usata per avere tracciamento immediato delle varie sottocategorie che la compongono, ed  arrivare  facilmente al vertice della gerarchia.

\textbf{Commit}:\\ E' un comando del software Git usato per salvare cambiamenti effettuati nella repository locale. Per includere i cambiamenti effettuati nel commit bisogna esplicitamente aggiungere i file aggiornati, modificati o creati, che si vogliono aggiungere alla repository.


\textbf{Committente}:\\	Chi ordina ad altri l’esecuzione di un lavoro, di una prestazione, o l’acquisto di una merce per conto proprio. E' il soggetto titolare del potere decisionale e di spesa relativo alla gestione dell'appalto.

%% ?? 
\textbf{Connettori}:\\ Micro-funzioni che aiutano l'utente a personalizzare la sua routine.

\textbf{Consumer}:\\ L'utilizzatore di servizi prodotti dal sistema, in questo caso i messaggi creati dal producer.

\textbf{Convenzione}:\\	Accordo collettivo volto a determinare le caratteristiche che deve contenere il software finale.

\textbf{Conflitto}:\\	Interferenza tra due file, si verifica quando il sistema automatico Git non sa quale sia il file che si vuole tenere dei due.

\textbf{Cross-platform}:\\	 Software per computer implementato su più piattaforme di elaborazione. Il software Cross-platform può essere diviso in due tipi; uno richiede la costruzione o la compilazione individuale per ogni piattaforma che supporta e l'altro può essere eseguito direttamente su qualsiasi piattaforma senza preparazione speciale

\textbf{Cruscotto}:\\ Vedi Dashboard.

\textbf{Custom}:\\	Realizzato su misura in base alle necessità o alle funzioni specifiche che è destinato a soddisfare.	


\newpage
\section{D}
\textbf{Dashboard}:	\\ Una schermata di gestione e monitoraggio dei dati a disposizione.

\textbf{Diagramma di Gantt}:\\	Serve a pianificare un insieme di attività in un certo periodo di tempo. La struttura è organizzata in un piano cartesiano in cui nelle ascisse si dispone la scala temporale dall’inizio alla fine del progetto, e nelle ordinate le cose da fare per portare a termine il progetto. Il tempo necessario per svolgere un compito è rappresentato visivamente sul diagramma con una barra colorata che va dalla data di inizio alla data di fine dell’attività.

\textbf{Diagramma UML}:\\ Schema grafico basato su UML (Unified Modeling Language) con lo scopo di rappresentare visivamente un sistema insieme ai suoi attori principali, ruoli, azioni, artefatti o classi, al fine di comprendere, alterare, mantenere o documentare meglio le informazioni riguardo al sistema.

\textbf{Distribuzione Linux}:\\	E' una distribuzione software del sistema operativo Linux, realizzata a partire dal kernel Linux, un sistema di base GNU e solitamente anche diversi altri applicativi (talvolta anch'essi parte di GNU). Tali distribuzioni appartengono quindi alla sotto-famiglia dei sistemi operativi GNU e, più in generale, alla famiglia dei sistemi Unix-like, perché ispirati a Unix e in certa misura compatibili con esso.


\newpage
\section{E}
\textbf{Efficacia}:\\ In diritto, idoneità di un atto a produrre gli effetti che conseguono al suo compimento, e dunque capacità dell'atto di rendere effettive le finalità perseguite dalle parti con la sua valida conclusione. 

\textbf{Efficienza}:\\ Capacità di raggiungere gli obiettivi fornendo adeguate prestazioni e minimizzando le risorse.

\newpage
\section{F}
\textbf{Fonte}:\\ Provenienza da cui giungono i dati. Essa può essere fisica, come ad esempio un Insegnante o un Allievo, oppure un sistema automatico come la libreria di PoS-tagging.

\textbf{FreeLing}:\\ Libreria C++ che fornisce funzionalità di analisi del linguaggio (analisi morfologica, rilevamento di entità con nome, PoS-tagging, parsing, etichettatura di ruolo semantica, ecc.) per una varietà di lingue.

\newpage
\section{G}
\textbf{Gerarchia}:\\ Sistema asimmetrico di graduazione e organizzazione delle cose di tipo piramidale.  
Come nel caso di "è il capo di...", "è parte di...", o "è meglio di...". Tali relazioni sono "asimmetriche" ovvero funzionano "in un senso" ma non funzionano "nell'altro".

\textbf{Git}:\\	 Software di controllo versione distribuito utilizzabile da interfaccia a riga di comando o GUI, creato da Linus Torvalds nel 2005.

\newpage
\section{I}
\textbf{Issue}:\\ Problema del quale si necessita di una soluzione. Una issue viene associata univocamente ad un task la cui soluzione deve essere assegnata tramite ticketing ad un membro del team.


\newpage
\section{K}
\textbf{Keyword}:\\	In un testo, le parole che vengono identificate come significative. Singolarmente una keyword è anche detta parola chiave.

\newpage
\section{L}
\textbf{Learning curve}:\\ Indica il rapporto tra tempo necessario per l'apprendimento e quantità di informazioni correttamente apprese. In informatica viene usata per descrivere la qualità di un programma. Infatti, tanto più un programma è intuitivo, ben progettato, organizzato e strutturato, minore sarà il tempo che un utente impiegherà per imparare a usarlo. 

\textbf{Librerie}:\\ Un insieme di funzioni o strutture dati predefinite e predisposte per essere collegate ad un software. Con lo scopo di fornire una collezione di entità di base pronte per l'uso.

\textbf{Linguaggio di markup}:\\  Insieme di regole che descrivono i meccanismi di rappresentazione (strutturali, semantici, presentazionali) di un testo, facendo uso di convenzioni rese standard, tali regole sono utilizzabili su più supporti.

\newpage
\section{M}
\textbf{Manutenibilità}:\\ Capacità di perfezionare o correggere un progetto, attraverso operazioni sistematiche.

\textbf{Maturità}:\\ E' una caratteristica, basata su uno standard di best practice, che facilita la valutazione della qualità di un processo software. 

\textbf{Merge}:\\	Comando di Git che incorpora le modifiche dai commit nominati (dal momento in cui le loro storie si sono discostate dal ramo corrente) nel ramo corrente. Questo comando viene utilizzato per incorporare le modifiche da un altro repository e può essere utilizzato manualmente per unire le modifiche da un ramo all'altro.

\textbf{Milestone}:\\	Indica importanti traguardi intermedi nello svolgimento del progetto. E deve corrispondere ad un incremento certo del prodotto.

\textbf{Microsoft Windows}:\\ Il nome del sistemo operativo prodotto da Microsoft Corporation dal 1985.

\textbf{ML}:\\	Machine learning. Tradotto Apprendimeto automatico vedi definizione sopra.

\textbf{Modello incrementale}:\\	Il modello di sviluppo software basato sulla successione di una serie di passi. Il ciclo dei passi può essere iterativo fino al soddisfacimento dei requisiti del cliente. Questo modello ha il vantaggio di creare prototipi che favoriscono l’interazione con il cliente così da favorire il dialogo e la validazione dei requisiti.


\newpage
\section{N}
\textbf{Nodi}:\\	Le reti Bayesiane rappresentano un grafo aciclico orientato dove:
\begin{itemize}
\item i nodi rappresentano le variabili;
\item gli archi rappresentano le relazioni di dipendenza statistica tra le variabili e le distribuzioni locali di probabilità dei nodi figlio rispetto ai valori dei nodi padre.
\end{itemize}

\textbf{Notazione a cammello}:\\	Pratica di scrivere parole composte o frasi unendo tutte le parole tra loro, ma lasciando le loro iniziali maiuscole.

\newpage
\section{O}
\textbf{Open Source}:\\	Viene utilizzato per riferirsi ad un software di cui i detentori dei diritti rendono pubblico il codice sorgente, favorendone il libero studio e permettendo a programmatori indipendenti di apportarvi modifiche ed estensioni. Questa possibilità è regolata tramite l'applicazione di apposite licenze d'uso. Il fenomeno ha tratto grande beneficio da Internet, perché esso permette a programmatori distanti di coordinarsi e lavorare allo stesso progetto.

\newpage
\section{P}
\textbf{Pattern publisher/subscriber}:\\	 Si riferisce a un design pattern o stile architetturale utilizzato per la comunicazione asincrona fra diversi processi, oggetti o altri.

\textbf{PDF}:\\	Portable Document Format. Formato di file basato su un linguaggio di descrizione di pagina sviluppato da Adobe Systems nel 1993 per rappresentare documenti di testo e immagini in modo indipendente dall'hardware e dal software utilizzati per generarli o per visualizzarli.

\textbf{Periodo}:\\	Intervallo di tempo dove viene descritto esattamente quello che si sta facendo nella fase attuale del progetto.

\textbf{Periodi di investimento}:\\	Si intende periodo dove vengono consumate tante risorse senza un esito immediato però si ha un buon risultato finale.

\textbf{Plugin}:\\	Programma non autonomo che interagisce con un software autonomo per ampliarne o estenderne le funzionalità originarie.

\textbf{PoS-tagging}:\\	Il procedimento per riconoscere quale sia la categoria lessicale e le relative sottocategorie di ogni parola nel contesto nel qualè usata.

\textbf{Porting}:\\	Processo di trasposizione, a volte anche con modifiche, di un componente software, volto a consentirne l'uso in un ambiente di esecuzione diverso da quello originale.

\textbf{Pull request}:\\	Consentono di comunicare agli altri membri del team, le modifiche che hai inviato a un repository GitHub. Una volta inviata una pull request, le parti interessate possono rivedere il set di modifiche, discutere i potenziali cambiamenti con i collaboratori e aggiungere commenti prima che le modifiche vengano unite nel ramo base(branch).

\textbf{Producer}:\\	Si occupa di creare e inoltrare le informazioni generate dal sistema.

\textbf{Project board}:\\	E' uno strumento messo a disposizione da Git che fa parte della struttura di gestione del progetto e ha i seguenti compiti:
\begin{itemize}
\item Essere responsabile per il successo o il fallimento del progetto;
\item Per fornire una direzione unificata al progetto e al Responsabile di Progetto;
\item Fornire le risorse e autorizzare i fondi per il progetto;
\item Fornire supporto visibile e sostenuto per l'Amministratore del Progetto;
\item Garantire una comunicazione efficace all'interno del team di progetto e con le parti interessate esterne.
\end{itemize}
\textbf{Proponente}:\\	Chi presenta una proposta, che presenta qualcosa affinché venga accettato, approvato.


\newpage
\section{Q}
\textbf{Qualità}:\\	Misura in cui un prodotto software soddisfa un determinato numero di aspettative rispetto sia al suo funzionamento sia alla sua struttura interna.


\newpage
\section{R}
\textbf{Repository}:\\	Archivio o un ambiente di un sistema informativo nel quale sono raccolti e conservati dati e informazioni corredati da descrizioni (metadati) in formato digitale, e direttamente accessibile dagli utenti. I repository rappresentano in qualche misura l’equivalente elettronico di una biblioteca.

\textbf{Requisito}:\\	Condizione necessaria per conseguire uno scopo.

\textbf{Risorsa}:\\	Persona, equipaggiamento, impianto, o qualsiasi altra cosa necessaria al compimento di una attività di progetto. Due dimensioni che caratterizzano una risorsa sono:
\begin{itemize}
\item disponibilità;
\item costo.
\end{itemize} 
Generalmente risulta variabile in relazione al grado di specializzazione della risorsa. Nel caso di risorse umane ad esempio può variare a seconda del livello di esperienza.


\newpage
\section{S}
\textbf{Skill}:\\	Capacità di fare bene qualcosa ovvero un'abilità acquisita o imparata.

\textbf{Slack}:\\	Software che rientra nella categoria degli strumenti di collaborazione aziendale utilizzato per inviare messaggi in modo istantaneo ai membri del team.

\textbf{Software-as-service}:\\	Modello di distribuzione del software applicativo dove un produttore di software sviluppa, opera e gestisce un'applicazione web che mette a disposizione dei propri clienti via Internet.

\textbf{Software di controllo di versione}:\\ E' il programma che si occupa della  gestione di versioni multiple di un insieme di informazioni.

\textbf{Sotto-caso d'uso}:\\
Un "sotto-caso d'uso" deve fornire lo stesso servizio generale del "super-caso d'uso", eventualmente producendo valore aggiuntivo, o fornendolo a qualche tipologia di attore aggiuntiva; o seguendo un procedimento parzialmente diverso per ottenere il risultato.

\textbf{Standard ISO}:\\	International Organization for Standardization. L'organizzazione internazionale che emette standard in tutti i campi interessati.

\textbf{Standard ISO$\backslash$IEC}:\\	International Electrotechnical Commission. E' un'organizzazione internazionale per la definizione di standard in materia di elettricità, elettronica e tecnologie correlate. Molti dei suoi standard sono definiti in collaborazione con l'ISO.

\textbf{Strategia}:\\	Capacità di raggiungere obiettivi importanti predisponendo, nel lungo termine e con lungimiranza, i mezzi atti a tale scopo.

\textbf{Specifiche UML}:\\	Unified Modeling Language. E' un linguaggio di modellazione e specifica basato sul paradigma orientato agli oggetti. Con specifiche  sono descritti i casi d'uso in cui sono specificati uno per uno gli elementi utili per lo sviluppo corretto del progetto.


\newpage
\section{T}
\textbf{Tag}:\\		Etichetta, tipicamente un comando inserito tra parentesi angolari.

\textbf{Telematica}:\\	 Quando l'informatica ha incontrato le tecnologie delle telecomunicazioni, è nata la telematica. Questo termine, dunque, designa l'insieme delle applicazioni che consentono il trasferimento di dati a distanza, per esempio per mezzo di una rete. 

\textbf{Template}:\\ E' un documento creato come riferimento per la struttura esterna o interna di processi o file, chiamato anche "modello" o "struttura base / scheletro", o più correntemente "modulo".

\textbf{Tool}:\\	Strumento software che esegue una specifica funzione.

\textbf{Troubleshooting}:\\	Consiste nell'identificazione del malfunzionamento e in una ricerca della sua causa attraverso un processo di eliminazione progressiva delle possibili cause conosciute.


\newpage
\section{V}
\textbf{Validazione}:\\	Processo mediante il quale ci si accerta che i requisiti siano stati rispettati.

\textbf{Verifica}:\\	Operazione di controllo per mezzo della quale si procede all'accertamento della correttezza dei risultati, nella forma e nei loro contenuti.

\textbf{Versione}:\\ Indica lo stato di avanzamento di un documento o di un processo software.

\textbf{Versionamento}:\\	Gestione di versioni multiple di un insieme di informazioni.


\end{document}