\documentclass[a4paper, oneside, openany, dvipsnames, table]{article}
\usepackage{../template/SWEightStyle}
\newcommand{\Titolo}{Manuale Utente}

\newcommand{\Gruppo}{SWEight}

\newcommand{\Approvatore}{Damien Ciagola}
\newcommand{\Redattori}{Alberto Bacco \newline Sebastiano Caccaro \newline Gheorghe Isachi \newline Gionata Legrottaglie}
\newcommand{\Verificatori}{Francesco Corti \newline Francesco Magarotto}

\newcommand{\pathimg}{../template/img/logoSWEight.png}

\newcommand{\Versionedoc}{1.0.0}

\newcommand{\Distribuzione}{\proponente \newline Prof. Vardanega Tullio \newline Prof. Cardin Riccardo \newline Gruppo SWEight}

\newcommand{\Uso}{Esterno}

\newcommand{\NomeProgetto}{Colletta}

\newcommand{\Mail}{SWEightGroup@gmail.com}

\newcommand{\DescrizioneDoc}{Questo documento si occupa di fornire le modalità di utilizzo del software Colletta commissionato}



\begin{document}
\copertina{}

%\definecolor{greySWEight}{RGB}{255, 71, 87}
\definecolor{greyROwSWEight}{RGB}{234, 234, 234}

\section*{Registro delle modifiche}
{
	\rowcolors{2}{greyROwSWEight}{white}
	\renewcommand{\arraystretch}{1.5}
	\centering
	\begin{longtable}{ c c C{4cm}  c  c }
		
		\rowcolor{greySWEight}
		\textcolor{white}{\textbf{Versione}} & \textcolor{white}{\textbf{Data}} & \textcolor{white}{\textbf{Descrizione}} & \textcolor{white}{\textbf{Nominativo}} & \textcolor{white}{\textbf{Ruolo}}\\
		1.2.2 & 2019-02-25 & Ampliamento sezione 5.4 e 3.2.5.2 & Alberto Bacco & \reda{} \\
		
		1.2.1 & 2019-02-23 & Aggiunta sezione 3.2.5.8 Checkstyle & Sebastiano Caccaro & \reda{} \\		
		
		1.2.0 & 2019-02-20 & Aggiunta scelte tecnologiche 3.2.4.2, da 3.4.5.4 a 3.4.5.7, 4.3.1.4, 4.3.2.2, 4.4.6 e figlie & Sebastiano Caccaro & \reda{} \\	
		
		1.1.5 & 2019-02-20 & Modifica sezione 2 & Alberto Bacco & \reda{} \\
		
		1.1.4 & 2019-02-18 & Correzione errori grammatica, spostate sottosezioni di asana da 4.3 a 5.2, & Alberto Bacco & \reda{} \\
		
		1.1.3 & 2019-02-14 & Riorganizzazione e correzione errori sezione 5 & Enrico Muraro & \reda{} \\
		
		1.1.2 & 2019-02-03 & Modifica sottosezione 4.1.10, 4.3.1.4, 4.3.1.5, 4.3.1.6 & Alberto Bacco& \reda{} \\	
		
		1.1.1 & 2019-01-31 & Modifica struttura e contenuti sezione 3  & Damien Ciagola & \reda{} \\	
		
		1.1.0 & 2019-01-27 & Sezione Qualità 4.2 & Sebastiano Caccaro & \reda{} \\	
		
		1.0.1 & 2019-01-25 & Parziale ristrutturazione della struttura del documento & Sebastiano Caccaro & \reda{} \\		
		
		1.0.0 & 2019-01-11 & Approvazione per il rilascio & Sebastiano Caccaro & \Res{} \\
		
		0.9.0 & 2019-01-9 & Verifica finale & Francesco Corti & \ver{} \\
		
		0.9.0 & 2019-01-8 & Aggiunta lista di controllo & Gionata Legrottaglie & \reda{} \\
		
		0.8.0 & 2018-12-23 & Correzioni errori ortografici & Gionata Legrottaglie & \reda{} \\
		
		0.7.0 & 2018-12-20 & Verifica documento & Francesco Corti & \ver{}\\
		
		0.6.0 & 2018-12-18 & Aggiunta sottosezione 5.2.2.2, 5.2.2.3, 5.2.2.4 & Francesco Magarotto & \reda{} \\
		
		0.5.2 & 2018-12-16 & Modifica sezione 4.1.5.3 & Alberto Bacco & \reda{} \\
		
		0.5.2 & 2018-12-16 & Modifica sezione 4.1.5.3 & Alberto Bacco & \reda{} \\
		
		0.5.2 & 2018-12-16 & Aggiunte sottosezioni  & Alberto Bacco & \reda{} \\
		
		0.5.1 & 2018-12-15 & Aggiunte sottosezioni 5.3, 5.4, 5.5, 5.6, 5.7, 5.8 & Alberto Bacco & \reda{} \\
		
		0.5.0 & 2018-12-15 & Aggiunta sezione 5 e sottosezioni 5.1, 5.2 & Gionata Legrottaglie & \reda{} \\
		
		0.4.1 & 2018-12-11 & Aggiunta sezione 4.1.7.3.1 & Francesco Magarotto & \reda{} \\ 
		
		0.4.0 & 2018-12-10 & Aggiunte sottosezioni 4.1.5, 4.1.6, 4.1.7, 4.1.8 & Gionata Legrottaglie & \reda{} \\ 
		0.4.0 & 2018-12-09 & Aggiunta sezione 4 e sottosezioni 4.1.1, 4.1.2, 4.1.3, 4.1.4 & Gionata Legrottaglie & \reda{} \\ 
		
		0.3.1 & 2018-12-07 & Aggiunta sottosezione 3.2 & Gionata Legrottaglie & \reda{} \\ 
		
		0.3.0 & 2018-12-06 & Aggiunta sezione 3 e sottosezione 3.1 & Gionata Legrottaglie & \reda{} \\ 
		
		0.2.0 & 2018-12-05 & Aggiunti i riferimenti & Gionata Legrottaglie & \reda{} \\ 
		
		0.1.0 & 2018-11-30 & Aggiunta introduzione & Gionata Legrottaglie & \reda{} \\
		
		0.0.1 & 2018-11-28 & Creazione scheletro del documento & Gionata Legrottaglie & \reda{}\\
		
	\end{longtable}

}
\newpage
\tableofcontents
\newpage

\newpage
\section{A}
\begin{itemize}
\item  \textbf{attività}:\\	Insieme coeso di task.
\end{itemize}

\newpage
\section{B}
\begin{itemize}
\item \textbf{broker}:\\	 Organizza e gestisce i messaggi arrivati dai vari applicativi.
\end{itemize}

\newpage
\section{C}
\begin{itemize}
\item \textbf{capitolato}:\\	Documento tecnico, in genere allegato ad un contratto di appalto, che vi fa riferimento per definire in quella sede le specifiche tecniche delle opere che andranno ad eseguirsi per effetto del contratto stesso, di cui è solamente parte integrante.
\end{itemize}

\begin{itemize}
\item \textbf{connettori}:\\	Sono delle micro-funzioni che aiuta l'utente a personalizzare la sua routine.
\end{itemize}

\begin{itemize}
\item \textbf{consumer}:\\	L'utilizzatore di servizi prodotti dal sistema, in questo caso i messaggi creati dal producer.
\end{itemize}

\begin{itemize}
\item \textbf{custom}:\\	Realizzato su misura in base alle necessità  o della funzione specifica che è destinato ad accontentare.	
\end{itemize}

\newpage
\section{D}
\begin{itemize}
\item \textbf{diagramma di Gantt}:\\	E' costruito partendo da un asse orizzontale - a rappresentazione dell'arco temporale totale del progetto, suddiviso in fasi incrementali (ad esempio, giorni, settimane, mesi) - e da un asse verticale - a rappresentazione delle mansioni o attività che costituiscono il progetto.
\end{itemize}


\newpage
\section{L}
\begin{itemize}
\item \textbf{librerie}:\\	Un insieme di funzioni o strutture dati predefinite e predisposte per essere collegate ad un software.
\end{itemize}


\newpage
\section{M}
\begin{itemize}
\item \textbf{manutenibilità}:\\	Rappresenta il requisito indispensabile del sistema per ottimizzare l'implementazione delle attività manutentive.

\end{itemize}

\begin{itemize}
\item \textbf{milestone}:\\		Indica importanti traguardi intermedi nello svolgimento del progetto.
\end{itemize}

\begin{itemize}
\item \textbf{modello incrementale}:\\	E' basato sulla successione dei seguenti passi 6 principali:
\begin{itemize}
\item[-]pianificazione;
\item[-]analisi dei requisiti;
\item[-]progetto;
\item[-]implementazione;
\item[-]prove;
\item[-]valutazione;
\end{itemize}
\end{itemize}

\newpage
\section{N}
\begin{itemize}
\item \textbf{nodi}:\\	Le reti Bayesiane rappresentano un grafo aciclico orientato dove:
\begin{itemize}
\item[-] i nodi rappresentano le variabili;
\item[-] gli archi rappresentano le relazioni di dipendenza statistica tra le variabili e le distribuzioni locali di probabilità dei nodi figlio rispetto ai valori dei nodi padre.
\end{itemize}
\end{itemize}

\newpage
\section{P}

\begin{itemize}
\item \textbf{pattern publisher/subscriber}:\\	 Si riferisce a un design pattern o stile architetturale utilizzato per la comunicazione asincrona fra diversi processi, oggetti o altri.
\end{itemize}

\begin{itemize}
\item \textbf{periodo}:\\	Intervallo di tempo dove viene descritto esattamente quello che si sta facendo nella fase attualedel progetto.
\end{itemize}

\begin{itemize}
\item \textbf{periodi di investimento}:\\	Si intende periodo dove vengono consumate tante risorse senza un esito immediato pero si ha un buon risultato finale.

\end{itemize}

\begin{itemize}
\item \textbf{plugin}:\\	Programma non autonomo che interagisce con un software autonomo per ampliarne o estenderne le funzionalità originarie.
\end{itemize}

\begin{itemize}
\item \textbf{producer}:\\	Si occupa di creare e inoltrare le informazioni generate dal sistema.
\end{itemize}

\begin{itemize}
\item \textbf{proponente}:\\	Chi presenta una proposta, che presenta qualcosa affinché venga accettato, approvato.
\end{itemize}

\newpage
\section{Q}
\begin{itemize}
\item \textbf{qualità}:\\	Misura in cui un prodotto software soddisfa un certo numero di aspettative rispetto sia al suo funzionamento sia alla sua struttura interna.
\end{itemize}

\newpage
\section{S}
\begin{itemize}
\item \textbf{skill}:\\	Capacità di fare bene qualcosa ovvero un'abilità acquisita o imparata.
\end{itemize}

\begin{itemize}
\item \textbf{software-as-service}:\\	Modello di distribuzione del software applicativo dove un produttore di software sviluppa, opera e gestisce un'applicazione web che mette a disposizione dei propri clienti via Internet.
\end{itemize}

\begin{itemize}
\item \textbf{strategie}:\\	Capacità di raggiungere obiettivi importanti predisponendo, nel lungo termine e con lungimiranza, i mezzi atti a tale scopo.
\end{itemize}


\newpage
\section{T}
\begin{itemize}
\item \textbf{troubleshooting}:\\	Consiste inizialmente nell'identificazione del malfunzionamento e quindi in una ricerca della sua causa attraverso un processo di eliminazione progressiva delle possibili cause conosciute.
\end{itemize}



\end{document}