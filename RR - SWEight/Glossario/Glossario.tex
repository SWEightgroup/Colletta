\documentclass[a4paper, oneside, openany, dvipsnames, table]{article}
\usepackage{../template/SWEightStyle}
\newcommand{\Titolo}{Verbale Riunione 2018-12-12}

\newcommand{\Gruppo}{SWEight}

\newcommand{\ACapoRedazione}{Francesco Magarotto}

\newcommand{\Verifica}{Francesco Corti}

\newcommand{\Approvazione}{Sebastiano Caccaro}

\newcommand{\Distribuzione}{Vardanega Tullio \newline Cardin Riccardo \newline Gruppo SWEight}

\newcommand{\Uso}{Interno}

\newcommand{\NomeProgetto}{Colletta}

\newcommand{\Mail}{SWEightGroup@gmail.com}

\newcommand{\DescrizioneDoc}{Questo documento si occupa di riportare quanto discusso nella riunione del 12-12-2018}



\begin{document}
\copertina{}

%\definecolor{greySWEight}{RGB}{255, 71, 87}
\definecolor{greyROwSWEight}{RGB}{234, 234, 234}

\section*{Registro delle modifiche}
{
	\rowcolors{2}{greyROwSWEight}{white}
	\renewcommand{\arraystretch}{1.5}
	\centering
	\begin{longtable}{ c c  C{4cm}  c  c }
		
		\rowcolor{greySWEight}
		\textcolor{white}{\textbf{Versione}} & \textcolor{white}{\textbf{Data}} & \textcolor{white}{\textbf{Descrizione}} & \textcolor{white}{\textbf{Nominativo}} & \textcolor{white}{\textbf{Ruolo}}\\
		
		1.0.2 & 2019-03-02 & Aggiunti nuovi termini del documento Piano di Progetto & Isachi Gheorghe &\reda{}\\
		
		1.0.1 & 2019-02-23 & Verifica del documento &  Francesco Corti & \ver{}\\
		
		1.0.1 & 2019-02-20 & Aggiunti nuovi termini del documento Norme di Progetto & Isachi Gheorghe &\reda{}\\
		
		1.0.0 & 2019-01-09 & Approvazione & Sebastiano Caccaro & \Res{}\\
						
		0.1.1 & 2019-01-08 & Verifica del documento & Bacco Alberto & \ver{}\\
		
		0.1.1 & 2019-01-04 & Aggiunti termini del documento Norme di Progetto & Isachi Gheorghe &\reda{}\\
		
		0.1.0 & 2019-01-01 & Aggiunti termini del documento Analisi dei Requisiti & Isachi Gheorghe &\reda{}\\
		
		0.0.4 & 2018-12-29 & Verifica del documento & Bacco Alberto & \ver{}\\
				
		0.0.4 & 2018-12-27 & Aggiunti termini del documento Piano di Qualifica & Isachi Gheorghe &\reda{}\\
				
		0.0.3 & 2018-12-26 &Aggiunti termini del documento Piano di Progetto & Isachi Gheorghe & \reda{}\\
				
		0.0.2 & 2018-12-17 & Aggiunti termini del documento Studio di Fattibilità & Isachi Gheorghe &\reda{}\\
		
		0.0.1 & 2018-12-15 & Scheletro del glossario & Damien Ciagola & \reda{}\\
		
	\end{longtable}

}
\newpage
\tableofcontents
\newpage

\newpage
\section{A}
\begin{itemize}
\item  \textbf{attività}:\\	Insieme coeso di task.
\end{itemize}

\newpage
\section{B}
\begin{itemize}
\item \textbf{broker}:\\	 Organizza e gestisce i messaggi arrivati dai vari applicativi.
\end{itemize}

\newpage
\section{C}
\begin{itemize}
\item \textbf{capitolato}:\\	Documento tecnico, in genere allegato ad un contratto di appalto, che vi fa riferimento per definire in quella sede le specifiche tecniche delle opere che andranno ad eseguirsi per effetto del contratto stesso, di cui è solamente parte integrante.
\end{itemize}

\begin{itemize}
\item \textbf{connettori}:\\	Sono delle micro-funzioni che aiuta l'utente a personalizzare la sua routine.
\end{itemize}

\begin{itemize}
\item \textbf{consumer}:\\	L'utilizzatore di servizi prodotti dal sistema, in questo caso i messaggi creati dal producer.
\end{itemize}

\begin{itemize}
\item \textbf{custom}:\\	Realizzato su misura in base alle necessità  o della funzione specifica che è destinato ad accontentare.	
\end{itemize}

\newpage
\section{D}
\begin{itemize}
\item \textbf{diagramma di Gantt}:\\	E' costruito partendo da un asse orizzontale - a rappresentazione dell'arco temporale totale del progetto, suddiviso in fasi incrementali (ad esempio, giorni, settimane, mesi) - e da un asse verticale - a rappresentazione delle mansioni o attività che costituiscono il progetto.
\end{itemize}


\newpage
\section{L}
\begin{itemize}
\item \textbf{librerie}:\\	Un insieme di funzioni o strutture dati predefinite e predisposte per essere collegate ad un software.
\end{itemize}


\newpage
\section{M}
\begin{itemize}
\item \textbf{manutenibilità}:\\	Rappresenta il requisito indispensabile del sistema per ottimizzare l'implementazione delle attività manutentive.

\end{itemize}

\begin{itemize}
\item \textbf{milestone}:\\		Indica importanti traguardi intermedi nello svolgimento del progetto.
\end{itemize}

\begin{itemize}
\item \textbf{modello incrementale}:\\	E' basato sulla successione dei seguenti passi 6 principali:
\begin{itemize}
\item[-]pianificazione;
\item[-]analisi dei requisiti;
\item[-]progetto;
\item[-]implementazione;
\item[-]prove;
\item[-]valutazione;
\end{itemize}
\end{itemize}

\newpage
\section{N}
\begin{itemize}
\item \textbf{nodi}:\\	Le reti Bayesiane rappresentano un grafo aciclico orientato dove:
\begin{itemize}
\item[-] i nodi rappresentano le variabili;
\item[-] gli archi rappresentano le relazioni di dipendenza statistica tra le variabili e le distribuzioni locali di probabilità dei nodi figlio rispetto ai valori dei nodi padre.
\end{itemize}
\end{itemize}

\newpage
\section{P}

\begin{itemize}
\item \textbf{pattern publisher/subscriber}:\\	 Si riferisce a un design pattern o stile architetturale utilizzato per la comunicazione asincrona fra diversi processi, oggetti o altri.
\end{itemize}

\begin{itemize}
\item \textbf{periodo}:\\	Intervallo di tempo dove viene descritto esattamente quello che si sta facendo nella fase attualedel progetto.
\end{itemize}

\begin{itemize}
\item \textbf{periodi di investimento}:\\	Si intende periodo dove vengono consumate tante risorse senza un esito immediato pero si ha un buon risultato finale.

\end{itemize}

\begin{itemize}
\item \textbf{plugin}:\\	Programma non autonomo che interagisce con un software autonomo per ampliarne o estenderne le funzionalità originarie.
\end{itemize}

\begin{itemize}
\item \textbf{producer}:\\	Si occupa di creare e inoltrare le informazioni generate dal sistema.
\end{itemize}

\begin{itemize}
\item \textbf{proponente}:\\	Chi presenta una proposta, che presenta qualcosa affinché venga accettato, approvato.
\end{itemize}

\newpage
\section{Q}
\begin{itemize}
\item \textbf{qualità}:\\	Misura in cui un prodotto software soddisfa un certo numero di aspettative rispetto sia al suo funzionamento sia alla sua struttura interna.
\end{itemize}

\newpage
\section{S}
\begin{itemize}
\item \textbf{skill}:\\	Capacità di fare bene qualcosa ovvero un'abilità acquisita o imparata.
\end{itemize}

\begin{itemize}
\item \textbf{software-as-service}:\\	Modello di distribuzione del software applicativo dove un produttore di software sviluppa, opera e gestisce un'applicazione web che mette a disposizione dei propri clienti via Internet.
\end{itemize}

\begin{itemize}
\item \textbf{strategie}:\\	Capacità di raggiungere obiettivi importanti predisponendo, nel lungo termine e con lungimiranza, i mezzi atti a tale scopo.
\end{itemize}


\newpage
\section{T}
\begin{itemize}
\item \textbf{troubleshooting}:\\	Consiste inizialmente nell'identificazione del malfunzionamento e quindi in una ricerca della sua causa attraverso un processo di eliminazione progressiva delle possibili cause conosciute.
\end{itemize}



\end{document}