\section{Progettazione}
Dopo aver terminato la fase di Analisi si passerà a quella di Progettazione che vede come protagonisti i progettisti, durante la quale questi devono trovare una soluzione soddisfacente al problema, definire un’architettura logica e i vari diagrammi che la rappresentano.
La progettazione permette di: 
\begin{itemize}
\item[•] Ottimizzare l'uso delle risorse;
\item[•] Garantire la qualità del prodotto sviluppato;
\item[•] Suddividere il problema principale in tanti sotto problemi di complessità minore;
\end{itemize}

\subsection{Architettura logica}
Bisogna definire un'architettura logica del prodotto che dovrà: 
\begin{itemize}
\item[•] Soddisfare i requisiti definiti nel documento di Analisi dei Requisiti v.1.0.0;
\item[•] Essere sicura in caso di malfunzionamenti o intrusioni;
\item[•] Essere modulare e formato da componenti riutilizzabili;
\item[•] Essere modulare e formato da componenti riutilizzabili;
\item[•] Essere affidabile;
\item[•] Essere comprensibile per future manutenzioni.
\end{itemize}
\subsection{Diagrammi UML}
Con l'obiettivo di rendere chiare le soluzioni progettuali utilizzate, è necessario l'utilizzo di {diagrammi UML}\ped{G}. Quest'ultimi devono essere realizzati utilizzando lo standard 2.0.

È richiesta la realizzazione di:
\begin{itemize}
\item[•] Diagrammi delle attività: descrivono un processo o un algoritmo;
\item[•] Diagrammi dei package;
\item[•] Diagrammi di sequenza: rappresentano una sequenza di processi o funzioni;
\item[•] Diagrammi di classi: rappresentano le classi utilizzate e le loro relazioni.
\end{itemize}
In caso vengano utilizzati dei Design Pattern sarà necessario accompagnarli con una descrizione ed un diagramma UML.

\subsubsection{Astah}
Per la realizzazione dei diagrammi UML si fa uso del software Astah, selezionando la voce "Use case diagram". 
I casi d'uso verranno esportati come immagini secondo le specifiche sotto riportate. Per esportare i diagrammi in Astah, dopo aver completato il caso d'uso, selezionare: $Tools \rightarrow Export\ image \rightarrow Current\ Diagram \rightarrow PNG$.
L'immagine rappresentante l'UML deve avere le seguenti caratteristiche:
\begin{itemize}
\item Nome file: UC\texttt{Codice Identificativo}, privo di "."\\
Ad esempio: UC4.2.png $\rightarrow$ UC42.png;
\item Formato file immagine \texttt{.png};
\item Massima larghezza 17cm;
\item Didascalia figura con codice caso d'uso;
\item Il file deve essere salvato all'interno del folder img, destinato alle immagini.
\end{itemize}

\subsection{Design Pattern}

I progettisti devono utilizzare il design pattern che ritengono più adatto al contesto per rendere l'applicazione più sicura ed efficiente possibile.

Ogni utilizzo di design pattern deve essere brevemente descritto ed accompagnato da un diagramma UML che ne esemplifica il funzionamento.

\section{Codifica}
In questa sezione vengono descritte le norme che i programmatori devono seguire con l'obiettivo di scrivere codice leggibile, affidabile e mantenibile.

Questa sezione verrà aggiornata durante la fase di progettazione

