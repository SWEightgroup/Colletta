\section{Processi primari}
\subsection{Fornitura}
In questa sezione vengono descritte le norme che il team deve rispettare al fine di diventare fornitori nei confronti del proponente nell'ambito della Progettazione, Sviluppo e Validazione del prodotto.
\subsubsection{Studio di fattibilità}
In seguito alla pubblicazione dei capitolati è compito del Responsabile di Progetto convocare l'intero team al fine di discutere degli aspetti positivi e negativi dei vari capitolati.

Successivamente, gli Analisti devono effettuare lo Studio di Fattibilità per ogni capitolato, attenendosi anche a quello emerso dalla riunione precedente; per ogni capitolato deve essere scritto un documento contenente:
\begin{itemize}
\item \textbf{Informazioni sul capitolato};
\item \textbf{Descrizione}: una sintesi del prodotto da realizzare secondo le richieste del capitolato;
\item \textbf{Dominio applicativo}: il contesto in cui l'applicazione opera;
\item \textbf{Dominio tecnologico}: rappresenta il dominio tecnologico richiesto dal capitolato;
\item \textbf{Considerazioni del gruppo}: racchiude tutte le valutazioni dei membri del gruppo che hanno portato ad una accettazione o ad un rifiuto;
\item \textbf{Valutazione finale}: descrive le motivazioni decisive che hanno portato il gruppo a scegliere o a rigettare il capitolato.
\end{itemize}

%I documenti devono essere scritti sfruttando il linguaggio di %markup\ped{G} \LaTeX utilizzando il modello che lo stile del %documento fonrnitoli dall'Amministratore.

Infine, questi documenti devono essere racchiusi in un unico documento: Studio di Fattibilità v 1.0.0, che verrà sottoposto a verifica dai componenti preposti a tale compito.

\subsection{Sviluppo}
\subsubsection{Analisi dei requisiti}
Gli Analisti devono redigere l'Analisi dei Requisiti al fine di:
\begin{itemize}
    \item[•] Fornire ai Progettisti riferimenti affidabili e precisi;
    \item[•] Facilitare le revisioni del codice;
    \item[•] Descrivere lo scopo del progetto;
    \item[•] Fissare le funzionalità ed i requisiti concordati con il proponente;
    \item[•] Fornire ai Verificatori riferimenti per l'attività di test circa i casi d'uso principali e alternativi.
\end{itemize}

Deve essere redatto, sempre dagli stessi, un documento che elenchi i requisiti, classificandoli come di seguito riportato.

\paragraph{Classificazione dei requisiti}\mbox{}\\
I vari {requisiti}\ped{G} potranno provenire dalle seguenti fonti:
\begin{itemize}
    \item[•] \textbf{Capitolato}\ped{G}: il requisito è stato scritto esplicitamente nel documento fornito dal proponente;
    \item[•] \textbf{Verbali interni e studio di fattibilità}: il requisito è emerso durante la discussione del capitolato tra gli Analisti;
    \item[•] \textbf{Casi d'uso}\ped{G}: il requisito è il risultato dell'analisi di uno o più casi d'uso.
\end{itemize}

I vari requisiti devono essere classificati secondo la seguente {convenzione}\ped{G}:
\begin{center}
R-[Importanza][Tipo][Identificativo]
\end{center}
Dove:
\begin{itemize}
\item[•] Importanza, indica se il Requisito è:
\begin{itemize}
\item[•] \textbf{1}: Requisito obbligatorio;
\item[•] \textbf{2}: Requisito desiderabile;
\item[•] \textbf{3}: Requisito opzionale.
\end{itemize}
\item[•] Tipo, indica se il requisito è:
\begin{itemize}
\item \textbf{F}: Requisito funzionale;
\item \textbf{Q}: Requisito di qualità;
\item \textbf{P}: Requisito prestazionale;
\item \textbf{V}: Requisito di vincolo.
\end{itemize}
\item[•] Identificativo: ovvero un codice univoco che contraddistingue il requisito.
\end{itemize}
Ogni identificativo di requisito non deve essere più modificato nel tempo, per evitare incongruenze tra i documenti.
Deve essere realizzata una tabella che classifichi tutti i requisiti, incorporando una descrizione e la provenienza del requisito, indicata con il termine fonte.
\begin{table}[H]
\centering
\begin{tabularx}{\linewidth}{| c | c | c | c | Z |}
\hline
\textbf{Identificativo} & \textbf{Importanza} & \textbf{Tipo} & \textbf{Fonte} & \textbf{Descrizione} \\
\hline
\textbf{3V001} & Requisito obbligatorio & Di vincolo & Capitolato & Gli sviluppatori devono poter accedere ai dati raccolti gratuitamente \\
\hline
\end{tabularx}
\caption{Esempio tabella classificazione requisiti}
\end{table}
\subsubsection{Casi d'uso}
Dopo la stesura dei requisiti, gli Analisti devono analizzare tutti i casi d'uso (d'ora in poi UC); ognuno di essi deve essere composto da un codice identificativo univoco, eventuali diagrammi UML e infine dalla sua descrizione testuale. In particolare, per quanto riguarda l'identificazione si procederà come descritto di seguito:
\begin{center}
UC[\texttt{Codice identificativo}]
\end{center}
Dove \texttt{Codice identificativo} consiste in codice numerico {gerarchico}\ped{G} e univoco.
Nel caso si trattasse di un {sottocaso d'uso}\ped{G}, pertanto generato da un caso d'uso generico, si utilizzerà il seguente costrutto:
\begin{center}
$\text{UC}\underbrace{[\texttt{Codice identificativo UC generico}].[\texttt{Codice indentificativo UC specifico}]}_{\text{Codice   identificativo}}$
\end{center}
L'utilizzo del "." (punto) viene utilizzato per determinare la profondità della gerarchia dei casi d'uso creatasi.
Per quanto riguarda la descrizione testuale dei casi d'uso i membri del gruppo interessati seguiranno il seguente modello:
\begin{figure}[H]
\centering
\includegraphics[scale=0.7]{example-image-b}
\caption{Caso d'uso Codice univoco o breve descrizione nel caso della panoramica attore}
\end{figure}
UC \texttt{Codice univoco} - Titolo del caso d'uso, la cui lunghezza deve essere ristretta
\begin{itemize}
\item[•] \textbf{Attore}: gli attori coinvolti nell'interazione con il sistema;
\item[•] \textbf{Precondizione}: definisce lo stato del sistema, pertanto anche le condizioni che devono essere vere, prima dell'esecuzione del caso d'uso;
\item[•] \textbf{Postcondizione}: definisce lo stato del sistema dopo il verificarsi del caso d'uso; 
\item[•] \textbf{Flusso degli eventi}: (Non obbligatorio) il susseguirsi di eventi che conducono alla postcondizione. Deve essere realizzato con un elenco numerato, facendo riferimento eventualmente ad ulteriori casi d'uso, oppure con una breve descrizione testuale;
\item[•] \textbf{Estensioni}: eventuali estensioni coinvolte;
\item[•] \textbf{Inclusioni}: eventuali inclusioni coinvolte;
\item[•] \textbf{Flusso degli eventi alternativo}: (Non obbligatorio) il susseguirsi alternativo di eventi che alternativi al flusso principale. 
\end{itemize}



