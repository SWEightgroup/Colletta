\section{Introduzione}
\subsection{Scopo del documento}
Il documento ha lo scopo di definire le norme che i membri del gruppo SWEight dovranno seguire e rispettare durante lo svolgimento del progetto “Colletta”. 
I membri sono tenuti a leggere il documento per garantire uniformità al fine di ottenere una migliore efficacia ed efficienza delle attività. I partecipanti potranno contattare l'\textit{Amministratore di Progetto} per eventuali suggerimenti e opinioni riguardanti le norme di progetto. 
L'\textit{Amministratore di Progetto} si dovrà consultare con i membri del team ed avrà la responsabilità di accettare o rifiutare eventuali suggerimenti proposti, argomentandone la decisione.
Il documento pone l'accento sui seguenti punti:
\begin{itemize}
\item[•] Interazioni tra il team di sviluppo ed esterni;
\item[•] Organizzazione dell'ambiente di lavoro;
\item[•] Modalità di lavoro durante lo sviluppo del progetto;
\item[•] Stesura documenti e convenzioni di scrittura utilizzate;
\item[•] Tecniche di analisi ed eventuali correzione degli errori;	
\end{itemize}
\subsection{Scopo del progetto}
Lo scopo del progetto è la realizzazione di una piattaforma interattiva per la raccolta dati relativi ad esercizi di analisi grammaticale, che verranno impiegati come dati sorgente in algoritmi di apprendimento automatico. %Il prodotto da realizzare dovrà impiegare tecnologie che funzionino sia in ambiente desktop che mobile.	
\subsection{Glossario}
Per prevenire ed evitare qualsiasi dubbio e per permettere una maggiore chiarezza e comprensione del testo su termini ambigui, abbreviazioni e acronimi utilizzati nei vari documenti, essi sono stati raccolti nel Glossario 1.0 nel quale si possono trovare tutte le informazioni desiderate. Al fine di rendere evidente un termine presente nel Glossario, esso verrà marcato con il pedice \ped{G}.
\section{Riferimenti}

\subsection{Normativi}
\begin{itemize}
\item[•]\textbf{standard {ISO}\ped{G}-8601}:\\
\url{https://en.wikipedia.org/wiki/ISO_8601};
\item[•]\textbf{standard ISO/{IEC}\ped{G} 12207}\\
\url{https://en.wikipedia.org/wiki/ISO/IEC_12207};
\item[•]\textbf{specifiche {UML}\ped{G} 2.0}:\\
\url{http://www.omg.org/spec/UML/2.0/}.

\end{itemize} 
\subsection{Informativi}
\begin{itemize}
%\item \textbf{Capitolato d’appalto C1:} Butterfly: monitor per processi CI/CD\\
%\url{https://www.math.unipd.it/~tullio/IS-1/2018/Progetto/C1.pdf}
\item \textbf{Capitolato d’appalto C2:} Colletta: piattaforma raccolta dati di analisi di testo\\
\url{https://www.math.unipd.it/~tullio/IS-1/2018/Progetto/C2.pdf}
\item \textbf{Piano di progetto} v 1.0.0;
\item \textbf{Dispense} \\
\url{http://www.math.unipd.it/~tullio/IS-1/2018/Dispense/L05.pdf}
\item \textbf{Software Engineering (10th edition)}\\ Ian Sommerville, Pearson Education;

%\item \textbf{Capitolato d’appalto C3:} G\&{B}: monitoraggio intelligente di processi DevOps\\
%\url{https://www.math.unipd.it/~tullio/IS-1/2018/Progetto/C3.pdf}
%\item \textbf{Capitolato d’appalto C4:} MegAlexa: arricchitore di skill di Amazon Alexa\\
%\url{https://www.math.unipd.it/~tullio/IS-1/2018/Progetto/C4.pdf}
%\item \textbf{Capitolato d’appalto C5:} P2PCS: piattaforma di peer-to-peer car sharing	\\
%\url{https://www.math.unipd.it/~tullio/IS-1/2018/Progetto/C5.pdf}
%\item \textbf{Capitolato d’appalto C6:} Soldino: piattaforma Ethereum per pagamenti IVA\\
%\url{https://www.math.unipd.it/~tullio/IS-1/2018/Progetto/C6.pdf}
\end{itemize}
