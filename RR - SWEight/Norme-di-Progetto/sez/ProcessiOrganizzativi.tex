\subsection{Obiettivo}
Gli obiettivi del processo di gestione sono:
\begin{itemize}
\item[•] raggiungere l’efficienza e l’efficacia seguendo un approccio sistematico dei processi software al fine di raggiungere gli obiettivi di progetto rispettando le scadenze;
\item[•] introdurre nuovi processi o migliorare quelli già esistenti.
\end{itemize}

\subsection{Incontri}

\subsubsection{Interni}
\label{sec:interni}
Una riunione dell'intero team è richiesta in caso di necessità del \textit{Responsabile di progetto}. Nel momento in cui un membro del gruppo abbia la necessità di incontrarsi con il team, deve inviare una richiesta al \textit{Responsabile di progetto} che la valuta e organizza un eventuale incontro. Questi sono fissati solo per discutere di argomenti attinenti al progetto.
Le riunioni possono essere realizzate anche in forma {telematica}\ped{G}, adottando le stesse procedure e regole utilizzate per le riunioni fisiche.
Dopo ogni riunione dovrà essere obbligatoriamente stilato un verbale che verrà strutturato secondo quanto scritto nel paragrafo \hyperref[sec:verbali]{§4.1.2.3}.
%Francesco Corti, scrivere che anche qui va fatto un verbale e aggiungere hyperreff a sezione verbale 
%va aggiunto un sub 
\subsubsection{Esterni}
Il \textit{Responsabile di progetto} è l'unico a poter fissare gli incontri con il Proponente o il Committente e successivamente informare i componenti del team. Non è necessaria la presenza dell'intero gruppo di lavoro durante gli incontri con gli esterni. Il numero di partecipanti sarà quindi limitato a 2 rappresentanti del team di sviluppo.  
\\
Il \textit{Responsabile di Progetto} deve essere sempre presente agli incontri con gli esterni mentre il secondo partecipante del team è scelto in base all'ordine del giorno. 
\newline
%Francesco Corti va il link alla sezione verbali
% 
Ogni riunione comprende la stesura di un verbale ufficiale contenente le seguenti informazioni:
\begin{itemize}
\item[•] Data e ora;
\item[•] Luogo;
\item[•] Partecipanti esterni;
\item[•] Partecipanti interni;
\item[•] Ordine del giorno;
\item[•] Domande e risposte.
\end{itemize}

\subsection{Comunicazione}

%va aggiunto un sub 
\subsubsection{Interna}

Per le comunicazioni interne al gruppo è stato adottato {Slack}\ped{G}, un'applicazione di messaggistica multipiattaforma con funzionalità specifiche per gruppi di lavoro.  
\newline
Su Slack si possono integrare {Bot}\ped{G}, creare canali tematici per rendere più efficiente lo scambio e il reperimento delle informazioni, fare ricerche di messaggi vecchi e condividere file di grosse dimensioni.   
\newline
In Slack, sono stati creati dei canali per dividere le comunicazioni con temi comuni:
\begin{itemize}
\item[•] \textbf{Analisti}: contenente le comunicazioni relative all'analisi dei requisiti;
\item[•] \textbf{Colletta}: relative alle comunicazioni generali;
\item[•] \textbf{Norme di progetto}: contenente le comunicazioni relative al documento "Norme di progetto";
\item[•] \textbf{Random}: contiene informazioni relative a domande dei componenti che possono riguardare informazioni di carattere personale;
\item[•] \textbf{Verifica}: per le comunicazioni relative alle attività di verifica;
\item[•] \textbf{Link}: contenente tutti i link utili.
\end{itemize}
Un altro strumento utilizzato per la comunicazione interna è Hangouts, il quale permette di realizzare videochiamate di gruppo utilizzando il browser web. Pertanto, come detto in \hyperref[sec:interni]{§5.1.1.1}, Hangouts è lo strumento per la realizzazione di incontri interni anche in remoto. \uppercase{è} delegato al capo progetto il compito di creare la chiamata di gruppo e aggiungere i vari componenti.
%va aggiunto un sub 
\subsubsection{Esterna}
Per le comunicazioni esterne è stata creata una casella di posta elettronica \href{SWEight@gmail.com}.
\newline 
Tale indirizzo deve essere l'unico canale di comunicazione esistente tra il gruppo di lavoro e l'esterno. 
\newline
Come descritto nel capitolato d'appalto: 
\begin{itemize}
\item[•] Il Proponente può essere contattato in qualsiasi momento, solo dal \textit{Responsabile di Progetto}, all'indirizzo e-mail \href{tech@mivoq.it} , al quale risponde il reparto tecnologico dell'azienda;
\item[•] Le e-mail devono contenere in oggetto la sigla “UNIPD-SWE" e dovranno essere indirizzate all'attenzione di Giulio Paci, che sarà il referente principale;
\item[•] Ogni e-mail ricevuta su questo indirizzo viene inoltrata automaticamente alla casella personale di ciascun membro, tramite l'utilizzo di filtri di Gmail;
\item[•] In alternativa, per le comunicazioni ritenute urgenti è possibile telefonare al numero 0490998335.
\end{itemize}

\subsection{Gestione delle responsabilità e ruoli}

\subsubsection{Introduzione}
I ruoli di progetto rappresentano le figure professionali che lavorano al progetto. Ogni membro del gruppo, in un dato momento, dovrà ricoprire un determinato ruolo.  
\newline
La rotazione è obbligatoria e si terrà conto delle preferenze personali di ognuno. Ogni membro deve rispettare il proprio ruolo e svolgere le attività assegnategli, secondo quanto stabilito nel \PdP. Si avrà l'accortezza di evitare situazioni di conflitto di interesse come, ad esempio, essere \textit{Verificatori} di documenti prodotti da sé stessi, compromettendone cosi la qualità.

\subsubsection{Responsabile di progetto}
Il \textit{Responsabile di Progetto}, o Project Manager, è una figura necessaria alla gestione dell’intero progetto. Egli raccoglie su di sé le responsabilità decisionali di scelta e approvazione e costituisce il centro di coordinamento per l'intero progetto; in particolare, rappresenta il gruppo di lavoro nei confronti del Committente e del Proponente. 
\newline
In particolare, ha l'incarico di:
\begin{itemize}
\item[•] Organizzare incontri interni ed esterni;
\item[•] Pianificare le attività svolte dal gruppo;
\item[•] Individuare per ciascun compito un membro del gruppo per svolgerlo;
\item[•] Analizzare, monitorare e gestire i rischi.
\end{itemize}

\subsubsection{Amministratore di Progetto}
L’\textit{Amministratore di Progetto} è la figura professionale che si deve gestire l’ambiente di lavoro, al fine di aumentare l’efficienza e portare qualità. 
\newline
Ha l'incarico di:
\begin{itemize}
\item[•] Ricercare nuovi strumenti che migliorino l’efficienza;
\item[•] Gestire la documentazione di progetto;
\item[•] Occuparsi del controllo di versione del prodotto;
\item[•] Occuparsi della configurazione del prodotto.
\end{itemize}

\subsubsection{Analista}
L'\textit{Analista} è la figura che svolge le attività di analisi al fine di comprendere appieno il dominio del problema.  
\newline
Ha l’incarico di:
\begin{itemize}
\item[•] Analizzare i requisiti del prodotto;
\item[•] Analizzare i requisiti di dominio;
\item[•] Redigere il documento Studio di Fattibilità;
\item[•] Redigere il documento Analisi dei Requisiti.
\end{itemize}

\subsubsection{Progettista}
Il \textit{Progettista} è il responsabile delle scelte architetturali del progetto e ne influenza gli aspetti tecnici e tecnologici.  
\newline
Partendo dalle attività dell'\textit{Analista}, il \textit{Progettista} ha il compito di trovare una possibile soluzione per i problemi e i requisiti precedentemente individuati. 
\newline
Ha l’incarico di:
\begin{itemize}
\item[•] Comprendere a fondo i requisiti nel documento Analisi dei Requisiti;
\item[•] Comprendere a fondo i requisiti nel documento Analisi dei Requisiti;
\item[•] Redigere la documentazione tecnica per il prodotto software;
\item[•] Redigere il documento Manuale Sviluppatore.
\end{itemize}

\subsubsection{Programmatore}
Il \textit{Programmatore} è la figura che provvederà alla codifica della soluzione, studiata e spiegata dal \textit{Progettista}.  
\newline
Ha l'incarico di:
\begin{itemize}
\item[•] Scrivere il codice del prodotto software che rispetti le decisioni del \textit{Progettista};
\item[•] Redigere il documento Manuale Utente.
\end{itemize}

\subsubsection{Verificatore}
Il \textit{Verificatore} è una figura presente per l'intero ciclo di vita del software e controlla che le attività svolte siano conformi alle attese e alle norme prestabilite.  
\newline
Ha  di:
\begin{itemize}
\item[•] Verificare che ciascuna attività svolta sia conforme alle norme stabilite nel progetto;
\item[•] Controllare che, per ogni stadio del ciclo di vita del prodotto, questo sia conforme al Piano di Qualifica.
\end{itemize}

\subsection{Pianificazione}
Il \textit{Responsabile di progetto} decide le scadenze tenendo conto degli impegni lavorativi e scolastici di ogni membro. Stima i costi e le risorse necessarie, pianifica le attività e le assegna alle persone. 
\newline
Il \textit{Responsabile di progetto} deve definire il piano di progetto, in cui descrive attività e compiti utili/necessari all'esecuzione di processo.  
\newline
Il piano deve:
\begin{itemize}
\item[•] Fissare un tempo di completamento dei compiti;
\item[•] Dare una stima del tempo richiesto dei compiti;
\item[•] Adeguare le risorse necessarie per eseguire i compiti;
\item[•] Allocare i compiti;
\item[•] Assegnare le responsabilità; 
\item[•] Analizzare i rischi;
\item[•] Definire delle metriche per il controllo della qualità;
\item[•] Associare dei costi al processo di esecuzione;
\item[•] Fornire un'infrastruttura.
\end{itemize}

\subsection{Monitoraggio del piano}
Il \textit{Responsabile di progetto} supervisiona l'esecuzione del progetto fornendo report interni sulla progressione del processo. 
\newline
Egli deve investigare, analizzare e risolvere i problemi scoperti durante l'esecuzione ed eventualmente può rivedere la pianificazione temporale per far fronte a tali problemi e a cambi di strategia.
\begin{itemize}
\item[•] I report redatti, in conclusione di ogni periodo, confluiranno nel Piano di Qualifica;
\item[•] Al termine del periodo di riferimento, il \textit{Responsabile di progetto} deve confrontare i risultati ottenuti con gli obiettivi prefissati e valutare strumenti, attività e processi impiegati per il completamento dei processi;
\item[•] La valutazione finale e il tracciamento di strumenti e tecnologie dovranno confluire nel Piano di Qualifica, eventuali rischi e piani di contingenza utilizzati dovranno essere riportati nel Piano di Progetto.
\end{itemize}

\subsection{Gestione dell'infrastruttura}

\subsubsection{Introduzione}
Inizializzare, implementare e gestire processi software, richiede l’istanziazione di un’infrastruttura ad hoc per l'intero progetto. In questa sezione verrà normato e descritto l'ambiente di lavoro e gli strumenti o software utilizzati.

\subsubsection{Versionamento}

%va messo un altro sub 
\paragraph{Introduzione}\mbox{}\\
Git, il più usato e conosciuto {software di controllo di {versione}\ped{G} open source, è stato scelto per il controllo di versione, GitHub invece è la piattaforma di web hosting. 
\newline
Il repository dedicato ai documenti è formata dalle seguenti subdirectory:
\begin{itemize}
\item[•] \textbf{RR}: contiene tutti i documenti da consegnare per la Revisione dei Requisiti;
\item[•] \textbf{RP}: contiene tutti i documenti da consegnare per la Revisione di Progettazione;
\item[•] \textbf{RQ}: contiene tutti i documenti da consegnare per la Revisione di Qualifica;
\item[•] \textbf{RA}: contiene tutti i documenti da consegnare per la Revisione di Accettazione.
\end{itemize}

\paragraph{Github Desktop}\mbox{}\\
Il {client}\ped{G} che viene utilizzato per il controllo di versione è Github Desktop, esso permette di gestire attraverso un'interfaccia grafica la {repository}\ped{G} al fine di consentire anche a coloro che non hanno familiarità con tale tool di apprenderlo con una {learning curve}\ped{G} meno ripida. Per utilizzare il programma è necessario recarsi al sito: \url{https://desktop.github.com/} disponibile sia per {Microsoft Windows}\ped{G} e {Apple MacOS}\ped{G}. Per quanto riguarda le varie {distribuzioni Linux}\ped{G} non è disponibile un client ufficiale pertanto è disponibile un {porting}\ped{G} non ufficiale all'indirizzo: \url{https://github.com/shiftkey/desktop}

\paragraph{Project Board}\mbox{}\\
\label{sec:projectboard}
Per la gestione delle {issue}\ped{G} si fa utilizzo di {project board}\ped{G} che permettono una gestione semplice del ciclo di vita dei documenti. 
In particolare, semplicemente trascinando la {card}\ped{G} da una colonna ad un'altra si andrà ad indicare in che stato si trova quella determinata issue.
Le colonne, indicanti lo stato in cui si trova la issue, utilizzate all'interno del ciclo di vita dei documenti sono le seguenti:
\begin{itemize}
\item[•] \textbf{To do}: indica che l'attività deve essere ancora svolta;
\item[•] \textbf{In progress}: indica che l'attività è in svolgimento;
\item[•] \textbf{Needs review}: indica che l'attività necessita di verifica;
\item[•] \textbf{Done}: indica che l'attività è conclusa;
\end{itemize}
\paragraph{Messaggi di commit}\mbox{}\\
Ogni volta che si effettuano modifiche sui file del repository locale per poi esser caricate in
quello remoto, bisogna specificarne le motivazioni. \uppercase{è} importante che i messaggi siano esplicativi e che non creino ambiguità, pertanto si scrivano le varie modifiche facendo uso di un elenco puntato.
Per la chiusura delle issue si può utilizzare la sintassi con le {keyword}\ped{G} direttamente all'interno dei {commit}\ped{G}, per fare ciò le keyword devono essere collocate alla fine del messaggio di commit. Per il loro utilizzo si rimanda alla documentazione di {Git}\ped{G}: \url{https://help.github.com/articles/closing-issues-using-keywords/}. 

\paragraph{Merge}\mbox{}\\
Nel caso in cui, nel tentativo di effettuare l'upload nella repository remota, si verifichi un {conflitto}\ped{G} è compito del componente del gruppo informare l'\textit{Amministratore di Progetto}. Nel caso in cui non si fosse in grado di effettuare il {merge}\ped{G} sfruttando il tool automatico messo a disposizione da client di versionamento, chiedere all'\textit{Amministratore di Progetto} di risolvere manualmente la cosa il prima possibile.
\subsubsection{Gestione delle milestone}
Per la gestione delle {milestone}\ped{G} è utilizzato il sistema di gestione offerto da GitHub, che permette di associare le milestone alle issue e alle pull request.
\paragraph{Creazione di una milestone}\mbox{}\\
Per la creazione di una milestone all'interno della repository da GitHub.com, si deve:
\begin{enumerate}
\item Selezionare la voce "Issues" oppure "Pull requests";
\item Selezionare la voce "Milestone" posto vicino alla barra di ricerca;
\item Selezionare la voce "New Milestone";
\item Inserire il nome della milestone, la descrizione associata ed eventuali tag;
\item Selezionare l'opzione "Salva cambiamenti".
\end{enumerate}
\paragraph{Modifica di una milestone}\mbox{}\\
Per modificare una milestone sempre su GitHub si deve:
\begin{enumerate}
\item Selezionare la voce "Issues" oppure "Pull requests";
\item Selezionare la voce "Milestone" posto vicino alla barra di ricerca;
\item Selezionare la voce "Edit";
\item Modificare il nome della milestone, la descrizione associata o eventuali tag;
\item Selezionare l'opzione "Save changes".
\end{enumerate}

\subsubsection{Strumenti di condivisione}
Un altro servizio utilizzato dal team è Google Drive, servizio web in {ambiente cloud}\ped{G}, di memorizzazione e sincronizzazione online. Questo servizio ha anche un tool aggiuntivo per l'integrazione Slack, che permette al gruppo un rapido scambio di documenti.
\subsection{Miglioramento continuo dei processi}
Processi, attività o compiti istanziati possono non essere definitivi. Se al termine di una delle fasi di progetto, un membro del gruppo dovesse trovare difficoltà, oppure se trovasse procedure più efficienti ed efficaci di quelle in uso, può decidere di proporre al \textit{Responsabile di Progetto} crearne di nuovi o modificare quelli già esistenti.  
\newline
Il \textit{Responsabile di progetto} organizzerà un incontro con tutti i membri di SWEight e procederà ad esporre i nuovi spunti suggeriti, ascoltando i pareri di tutti. 
\newline
La decisione finale spetta comunque al \textit{Responsabile di progetto}, che dopo aver valutato l'impatto di qualsiasi modifica al piano ed aver controllato che ciò non comporti l'impossibilità nel raggiungere gli obiettivi prefissati nei tempi stabiliti, potrà decidere di attuarla.

\subsection{Training process}
È il processo che fornisce e mantiene la formazione del personale. Acquisizione, supporto, sviluppo esecuzione, mantenimento del prodotto software sono largamente dipendenti dalle conoscenze e dalle capacità dei membri del gruppo. 
\newline
Per questo ognuno deve procedere in modo autonomo con lo studio individuale delle tecnologie che verranno utilizzate nel corso del progetto, prendendo come riferimento, oltre al materiale indicato nella sottosezione Riferimenti normativi, anche la seguente documentazione:
\begin{itemize}
\item[•] \textbf{Latex}: \url{https://www.latex-project.org};
\item[•] \textbf{GitHub}: \url{https://guides.github.com};
\item[•] \textbf{Git}: \url{https://git-scm.com/doc};
\item[•] \textbf{Hangout}: \url{https://support.google.com/hangouts/?hl=en};
\item[•] \textbf{Texmaker}: \url{http://www.xm1math.net/texmaker/doc.html};
\item[•] \textbf{Slack}: \url{https://get.slack.help/hc/en-us};
\item[•] \textbf{Astah}: \url{http://astah.net/manual}.
\end{itemize}