\section{Capitolato C5}
\subsection{Informazioni sul Capitolato}
\begin{itemize}
	\item \textbf{Nome:} P2PCS: piattaforma di peer-to-peer car sharing;
	\item \textbf{Proponente}: Gaiago;
	\item \textbf{Committenti}: Prof. Tullio Vardanega, Prof. Riccardo Cardin.
\end{itemize}
\subsection{Descrizione}
L'obbiettivo del capitolato è lo sviluppo di un'applicazione mobile Android\textsuperscript{TM} che offre un servizio di 
car-sharing tramite una piattaforma che, sfruttando la geolocalizzazione, permetta agli utenti di condividere l'auto conoscendone la posizione in tempo reale. 
L'intento è quello di massimizzare il tempo di utilizzo del mezzo tramite calendarizzazione, 
riducendone i costi. 
Il proponente del capitolato mette a disposizione delle librerie\ped{G} che riducono la 
complessità del progetto, semplificandone lo sviluppo.

\subsection{Dominio Applicativo}
Il dominio applicativo a cui P2PCS fa riferimento è quello del commercio elettronico Consumer to Consumer, dove gli utenti interagiscono tra di loro mettendo a disposizione un oggetto che vogliono vendere o prestare.

\subsection{Dominio Tecnologico}
\begin{itemize}
\item[•] \texttt{Hensin Moven} che fornisce il servizio per lo sviluppo e l'utilizzo dell'applicativo;
\item[•] \texttt{Android SDK} per la realizzazione dell'applicazione nativa o \texttt{Apache Cordova} per
 la realizzazione di un'applicazione web-based multi-piattaforma;
\item[•] \texttt{Node.js} come piattaforma per la realizzazione dell'applicazione lato server;
\item[•] \texttt{JavaScript} come linguaggio di scripting utilizzato all'interno della piattaforma Node.js;
\item[•] \texttt{Google Cloud Platform} per fornire una soluzione cloud dove memorizzare i dati;
\item[•] \texttt{Git} come tool di versionamento;
\item[•] \texttt{Google Location Services} per il tracciamento del mezzo durante gli spostamenti;
\end{itemize}

\subsection{Considerazioni del gruppo}
Gli aspetti ritenuti positivi sono: 
\begin{itemize}
\item[•] Utilizzo di Octalysis, basato sulla gamification;
\item[•] Creazione applicazioni Android\textsuperscript{TM} con tecnologie all'avanguardia;
\item[•] Utilizzo di un sistema cloud per l'immagazzinamento dei dati come Hensin;
\end{itemize}

Le principali criticità sono:
\begin{itemize}
\item[•] Le tecnologie adottate sono sconosciute alla maggior parte del gruppo, pertanto il loro apprendimento non è adeguato al tempo messo a disposizione. 
\item[•] Mancato interesse verso il capitolato proposto;
\end{itemize}

\subsection{Valutazione Finale}
A causa del numero di tecnologie sconosciute alla maggior parte dei membri del gruppo e vista anche la mancanza di interesse verso la maggior parte di esse, il capitolato è stato rigettato.

