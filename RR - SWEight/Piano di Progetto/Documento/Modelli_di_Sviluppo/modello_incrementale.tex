L'analisi dei rischi ha stimato un basso rischio di cambio dei requisiti da parte della proponente.
Ciò significa che è possibile adottare un modello di sviluppo che prevede un'approfondita fase di 
analisi e progettazione iniziale.
La scelta è quindi ricaduta sul modello di sviluppo incrementale, che, inoltre, presenta altre
caratteristiche congeniali alle intenzioni del gruppo \gruppo \space :
\begin{itemize}
    \item Lo sviluppo del prodotto software è diviso in varie attività singole, comportando i seguenti vantaggi:
    \begin{itemize}
    	\item Maggior semplicità dello sviluppo di un'attività;
    	\item Maggior semplicità in fase di test e troubleshooting;
    	\item Migliore parallelizzazione del lavoro;
    	\item Minore probabilità di incorrere in ritardi;
    	\item Maggiore facilità nel rilascio;
    \end{itemize}
    \item Le attività sono ordinate in ordine di importanza, e vengono sviluppati prima i requisiti di maggiore rilevanza per il committente;
    \item Creazione di {milestone}\ped{G} per suddividere meglio il lavoro e per ottenere riscontri dal committente sul soddisfacimento dei requisiti.
\end{itemize}