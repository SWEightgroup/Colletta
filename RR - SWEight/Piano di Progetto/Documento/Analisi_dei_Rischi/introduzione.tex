Per realizzare un prodotto software di {qualità}\ped{G}, è necessario
comprendere a fondo i rischi che possono occorrere durante lo svolgimento
del progetto, al fine di minimizzare l'impatto di essi sull'avanzamento
delle {attività}\ped{G}. \newline
Per ogni rischio è usata la seguente procedura:
\begin{enumerate}
    \item \textbf{Identificazione: }
           Identificazione dei rischi che si possono presentare, assegnando a ciascuno di essi un nome univoco;
    \item \textbf{Analisi: } Studio approfondito di ogni rischio, individuandone le possibili conseguenze e categorizzando ciascuno di essi per:
           \begin{itemize}
               \item Probabilità;
               \item Gravità.
           \end{itemize}
           Ognuno di questi due indici può assumere i seguenti valori:
           \begin{itemize}
               \item Bassa;
               \item Media;
               \item Alta;
           \end{itemize}
    \item \textbf{Rilevazione: }Strategie atte a individuare e prevenire l'occorrenza di un rischio;
    \item \textbf{Mitigazione: }{Strategie}\ped{G}\space da adottare in caso di occorrenza di un dato rischio;
    \item \textbf{Controllo: }Raffinamento delle strategie di gestione del rischio mediante riscontro con i rischi
                              finora incontrati.
\end{enumerate}                         
Nella seguente tabella sono riportati i rischi individuati secondo i seguente indici:
\begin{itemize}
	\item \textbf{Descrizione: }Breve descrizione del rischio; 
	\item \textbf{Probabilità};
    \item \textbf{Gravità};
    \item \textbf{Contromisure: }Breve descrizione delle strategie di mitigazione.

\end{itemize}

