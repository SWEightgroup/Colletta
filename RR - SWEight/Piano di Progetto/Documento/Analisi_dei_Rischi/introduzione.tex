Per realizzare un prodotto sofware di qualità, è necessario
compredere a fondo i rischi che possono occorrere durante lo svolgimento
del progetto, al fine di minimizzare l'impatto di essi sull'avanzamento
delle attività. \newline
Per ogni rischio è usata la seguente procedura:
\begin{enumerate}
    \item \textbf{Identificazione: }
           Identificazione dei rischi che si possono presentare, assegnando a ciascuno di essi un nome univoco e
           una categoria fra le seguenti:
           \begin{itemize}
               \item Rischi relativi all'organico del gruppo \gruppo\space, indicati con \textbf{G}
               \item Rischi tecnologici, indicati con \textbf{T}
               \item Rischi relativi all'organizzazione del lavoro, indicati con \textbf{O}
               \item Rischi relativi ai requisiti, indicati con \textbf{R}
           \end{itemize}
    \item \textbf{Analisi: } Studio approfondito di ogni rischio, individuandone le possibili conseguenze e categorizzando ciascuno di essi per:
           \begin{itemize}
               \item Probabiltià
               \item Gravità
           \end{itemize}
           Ognuno di questi due indici può assumere i seguenti valori:
           \begin{itemize}
               \item Bassa, indicata con \textbf{B}
               \item Media, indicata con \textbf{M}
               \item Alta, indicata con \textbf{A}
           \end{itemize}
    \item \textbf{Rilevazione: }Strategie atte a individuare e prevenire l'occorrenza di un rischio
    \item \textbf{Mitigazione: }Strategie da adottare in caso di occorrenza di una dato rischio
    \item \textbf{Controllo: }Raffinamento delle strategie di gestione del rischio mediante riscontro con i rischi
                              finora incontrati
\end{enumerate}                         
Nella seguente tabella sono riportati i rischi individuati secondo i seguente indici:
\begin{itemize}
    \item \textbf{Tipo: } Indicato con \textbf{T}, può essere G,T,O,R
    \item \textbf{Probabilità: } Indicata con \textbf{P}, può essere B,M,A
    \item \textbf{Gravità: }Indicata con \textbf{G}, può essere B,M,A
    \item \textbf{Descrizione: }Breve descrizone del rischio
    \item \textbf{Rilevazione: }Breve descrizione delle strategie di rilevazione
    \item \textbf{Mitigazione: }Breve descrizione delle strategie di mitigazione
\end{itemize}

