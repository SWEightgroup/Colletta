
\begin{table}[H]
\renewcommand{\arraystretch}{1.5}
\begin{tabular}{| C{.025\textwidth}| C{.2\textwidth}L{.6\textwidth}C{.15\textwidth}|}
\hline
\rowcolor{greySWEight}
	\textcolor{white}{\textbf{\#}}&
    \textcolor{white}{\textbf{Nome}} & \textcolor{white}{\textbf{Descrizione}}&
    \textcolor{white}{\textbf{Livello di Rischio}}\\
\hline
\multirow{2}*{1}&
 \textbf  	
 	{Conflitti fra i membri del gruppo}&
    Per molti membri del gruppo questa è la prima esperienza di lavoro in gruppo con un certo
    numero di persone.Ciò potrebbe causare inconvenienti di natura interpersonale.&
    
    Probabilità: \newline \textbf{Bassa}\newline
    Gravità: \newline \textbf{Alta}\\
    
    \cellcolor{white}&Contromisure&
    \multicolumn{2}{L{.77\textwidth} |}{
    Ogni problema andrà tempestivamente riportato al responsabile. Ove non sia possibile
    trovare una soluzione, il responsabile cercherà di assegnare ruoli e attività che 
    minimizzino l'interazione fra i membri in causa.
    }\\
    \hline \hline
 
% \textbf
% 	{Assenza prolungata di un membro del gruppo}&
%    \'E possibile che, a causa di problemi di salute o familiari, un membro del gruppo possa
%    non poter svolgere le sue mansioni per un certo periodo di tempo. &
%    Probabilità: \newline \textbf{Bassa}\newline
%    Gravità: \newline \textbf{Alta}\\
%    
%    Contromisure&
%    \multicolumn{2}{L{.825\textwidth}}{
%    A seconda della natura del problema e delle attività lasciate in sospeso, il responsabile
%    può ridistribuire il carico di lavoro del membro assente o posticiparle e rivedere
%    la pianificazione
%    }\\
%    \hline \hline
%
%\textbf
%    {Incompatibilità orari dei membri del gruppo}&
%   A causa di ubicazione geografica e diversi impegni universitari e lavorativi
%   dei vari membri del gruppo, può essere complicato incontrarsi di persona per
%   discutere del progetto.&
%   Probabilità: \newline \textbf{Alta}\newline
%   Gravità: \newline \textbf{Bassa}\\
%   
%   Contromisure&
%   \multicolumn{2}{L{.825\textwidth}}{     
%   Creazione di una tabella oraria con gli impegni di ogni membro del gruppo. \newline
%   Ogni riunione avrà uno scopo ben preciso, e ogni membro è tenuto a preparasi
%   attentamente per sfruttare al meglio il tempo disponibile. \newline
%   In caso non sia possibile organizzare un incontro fisico, è sempre possibile
%   discutere in videoconferenza.
%   Una riunione può essere svolta anche in assenza 2 membri.
%   }\\
   \hline \hline
\end{tabular}
\end{table}



