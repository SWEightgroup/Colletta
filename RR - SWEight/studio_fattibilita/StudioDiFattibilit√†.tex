%generare il pdf con il comando: pdflatex main.tex
\documentclass[a4paper, oneside, openany, dvipsnames, table]{article}
\usepackage{../template/SWEightStyle}
\newcommand{\Titolo}{Manuale Utente}

\newcommand{\Gruppo}{SWEight}

\newcommand{\Approvatore}{Damien Ciagola}
\newcommand{\Redattori}{Alberto Bacco \newline Sebastiano Caccaro \newline Gheorghe Isachi \newline Gionata Legrottaglie}
\newcommand{\Verificatori}{Francesco Corti \newline Francesco Magarotto}

\newcommand{\pathimg}{../template/img/logoSWEight.png}

\newcommand{\Versionedoc}{1.0.0}

\newcommand{\Distribuzione}{\proponente \newline Prof. Vardanega Tullio \newline Prof. Cardin Riccardo \newline Gruppo SWEight}

\newcommand{\Uso}{Esterno}

\newcommand{\NomeProgetto}{Colletta}

\newcommand{\Mail}{SWEightGroup@gmail.com}

\newcommand{\DescrizioneDoc}{Questo documento si occupa di fornire le modalità di utilizzo del software Colletta commissionato}


\begin{document}
\copertina{} %Richiamo funzione
%%%%%%%%%%%%%%%%%%%%%%%%%%%%%%%%%%%%%%%%%%%%%%%%%%%%%%%%%%%%%%%%%%%%%%%%%%%%%%%%%%%%%%%%%%%%%%%%%%%%%%%
%SOMMARIO
\definecolor{greySWEight}{RGB}{255, 71, 87}
\definecolor{greyROwSWEight}{RGB}{234, 234, 234}

\section*{Registro delle modifiche}
{
	\rowcolors{2}{greyROwSWEight}{white}
	\renewcommand{\arraystretch}{1.5}
	\centering
	\begin{longtable}{ c c C{4cm}  c  c }
		
		\rowcolor{greySWEight}
		\textcolor{white}{\textbf{Versione}} & \textcolor{white}{\textbf{Data}} & \textcolor{white}{\textbf{Descrizione}} & \textcolor{white}{\textbf{Nominativo}} & \textcolor{white}{\textbf{Ruolo}}\\
		1.2.2 & 2019-02-25 & Ampliamento sezione 5.4 e 3.2.5.2 & Alberto Bacco & \reda{} \\
		
		1.2.1 & 2019-02-23 & Aggiunta sezione 3.2.5.8 Checkstyle & Sebastiano Caccaro & \reda{} \\		
		
		1.2.0 & 2019-02-20 & Aggiunta scelte tecnologiche 3.2.4.2, da 3.4.5.4 a 3.4.5.7, 4.3.1.4, 4.3.2.2, 4.4.6 e figlie & Sebastiano Caccaro & \reda{} \\	
		
		1.1.5 & 2019-02-20 & Modifica sezione 2 & Alberto Bacco & \reda{} \\
		
		1.1.4 & 2019-02-18 & Correzione errori grammatica, spostate sottosezioni di asana da 4.3 a 5.2, & Alberto Bacco & \reda{} \\
		
		1.1.3 & 2019-02-14 & Riorganizzazione e correzione errori sezione 5 & Enrico Muraro & \reda{} \\
		
		1.1.2 & 2019-02-03 & Modifica sottosezione 4.1.10, 4.3.1.4, 4.3.1.5, 4.3.1.6 & Alberto Bacco& \reda{} \\	
		
		1.1.1 & 2019-01-31 & Modifica struttura e contenuti sezione 3  & Damien Ciagola & \reda{} \\	
		
		1.1.0 & 2019-01-27 & Sezione Qualità 4.2 & Sebastiano Caccaro & \reda{} \\	
		
		1.0.1 & 2019-01-25 & Parziale ristrutturazione della struttura del documento & Sebastiano Caccaro & \reda{} \\		
		
		1.0.0 & 2019-01-11 & Approvazione per il rilascio & Sebastiano Caccaro & \Res{} \\
		
		0.9.0 & 2019-01-9 & Verifica finale & Francesco Corti & \ver{} \\
		
		0.9.0 & 2019-01-8 & Aggiunta lista di controllo & Gionata Legrottaglie & \reda{} \\
		
		0.8.0 & 2018-12-23 & Correzioni errori ortografici & Gionata Legrottaglie & \reda{} \\
		
		0.7.0 & 2018-12-20 & Verifica documento & Francesco Corti & \ver{}\\
		
		0.6.0 & 2018-12-18 & Aggiunta sottosezione 5.2.2.2, 5.2.2.3, 5.2.2.4 & Francesco Magarotto & \reda{} \\
		
		0.5.2 & 2018-12-16 & Modifica sezione 4.1.5.3 & Alberto Bacco & \reda{} \\
		
		0.5.2 & 2018-12-16 & Modifica sezione 4.1.5.3 & Alberto Bacco & \reda{} \\
		
		0.5.2 & 2018-12-16 & Aggiunte sottosezioni  & Alberto Bacco & \reda{} \\
		
		0.5.1 & 2018-12-15 & Aggiunte sottosezioni 5.3, 5.4, 5.5, 5.6, 5.7, 5.8 & Alberto Bacco & \reda{} \\
		
		0.5.0 & 2018-12-15 & Aggiunta sezione 5 e sottosezioni 5.1, 5.2 & Gionata Legrottaglie & \reda{} \\
		
		0.4.1 & 2018-12-11 & Aggiunta sezione 4.1.7.3.1 & Francesco Magarotto & \reda{} \\ 
		
		0.4.0 & 2018-12-10 & Aggiunte sottosezioni 4.1.5, 4.1.6, 4.1.7, 4.1.8 & Gionata Legrottaglie & \reda{} \\ 
		0.4.0 & 2018-12-09 & Aggiunta sezione 4 e sottosezioni 4.1.1, 4.1.2, 4.1.3, 4.1.4 & Gionata Legrottaglie & \reda{} \\ 
		
		0.3.1 & 2018-12-07 & Aggiunta sottosezione 3.2 & Gionata Legrottaglie & \reda{} \\ 
		
		0.3.0 & 2018-12-06 & Aggiunta sezione 3 e sottosezione 3.1 & Gionata Legrottaglie & \reda{} \\ 
		
		0.2.0 & 2018-12-05 & Aggiunti i riferimenti & Gionata Legrottaglie & \reda{} \\ 
		
		0.1.0 & 2018-11-30 & Aggiunta introduzione & Gionata Legrottaglie & \reda{} \\
		
		0.0.1 & 2018-11-28 & Creazione scheletro del documento & Gionata Legrottaglie & \reda{}\\
		
	\end{longtable}

}
\newpage
\tableofcontents
\newpage
%%%%%%%%%%%%%%%%%%%%%%%%%%%%%%%%%%%%%%%%%%%%%%%%%%%%%%%%%%%%%%%%%%%%%%%%%%%%%%%%%%%%%%%%%%%%%%%%%%%%%%%
%PARAGRAFI
%%%%%%%%%%%%%%%%%%%%%%%%%%%%%%%%%%%%%%%%%%%%%%%%%%%%%%%%%%%%%%%%%%%%%%%%%%%%%%%%%%%%%%%%%%%%%%%%%%%%%%%
\subsection{Document goal}
The purpose of this document is to provide all the necessary information to extend, correct and improve Colletta.
There will be additional information regarding setting up the development environment to work in an environment that is as consistent as possible with that used
by the other members of group SWEight, but can be ignored if you only want to use part of the product.
This guide was written taking into account the Microsoft Windows and Linux operating systems. If other systems are used, compatibility issues may arise, even if it's unlikely. In this case refer to the git page. This document will grow as the product will be fully
developed.

\subsection{Product goal}
The purpose of the product is the creation of a collaborative data collection platform where users can prepare and/or perform small grammar exercises. 
The front-end of the system consists of a web application developed with React and Redux, while the back-end is a Spring Boot application written in Java, which will handle HTTP Requests sent from the front-end. 

\subsection{References}


\subsubsection{Installation references}

\begin{itemize}
\item \textbf{Git}: \url{https://git-scm.com/}
\item \textbf{Node.js}: \url{https://nodejs.org/en/}
\item \textbf{NPM}: \url{https://www.npmjs.com/}
\item \textbf{Oracle JDK}: \url{https://www.oracle.com/technetwork/java/javase/downloads/index.html}
\item \textbf{OpenJDK}: \url{https://openjdk.java.net/}
\item \textbf{Maven}: \url{https://maven.apache.org/}
\item \textbf{Lombok}: \url{https://projectlombok.org/}
\item \textbf{VSCode}: \url{https://code.visualstudio.com/} 

\end{itemize}

\subsubsection{Legal references}
\begin{itemize}
\item \textbf{MIT License}: \url{https://opensource.org/licenses/MIT}
\end{itemize}

%\subsubsection{Informative references}

%primo imput sarà capitolato scelto!
%poi in ordine
\section{Capitolato C1}
\subsection{Informazioni sul Capitolato}
\begin{itemize}
	\item \textbf{Nome:} Butterly;
	\item \textbf{Proponente}: Imola Informatica;
	\item \textbf{Committenti}: Prof. Tullio Vardanega, Prof. Riccardo Cardin.
\end{itemize}

\subsection{Descrizione}
Il progetto Butterfly prevede la realizzazione di un insieme di plugin\ped{G} che permettano di realizzare un'infrastruttura per la gestione automatizzata delle segnalazioni provenienti dagli applicativi (i.e. SonarQube, GitLab, RedMine). Questi sono utilizzati durante lo sviluppo di software al fine di fornire una migliore accessibilità e standardizzazione dei messaggi. Tali messaggi dovranno essere inseriti all'interno di topic da un broker\ped{G}, che si occuperà della loro gestione. Il proponente del capitolato fa riferimento ad un pattern Publisher/Subscriber\ped{G} per l'implementazione dei componenti richiesti, classificati in categorie:
\begin{itemize}
\item[•]I producer\ped{G} che avranno il compito di recuperare le segnalazioni e pubblicarle;
%\item[•]Il broker impiegato per la creazione e la gestione dei topic;
\item[•] I consumer\ped{G} che avranno il compito di abbonarsi ai topic adeguati, recuperarne i messaggi e procedere al loro inoltro verso i destinatari finali mediante applicazioni atte alla comunicazione, quali Telegram, Slack ed email;
\item[•]Il componente custom\ped{G} specifico per la gestione del personale cheù idonea in quel determinato momento; si occuperà di inoltrare i messaggi alla persona pi
\end{itemize}
\subsection{Dominio Applicativo}
Il dominio applicativo a cui fa riferimento il capitolato proposto è quello della comunicazione automatizzata, infatti i messaggi provenienti dai vari applicativi devono essere inoltrati automaticamente al team di sviluppo.
\subsection{Dominio Tecnologico}
\begin{itemize}
\item[•] \texttt{Apache Kafka}: per la realizzazione del broker;
\item[•] \texttt{NodeJS}, \texttt{Python}, \texttt{Java}: come linguaggi di riferimento per la realizzazione dei componenti;
\item[•] The twelve-factor app, una metodologia per la realizzazione di \texttt{software-as-serivice}\ped{G}
\item[•] \texttt{Docker} per per l’istanziazione di
tutti i componenti;
\item[•] Strumenti per il testing;
\item[•] Utilizzo di un'architettura \texttt{REST}
\item[•] \texttt{Git} come tool di versionamento; 
\textsl{•}\end{itemize}
\subsection{Considerazioni del gruppo}

Il progetto presenta alcuni aspetti positivi che non sono richiesti esplicitamente negli altri capitolati, quali l'utilizzo di strumenti di testing e per l'istanziazione delle componenti. 
Le principali criticità constatate sono:
\begin{itemize}
\item[•] Le tecnologie richieste come obbligatorie sono numerose e sconosciute alla maggior parte del gruppo, pertanto l'apprendimento di queste non si confà al tempo a disposizione dai singoli membri del gruppo; 
\item[•] Mancato interesse verso il capitolato proposto;
\item[•] Distanza tra la sede dell'azienda e la sede dell'università;
\end{itemize}

\subsection{Valutazione Finale}
Considerata la distanza tra la sede dell'azienda e l'università, che non permette incontri fisici con il proponente del capitolato, e viste le numerose tecnologie sconosciute alla totalità dei componenti del gruppo, il capitolato proposto è stato rigettato.


\section{Capitolato C3}

\subsection{Informazioni sul Capitolato}

\begin{itemize}
	\item \textbf{Nome:} G\&{B} : monitoraggio intelligente di processi DevOps;
	\item \textbf{Proponente}: Zucchetti;
	\item \textbf{Committenti}: Prof. Tullio Vardanega, Prof. Riccardo Cardin.
\end{itemize}

\subsection{Descrizione}
Il capitolato ha per oggetto lo sviluppo di un plug-in di Grafana, scritto in linguaggio JavaScript. Questo crea da un file .Json la rete Bayesiana e permette di associare dei valori ai nodi\ped{G}, a seconda dei dati prelevati dal {flusso}\ped{G} di monitoraggio.
La rete Bayesiana viene poi modificata a seconda dei dati rilevati. 
L'applicativo pertanto è composto dalle seguenti parti: 

\begin{itemize}

\item[•] Interfaccia per la visualizzazione di grafici e dashboard;
\item[•] Rete Bayesiana applicata ai file di input.

\end{itemize}

\subsection{Dominio Applicativo}
Il dominio applicativo a cui G\&{B} fa riferimento è quello dell'immagazzinamento ed elaborazione dei dati tramite librerie offerte dall'azienda. 

\subsection{Dominio Tecnologico}
\begin{itemize}

\item[•] \texttt{Grafana}: piattaforma che consente di raccogliere dati telemetrici e visualizzarli in una dashboard;
\item[•] \texttt{Git}: tool di versionamento;
\item[•] \texttt{Rete Bayesiana}: di verifica dell'applicazione che fornisce informazioni sullo stato del sistema;
\item[•] \texttt{Json}: per la gestione dei dati nei file.

\end{itemize}

\subsection{Considerazioni del gruppo}


Gli aspetti ritenuti positivi sono: 
\begin{itemize}
\item[•] L'utilizzo del framework innovativo Grafana per la creazione di grafici e dashboard;
\item[•] Creazione di una rete Bayesiana basata sui principi dell'intelligenza artificiale.
\end{itemize}

Le principali criticità sono costituite da: 
\begin{itemize}

\item[•]La gestione della rete Bayesiana tramite regole temporali precedentemente stabilite comporta una tolleranza agli errori molto bassa;

\item[•] L'utilizzo e la comprensione di una libreria come Grafana implica un'elevata quantità di risorse.
\end{itemize}

\subsection{Valutazione Finale}
Il capitolato ha suscitato un discreto interesse per i temi molto innovativi. 
La quantità di nozioni sconosciute, e quindi l'eccessivo ammontare di ore per lo studio delle librerie, ne ha portato all'esclusione. 

\section{Capitolato C5}
\subsection{Informazioni sul Capitolato}
\begin{itemize}
	\item \textbf{Nome:} P2PCS: piattaforma di peer-to-peer car sharing;
	\item \textbf{Proponente}: Gaiago;
	\item \textbf{Committenti}: Prof. Tullio Vardanega, Prof. Riccardo Cardin.
\end{itemize}

\subsection{Descrizione}
L'obbiettivo del capitolato è lo sviluppo di un'applicazione mobile Android$^{TM}$ che offra un servizio di car-sharing tramite una piattaforma che sfruttando la geo-localizzazione permetta agli utenti di condividere un'auto con l'intento di massimizzare il tempo di utilizzo del mezzo quando quest ultimo è inutilizzato tramite calendarizzazione, con l'intento di ridurre i costi. Il proprietario del mezzo è sempre a conoscenza della posizione del mezzo tramite il servizio di geo-localizzazione. L'applicativo pertanto è composto dalle seguenti parti:
\begin{itemize}
\item[•] Interfacce utente per la calanderizzazione, localizzazione del mezzo e prenotazione;
\end{itemize}

\subsection{Dominio Applicativo}

\subsection{Dominio Tecnologico}


\subsection{Aspetti Positivi}

\subsection{Potenziali Criticità}

\subsection{Valutazione Finale}


\section{Capitolato C6}
\subsection{Informazioni sul Capitolato}
\begin{itemize}
	\item \textbf{Nome:} Soldino;
	\item \textbf{Proponente}: Red Babel;
	\item \textbf{Committenti}: Prof. Tullio Vardenega, Prof. Riccardo Cardin.
\end{itemize}

\subsection{Descrizione}
Il capitolato ha per oggetto lo sviluppo di una piattaforma, Soldino, gestita dal governo. La piattaforma permette ai gestori di imprese di vendere e comprare merci, o servizi e inoltre permette di ricevere e registrare le tasse relative ai beni acquistati. \newline
I cittadini possono comprare oggetti o servizi utilizzando del denaro prodotto dal governo, la moneta utilizzata in questa piattaforma sarà ECR20 chiamata Cubit. Possono inoltre diventare gestori di imprese iscrivendosi alla lista di gestori di imprese gestita dal governo.  
Il capitolato è pertanto composto da due parti: 

\begin{itemize}

\item[•] Pagina Web che agisca da interfaccia per comunicare con la EVM, Ethereum Virtual Machine;
\item[•] Smart contracts, contenente tutte le transazioni che vengono gestite dalla rete Ethereum;

\end{itemize}

\subsection{Dominio Applicativo}
Il dominio applicativo a cui Soldino fa riferimento è quello del commercio online tramite criptovaluta. 

\subsection{Dominio Tecnologico}
\begin{itemize}

\item[•] \texttt{Unit} test di integrazione in ambiente locale ;
\item[•] \texttt{Git} tool di versionamento;
\item[•] \texttt{EVM} permette l’esecuzione di codici complessi, smart contracts, al di sopra della piattaforma Ethereum;
\item[•] \texttt{Truffle} come framework che permette la gestione dei contratti di Ethereum;
\item[•] \texttt{MetaMask} 
permette l'accesso alla rete di Ethereum usando un nodo pubblico; 
\item[•] \texttt{Ropsten} permette di testare le funzionalità di Ethereum su una rete. 
\item[•] \texttt{Raiden} permette di trasferire token ERC20 quasi istantaneamente alla rete Ethereum;  

\end{itemize}

\subsection{Aspetti Positivi}

Gli aspetti ritenuti positivi sono: 
\begin{itemize}

\item[•] Lo studio di un framework come Truffle per la gestione degli smart contracts di Ethereum;
\item[•] L'utilizzo di Ropsten che fornisce un ambiente per testare le transazione che successivamente avverrano nella rete principale di Ethereum;
\item[•] Lo studio della gestione, e della validazione, delle transazioni in ambiente Ethereum; 

\end{itemize}

\subsection{Potenziali Criticità}
Le principali criticità sono costituite da: 
\begin{itemize}

\item[•] Il progetto deve usare almeno gli ambienti: Local, Test, Staging e Production. Questo comporta un dispendio di tempo molto alto. 

\item[•] Lo studio di tre framework: Truffle, Ropsten e l'estensione web MetaMask comporta un tempo considerevole;

\end{itemize}


\subsection{Valutazione Finale}
Il capitolato ha suscitato dell'interesse per i suoi temi innovativi, ovvero la gestione dei pagamenti tramite criptovalute. A causa della mancata conoscenza e quindi dell'eccessivo carico di studio riguardante l'utilizzo di framework per la gestione dei pagamenti e la in sicurezza di essi nell' infrastruttura Ethereum è stato scartato. 

\end{document}
