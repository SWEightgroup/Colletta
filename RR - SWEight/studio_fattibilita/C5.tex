\section{Capitolato C5}
\subsection{Informazioni sul Capitolato}
\begin{itemize}
	\item \textbf{Nome:} P2PCS: piattaforma di peer-to-peer car sharing;
	\item \textbf{Proponente}: Gaiago;
	\item \textbf{Committenti}: Prof. Tullio Vardanega, Prof. Riccardo Cardin.
\end{itemize}

\subsection{Descrizione}
L'obbiettivo del capitolato è lo sviluppo di un'applicazione mobile Android$^{TM}$ che offra un servizio di car-sharing tramite una piattaforma che sfruttando la geo-localizzazione permetta agli utenti di condividere un'auto con l'intento di massimizzare il tempo di utilizzo del mezzo quando quest ultimo è inutilizzato tramite calendarizzazione, con l'intento di ridurre i costi. Il proprietario del mezzo è sempre a conoscenza della posizione del mezzo tramite il servizio di geo-localizzazione. L'applicativo pertanto è composto dalle seguenti parti:
\begin{itemize}
\item[•] Interfacce utente per la calanderizzazione, localizzazione del mezzo e prenotazione;
\end{itemize}

\subsection{Dominio Applicativo}

\subsection{Dominio Tecnologico}


\subsection{Aspetti Positivi}

\subsection{Potenziali Criticità}

\subsection{Valutazione Finale}

