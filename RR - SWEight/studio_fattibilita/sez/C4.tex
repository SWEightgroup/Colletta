\section{Capitolato C4}
\subsection{Informazioni sul Capitolato}
\begin{itemize}
	\item \textbf{Nome:} MegAlexa;
	\item \textbf{Proponente}: Zero12;
	\item \textbf{Committenti}: Prof. Tullio Vardanega, Prof. Riccardo Cardin;
\end{itemize}

\subsection{Descrizione}
L'obiettivo del capitolato è lo sviluppo di un applicativo Web e Mobile che deve essere in grado di creare delle routine personalizzate per gli utenti, ed essere gestibile attraverso Alexa di Amazon.
L'utente, registrato alla piattaforma, avrà a disposizione dei connettori\ped{G} che potrà inserire all'interno di una routine. Questa routine verrà in seguito eseguita tramite controllo vocale attraverso delle funzioni fornite da Alexa chiamate skill\ped{G}.

Alcuni esempi di connettori all'interno della routine sono:
\begin{itemize}
\item[•] Lettura di feed rss;
\item[•] Controllo del calendario o della posta;
\item[•] Avvio di musica o podcast;
\end{itemize}


\subsection{Dominio Applicativo}
Il dominio applicativo di MegAlexa è formato dagli utenti possessori di un dispositivo Alexa di Amazon.

\subsection{Dominio Tecnologico}
\begin{itemize}
\item[•] Amazon Web Services come servizio cloud per l'immagazzinamento dei dati e Node.js come linguaggio di programmazione;
\item[•] Swift per lo sviluppo di una applicazione moblie iOS;
\item[•] Kotlin per la realizzazione di una applicazione mobile Android;
\item[•] Git come tool di versionamento;
\end{itemize}

\subsection{Aspetti Positivi}
Gli aspetti ritenuti positivi sono:
\begin{itemize}
\item[•] L'utilizzo di tecnologie cloud come Amazon Web Services e le API fornite da esso;
\end{itemize}

\subsection{Potenziali Criticità}
\begin{itemize}
\item[•] La maggior parte del gruppo non ha esperienza con sviluppo di applicazioni iOS o Android;
\item[•] Necessità di apprendere il funzionamento delle skill di Alexa;
\end{itemize}

\subsection{Valutazione Finale}
Il capitolato non è stato scelto a causa delle tecnologie di sviluppo poco conosciute dal gruppo. Il campo di applicazione, inoltre, non è stato sufficientemente accattivante.
