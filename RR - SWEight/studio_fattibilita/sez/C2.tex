\section{Capitolato C2}
\subsection{Informazioni sul Capitolato}
\begin{itemize}
	\item \textbf{Nome:} Colletta;
	\item \textbf{Proponente}: Mivoq S.R.L;
	\item \textbf{Committenti}: Prof. Tullio Vardanega, Prof. Riccardo Cardin.
\end{itemize}

\subsection{Descrizione}
Il capitolato proposto da MIVOQ richiede di  realizzare una piattaforma collaborativa di raccolta dati in cui gli utenti possano predisporre e svolgere piccoli esercizi di analisi grammaticale. Gli insegnanti sono agevolati nella realizzazione e nella correzione dei compiti grazie ad una libreria che esegue l’analisi grammaticale automaticamente.
Gli esercizi corretti, revisionati dagli insegnanti, devono essere raccolti e catalogati in una base di dati per fornire, a ricercatori/sviluppatori di accedervi per migliorare gli algoritmi di apprendimento automatico supervisionato per la realizzazione automatica dei sopracitati esercizi.


\subsection{Dominio Applicativo}
Il dominio applicativo a cui il capitolato fa riferimento è quello della raccolta dati, anche se questa viene realizzata implicitamente fornendo un servizio utile all’utente. Infatti il prodotto risulta vantaggioso sia a insegnanti che a studenti come fonte di esercizi di analisi grammaticale.

\subsection{Dominio Tecnologico}
Per quanto riguarda la realizzazione dell’applicativo si farà uso di:

\begin{itemize}
\item[•] Sistema cloud come Firebase per l'immagazzinamento dei dati;

\item[•] Utilizzo di una libreria come FreeLing o Hunpos come strumento di pos-tagging;

\item[•] Linguaggio di scripting lato server, come PHP o Java;

\item[•] Linguaggio di scripting lato client come JavaScript, impiegando librerie quali AngularJS o jQuery;

\item[•] Utilizzo di framework quali Foundation o Twitter Bootstrap per la realizzazione della parte grafica dell’applicativo;

\end{itemize}

\subsection{Potenziali Criticità}
Durante l’analisi del capitolato sono emerse le seguenti criticità:
la quantità di attori interessati e le funzionalità messe a disposizione per ognuno di essi, potrebbe portare ad uno sforamento del limite di ore proposte e del budget a disposizione.

\subsection{Valutazione Finale}
Il capitolato d’appalto presenta alcune caratteristiche positive che lo hanno portato ad essere
la scelta finale del gruppo:
\begin{itemize}
\item[•] interesse nel dominio tecnologico;
\item[•] acquisizione di esperienza nello sviluppo di applicazioni web o mobile, con l’utilizzo di tecnologie
ampiamente richieste nell’ambito lavorativo;
\item[•] discreto interesse nell'ambito proposto e nel prodotto finale 

\end{itemize}

