\section{Capitolato C6}
\subsection{Informazioni sul Capitolato}
\begin{itemize}
	\item \textbf{Nome:} Soldino ;
	\item \textbf{Proponente}: Red Babel;
	\item \textbf{Committenti}: Prof. Tullio Vardenega, Prof. Riccardo Cardin.
\end{itemize}

\subsection{Descrizione}
Il capitolato ha per oggetto lo sviluppo di un servizio che permette di mettere in vendita o di comprare oggetti mediante l'uso della tecnologia Ethereum attraverso la valuta Cubit. L'applicativo pertanto è composto dalle seguenti parti: 

\begin{itemize}

\item[•] Interfaccia Web che agisca da interfaccia per comunicare con la rete Ethereum;
\item[•] Un database contenente tutti gli smart contract che registrano e verificano le transazioni avvenute;

\end{itemize}

\subsection{Dominio Applicativo}
Il dominio applicativo a cui Soldino fa riferimento è quello del commercio online tramite criptovaluta. 

\subsection{Dominio Tecnologico}
\begin{itemize}

\item[•] \texttt{Unit} per i test di integrazione in ambiente locale ;
\item[•] \texttt{Git} come tool di versionamento;
\item[•] \texttt{EVM} permette l’esecuzione di codici complessi, smart contracts, al di sopra della piattaforma Ethereum.;
\item[•] \texttt{Truffle} come framework che permette la gestione dei contratti di Ethereum;
\item[•] \texttt{MetaMask} 
che permette l'accesso alla rete di Ethereum usando un nodo pubblico; 
\item[•] \texttt{Ropsten} che permette di testare le funzionalità di Ethereum su una rete. 

\end{itemize}

\subsection{Aspetti Positivi}

Gli aspetti ritenuti positivi sono: 
\begin{itemize}

\item[•] Lo studio di un framework come Truffle per la gestione degli smart contracts di Ethereum;
\item[•] L'utilizzo di Ropsten che fornisce un ambiente per testare le transazione che successivamente avverrano nella rete principale di Ethereum;

\end{itemize}

\subsection{Potenziali Criticità}
Le principali criticità sono costituite da: 
\begin{itemize}

\item[•]La gestione dei pagamenti in tempo reale tramite piattaforma Ethereum;

\item[•] Lo studio di tre framework: Truffle, Ropsten e l'estensione web MetaMask comporta un tempo molto alto ;

\end{itemize}


\subsection{Valutazione Finale}
Il capitolato ha suscitato del discreto interesse che è però sciamato a causa della mancata conoscenza e quindi dell'eccessivo carico di studio riguardante l'utilizzo di framework per pagamenti nella rete Ethereum. 