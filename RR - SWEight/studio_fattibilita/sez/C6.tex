\section{Capitolato C6}
\subsection{Informazioni sul Capitolato}
\begin{itemize}
	\item \textbf{Nome:} Soldino;
	\item \textbf{Proponente}: Red Babel;
	\item \textbf{Committenti}: Prof. Tullio Vardanega, Prof. Riccardo Cardin.
\end{itemize}

\subsection{Descrizione}
Il capitolato ha per oggetto lo sviluppo di una piattaforma che permette ai gestori di imprese di \newline 
vendere/comprare merci o servizi, e di ricevere/registrare le tasse relative ai beni acquistati.\newline
I cittadini possono comprare oggetti o servizi utilizzando del denaro prodotto dal governo, la moneta utilizzata in questa piattaforma sarà ECR20 chiamata Cubit. Essi possono inoltre diventare gestori di imprese iscrivendosi alla lista di gestori di imprese gestita dal governo.  
Il capitolato è pertanto composto da due parti: 

\begin{itemize}

\item[•] Pagina Web che agisca da interfaccia per comunicare con la EVM, Ethereum Virtual Machine;
\item[•] Smart contracts, contenente tutte le transazioni che vengono gestite dalla rete Ethereum;

\end{itemize}

\subsection{Dominio Applicativo}
Il dominio applicativo a cui Soldino fa riferimento è quello del commercio online tramite criptovaluta. 

\subsection{Dominio Tecnologico}
\begin{itemize}

\item[•] \texttt{Git}: tool di versionamento;
\item[•] \texttt{EVM}: permette l’esecuzione di codici complessi, smart contracts, al di sopra della piattaforma Ethereum;
\item[•] \texttt{Truffle}: come framework che permette la gestione dei contratti di Ethereum;
\item[•] \texttt{MetaMask}: 
permette l'accesso alla rete di Ethereum usando un nodo pubblico; 
\item[•] \texttt{Ropsten}: permette di testare le funzionalità di Ethereum su una rete. 
\item[•] \texttt{Raiden}: permette di trasferire token ERC20 quasi istantaneamente alla rete Ethereum;  

\end{itemize}

\subsection{Considerazioni del gruppo}

Gli aspetti ritenuti positivi sono: 
\begin{itemize}

\item[•] Lo studio del framework Truffle;
\item[•] L'utilizzo di Ropsten che fornisce un ambiente per testare le transazione che successivamente avverranno nella rete principale di Ethereum;
\item[•] Lo studio della gestione, e della validazione, delle transazioni in ambiente Ethereum; 

\end{itemize}

Le principali criticità sono: 
\begin{itemize}

\item[•] Il progetto utilizza ambienti come: Local, Test, Staging e Production. Questo comporta un dispendio di tempo molto alto. 

\item[•] Lo studio di tre framework: Truffle, Ropsten e l'estensione web MetaMask comporta uno studio eccessivo;

\end{itemize}


\subsection{Valutazione Finale}
La gestione dei pagamenti tramite criptovalute ha suscitato dell'interesse per i suoi temi innovativi. 
A causa della mancata conoscenza e dell'eccessivo carico di studio dei framework per la gestione dei pagamenti, il seguente capitolato è stato rigettato. 