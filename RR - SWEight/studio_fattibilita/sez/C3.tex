\section{Capitolato C3}

\subsection{Informazioni sul Capitolato}

\begin{itemize}
	\item \textbf{Nome:} G\&{B} : monitoraggio intelligente di processi DevOps;
	\item \textbf{Proponente}: Zucchetti;
	\item \textbf{Committenti}: Prof. Tullio Vardanega, Prof. Riccardo Cardin.
\end{itemize}

\subsection{Descrizione}
Il capitolato ha per oggetto lo sviluppo di un plug-in di Grafana, scritto in linguaggio JavaScript. Questo crea da un file json la rete Bayesiana e permette di associare dei valori ai {nodi}\ped{G}, a seconda dei dati prelevati dal {flusso}\ped{G} di monitoraggio.
La rete Bayesiana viene poi modificata a seconda dei dati rilevati. 
L'applicativo pertanto è composto dalle seguenti parti: 

\begin{itemize}

\item[•] Interfaccia per la visualizzazione di grafici e dashboard;
\item[•] Rete Bayesiana applicata ai file di input.

\end{itemize}

\subsection{Dominio Applicativo}
Il dominio applicativo a cui G\&{B} fa riferimento è quello dell'immagazzinamento ed elaborazione dei dati tramite librerie offerte dall'azienda. 

\subsection{Dominio Tecnologico}
\begin{itemize}

\item[•] \textbf{Grafana}: piattaforma che consente di raccogliere dati telemetrici e visualizzarli in una dashboard;
\item[•] \textbf{Git}: tool di versionamento;
\item[•] \textbf{Rete Bayesiana}: di verifica dell'applicazione che fornisce informazioni sullo stato del sistema;
\item[•] \textbf{Json}: per la gestione dei dati nei file.

\end{itemize}

\subsection{Considerazioni del gruppo}


Gli aspetti ritenuti positivi sono: 
\begin{itemize}
\item[•] L'utilizzo del framework innovativo Grafana per la creazione di grafici e dashboard;
\item[•] Creazione di una rete Bayesiana basata sui principi dell'intelligenza artificiale.
\end{itemize}

Le principali criticità sono costituite da: 
\begin{itemize}

\item[•]La gestione della rete Bayesiana tramite regole temporali precedentemente stabilite comporta una tolleranza agli errori molto bassa;

\item[•] L'utilizzo e la comprensione di una libreria come Grafana implica un'elevata quantità di risorse.
\end{itemize}

\subsection{Valutazione Finale}
Il capitolato ha suscitato un discreto interesse per i temi molto innovativi. 
La quantità di nozioni sconosciute, e quindi l'eccessivo ammontare di ore per lo studio delle librerie, ne ha portato all'esclusione. 
