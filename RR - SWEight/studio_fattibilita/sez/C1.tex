\section{Capitolato C1}
\subsection{Informazioni sul Capitolato}
\begin{itemize}
	\item \textbf{Nome:} Butterly;
	\item \textbf{Proponente}: Imola Informatica;
	\item \textbf{Committenti}: Prof. Tullio Vardanega, Prof. Riccardo Cardin.
\end{itemize}

\subsection{Descrizione}
Il progetto Butterfly prevede la realizzazione di un insieme di \texttt{plugin}\ped{G} che permettano di realizzare un'infrastruttura unificata e flessibile per la gestione automatizzata delle segnalazioni provenienti dagli applicativi (i.e. SonarQube, GitLab, RedMine) utilizzati durante lo sviluppo di software al fine di fornire una migliore accessibilità e una standardizzazione dei messaggi. In particolare, tali messaggi dovranno essere inseriti all'interno di topic da un \texttt{broker}\ped{G}, che si occuperà anche della loro gestione. Il proponente del capitolato fa riferimento ad un pattern \texttt{Publisher/Subscriber}\ped{G} per l'implementazione di tali componenti, che vengono classificati in tre categorie:
\begin{itemize}
\item[•]I componenti producer che avranno il compito di recuperare le segnalazioni e pubblicarle;
\item[•]Il broker impiegato per la creazione e la gestione dei topic;
\item[•]I componenti consumer che avranno il compito di abbonarsi ai topic adeguati, recuperarne i messaggi e procedere al loro inoltro verso i destinatari finali mediante applicazioni atte alla comunicazione, quali Telegram, Slack e email;
\item[•]Il componente custom specifico per la gestione del personale che si occuperà di inoltrare i messaggi alla persona più idonea in quel determinato momento;
\end{itemize}
\subsection{Dominio Applicativo}
Il dominio applicativo a cui fa riferimento il capitolato proposto è quello della comunicazione automatizzata (automated communication), infatti i messaggi provenienti dai vari applicativi devono essere inoltrati automaticamente al team di sviluppo.
\subsection{Dominio Tecnologico}
\begin{itemize}
\item[•]Apache Kafka, per la realizzazione del broker;
\item[•]NodeJS, Python, Java come linguaggi di riferimento per la realizzazione dei componenti;
\item[•]The twelve-factor app, una metodologia per la realizzazione di \texttt{software-as-serivice}\ped{G}
\item[•]Docker per per l’istanziazione di
tutti i componenti;
\item[•]Strumenti per il testing (ad esempio JUnit);
\item[•]
\end{itemize}


\subsection{Aspetti Positivi}

\subsection{Potenziali Criticità}

\subsection{Valutazione Finale}

