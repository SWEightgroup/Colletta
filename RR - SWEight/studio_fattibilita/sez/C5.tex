\section{Capitolato C5}
\subsection{Informazioni sul Capitolato}
\begin{itemize}
	\item \textbf{Nome:} P2PCS: piattaforma di peer-to-peer car sharing;
	\item \textbf{Proponente}: Gaiago;
	\item \textbf{Committenti}: Prof. Tullio Vardanega, Prof. Riccardo Cardin.
\end{itemize}
\subsection{Descrizione}
L'obbiettivo del capitolato è lo sviluppo di un'applicazione mobile Android\textsuperscript{TM} che offra un servizio di car-sharing tramite una piattaforma che, sfruttando la geolocalizzazione, permetta agli utenti di condividere un'auto con l'intento di massimizzare il tempo di utilizzo del mezzo quando quest'ultimo è inutilizzato tramite calendarizzazione, con l'intento di ridurre i costi. Il proprietario dell'auto è sempre a conoscenza della posizione di quest'ultima tramite il suddetto servizio. L'applicativo pertanto è composto dalle seguenti parti:
\begin{itemize}
\item[•] Interfacce utente per la calendarizzazione, localizzazione del mezzo e prenotazione;
\item[•] \texttt{Servizi Google}\ped{G} per l'immagazzinamento delle prenotazioni;
\end{itemize}
\subsection{Dominio Applicativo}
Il dominio applicativo a cui P2PCS fa riferimento è quello del commercio elettronico Consumer to Consumer, dove gli utenti interagiscono tra di loro mettendo a disposizione un oggetto che vogliono vendere o prestare.
\subsection{Dominio Tecnologico}
\begin{itemize}
\item[•] \texttt{Henshin} per il servizio cloud di immagazzinamento dei dati;
\item[•] \texttt{Android SDK} per la realizzazione dell'applicazione nativa o \texttt{Apache Cordova} per la realizzazione di un'applicazione web-based multi-piattaforma facendo uso di \texttt{NodeJS};
\item[•] \texttt{Git} come tool di versionamento;
\item[•] \texttt{Google Location Services} per la tracciatura del mezzo durante gli spostamenti;
\end{itemize}
\subsection{Aspetti Positivi}
Gli aspetti ritenuti positivi sono: 
\begin{itemize}
\item[•] Utilizzo di Octalysis, basato sulla gamification;
\item[•] Creazione applicazioni Android\textsuperscript{TM} con tecnologie all'avanguardia;
\item[•] Utilizzo di un sistema cloud per l'immagazzinamento dei dati come Hensin;
\end{itemize}
\subsection{Potenziali Criticità}
Le principali criticità constatate sono:
\begin{itemize}
\item[•] Le tecnologie adottate sono sconosciute alla maggior parte del gruppo, pertanto l'apprendimento di queste non si confà al tempo a disposizione dai singoli membri del gruppo; 
\item[•] Mancato interesse verso il capitolato proposto;
\end{itemize}
\subsection{Valutazione Finale}
A causa del numero di tecnologie sconosciute alla maggior parte dei membri del gruppo e vista anche la mancanza di interesse verso la maggior parte di esse, il capitolato è stato rigettato.

