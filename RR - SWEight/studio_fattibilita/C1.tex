\section{Capitolato C1}
\subsection{Informazioni sul Capitolato}
\begin{itemize}
	\item \textbf{Nome:} Butterfly;
	\item \textbf{Proponente}: Imola Informatica;
	\item \textbf{Committenti}: Prof. Tullio Vardanega, Prof. Riccardo Cardin.
\end{itemize}

\subsection{Descrizione}
Il progetto Butterfly prevede la realizzazione di un insieme di plugin\ped{G} che permettano di realizzare un'infrastruttura per la gestione automatizzata delle segnalazioni provenienti dagli applicativi (i.e. SonarQube, GitLab, RedMine). Questi sono utilizzati durante lo sviluppo di software al fine di fornire una migliore accessibilità e standardizzazione dei messaggi. Tali messaggi dovranno essere inseriti all'interno di topic da un broker\ped{G}, che si occuperà della loro gestione. Il proponente del capitolato fa riferimento ad un pattern Publisher/Subscriber\ped{G} per l'implementazione dei componenti richiesti, classificati in categorie:
\begin{itemize}
\item[•]I producer\ped{G} che avranno il compito di recuperare le segnalazioni e pubblicarle;
%\item[•]Il broker impiegato per la creazione e la gestione dei topic;
\item[•] I consumer\ped{G} che avranno il compito di abbonarsi ai topic adeguati, recuperarne i messaggi e procedere al loro inoltro verso i destinatari finali mediante applicazioni atte alla comunicazione, quali Telegram, Slack ed email;
\item[•]Il componente custom\ped{G} specifico per la gestione del personale cheù idonea in quel determinato momento; si occuperà di inoltrare i messaggi alla persona pi
\end{itemize}
\subsection{Dominio Applicativo}
Il dominio applicativo a cui fa riferimento il capitolato proposto è quello della comunicazione automatizzata, infatti i messaggi provenienti dai vari applicativi devono essere inoltrati automaticamente al team di sviluppo.
\subsection{Dominio Tecnologico}
\begin{itemize}
\item[•] \texttt{Apache Kafka}: per la realizzazione del broker;
\item[•] \texttt{NodeJS}, \texttt{Python}, \texttt{Java}: come linguaggi di riferimento per la realizzazione dei componenti;
\item[•] The twelve-factor app, una metodologia per la realizzazione di \texttt{software-as-serivice}\ped{G}
\item[•] \texttt{Docker}: per l’istanziazione di
tutti i componenti;
\item[•] Strumenti per il testing;
\item[•] Utilizzo di un'architettura \texttt{REST};
\item[•] \texttt{Git} come tool di versionamento; 

\end{itemize}
\subsection{Considerazioni del gruppo}

Il progetto presenta alcuni aspetti positivi che non sono richiesti esplicitamente negli altri capitolati, quali l'utilizzo di strumenti di testing e per l'istanziazione delle componenti. 
Le principali criticità constatate sono:
\begin{itemize}
\item[•] Le tecnologie richieste come obbligatorie sono numerose e sconosciute alla maggior parte del gruppo, pertanto l'apprendimento di queste non si confà al tempo a disposizione dai singoli membri del gruppo; 
\item[•] Mancato interesse verso il capitolato proposto;
\item[•] Distanza tra la sede dell'azienda e la sede dell'università;
\end{itemize}

\subsection{Valutazione Finale}
Considerata la distanza tra la sede dell'azienda e l'università, che non permette incontri fisici con il proponente del capitolato, e viste le numerose tecnologie sconosciute alla totalità dei componenti del gruppo, il capitolato proposto è stato rigettato.

