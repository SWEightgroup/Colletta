\documentclass[a4paper, oneside, openany, dvipsnames, table]{article}
\usepackage{../../template/SWEightStyle}
\usepackage{hyperref}
\newcommand{\Titolo}{Verbale Riunione 2018-12-12}

\newcommand{\Gruppo}{SWEight}

\newcommand{\ACapoRedazione}{Francesco Magarotto}

\newcommand{\Verifica}{Francesco Corti}

\newcommand{\Approvazione}{Sebastiano Caccaro}

\newcommand{\Distribuzione}{Vardanega Tullio \newline Cardin Riccardo \newline Gruppo SWEight}

\newcommand{\Uso}{Interno}

\newcommand{\NomeProgetto}{Colletta}

\newcommand{\Mail}{SWEightGroup@gmail.com}

\newcommand{\DescrizioneDoc}{Questo documento si occupa di riportare quanto discusso nella riunione del 12-12-2018}


\begin{document}
\copertina{}

\definecolor{greySWEight}{RGB}{255, 71, 87}
\definecolor{greyROwSWEight}{RGB}{234, 234, 234}

\section*{Registro delle modifiche}
{
	\rowcolors{2}{greyROwSWEight}{white}
	\renewcommand{\arraystretch}{1.5}
	\centering
	\begin{longtable}{ c c  C{4cm}  c  c }
		
		\rowcolor{greySWEight}
		\textcolor{white}{\textbf{Versione}} & \textcolor{white}{\textbf{Data}} & \textcolor{white}{\textbf{Descrizione}} & \textcolor{white}{\textbf{Nominativo}} & \textcolor{white}{\textbf{Ruolo}}\\
		
		1.0.2 & 2019-03-02 & Aggiunti nuovi termini del documento Piano di Progetto & Isachi Gheorghe &\reda{}\\
		
		1.0.1 & 2019-02-23 & Verifica del documento &  Francesco Corti & \ver{}\\
		
		1.0.1 & 2019-02-20 & Aggiunti nuovi termini del documento Norme di Progetto & Isachi Gheorghe &\reda{}\\
		
		1.0.0 & 2019-01-09 & Approvazione & Sebastiano Caccaro & \Res{}\\
						
		0.1.1 & 2019-01-08 & Verifica del documento & Bacco Alberto & \ver{}\\
		
		0.1.1 & 2019-01-04 & Aggiunti termini del documento Norme di Progetto & Isachi Gheorghe &\reda{}\\
		
		0.1.0 & 2019-01-01 & Aggiunti termini del documento Analisi dei Requisiti & Isachi Gheorghe &\reda{}\\
		
		0.0.4 & 2018-12-29 & Verifica del documento & Bacco Alberto & \ver{}\\
				
		0.0.4 & 2018-12-27 & Aggiunti termini del documento Piano di Qualifica & Isachi Gheorghe &\reda{}\\
				
		0.0.3 & 2018-12-26 &Aggiunti termini del documento Piano di Progetto & Isachi Gheorghe & \reda{}\\
				
		0.0.2 & 2018-12-17 & Aggiunti termini del documento Studio di Fattibilità & Isachi Gheorghe &\reda{}\\
		
		0.0.1 & 2018-12-15 & Scheletro del glossario & Damien Ciagola & \reda{}\\
		
	\end{longtable}

}
\newpage
\tableofcontents
\newpage

\section{Informazioni Generali}
\begin{itemize}
\item \textbf{Motivazione:} Discussione sulla corrispondenza con MIVOQ e sull'analisi dei requisiti;
\item \textbf{Luogo:} Online usando Hangouts;
\item \textbf{Data:} 2018-12-21;
\item \textbf{Partecipanti del gruppo:}
	\begin{itemize}
	\item Ciagola Damien;
	\item Corti Francesco;
	\item Muraro Enrico;
	\item Sebastiano Caccaro;
	\item Gionata Legrottaglie;
	\item Magarotto Francesco.
	\end{itemize} 
\item \textbf{Ora:} 15:30 - 16:30;
\item \textbf{Segretario:} Muraro Enrico.
\end{itemize}

\section{Ordine del Giorno}
\begin{itemize}
\item Discussione della risposta ricevuta da MIVOQ riguardante l'email inviata in precedenza;
\item Ricerca dei casi d'uso principali.
\end{itemize}

\section{Resoconto}
\subsection{Discussione della risposta da MIVOQ} In base alla risposta ricevuta, si è deciso di fornire la possibilità di iscrizione al sistema anche agli sviluppatori. I dati saranno quindi scaricabili accedendo all'applicativo, viene inoltre data la possbilità di filtrare i dati prima di scaricarli.
\subsection{Ricerca dei casi d'uso principali} Sono stati discussi e trovati i casi d'uso principali di ogni attore del sistema. In particolare i casi d'uso principali di:
\begin{itemize}
\item Utente non autenticato:
	\begin{itemize}
		\item Autenticazione;
		\item Registrazione.
	\end{itemize}
\item Insegnante:
	\begin{itemize}
		\item Visualizzazione del profilo personale;
		\item Visualizzazione degli esercizi svolti dai propri allievi;
		\item Inserimento di esercizi.
	\end{itemize}
\item Allievo:
	\begin{itemize}
		\item Visualizzazione della propria dashboard;
		\item Ricerca di esercizi;
		\item Svolgimento di esercizi.
	\end{itemize}
\item Sviluppatore
	\begin{itemize}
		\item Creazione filtro dei dati;
		\item Scaricamento dei dati.
	\end{itemize}
\end{itemize}

\newpage
\end{document}