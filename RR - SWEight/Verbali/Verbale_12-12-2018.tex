\documentclass[a4paper, oneside, openany, dvipsnames, table]{article}
\usepackage{../template/SWEightStyle}
\usepackage{hyperref}
\newcommand{\Titolo}{Verbale Riunione 2018-12-12}

\newcommand{\Gruppo}{SWEight}

\newcommand{\ACapoRedazione}{Francesco Magarotto}

\newcommand{\Verifica}{Francesco Corti}

\newcommand{\Approvazione}{Sebastiano Caccaro}

\newcommand{\Distribuzione}{Vardanega Tullio \newline Cardin Riccardo \newline Gruppo SWEight}

\newcommand{\Uso}{Interno}

\newcommand{\NomeProgetto}{Colletta}

\newcommand{\Mail}{SWEightGroup@gmail.com}

\newcommand{\DescrizioneDoc}{Questo documento si occupa di riportare quanto discusso nella riunione del 12-12-2018}


\begin{document}
\copertina{}

\definecolor{greySWEight}{RGB}{255, 71, 87}
\definecolor{greyROwSWEight}{RGB}{234, 234, 234}

\section*{Registro delle modifiche}
{
	\rowcolors{2}{greyROwSWEight}{white}
	\renewcommand{\arraystretch}{1.5}
	\centering
	\begin{longtable}{ c c  C{4cm}  c  c }
		
		\rowcolor{greySWEight}
		\textcolor{white}{\textbf{Versione}} & \textcolor{white}{\textbf{Data}} & \textcolor{white}{\textbf{Descrizione}} & \textcolor{white}{\textbf{Nominativo}} & \textcolor{white}{\textbf{Ruolo}}\\
		
		1.0.2 & 2019-03-02 & Aggiunti nuovi termini del documento Piano di Progetto & Isachi Gheorghe &\reda{}\\
		
		1.0.1 & 2019-02-23 & Verifica del documento &  Francesco Corti & \ver{}\\
		
		1.0.1 & 2019-02-20 & Aggiunti nuovi termini del documento Norme di Progetto & Isachi Gheorghe &\reda{}\\
		
		1.0.0 & 2019-01-09 & Approvazione & Sebastiano Caccaro & \Res{}\\
						
		0.1.1 & 2019-01-08 & Verifica del documento & Bacco Alberto & \ver{}\\
		
		0.1.1 & 2019-01-04 & Aggiunti termini del documento Norme di Progetto & Isachi Gheorghe &\reda{}\\
		
		0.1.0 & 2019-01-01 & Aggiunti termini del documento Analisi dei Requisiti & Isachi Gheorghe &\reda{}\\
		
		0.0.4 & 2018-12-29 & Verifica del documento & Bacco Alberto & \ver{}\\
				
		0.0.4 & 2018-12-27 & Aggiunti termini del documento Piano di Qualifica & Isachi Gheorghe &\reda{}\\
				
		0.0.3 & 2018-12-26 &Aggiunti termini del documento Piano di Progetto & Isachi Gheorghe & \reda{}\\
				
		0.0.2 & 2018-12-17 & Aggiunti termini del documento Studio di Fattibilità & Isachi Gheorghe &\reda{}\\
		
		0.0.1 & 2018-12-15 & Scheletro del glossario & Damien Ciagola & \reda{}\\
		
	\end{longtable}

}
\newpage
\tableofcontents
\newpage

\section{Informazioni Generali}
\begin{itemize}
\item \textbf{Motivazione:} Discussione sull'analisi dei requisiti
\item \textbf{Luogo:} Torre Archimede
\item \textbf{Data:} 2018-12-12
\item \textbf{Partecipanti del gruppo:}
	\begin{itemize}
	\item Bacco Alberto
	\item Ciagola Damien
	\item Corti Francesco
	\item Munaro Enrico
	\item Magarotto Francesco
	\end{itemize} 
\item \textbf{Ora: 12:30 - 14.30}
\item \textbf{Segretario: Magarotto Francesco}
\end{itemize}

\section{Ordine del Giorno}
\begin{itemize}
\item \textbf{Suddivisione contenuti per la realizzazione dell'Analisi dei Requisiti}
\item \textbf{Stesura email contenente le domande per il proponente del capitolato (Mivoq S.r.l)}
\item \textbf{Attività di verifica sullo studio di fattbilità}
\end{itemize}

\section{Resoconto}
\begin{itemize}
\item \textbf{Organigramma:} Come concordato con il capo progetto, l'attività di verifica relativa all'analisi dei requisiti è delegata agli analisti, che la svolgeranno in modo tale che nessun testo prodotto sia verificato dalla medesima persona.
\item \textbf{Stesura testo email per Mivoq: } Il capitolato ha destato alcuni dubbi che potrebbero compromettere la corretta analisi dei requisiti. Si è ritenuto strettamente necessario contattare l'azienda proponente e pertanto è stata scritta una bozza contenente le domande interessate. Il testo è stato condiviso sul canale Slack del gruppo e può essere soggetto a modifiche, fino al giorno 2018-12-16.
\item \textbf{Suddivisione contenuti per Analisi dei Requisiti:} Facendo riferimento allo standard e a documenti di anni precedenti, è stata realizzata una scaletta per la realizzazione del suddetto documento. Ogni elemento della scaletta è stato assegnato ad un analista che lo completerà entro il giorno 2018-12-16, per permetterne la verifica il giorno 2018-12-17.
La realizzazione dei casi d'uso è delegata all'interno team di analisti.
\end{itemize}
\subsection{Organigramma}
\begin{itemize}
\item Come concordato con il capo progetto, l'organigramma prevede i seguenti ruoli, che saranno soggetti a cambiamenti in brevissimo tempo:
	\begin{itemize}
	\item \textbf{Responsabile:} Caccaro Sebastiano
	\item \textbf{Analisti:} Ciagola Damien, Corti Francesco, Magarotto Francesco, Muraro Enrico, Legrottaglie Gionata
	\item \textbf{Amministratore:} Legrottaglie Gionata, Bacco Alberto
	\item \textbf{Verificatori:} Isachi Gheorghe, Bacco Alberto, Magarotto Francesco
	\end{itemize}
\end{itemize}

\subsection{Norme di Progetto}
\begin{itemize}
\item \textbf{Attività di verifica:} Conclusa la stesura del documento è necessaria un'attività di verifica. I componenti designati a tale attività sono Isachi e Magarotto.
\end{itemize}
\subsection{Analisi Capitolati}
\begin{itemize}
\item \textbf{Scelta capitolato:} 
\end{itemize}

\subsection{Pianificazione Generale}
\begin{itemize}
\item \textbf{Scadenze:} Confermata l’intenzione di presentare la RR entro il 14 Gennaio.
\item \textbf{Documenti:} Fatto il punto sui documenti da redigere, analizzate brevemente alcune delle RR degli anni precedenti. Si è quindi deciso di procedere con zero latency, partendo il prima possibile con la fase di analisi.
\end{itemize}


\newpage
\end{document}