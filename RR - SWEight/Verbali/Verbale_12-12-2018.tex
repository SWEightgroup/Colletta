\documentclass[a4paper, oneside, openany, dvipsnames, table]{article}
\usepackage{../template/SWEightStyle}
\usepackage{hyperref}
\newcommand{\Titolo}{Manuale Utente}

\newcommand{\Gruppo}{SWEight}

\newcommand{\Approvatore}{Damien Ciagola}
\newcommand{\Redattori}{Alberto Bacco \newline Sebastiano Caccaro \newline Gheorghe Isachi \newline Gionata Legrottaglie}
\newcommand{\Verificatori}{Francesco Corti \newline Francesco Magarotto}

\newcommand{\pathimg}{../template/img/logoSWEight.png}

\newcommand{\Versionedoc}{1.0.0}

\newcommand{\Distribuzione}{\proponente \newline Prof. Vardanega Tullio \newline Prof. Cardin Riccardo \newline Gruppo SWEight}

\newcommand{\Uso}{Esterno}

\newcommand{\NomeProgetto}{Colletta}

\newcommand{\Mail}{SWEightGroup@gmail.com}

\newcommand{\DescrizioneDoc}{Questo documento si occupa di fornire le modalità di utilizzo del software Colletta commissionato}


\begin{document}
\copertina{}

\definecolor{greySWEight}{RGB}{255, 71, 87}
\definecolor{greyROwSWEight}{RGB}{234, 234, 234}

\section*{Registro delle modifiche}
{
	\rowcolors{2}{greyROwSWEight}{white}
	\renewcommand{\arraystretch}{1.5}
	\centering
	\begin{longtable}{ c c C{4cm}  c  c }
		
		\rowcolor{greySWEight}
		\textcolor{white}{\textbf{Versione}} & \textcolor{white}{\textbf{Data}} & \textcolor{white}{\textbf{Descrizione}} & \textcolor{white}{\textbf{Nominativo}} & \textcolor{white}{\textbf{Ruolo}}\\
		1.2.2 & 2019-02-25 & Ampliamento sezione 5.4 e 3.2.5.2 & Alberto Bacco & \reda{} \\
		
		1.2.1 & 2019-02-23 & Aggiunta sezione 3.2.5.8 Checkstyle & Sebastiano Caccaro & \reda{} \\		
		
		1.2.0 & 2019-02-20 & Aggiunta scelte tecnologiche 3.2.4.2, da 3.4.5.4 a 3.4.5.7, 4.3.1.4, 4.3.2.2, 4.4.6 e figlie & Sebastiano Caccaro & \reda{} \\	
		
		1.1.5 & 2019-02-20 & Modifica sezione 2 & Alberto Bacco & \reda{} \\
		
		1.1.4 & 2019-02-18 & Correzione errori grammatica, spostate sottosezioni di asana da 4.3 a 5.2, & Alberto Bacco & \reda{} \\
		
		1.1.3 & 2019-02-14 & Riorganizzazione e correzione errori sezione 5 & Enrico Muraro & \reda{} \\
		
		1.1.2 & 2019-02-03 & Modifica sottosezione 4.1.10, 4.3.1.4, 4.3.1.5, 4.3.1.6 & Alberto Bacco& \reda{} \\	
		
		1.1.1 & 2019-01-31 & Modifica struttura e contenuti sezione 3  & Damien Ciagola & \reda{} \\	
		
		1.1.0 & 2019-01-27 & Sezione Qualità 4.2 & Sebastiano Caccaro & \reda{} \\	
		
		1.0.1 & 2019-01-25 & Parziale ristrutturazione della struttura del documento & Sebastiano Caccaro & \reda{} \\		
		
		1.0.0 & 2019-01-11 & Approvazione per il rilascio & Sebastiano Caccaro & \Res{} \\
		
		0.9.0 & 2019-01-9 & Verifica finale & Francesco Corti & \ver{} \\
		
		0.9.0 & 2019-01-8 & Aggiunta lista di controllo & Gionata Legrottaglie & \reda{} \\
		
		0.8.0 & 2018-12-23 & Correzioni errori ortografici & Gionata Legrottaglie & \reda{} \\
		
		0.7.0 & 2018-12-20 & Verifica documento & Francesco Corti & \ver{}\\
		
		0.6.0 & 2018-12-18 & Aggiunta sottosezione 5.2.2.2, 5.2.2.3, 5.2.2.4 & Francesco Magarotto & \reda{} \\
		
		0.5.2 & 2018-12-16 & Modifica sezione 4.1.5.3 & Alberto Bacco & \reda{} \\
		
		0.5.2 & 2018-12-16 & Modifica sezione 4.1.5.3 & Alberto Bacco & \reda{} \\
		
		0.5.2 & 2018-12-16 & Aggiunte sottosezioni  & Alberto Bacco & \reda{} \\
		
		0.5.1 & 2018-12-15 & Aggiunte sottosezioni 5.3, 5.4, 5.5, 5.6, 5.7, 5.8 & Alberto Bacco & \reda{} \\
		
		0.5.0 & 2018-12-15 & Aggiunta sezione 5 e sottosezioni 5.1, 5.2 & Gionata Legrottaglie & \reda{} \\
		
		0.4.1 & 2018-12-11 & Aggiunta sezione 4.1.7.3.1 & Francesco Magarotto & \reda{} \\ 
		
		0.4.0 & 2018-12-10 & Aggiunte sottosezioni 4.1.5, 4.1.6, 4.1.7, 4.1.8 & Gionata Legrottaglie & \reda{} \\ 
		0.4.0 & 2018-12-09 & Aggiunta sezione 4 e sottosezioni 4.1.1, 4.1.2, 4.1.3, 4.1.4 & Gionata Legrottaglie & \reda{} \\ 
		
		0.3.1 & 2018-12-07 & Aggiunta sottosezione 3.2 & Gionata Legrottaglie & \reda{} \\ 
		
		0.3.0 & 2018-12-06 & Aggiunta sezione 3 e sottosezione 3.1 & Gionata Legrottaglie & \reda{} \\ 
		
		0.2.0 & 2018-12-05 & Aggiunti i riferimenti & Gionata Legrottaglie & \reda{} \\ 
		
		0.1.0 & 2018-11-30 & Aggiunta introduzione & Gionata Legrottaglie & \reda{} \\
		
		0.0.1 & 2018-11-28 & Creazione scheletro del documento & Gionata Legrottaglie & \reda{}\\
		
	\end{longtable}

}
\newpage
\tableofcontents
\newpage

\section{Informazioni Generali}
\begin{itemize}
\item \textbf{Motivazione:} Discussione sull'analisi dei requisiti;
\item \textbf{Luogo:} Torre Archimede;
\item \textbf{Data:} 2018-12-12;
\item \textbf{Partecipanti del gruppo:}
	\begin{itemize}
	\item Bacco Alberto;
	\item Ciagola Damien;
	\item Corti Francesco;
	\item Munaro Enrico;
	\item Magarotto Francesco.
	\end{itemize} 
\item \textbf{Ora:} 12:30 - 14.30;
\item \textbf{Segretario:} Magarotto Francesco.
\end{itemize}

\section{Ordine del Giorno}
\begin{itemize}
\item Analisi contenuti per la realizzazione dell'Analisi dei Requisiti;
\item Stesura email contenente le domande per il proponente del capitolato (MIVOQ S.r.l);
\item Attività di verifica sullo studio di fattibilità.
\end{itemize}

\section{Resoconto}
\begin{itemize}
\item \textbf{Organigramma}: L'attività di verifica relativa all'analisi dei requisiti è delegata ai verificatori e ad alcuni analisti. Quest'ultimi svolgeranno l'attività di verifica in modo tale che nessun testo prodotto sia verificato dalla medesima persona che l'ha redatto. 
\item \textbf{Stesura testo email per MIVOQ}: Il capitolato ha destato alcuni dubbi che potrebbero compromettere la corretta analisi dei requisiti. Si è ritenuto strettamente necessario contattare l'azienda proponente e, pertanto, è stata scritta una bozza contenente le domande interessate. Il testo è stato condiviso sul canale Slack del gruppo e può essere soggetto a modifiche, fino al giorno 2018-12-16.
\item \textbf{Suddivisione contenuti per Analisi dei Requisiti:} Facendo riferimento allo standard e a documenti di anni precedenti, è stata realizzata una scaletta per la realizzazione del suddetto documento.
La realizzazione dei casi d'uso è delegata all'interno team di analisti che procederanno collettivamente alla stesura del documento.
\end{itemize}
\subsection{Organigramma}
\begin{itemize}
\item In questo periodo del progetto, vengono incaricati della stesura dell'Analisi dei Requisiti i seguenti membri:
	\begin{itemize}
	\item 	Ciagola Damien;
	\item	Corti Francesco;
	\item	Magarotto Francesco;
	\item	Muraro Enrico;
	\item	Legrottaglie Gionata.
	\end{itemize}
\end{itemize}

\subsection{Norme di Progetto}
\begin{itemize}
\item \textbf{Attività di verifica}: Conclusa la stesura del documento è necessaria un'attività di verifica. I componenti designati a tale attività sono Isachi, Magarotto (escluse le parti che ha redatto) e Corti.
\end{itemize}
\subsection{Analisi dei Requisiti}
\begin{itemize}
\item \textbf{Stesura lista dei requisiti}: Gli analisi procederanno collettivamente a stendere la lista dei requisiti e a catalogarli. Successivamente, dopo aver verificato che quest'ultima sia completa, si procederà alla suddivisione dei requisiti per la realizzazione.
Per le convenzioni riguardanti la stesura dei casi d'uso si rimanda alle Norme di Progetto 1.0.0
\end{itemize}

\subsection{Pianificazione Generale}
\begin{itemize}
\item \textbf{Scadenze:} 
\begin{itemize}
\item[•] 2018-12-16 16:00 - Termine ultimo per l'aggiunta di ulteriori domande da porre al proponente del capitolato;
\item[•] 2018-12-16 23:59 - Termine ultimo per la stesura dell'elenco dei requisiti;
\item[•] 2018-12-18 - Termine ultimo per la verifica della lista dei requisiti;
\item[•] 2019-1-14 17:00 - Consegna documenti per Revisione dei Requisiti.
\end{itemize}
\item[•] \textbf{Documenti:} Fatto il punto sui documenti da redigere, analizzate brevemente alcune delle RR degli anni precedenti. Si è quindi deciso di procedere con zero latency, con l'obbiettivo di completare l'analisi dei requisiti almeno 5 giorni prima della consegna dei documenti per permetterne una verifica accurata.
\end{itemize}


\newpage
\end{document}