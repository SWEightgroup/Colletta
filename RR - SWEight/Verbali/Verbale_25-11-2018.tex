\documentclass[a4paper, oneside, openany, dvipsnames, table]{article}
\usepackage{../template/SWEightStyle}
\newcommand{\Titolo}{Verbale Riunione 2018-12-12}

\newcommand{\Gruppo}{SWEight}

\newcommand{\ACapoRedazione}{Francesco Magarotto}

\newcommand{\Verifica}{Francesco Corti}

\newcommand{\Approvazione}{Sebastiano Caccaro}

\newcommand{\Distribuzione}{Vardanega Tullio \newline Cardin Riccardo \newline Gruppo SWEight}

\newcommand{\Uso}{Interno}

\newcommand{\NomeProgetto}{Colletta}

\newcommand{\Mail}{SWEightGroup@gmail.com}

\newcommand{\DescrizioneDoc}{Questo documento si occupa di riportare quanto discusso nella riunione del 12-12-2018}


\begin{document}
\copertina{}

\definecolor{greySWEight}{RGB}{255, 71, 87}
\definecolor{greyROwSWEight}{RGB}{234, 234, 234}

\section*{Registro delle modifiche}
{
	\rowcolors{2}{greyROwSWEight}{white}
	\renewcommand{\arraystretch}{1.5}
	\centering
	\begin{longtable}{ c c  C{4cm}  c  c }
		
		\rowcolor{greySWEight}
		\textcolor{white}{\textbf{Versione}} & \textcolor{white}{\textbf{Data}} & \textcolor{white}{\textbf{Descrizione}} & \textcolor{white}{\textbf{Nominativo}} & \textcolor{white}{\textbf{Ruolo}}\\
		
		1.0.2 & 2019-03-02 & Aggiunti nuovi termini del documento Piano di Progetto & Isachi Gheorghe &\reda{}\\
		
		1.0.1 & 2019-02-23 & Verifica del documento &  Francesco Corti & \ver{}\\
		
		1.0.1 & 2019-02-20 & Aggiunti nuovi termini del documento Norme di Progetto & Isachi Gheorghe &\reda{}\\
		
		1.0.0 & 2019-01-09 & Approvazione & Sebastiano Caccaro & \Res{}\\
						
		0.1.1 & 2019-01-08 & Verifica del documento & Bacco Alberto & \ver{}\\
		
		0.1.1 & 2019-01-04 & Aggiunti termini del documento Norme di Progetto & Isachi Gheorghe &\reda{}\\
		
		0.1.0 & 2019-01-01 & Aggiunti termini del documento Analisi dei Requisiti & Isachi Gheorghe &\reda{}\\
		
		0.0.4 & 2018-12-29 & Verifica del documento & Bacco Alberto & \ver{}\\
				
		0.0.4 & 2018-12-27 & Aggiunti termini del documento Piano di Qualifica & Isachi Gheorghe &\reda{}\\
				
		0.0.3 & 2018-12-26 &Aggiunti termini del documento Piano di Progetto & Isachi Gheorghe & \reda{}\\
				
		0.0.2 & 2018-12-17 & Aggiunti termini del documento Studio di Fattibilità & Isachi Gheorghe &\reda{}\\
		
		0.0.1 & 2018-12-15 & Scheletro del glossario & Damien Ciagola & \reda{}\\
		
	\end{longtable}

}
\newpage
\tableofcontents
\newpage

\section{Informazioni Generali}
\begin{description}
\item [Motivazione:] Discussione generale sul progetto
\item [Luogo:] Bar Cinesi, Via Belzoni
\item [Data:] 2018-11-25
\item [Partecipanti del gruppo:] \hfill
	\begin{itemize}
	\item Bacco Alberto
	\item Caccaro Sebastiano
	\item Ciagola Damien
	\item Corti Francesco
	\item Isachi Gheorghe
	\item Legrottaglie Gionata
	\end{itemize} 
\item [Ora:] 15:00 - 17:00
\item [Segretario:] Caccaro Sebastiano
\end{description}

\section{Ordine del Giorno}
\begin{description}
\item [Organigramma:] Discussione sull'assegnazione dei ruoli per il periodo precedente la RR
\item [Norme di progetto:] Individuati gli strumenti principali per la codifica dei documenti e per l'assegnazione e monitoraggio dei task.
\item [Analisi capitolati:] Analizzati capitolati. Scelto e analizzato il capitolato C2 (Colletta)
\item [Pianificazione generale:] Analisi delle scadenze e dei documenti da redigere per la RR
\end{description}

\section{Resoconto}


\subsection{Organigramma}
\begin{itemize}
\item Stabilita la seguente bozza di organigramma per la RR, ancora non definitiva e soggetta a futuri cambiamenti.
	\begin{description}
	\item [Responsabile:] Caccaro Sebastiano
	\item [Analisti:] Ciagola Damien, Corti Francesco, Magarotto Francesco, Muraro Enrico
	\item [Amministratore:] Legrottaglie Gionata
	\item [Verificatori:] Isachi Gheorghe, Bacco Alberto
	\end{description}
\end{itemize}

\subsection{Norme di Progetto}
\begin{description}
\item [Redazione documenti:] Per la redazione dei documenti verrà usato Latex, le norme di codifica verranno stabilite da Corti e Ciagola che hanno già familiarità con il linguaggio sopracitato.
\item [Gestione Codice:] scelta la piattaforma GitHub. Creata la seguente repo. https://github.com/SWEightgroup/Colletta/
\item [Gestione Task:] Individuate opzioni per la gestione l'assegnazione di scadenze e attività. Sono state presi in considerazione l’issue tracking system integrato in GitHub (opportunamente configurato) e Asana. Maggiori approfondimenti sono richiesti.
\end{description}

\subsection{Analisi Capitolati}
\begin{description}
\item [Scelta capitolato:] Scelto il capitolato C2. Si è trattato più di una conferma, in quanto si era già discusso di ciò via chat su Telegram e brevemente a voce in altre occasioni.
\item [Discussione sul capitolato scelto:] Fatta un’overview informale del capitolato C2. Compresi i requisiti principali e lo scopo del progetto.
\end{description}

\subsection{Pianificazione Generale}
\begin{description}
\item [Scadenze:] Confermata l’intenzione di presentare la RR entro il 14 Gennaio.
\item [Documenti:] Fatto il punto sui documenti da redigere, analizzate brevemente alcune delle RR degli anni precedenti. Si è quindi deciso di procedere con zero latency, partendo il prima possibile con la fase di analisi.
\end{description}


\newpage
\end{document}