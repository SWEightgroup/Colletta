 
\renewcommand{\arraystretch}{1.5}
\def\tabularxcolumn#1{m{#1}}
\begin{tabularx}{\textwidth}{cXccc}
 
\rowcolor{greySWEight}
   \textcolor{white}{\textbf{Identificativo}} &
   \textcolor{white}{\textbf{Descrizione}}&
   \textcolor{white}{\textbf{Importanza}}&
   \textcolor{white}{\textbf{Tipo}}&
   \textcolor{white}{\textbf{Fonte}}\endhead
 
%%################## INIZIO ENTRY ######################################    
%Identificativo
RF4242P &
 
%Descrizione
Questa è una descrizione lunga. Questa è una descrizione lunga. Questa è una descrizione lunga. Questa è una descrizione lunga. Questa è una descrizione lunga. Questa è una descrizione lunga. &
 
%Importanza
Altissima! &
 
%Tipo
lol &
 
%Fonte
Boh \\
%Puoi anche mettere \hline al posto di \\ se vuoi la barra orizzontale,
%ma secondo me è bello gia così
%%##################### FINE ENTRY ######################################
 
%%################## INIZIO ENTRY ######################################    
%Identificativo
RF4242P &
 
%Descrizione
Questa è una descrizione lunga. Questa è una descrizione lunga. Questa è una descrizione lunga. Questa è una descrizione lunga. Questa è una descrizione lunga. Questa è una descrizione lunga. &
 
%Importanza
Altissima! &
 
%Tipo
lol &
 
%Fonte
Boh \\
%Puoi anche mettere \hline al posto di \\ se vuoi la barra orizzontale,
%ma secondo me è bello gia così
%%##################### FINE ENTRY ######################################
   
 
\caption{Questa è una tabella} \label{tab:tabellarischi}
\end{tabularx}