\subsection{Sviluppatori}
\subsubsection{Panoramica Sviluppatori}
\begin{figure}[H]
\centering
\includegraphics[width=17cm]{img/UC4x.png} 
\caption{Panoramica casi d'uso che interessano lo Sviluppatore, per le versioni più specifiche dei vari casi si vedano sotto}\label{fig:4x}
\end{figure}
\subsubsection{UC 4.1 - Visualizzazione elenco frasi}
\begin{figure}[H]
\centering
\includegraphics[width=17cm]{img/UC410.png} 
\caption{Caso d'uso 4.1}\label{fig:410}
\end{figure}
\begin{itemize}
\item[•]\textbf{Attori}: Sviluppatore;
\item[•]\textbf{Descrizione}: Lo sviluppatore visualizza un elenco di frasi accompagnate dalla data di creazione e dall’autore ordinate cronologicamente;
\item[•]\textbf{Precondizione}:  L'amministratore ha approvato l'utenza dello sviluppatore, lo sviluppatore si è autenticato nel sistema e visualizza la propria dashboard\ped{G}.
\item[•]\textbf{Postcondizione}: Lo sviluppatore visualizza l'elenco delle preposizioni ordinate cronologicamente;
\item[•]\textbf{Flusso degli eventi principale}: Lo sviluppatore ha selezionato la voce relativa alla pagina contenente l’elenco delle frasi e ne visualizza il contenuto; 
\end{itemize}
\subsubsection{UC 4.2 - Filtraggio dei dati}
\begin{figure}[H]
\centering
\includegraphics[width=17cm, height=10cm]{img/UC420.png} 
\caption{Caso d'uso 4.2}\label{fig:420}
\end{figure}
\begin{figure}[H]
\centering
\includegraphics[width=17cm, height=10cm]{img/UC426.png} 
\caption{Caso d'uso 4.2.6 relativo alla rimozione dei filtri}\label{fig:426}
\end{figure}
\begin{itemize}
\item[•]\textbf{Attori}: Sviluppatore;
\item[•]\textbf{Descrizione}: Lo sviluppatore applica dei filtri ai dati, ottenendo solo quelli d'interesse;
\item[•]\textbf{Precondizione}: Lo sviluppatore visualizza le proposizioni in ordine cronologico;
\item[•]\textbf{Postcondizione}: Lo sviluppatore visualizza i dati che rispettano i filtri scelti;
\item[•]\textbf{Flusso degli eventi principale}:
\begin{enumerate}
\item UC 4.2.1 - Inserimento filtro periodo temporale;
\item UC 4.2.2 - Inserimento filtro fonte\ped{G};
\item UC 4.2.3 - Selezione filtro lingue;
\item UC 4.2.4 - Ricerca parole chiave.
\end{enumerate}
\item[•]\textbf{Estensioni}:
\begin{enumerate}
	\item UC 4.2.6 - Rimozione filtro;
\end{enumerate}
\item[•]\textbf{Inclusioni}:
\begin{enumerate}
\item UC 4.2.5 - Visualizzazione dati filtrati
\end{enumerate}
\end{itemize}
\subsubsection{UC 4.2.1 - Inserimento filtro periodo temporale}
\begin{itemize}
\item[•]\textbf{Attori}: Sviluppatore;
\item[•]\textbf{Descrizione}: Lo sviluppatore seleziona un periodo temporale, ottenendo i dati salvati in un determinato arco di tempo;
\item[•]\textbf{Precondizione}: Lo sviluppatore visualizza le proposizioni in base ai filtri stabiliti;
\item[•]\textbf{Postcondizione}: Lo sviluppatore visualizza le frasi in base al filtraggio inserito;
\item[•]\textbf{Flusso degli eventi principale}: 
\begin{enumerate}
\item Scelta data inizio periodo;
\item Scelta data fine periodo;
\end{enumerate}
\item[•]\textbf{Estensioni}: 
\begin{enumerate}
	\item Visualizzazione messaggio di errore nel caso si scegliesse una data non idonea;
	\item UC 4.2.6 - Rimozione filtro;
\end{enumerate}
\item[•]\textbf{Inclusioni}:
\begin{enumerate}
\item UC 4.2.5 - Visualizzazione dati filtrati.
\end{enumerate}
\end{itemize}

\subsubsection{UC 4.2.2 - Inserimento filtro fonte}
\begin{itemize}
\item[•]\textbf{Attori}: Sviluppatore;
\item[•]\textbf{Descrizione}: Lo sviluppatore seleziona una o più fonti degli esercizi, come ad esempio le correzione provenienti da determinati docenti o dal sistema di correzione automatico;
\item[•]\textbf{Precondizione}: Lo sviluppatore visualizza le proposizioni in base ai filtri stabiliti;
\item[•]\textbf{Postcondizione}: Lo sviluppatore ha impostato i valori del filtro fonte;
\item[•]\textbf{Flusso degli eventi principale}: Lo sviluppatore seleziona da un elenco le fonti di cui è interessato visualizzare i dati;
\item[•]\textbf{Estensioni}: 
\begin{enumerate}
	\item UC 4.2.6 - Rimozione filtro;
\end{enumerate}
\item[•]\textbf{Inclusioni}:
\begin{enumerate}
\item UC 4.2.5 - Visualizzazione dati filtrati.
\end{enumerate}
\end{itemize}

\subsubsection{UC 4.2.3 -  Selezione filtro lingue}
\begin{itemize}
\item[•]\textbf{Attori}: Sviluppatore;
\item[•]\textbf{Descrizione}: Lo sviluppatore seleziona una o più lingue da un elenco di lingue predefinito al fine di ottenere solamente le proposizioni scritte in tali lingue;
\item[•]\textbf{Precondizione}: Lo sviluppatore visualizza le proposizioni in base ai filtri stabiliti;
\item[•]\textbf{Postcondizione}: Lo sviluppatore ha stabilito i valori di filtraggio relativi alle lingue;
\item[•]\textbf{Flusso degli eventi principale}: Lo sviluppatore seleziona da un elenco le lingue di cui è interessato vedere le proposizioni memorizzate;
\item[•]\textbf{Estensioni}: 
\begin{enumerate}
	\item UC 4.2.6 - Rimozione filtro;
\end{enumerate}
\item[•]\textbf{Inclusioni}:
\begin{enumerate}
\item UC 4.2.5 - Visualizzazione dati filtrati.
\end{enumerate}
\end{itemize}

\subsubsection{UC 4.2.4 - Ricerca frase}
\begin{itemize}
\item[•]\textbf{Attori}: Sviluppatore;
\item[•]\textbf{Descrizione}: Lo sviluppatore scrive una o più parole chiave al fine di cernere le frasi contenenti tali parole;
\item[•]\textbf{Precondizione}: Lo sviluppatore visualizza le frasi in base al filtraggio scelto;
\item[•]\textbf{Postcondizione}: Lo sviluppatore ha impostato le parole chiave;
\item[•]\textbf{Flusso degli eventi principale}: Lo sviluppatore seleziona da un elenco le lingue di cui è interessato vedere le proposizioni memorizzate;
\item[•]\textbf{Estensioni}: 
\begin{enumerate}
	\item Visualizzazione messaggio di errore nel caso venga inserito un input non consentito;
	\item UC 4.2.6 - Rimozione filtro;
\end{enumerate}
\item[•]\textbf{Inclusioni}:
\begin{enumerate}
\item UC 4.2.5 - Visualizzazione dati filtrati
\end{enumerate}
\end{itemize}

\subsubsection{UC 4.2.5 - Visualizzazione dati filtrati}
\begin{itemize}
\item[•]\textbf{Attori}: Sviluppatore;
\item[•]\textbf{Descrizione}: Selezionato uno o più filtri, i dati vengono mostrati secondo i vincoli inseriti;
\item[•]\textbf{Precondizione}: I valori di uno o più filtri sono stati inseriti;
\item[•]\textbf{Postcondizione}: I dati mostrati rispettano tali vincoli;
\end{itemize}

\subsubsection{UC 4.2.6 - Rimozione filtro}
\begin{itemize}
\item[•]\textbf{Attori}: Sviluppatore;
\item[•]\textbf{Descrizione}: Lo sviluppatore cancella il valore del filtro selezionato.
\item[•]\textbf{Precondizione}: Il filtro contiene un valore ed è applicato un filtraggio ai dati rispettando le condizioni di tale filtro;
\item[•]\textbf{Postcondizione}: Il filtro è stato rimosso pertanto i dati visualizzati non rispetteranno più tale vincolo.
\item[•]\textbf{Flusso degli eventi}: Lo sviluppatore seleziona il filtro e lo rimuove.
 \end{itemize}
  
\subsubsection{UC 4.3 - Visualizzazione storico frase}
\begin{itemize}
\item[•]\textbf{Attori}: Sviluppatore;
\item[•]\textbf{Descrizione}: Lo sviluppatore seleziona una frase dall'elenco e ne visualizza lo storico contenente la correzione fornita dal sistema automatico, ed eventualmente quelle redatte anche da insegnanti diversi;
\item[•]\textbf{Precondizione}:  Lo sviluppatore visualizza le frasi in base ai filtri stabiliti;
\item[•]\textbf{Postcondizione}: Lo sviluppatore visualizza tutte le correzioni della frase;
\item[•]\textbf{Flusso degli eventi principale}: Lo sviluppatore seleziona la frase e appare lo storico della frase.
\end{itemize}

\subsubsection{UC 4.4 - Scaricamento dati}
\begin{figure}[H]
\centering
\includegraphics[width=17cm, height=10.5cm]{img/UC440.png} 
\caption{Caso d'uso 4.4}\label{fig:440}
\end{figure}
\begin{itemize}
\item[•]\textbf{Attori}: Sviluppatore;
\item[•]\textbf{Descrizione}: Lo sviluppatore esegue il download dei dati d'interesse;
\item[•]\textbf{Precondizione}: Lo sviluppatore visualizza i dati in ordine cronologico, e nel caso siano applicati filtri, vede solo i dati che rispettano tali filtri;
\item[•]\textbf{Postcondizione}:  Lo sviluppatore scarica un file contenente i dati visualizzati nel formato scelto;
\item[•]\textbf{Flusso degli eventi principale}:
\begin{enumerate}
\item UC 4.4.1 - Scelta contenuti download;
\item UC 4.4.2 - Scelta formato file;
\item UC 4.4.3 - Scelta formato compressione;
\item UC 4.4.4 - Ottenimento file.
\end{enumerate}
\end{itemize}

\subsubsection{UC 4.4.1 - Scelta contenuti download}
\begin{itemize}
\item[•]\textbf{Attori}: Sviluppatore;
\item[•]\textbf{Descrizione}: Lo sviluppatore sceglie gli elementi di interesse che vuole siano scaricati una volta completata la procedura di download;
\item[•]\textbf{Precondizione}: Lo sviluppatore visualizza le frasi in base ai filtri stabiliti;
\item[•]\textbf{Postcondizione}:  Lo sviluppatore visualizza i contenuti che intende scaricare;
\item[•]\textbf{Flusso degli eventi principale}:  Lo sviluppatore seleziona dall'elenco, eventualmente filtrato, delle frasi i dati d'interesse, come ad esempio i soli docenti di italiano di cui vuole scaricare i dati.
\end{itemize}

\subsubsection{UC 4.4.2 - Scelta formato file}
\begin{itemize}
\item[•]\textbf{Attori}: Sviluppatore;
\item[•]\textbf{Descrizione}:  Lo sviluppatore seleziona da un elenco prestabilito il formato dei file che intende scaricare;
\item[•]\textbf{Precondizione}: Lo sviluppatore non ha stabilito il formato dei file che intende scaricare;
\item[•]\textbf{Postcondizione}:  Lo sviluppatore ha stabilito il formato dei file che intende scaricare;
\item[•]\textbf{Flusso degli eventi principale}:  Lo sviluppatore seleziona dall'elenco prestabilito un'opzione che determina il formato dei file che contengono i dati precedentemente scelti.
\end{itemize}

\subsubsection{UC 4.4.2 - Scelta formato archivio}
\begin{itemize}
\item[•]\textbf{Attori}: Sviluppatore;
\item[•]\textbf{Descrizione}: Lo sviluppatore seleziona da un elenco prestabilito il formato dell'archivio compresso con cui vuole che i dati d'interesse siano compressi;
\item[•]\textbf{Precondizione}: Il formato dell'archivio contenente i dati d'interesse non è definito;
\item[•]\textbf{Postcondizione}: Il formato dell'archivio contenente i dati d'interesse è stato definito;
\item[•]\textbf{Flusso degli eventi principale}:  Lo sviluppatore seleziona dall'elenco prestabilito un'opzione che determina il formato dell'archivio che contiene i file relativi ai dati precedentemente scelti.
\end{itemize}

\subsubsection{UC 4.4.2 - Ottenimento file}
\begin{itemize}
\item[•]\textbf{Attori}: Sviluppatore;
\item[•]\textbf{Descrizione}: Lo sviluppatore riceve un archivio dati contenente i dati, salvati in file aventi estensione scelta dal programmatore;
\item[•]\textbf{Precondizione}: I contenuti di interesse, il formato del file e il formato dell'archivio sono stati scelti;
\item[•]\textbf{Postcondizione}: Lo sviluppatore ottiene il file richiesto e si procede al download di questo;
\item[•]\textbf{Flusso degli eventi principale}:  Lo sviluppatore seleziona dall'elenco prestabilito un'opzione che determina il formato dell'archivio che contiene i file relativi ai dati precedentemente scelti.
\end{itemize}

\subsubsection{UC 4.5 - Creazione modello}
\begin{figure}[H]
\centering
\includegraphics[width=17cm]{img/UC450.png} 
\caption{Caso d'uso 4.5}\label{fig:450}
\end{figure}
\begin{itemize}
\item[•]\textbf{Attori}: Sviluppatore;
\item[•]\textbf{Descrizione}: Lo sviluppatore crea un nuovo modello;
\item[•]\textbf{Precondizione}: Lo sviluppatore non ha creato il modello;
\item[•]\textbf{Postcondizione}: Lo sviluppatore ha creato un nuovo modello;
\item[•]\textbf{Flusso degli eventi principale}:  
\begin{enumerate}
	\item UC 4.5.1 - Scelta parametri funzione;
	\item UC 4.5.2 - Inserimento nome modello;
	\item UC 4.5.3 - Salvataggio modello.
\end{enumerate}
\end{itemize}

\subsubsection{UC 4.5.1 - Scelta parametri funzione}
\begin{itemize}
\item[•]\textbf{Attori}: Sviluppatore;
\item[•]\textbf{Descrizione}: Lo sviluppatore sceglie i parametri della funzione per la creazione di un nuovo modello con i relativi valori associati;
\item[•]\textbf{Precondizione}: Lo sviluppatore non ha scelto i parametri da inserire;
\item[•]\textbf{Postcondizione}: Lo sviluppatore ha inserito i parametri per la creazione di un nuovo modello;
\item[•]\textbf{Flusso degli eventi principale}:  Lo sviluppatore inserisce tutti i parametri necessari che verranno impiegati nella realizzazione del modello.
\item[•] \textbf{Estensioni}: 
\begin{enumerate}
	\item UC 4.5.4 - Valore parametro non valido.
\end{enumerate}
\end{itemize}

\subsubsection{UC 4.5.2 - Inserimento nome modello}
\begin{itemize}
\item[•]\textbf{Attori}: Sviluppatore;
\item[•]\textbf{Descrizione}: Lo sviluppatore inserisce il nome per il modello che intende salvare;
\item[•]\textbf{Precondizione}: Lo sviluppatore non ha inserito il nome del modello;
\item[•]\textbf{Postcondizione}: Lo sviluppatore ha inserito il nome del modello;
\item[•]\textbf{Flusso degli eventi principale}:  Lo sviluppatore inserisce una stringa che definisce il nome del modello che verrà memorizzato.
\item[•] \textbf{Estensioni}: 
\begin{enumerate}
	\item UC 4.5.5 - Visualizzazione messaggio errore nome duplicato.
\end{enumerate}
\end{itemize}

\subsubsection{UC 4.5.3 - Salvataggio modello}
\begin{itemize}
\item[•]\textbf{Attori}: Sviluppatore;
\item[•]\textbf{Descrizione}: Lo sviluppatore salva il modello di cui ha definito il nome;
\item[•]\textbf{Precondizione}: Lo sviluppatore ha inserito il nome del modello;
\item[•]\textbf{Postcondizione}: Lo sviluppatore ha salvato il modello;
\item[•]\textbf{Flusso degli eventi principale}:  Lo sviluppatore seleziona l'opzione relativa al salvataggio che comporta il salvataggio del modello.
\end{itemize}

\subsubsection{UC 4.5.5 - Visualizzazione messaggio errore nome duplicato}
\begin{itemize}
\item[•]\textbf{Attori}: Sviluppatore;
\item[•]\textbf{Descrizione}: Lo sviluppatore visualizza un messaggio di errore relativo all'inserimento di un modello avente lo stesso nome di uno già inserito;
\item[•]\textbf{Precondizione}: Lo sviluppatore ha inserito il nome del modello;
\item[•]\textbf{Postcondizione}: Lo sviluppatore visualizza il messaggio di errore e inserisce un nuovo nome;
\item[•]\textbf{Flusso degli eventi principale}:  Lo sviluppatore visualizza il messaggio di errore e può inserire una stringa che definisce il nome del modello che verrà memorizzato.
\end{itemize}

\subsubsection{UC 4.6 - Visualizzazione elenco modelli}
\begin{figure}[H]
\centering
\includegraphics[width=17cm, height=7cm]{img/UC460.png} 
\caption{Caso d'uso 4.6}\label{fig:460}
\end{figure}
\begin{itemize}
\item[•]\textbf{Attori}: Sviluppatore;
\item[•]\textbf{Descrizione}: Lo sviluppatore visualizza l'elenco dei modelli realizzati per lo svolgimento degli esercizi;
\item[•]\textbf{Precondizione}: Lo sviluppatore si è autenticato nel sistema;
\item[•]\textbf{Postcondizione}: Lo sviluppatore visualizza l'elenco dei modelli;
\item[•]\textbf{Flusso degli eventi principale}:  Lo sviluppatore seleziona la voce del menù relativa ai modelli e ne visualizza l'elenco.
\end{itemize}

\subsubsection{UC 4.7 - Scaricamento modello}
\begin{figure}[H]
\centering
\includegraphics[width=17cm, height=8cm]{img/UC470.png} 
\caption{Caso d'uso 4.7}\label{fig:470}
\end{figure}
\begin{itemize}
\item[•]\textbf{Attori}: Sviluppatore;
\item[•]\textbf{Descrizione}: Lo sviluppatore ottiene un file contenente il modello precedentemente creato;
\item[•]\textbf{Precondizione}: Lo sviluppatore visualizza l'elenco dei modelli (UC 4.6);
\item[•]\textbf{Postcondizione}: Lo sviluppatore ottiene un file contenente il modello selezionato;
\item[•]\textbf{Flusso degli eventi principale}:  Lo sviluppatore seleziona l'opzione di scaricamento sul nome del modello desiderato e ottiene un file contenete il modello.
\end{itemize}
