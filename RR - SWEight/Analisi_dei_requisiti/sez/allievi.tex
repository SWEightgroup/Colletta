\subsection{Allievi}
\subsubsection{Panoramica allievo}

\begin{figure}[H]
\centering
\includegraphics[width=17cm]{img/PanoramicaAllievi.png} 
\caption{Panoramica allievi}\label{fig:31}
\end{figure}


\subsubsection{UC 3.2 - Visualizzazione della dashboard}
\begin{figure}[H]
\centering
\includegraphics[width=17cm]{img/UC32.png} 
\caption{Caso d'uso UC 3.2}\label{fig:32}
\end{figure}
\begin{itemize}
\item[•]\textbf{Attori}: Allievo;
\item[•]\textbf{Descrizione}: L'allievo apre la propria dashboard;
\item[•]\textbf{Precondizione}: L'allievo si è autenticato;
\item[•]\textbf{Postcondizione}: L'allievo visualizza la dashboard e può vederne tutte le componenti;
\item[•]\textbf{Flusso degli eventi principale}:
\begin{enumerate}
\item UC 3.2.1 - Visualizzazione progressi;
\item UC 3.2.2 - Visualizzazione traguardi;
\item UC 3.2.3 - Visualizzazione valutazioni;
\item UC 3.2.4 - Inserimento insegnante preferito;
\item UC 3.2.5 - Visualizzazione insegnanti preferiti;
\item UC 3.2.6 - Selezione lingua esercizi.
\end{enumerate}
\end{itemize}

\subsubsection{UC 3.2.1 - Visualizzazione progressi}
\begin{itemize}
\item[•]\textbf{Attori}: Allievo;
\item[•]\textbf{Descrizione}: L'allievo visualizza i suoi progressi: numero di esercizi svolti, corretti ed errati;
\item[•]\textbf{Precondizione}: L'allievo ha visualizzato la propria dashboard;
\item[•]\textbf{Postcondizione}: L'allievo ha visualizzato i propri progressi.
\end{itemize}

\subsubsection{UC 3.2.2 - Visualizzazione traguardi}
\begin{itemize}
\item[•]\textbf{Attori}: Allievo;
\item[•]\textbf{Descrizione}: L'allievo visualizza i traguardi raggiunti;
\item[•]\textbf{Precondizione}: L'allievo visualizza la propria dashboard;
\item[•]\textbf{Postcondizione}: L'allievo visualizza tutti i traguardi raggiunti.
\end{itemize}

\subsubsection{UC 3.2.3 - Visualizzazione valutazioni}
\begin{itemize}
\item[•]\textbf{Attori}: Allievo;
\item[•]\textbf{Descrizione}: L'allievo visualizza le valutazioni di tutti gli esercizi svolti, sia esercizi assegnati che svolti indipendentemente;
\item[•]\textbf{Precondizione}: L'allievo visualizza la propria dashboard;
\item[•]\textbf{Postcondizione}: L'allievo visualizza tutte le valutazioni ricevute.
\end{itemize}

\subsubsection{UC 3.2.4 - Inserimento insegnante preferito}
\begin{itemize}
\item[•]\textbf{Attori}: Allievo;
\item[•]\textbf{Descrizione}: L'allievo inserisce il nome utente di un insegnante da prediligere quando riceve la correzione di un esercizio;
\item[•]\textbf{Precondizione}: L'allievo visualizza la propria dashboard;
\item[•]\textbf{Postcondizione}: L'allievo inserisce un insegnante preferito.
\end{itemize}

\subsubsection{UC 3.2.5 - Visualizzazione insegnanti preferiti}
\begin{figure}[H]
	\centering
	\includegraphics[width=17cm]{img/UC325.png} 
	\caption{Caso d'uso UC 3.2.5}\label{fig:325}
\end{figure}
\begin{itemize}
	\item[•]\textbf{Attori}: Allievo;
	\item[•]\textbf{Descrizione}: L'allievo visualizza la lista degli insegnanti preferiti che ha inserito;
	\item[•]\textbf{Precondizione}: L'allievo visualizza la propria dashboard;
	\item[•]\textbf{Postcondizione}: L'allievo visualizza gli insegnanti preferiti;
	\item[•]\textbf{Flusso degli eventi}:
	\begin{enumerate}
		\item UC 3.2.5.1 - Rimozione insegnante preferito.
	\end{enumerate}
\end{itemize}

\subsubsection{UC 3.2.5.1 - Rimozione insegnante preferito}
\begin{itemize}
	\item[•]\textbf{Attori}: Allievo;
	\item[•]\textbf{Descrizione}: L'allievo seleziona dalla lista visualizzata un insegnante da rimuovere dagli insegnanti preferiti;
	\item[•]\textbf{Precondizione}: L'allievo ha visualizzato la lista di insegnanti preferiti;
	\item[•]\textbf{Postcondizione}: L'allievo ha rimosso un insegnante preferito dalla lista.
\end{itemize}

\subsubsection{UC 3.2.6 - Selezione lingua esercizi}
\begin{itemize}
	\item[•]\textbf{Attori}: Allievo;
	\item[•]\textbf{Descrizione}: L'allievo seleziona la lingua in cui vuole svolgere gli esercizi e visualizzare le soluzioni;
	\item[•]\textbf{Precondizione}: L'allievo visualizza la propria dashboard;
	\item[•]\textbf{Postcondizione}: L'allievo ha selezionato una lingua per gli esercizi.
\end{itemize}

\subsubsection{UC 3.3 - Ricerca esercizio}
\begin{figure}[H]
\centering
\includegraphics[width=17cm]{img/UC33.png} 
\caption{Caso d'uso UC 3.3}\label{fig:33}
\end{figure}
\begin{itemize}
\item[•]\textbf{Attori}: Allievo;
\item[•]\textbf{Descrizione}: L'allievo ricerca un esercizio attraverso frasi suggerite dal sistema o attraverso l'inserimento di una frase libera;
\item[•]\textbf{Precondizione}: L'allievo si è autenticato;
\item[•]\textbf{Postcondizione}: L'allievo può navigare una lista di esercizi consigliati o inserire una frase;
\item[•]\textbf{Flusso degli eventi}:
\begin{enumerate}
\item UC 3.3.1 - Selezione esercizio;
\item UC 3.3.2 - Inserimento frase libera.
\end{enumerate}
\end{itemize}

\subsubsection{UC 3.3.1 - Selezione esercizio}
\begin{figure}
\centering
\includegraphics[width=17cm]{img/UC331.png} 
\caption{Caso d'uso UC 3.3.1}\label{fig:331}
\end{figure}

\begin{itemize}
\item[•]\textbf{Attori}: Allievo;
\item[•]\textbf{Descrizione}: L'allievo seleziona un esercizio nella lista delle frasi consigliate;
\item[•]\textbf{Precondizione}: L'allievo visualizza una lista di esercizi;
\item[•]\textbf{Postcondizione}: L'allievo seleziona un esercizio;
\item[•]\textbf{Flusso degli eventi}:
\begin{enumerate}
\item UC 3.3.1.1 - Selezione filtro insegnante.
\end{enumerate}
\end{itemize}


\subsubsection{UC 3.3.2 - Inserimento frase libera}
\begin{itemize}
	\item[•]\textbf{Attori}: Allievo;
	\item[•]\textbf{Descrizione}: L'allievo inserisce la propria frase in modo da ricevere l'analisi grammaticale di essa;
	\item[•]\textbf{Precondizione}: Il sistema offre la possibilità di inserire una frase;
	\item[•]\textbf{Postcondizione}: L'allievo inserisce una frase;
	\item[•]\textbf{Flusso degli eventi}:
	\begin{enumerate}
		\item UC 3.4.3 - Visualizzazione soluzione. % io metterei visualizza 
	\end{enumerate}
	\item[•]\textbf{Estensioni}:
	\begin{enumerate}
		\item UC 3.3.2.1 - Visualizzazione errore inserimento frase. %visualizza 
	\end{enumerate}
\end{itemize}


\subsubsection{UC 3.3.1.1 - Selezione filtro insegnante}

\begin{itemize}
\item[•]\textbf{Attori}: Allievo;
\item[•]\textbf{Descrizione}: L'allievo seleziona l'insegnante da cui vuole ricevere la correzione dell'esercizio, se nessun insegnante ha predisposto quella frase verrà utilizzato il sistema di correzione automatico. %Quando è possibile l'insegnante preferito viene selezionato automaticamente;
\item[•]\textbf{Precondizione}: L'allievo seleziona un esercizio;
\item[•]\textbf{Postcondizione}: L'allievo seleziona da chi vuole ricevere la correzione.
\end{itemize}

% anche qui metterei visualizza e non visualizzazione 

\subsubsection{UC 3.3.2.1 - Visualizzazione errore inserimento frase}
\begin{itemize}
\item[•]\textbf{Attori}: Allievo, Libreria di pos-tagging;
\item[•]\textbf{Descrizione}: La frase inserita non è accettata dalla libreria di pos-tagging. L'allievo visualizza un errore e può inserire una nuova frase;
\item[•]\textbf{Precondizione}: L'allievo ha provato ad inserire una frase;
\item[•]\textbf{Postcondizione}: L'allievo ha visualizzato il messaggio di errore e può inserire una nuova frase.
\end{itemize}

\subsubsection{UC 3.4 - Svolgimento esercizio}

\begin{figure}[H]
\centering
\includegraphics[width=17cm]{img/UC34.png} 
\caption{Caso d'uso 3.4}\label{fig:34}
\end{figure}

\begin{itemize}
\item[•]\textbf{Attori}: Allievo;
\item[•]\textbf{Descrizione}: L'allievo può svolgere l'esercizio scegliendo le classi grammaticali per ciascuna parola da una apposita lista;
\item[•]\textbf{Precondizione}: L'allievo seleziona un esercizio;
\item[•]\textbf{Postcondizione}: L'allievo svolge un esercizio;
\item[•]\textbf{Flusso degli eventi}:
\begin{enumerate}
\item UC 3.4.1 - Classificazione parola;
\item UC 3.4.3 - Visualizzazione soluzione.  %visualizza 
\end{enumerate}
\item[•]\textbf{Estensioni}:
\begin{enumerate}
\item UC 3.4.2 - Interruzione svolgimento esercizio.
\end{enumerate}
\end{itemize}

\subsubsection{UC 3.4.1 - Classificazione parola}
\begin{itemize}
\item[•]\textbf{Attori}: Allievo;
\item[•]\textbf{Descrizione}: L'allievo seleziona la classe grammaticale di una parola da una lista predefinita;
\item[•]\textbf{Precondizione}: L'allievo inizia a svolgere un esercizio;
\item[•]\textbf{Postcondizione}: L'allievo seleziona la classe grammaticale di una parola.
\end{itemize}

\subsubsection{UC 3.4.2 - Interruzione svolgimento esercizio}
\begin{itemize}
\item[•]\textbf{Attori}: Allievo;
\item[•]\textbf{Descrizione}: L'allievo interrompe  l'esercizio, scartando i dati inseriti fino a quel momento, e ritorna nella sezione di ricerca di un esercizio;
\item[•]\textbf{Precondizione}: L'allievo inizia a svolgere un esercizio;
\item[•]\textbf{Postcondizione}: L'allievo interrompel'esercizio, torna nella sezione di ricerca di un esercizio.
\end{itemize}

\subsubsection{UC 3.4.3 - Visualizzazione soluzione}

\begin{figure}[H]
\centering
\includegraphics[width=17cm]{img/UC343.png} 
\caption{Caso d'uso UC 3.4.3}\label{fig:343}
\end{figure}

\begin{itemize}
\item[•]\textbf{Attori}: Allievo, Libreria di pos-tagging;
\item[•]\textbf{Descrizione}: L'allievo visualizza la correzione secondo l'insegnante selezionato in precedenza. Se era stata inserita una frase libera la correzione viene eseguita dal sistema di correzione automatico;
\item[•]\textbf{Precondizione}: L'allievo ha svolto un esercizio oppure ha inserito una frase libera;
\item[•]\textbf{Postcondizione}: L'allievo visualizza la soluzione dell'esercizio;
\item[•]\textbf{Flusso degli eventi}:
\begin{enumerate}
\item UC 3.4.3.1 - Visualizzazione valutazione.  % visualizza 
\end{enumerate}
\end{itemize}

\subsubsection{UC 3.4.3.1 - Visualizzazione valutazione esercizio}  %visualizza 

\begin{itemize}
\item[•]\textbf{Attori}: Allievo;
\item[•]\textbf{Descrizione}: L'allievo riceve una valutazione in base al numero di parole classificate correttamente. Se sono presenti più soluzioni proposte dallo stesso insegnante viene presa la soluzione che porta alla valutazione più alta.
\item[•]\textbf{Precondizione}: L'allievo visualizza la soluzione dell'esercizio svolto;
\item[•]\textbf{Postcondizione}: L'allievo visualizza una valutazione sull'esercizio svolto.
\end{itemize}


