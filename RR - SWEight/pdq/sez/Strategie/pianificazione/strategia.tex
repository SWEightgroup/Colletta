Lo sviluppo del prodotto, come descritto nel \PdP , avviene secondo il {modello incrementale}\ped{G}. \'E utile distinguere due tipi di incremento:
\begin{itemize}
	\item \textbf{Programmato:} prefissato nel calendario;
	\item \textbf{Non programmato} insorge in seguito ad attività di verifica, sia manuali che automatiche. Può essere la correzione di un {bug}\ped{G}, di un errore errore ortografico, o, in più in generale, di una problematica in un prodotto.
\end{itemize}
Il mancato svolgimento di quest'ultimo tipo di incremento, può portare alla permanenza di problematiche nel prodotto a scapito della qualità. A tale fine, è necessario focalizzarsi su:
\begin{itemize}
	\item \textbf{Prevenzione:} misure atte ad evitare l'insorgenza di problematiche;
	\item \textbf{Individuazione:} misure atte a individuare tempestivamente possibili problematiche.
\end{itemize}
Tali misure possono essere {efficienti}\ped{G} ed {efficaci}\ped{G} solo se propriamente automatizzate: le tecniche adottate dal gruppo \gruppo \space e la loro evoluzione sono reperibili nelle \NdP . 
 
