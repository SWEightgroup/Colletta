Alcune metriche per il software sono più adatte per alcuni linguaggi di programmazione e meno per altri. Come indicato nel \PdP , il gruppo \gruppo \space sceglierà quali linguaggi usare nel periodo di Progettazione Architetturale; pertanto, le metriche presenti in questa sezione non sono da considerarsi complete o definitive.\newline
Per alcune metriche, può mancare un'indicazione di valori accettabili e ottimali: ciò significa che il team si riserva di definirli in futuri incrementi.

%\begin{table}[H]
	\rowcolors{2}{greyROwSWEight}{white}
	\renewcommand{\arraystretch}{1.5}
	%\centering
	\begin{longtable}{C{4cm} C{5cm} c c}
	
	\rowcolor{greySWEight}
	\textcolor{white}{\textbf{Nome}} &
	\textcolor{white}{\textbf{Descrizione}} &
	\textcolor{white}{\textbf{Accettabile}} &
	\textcolor{white}{\textbf{Ottimale}}\\
	\endhead
	%%% ENTRY
	\textbf{Number di metodi} &
	Numero medio di metodi nelle classi di un package &
	[3 , 9] &
	[3 , 7]\\
	
	%%% ENTRY
	\textbf{Numero di parametri} &
	Numero di parametri passati a un metodo &
	[0 , 8] &
	[0 , 5]\\
	
	%%% ENTRY
	\textbf{Funzioni di interfaccia per package} &
	Numero di metodi esposti da un package &
	[0 , 20] &
	[0 , 10]\\
	
	%%% ENTRY
	\textbf{Complessità ciclomatica} &
	Misura della complessità del codice &
	[0 , 17] &
	[0 , 10]\\
	
	%%% ENTRY
	\textbf{Campi dati per classe} &
	Numero di campi dati per classe &
	[0 , 15] &
	[0 , 10]\\
	
	%%% ENTRY
	\textbf{Commenti per linee di codice} &
	Rapporto fra linee di commento e linee di codice &
	[10\%, 100\%] &
	[15\% , 100\%]\\
	
	%%% ENTRY
	\textbf{Code coverage} &
	Percentuale delle linee di codice coperte dai test &
	&
	\\
	
	%%% ENTRY
	\textbf{Superamento test} &
	Percentuale dei test superati &
	[70\%, 100\%]&
	[95\%, 100\%]\\
	
	%%% ENTRY
	\textbf{Soddisfacimento requisiti obbligatori} &
	Percentuale dei requisiti obbligatori soddisfatti &
	[100\%, 100\%]&
	[100\%, 100\%]\\
	
	\rowcolor{white}
	\caption{Metriche sul software}\\	
	\end{longtable}
	
