Le metriche presentate in questa sezione monitorano lo stato dei processi del progetto analizzando l'uso che essi fanno di tempo e risorse finanziarie. Sono particolarmente utili per il \Res , che può quindi decidere di apportare modifiche alla pianificazione quando necessario.

\paragraph{Schedule Variance}\mbox{}\\[0,3cm]
La Schedule Variance indica se una certa attività o processo è in anticipo, in pari, o in ritardo rispetto alla data di scadenza prevista.
Se $SV < 0$ significa che l'attività o il processo è in pari o in anticipo, invece, se $SV \geq 0$ significa che l'attività è in ritardo.\\[0,2cm]
La Schedule Variance è calcolata come segue:
\[
SV = \textit{data conclusione effettiva} - \textit{data conclusione pianficata}
\]
\textbf{Parametri adottati:}
\begin{itemize}
	\item Range accettabile: $(-\infty , 3]$;
	\item Range ottimale: $(-\infty , 0]$.
\end{itemize}

\paragraph{Budget Variance}\mbox{}\\[0,3cm]
La Budget Variance misura, ad una determinata data, di quanto il costo effettivo in termini prettamente economici del progetto si discosta dal costo preventivato. Se $BV \geq 0\%$ significa che il progetto sta spendendo le risorse più velocemente di quanto pianificato.\\[0,2cm]
La Budget Variance è calcolata come segue:
\[
BV = \frac{\textit{costo effettivo} - \textit{costo preventivato}}{\textit{costo preventivato}}
\]
\textbf{Parametri adottati:}
\begin{itemize}
	\item Range accettabile: $(-\infty , 9\%]$;
	\item Range ottimale: $(-\infty , 1\%]$.
\end{itemize}