\section{Capitolato C2}
\subsection{Informazioni sul Capitolato}
\begin{itemize}
	\item \textbf{Nome:} Colletta;
	\item \textbf{Proponente}: Mivoq S.R.L.;
	\item \textbf{Committenti}: Prof. Tullio Vardanega, Prof. Riccardo Cardin.
\end{itemize}

\subsection{Descrizione}
Il capitolato ha per oggetto lo sviluppo di una piattaforma 
collaborativa, web o mobile, 
di raccolta dati, in cui gli utenti possano predisporre e svolgere piccoli esercizi di analisi grammaticale. Gli insegnanti sono agevolati nella realizzazione e nella correzione dei compiti grazie ad una libreria che esegue l’analisi grammaticale automaticamente.
Gli esercizi corretti, revisionati dagli insegnanti, devono essere raccolti e catalogati in una base di dati per fornire, a ricercatori e sviluppatori di accedervi per migliorare degli algoritmi di apprendimento automatico supervisionato che implementino automaticamente l'analisi grammaticale di frasi. 

\subsection{Dominio Applicativo}
Il dominio applicativo a cui il capitolato fa riferimento è quello della raccolta dati, questa viene realizzata implicitamente fornendo un servizio 
di analisi grammaticale all’utente. 

\subsection{Dominio Tecnologico}
Per quanto riguarda la realizzazione dell’applicativo si farà uso di:

\begin{itemize}
\item[•] \texttt{Firebase}: come sistema cloud per l'immagazzinamento dei dati;

\item[•] \texttt{Freeling} o \texttt{Hunpos}: librerie di pos-tagging\ped{G};

\item[•] \texttt{PHP} o \texttt{Java}: come linguaggi di scripting lato server;

\item[•] \texttt{JavaScript}: linguaggio di scripting lato client, impiegando librerie quali \texttt{AngularJS} o \texttt{jQuery};

\item[•] \texttt{Foundation} o \texttt{Twitter}: framework per la realizzazione della parte grafica dell’applicativo;

\end{itemize}

\subsection{Considerazioni del gruppo}
Durante l’analisi del capitolato è emersa la seguente criticità:
la quantità di attori interessati e le funzionalità messe a disposizione per ognuno di essi, potrebbe portare ad un eccedere del limite di ore proposte e del budget messo a disposizione.

\subsection{Valutazione Finale}

Il capitolato d’appalto presenta alcune caratteristiche che lo hanno portato ad essere
la scelta finale del gruppo:
\begin{itemize}
\item[•] acquisizione di esperienza nello sviluppo di applicazioni web o mobile, con l’utilizzo di tecnologie;
ampiamente richieste nell’ambito lavorativo;
\item[•] interesse nel dominio tecnologico;
\item[•] disponibilità da parte dell'azienda proponente a chiarimenti e incontri futuri; 

\end{itemize}

