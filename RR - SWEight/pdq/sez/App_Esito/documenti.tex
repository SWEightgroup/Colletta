\paragraph{Indice di Gulpease}\mbox{}\\[0,3cm]
\begin{table}[H]
	\centering
	\begin{tabular}{cccc}
	\rowcolor{greySWEight}
	\textcolor{white}{\textbf{Documento}} & 
	\textcolor{white}{\textbf{Abbreviazione}} &
	\textcolor{white}{\textbf{Valore Indice}}&
	\textcolor{white}{\textbf{Riscontro}}\\
	
	\textbf{Analisi dei Requisiti} & ADR & 58,84 & \textcolor{ForestGreen}{Ottimale} \\
	\textbf{Glossario} & GLO & 51,12 & \textcolor{YellowOrange}{Accettabile} \\
	\textbf{Piano di Progetto} & PDQ & 54,89 & \textcolor{YellowOrange}{Accettabile} \\
	\textbf{Piano di Qualifica} & PDP & 56,67 & \textcolor{ForestGreen}{Ottimale} \\
	\textbf{Norme di Progetto} & NDP & 55,3 & \textcolor{ForestGreen}{Ottimale} \\
	\textbf{Studio di Fattibilità} & SDF & 56,77 & \textcolor{ForestGreen}{Ottimale} \\

	\end{tabular}
	\caption{Indice di Gulpease nel periodo di Analisi}
\end{table}
\begin{figure}[H]
	\includegraphics[width=1\linewidth]{sez/App_Esito/graph/AN_Gulp.pdf}
	\caption{Indice di Gulpease nel periodo di Analisi}
\end{figure}

%\textcolor{ForestGreen}{Ottimale}
%\textcolor{YellowOrange}{Accettabile}
%\textcolor{BrickRed}{Non accettabile}