Il modello ISO/IEC 1554, meglio conosciuto come SPICE (Software Process Improvement and Capability Determination), è lo standard di riferimento per valutare in modo oggettivo la qualità dei processi nello sviluppo del software. \newline
Sono definiti:
\begin{itemize}
	\item Una serie di sei livelli utilizzati per classificare la capacità\ped{G} e la maturità\ped{G} del processo software. Ogni livello è caratterizzato dal soddisfacimento degli attributi associati:
	\begin{enumerate} 
	\item \textbf{Incompleto}: il processo non è stato implementato o non ha raggiunto il successo desiderato;
	\item \textbf{Eseguito}: il processo è implementato e ha realizzato il suo obiettivo (conformità);
		Attributi:
		\begin{itemize}
	    	\item Process performance;
    		\end{itemize}
	\item \textbf{Gestito}: il processo è gestito e il prodotto finale è verificato, controllato e manutenuto (affidabilità);
		Attributi:
		\begin{itemize}	
	    	\item Performance management;
	    	\item Work product management;
		\end{itemize}
    	\item \textbf{Stabilito}: il processo è basato sullo standard di processo (standardizzazione);
		Attributi:
		\begin{itemize}
		\item Process definition;
		\item Process deployment;
		\end{itemize}
    	\item \textbf{Predicibile}: il processo è consistente e rispetta limiti definiti (strategico);
		Attributi:
		\begin{itemize}
		\item Process measurement;
		\item Process control;
		\end{itemize}
    	\item \textbf{Ottimizzato}: il processo segue un  miglioramento continuo per rispettare tutti gli obiettivi di progetto;
		Attributi:
		\begin{itemize}
		\item Process innovation;
		\item Process optimization;
		\end{itemize}
    	\end{enumerate}
	
	Ogni attributo riceve una valutazione nella seguente scala, andando a definire il rispettivo livello di capacità del processo:
	\begin{itemize}
	    \item \textbf{N: }non raggiunto (0 - 15\%);
	    \item \textbf{P: }parzialmente raggiunto (>15\% - 50\%);
	    \item \textbf{L: }largamente raggiunto (>50\%- 85\%);
	    \item \textbf{F: }pienamente raggiunto (>85\% - 100\%);
	\end{itemize}
	\item Delle linee guida per effettuare delle \textbf{stime}, eseguite tramite:
		\begin{itemize}
			\item \textbf{Processi di misurazione}, descritti nel \PdP ;
			\item \textbf{Modello di misurazione}, descritto in questo documento;
			\item \textbf{Strumenti di misurazione}, descritti nelle \NdP ;
		\end{itemize}
	\item Una serie di \textbf{competenze} che chi effettua misurazioni deve possedere. La mancanza di esperienza degli elementi del gruppo \gruppo , fa sì che nessun membro possieda queste skill, rendendo così impossibile la piena adesione allo standard. Tuttavia, ogni componente è chiamato a studiare SPICE e a applicare al meglio le indicazioni descritte in questo documento e nelle \NdP , al fine di perseguire un livello di qualità accettabile.
	
\end{itemize}
