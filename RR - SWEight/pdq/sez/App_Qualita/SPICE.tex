Il modello ISO/IEC 1554, meglio conosciuto come SPICE (Software Process Improvement and Capability Determination), è lo standard di riferimento per valutare in modo oggettivo la qualità dei processi nello sviluppo del software. \newline
Sono definiti:
\begin{itemize}
	\item Una serie di \textbf{attributi} per ogni processo, che ne vanno a determinare la sua capability (capacità):
	\begin{itemize}
	    \item Process performance;
	    \item Performance management;
	    \item Work product management;
	    \item Process definition;
	    \item Process deployment;
	    \item Process measurement;
	    \item Process control;
	    \item Process innovation;
	    \item Process optimization;
	\end{itemize}
	
	Ognuno di questi attributi riceve una valutazione nella seguente scala:
	\begin{itemize}
	    \item \textbf{N: }Non raggiunto (0 - 15\%);
	    \item \textbf{P: }Parzialmente raggiunto (>15\% - 50\%);
	    \item \textbf{L: }Largamente raggiunto (>50\%- 85\%);
	    \item \textbf{F: }Pienamente raggiunto (>85\% - 100\%);
	\end{itemize}
	\item Sei \textbf{livelli di capacità} dei processi:
		\begin{enumerate}[start=0]
			\item Incompleto;
			\item Eseguito;
			\item Gestito;
			\item Stabilito;
			\item Predicibile;
			\item Ottimizzato;
		\end{enumerate}
	\item Delle linee guida per effettuare delle \textbf{stime}, eseguite tramite:
		\begin{itemize}
			\item \textbf{Processi di misurazione}, descritti nel \PdP ;
			\item \textbf{Modello di misurazione}, descritto in questo documento;
			\item \textbf{Strumenti di misurazione}, descritti nelle \NdP ;
		\end{itemize}
	\item Una serie di \textbf{competenze} che chi effettua misurazioni deve possedere. La mancanza di esperienza degli elementi del gruppo \gruppo , fa sì che nessun membro possieda queste skill, rendendo così impossibile la piena adesione allo standard. Tuttavia, ogni componente è chiamato a studiare SPICE e a applicare al meglio le indicazioni descritte in questo documento e nelle \NdP , al fine di perseguire un livello di qualità accettabile.
	
\end{itemize}
