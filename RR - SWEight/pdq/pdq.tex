%generare il pdf con il comando: pdflatex main.tex
\documentclass[a4paper, oneside, openany, dvipsnames, table]{article}
\usepackage{../template/SWEightStyle}
\newcommand{\Titolo}{Verbale Riunione 2018-12-12}

\newcommand{\Gruppo}{SWEight}

\newcommand{\ACapoRedazione}{Francesco Magarotto}

\newcommand{\Verifica}{Francesco Corti}

\newcommand{\Approvazione}{Sebastiano Caccaro}

\newcommand{\Distribuzione}{Vardanega Tullio \newline Cardin Riccardo \newline Gruppo SWEight}

\newcommand{\Uso}{Interno}

\newcommand{\NomeProgetto}{Colletta}

\newcommand{\Mail}{SWEightGroup@gmail.com}

\newcommand{\DescrizioneDoc}{Questo documento si occupa di riportare quanto discusso nella riunione del 12-12-2018}


\begin{document}
\copertina{}
\definecolor{greySWEight}{RGB}{255, 71, 87}
\definecolor{greyROwSWEight}{RGB}{234, 234, 234}

\section*{Registro delle modifiche}
{
	\rowcolors{2}{greyROwSWEight}{white}
	\renewcommand{\arraystretch}{1.5}
	\centering
	\begin{longtable}{ c c  C{4cm}  c  c }
		
		\rowcolor{greySWEight}
		\textcolor{white}{\textbf{Versione}} & \textcolor{white}{\textbf{Data}} & \textcolor{white}{\textbf{Descrizione}} & \textcolor{white}{\textbf{Nominativo}} & \textcolor{white}{\textbf{Ruolo}}\\
		
		1.0.2 & 2019-03-02 & Aggiunti nuovi termini del documento Piano di Progetto & Isachi Gheorghe &\reda{}\\
		
		1.0.1 & 2019-02-23 & Verifica del documento &  Francesco Corti & \ver{}\\
		
		1.0.1 & 2019-02-20 & Aggiunti nuovi termini del documento Norme di Progetto & Isachi Gheorghe &\reda{}\\
		
		1.0.0 & 2019-01-09 & Approvazione & Sebastiano Caccaro & \Res{}\\
						
		0.1.1 & 2019-01-08 & Verifica del documento & Bacco Alberto & \ver{}\\
		
		0.1.1 & 2019-01-04 & Aggiunti termini del documento Norme di Progetto & Isachi Gheorghe &\reda{}\\
		
		0.1.0 & 2019-01-01 & Aggiunti termini del documento Analisi dei Requisiti & Isachi Gheorghe &\reda{}\\
		
		0.0.4 & 2018-12-29 & Verifica del documento & Bacco Alberto & \ver{}\\
				
		0.0.4 & 2018-12-27 & Aggiunti termini del documento Piano di Qualifica & Isachi Gheorghe &\reda{}\\
				
		0.0.3 & 2018-12-26 &Aggiunti termini del documento Piano di Progetto & Isachi Gheorghe & \reda{}\\
				
		0.0.2 & 2018-12-17 & Aggiunti termini del documento Studio di Fattibilità & Isachi Gheorghe &\reda{}\\
		
		0.0.1 & 2018-12-15 & Scheletro del glossario & Damien Ciagola & \reda{}\\
		
	\end{longtable}

}
\newpage
\tableofcontents
\newpage


\section{Introduzione}
	\subsection{Scopo del documento}
		In questo documento è illustrata la {strategia}\ped{G} di {verifica}\ped{G} e {validazione}\ped{G} del gruppo \gruppo . Tale strategia è fondamentale per dare una misurazione oggettiva e quantificabile del livello di {qualità}\ped{G} di quanto viene prodotto. \newline
Ciò è vantaggioso sia per il gruppo \gruppo , che può facilmente individuare difetti durante lo svolgimento del progetto, sia per il {committente}\ped{G}, che può costantemente monitorare la qualità del prodotto in base a criteri oggettivi e prestabiliti.


	\subsection{Scopo del prodotto}
		Lo scopo del progetto è realizzare una piattaforma collaborativa di raccolta dati in cui gli utenti possano predisporre
e/o svolgere piccoli esercizi di grammatica (per esempio esercizi di analisi grammaticale) e i dati raccolti
siano relativi sia agli esercizi predisposti che al loro svolgimento da parte degli utenti. I dati raccolti devono
essere utilizzabili da sviluppatori e ricercatori al fine di insegnare ad un elaboratore a svolgere i medesimi esercizi
mediante tecniche di apprendimento automatico supervisionato.

%Questa parte è un copia-incolla cafonissimo dal capitolato della mivoc
%Siccome è proprio quello che vogliono, non mi sembrava il caso di andare a modifcarla
	\subsection{Glossario}
		Nel documento è possibile incontrare termini tecnici, i quali potrebbero non essere immediatamente chiari al lettore. Per disambiguarne il significato, essi sono stati marcati con una \ped{G} a pedice e la loro definizione è reperibile nel glossario fornito separatamente.
	\subsection{Riferimenti}
		\subsubsection{Riferimenti normativi}
			\begin{itemize}
	\item \textbf{Norme di Progetto:} \NdP, §4.2 Qualità;
	\item \textbf{Capitolato d'appalto C2: } Colletta \newline
		  \url{https://www.math.unipd.it/~tullio/IS-1/2018/Progetto/C2.pdf}.
\end{itemize}
		\subsubsection{Riferimenti informativi}
			\begin{itemize}
	\item \textbf{Piano di Progetto:} \PdP;
	\item \textbf{Slide del corso di Ingegneria del Software:}\newline
		  		 \url{https://www.math.unipd.it/~tullio/IS-1/2018};
	\item \textbf{Ian Sommerville, Software Engineering, Nona edizione:}
		\begin{itemize}
		  	\item Capitolo 24: Quality management;
		  	\item Capitolo 26: Process improvement;
		\end{itemize}
	\item \textbf{Standard ISO/IEC 9126:}\newline
		  		  \url{https://it.wikipedia.org/wiki/ISO/IEC_9126}\newline
				  \url{https://en.wikipedia.org/wiki/ISO/IEC_9126};
	\item \textbf{Standard ISO/IEC 15504:}\newline
				  \url{https://en.wikipedia.org/wiki/ISO/IEC_15504}
	\item \textbf{VARI:};
	\item \textbf{ED:};
	\item \textbf{EVENTUALI:};
	\item \textbf{:};
	\item \textbf{:};
	\item \textbf{:};
	\item \textbf{:};
		
\end{itemize}
	\newpage
\section{Strategie di verifica}
	\subsection{Obiettivi di qualità}
		\subsubsection{Qualità di processo}
		\subsubsection{Qualità di prodotto}
		\subsubsection{Organizzazione????}
		\subsubsection{Pianficazione strategica temporale?????}
	\subsection{Risorse}
	\subsection{Responsabilità}
	\subsection{Misure e metriche}
		\subsubsection{Metriche processi}
		\subsubsection{Metriche documenti}
		\subsubsection{Metriche software}
\section{Gestione amministrativa della revisione}

\section{Appendice - Esito Verifica}

\end{document}