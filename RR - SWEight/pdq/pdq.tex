%generare il pdf con il comando: pdflatex main.tex
\documentclass[a4paper, oneside, openany, dvipsnames, table]{article}
\usepackage{../template/SWEightStyle}
\usepackage{cleveref}

\newcommand{\Titolo}{Manuale Utente}

\newcommand{\Gruppo}{SWEight}

\newcommand{\Approvatore}{Damien Ciagola}
\newcommand{\Redattori}{Alberto Bacco \newline Sebastiano Caccaro \newline Gheorghe Isachi \newline Gionata Legrottaglie}
\newcommand{\Verificatori}{Francesco Corti \newline Francesco Magarotto}

\newcommand{\pathimg}{../template/img/logoSWEight.png}

\newcommand{\Versionedoc}{1.0.0}

\newcommand{\Distribuzione}{\proponente \newline Prof. Vardanega Tullio \newline Prof. Cardin Riccardo \newline Gruppo SWEight}

\newcommand{\Uso}{Esterno}

\newcommand{\NomeProgetto}{Colletta}

\newcommand{\Mail}{SWEightGroup@gmail.com}

\newcommand{\DescrizioneDoc}{Questo documento si occupa di fornire le modalità di utilizzo del software Colletta commissionato}


\begin{document}
\copertina{}
\definecolor{greySWEight}{RGB}{255, 71, 87}
\definecolor{greyROwSWEight}{RGB}{234, 234, 234}

\section*{Registro delle modifiche}
{
	\rowcolors{2}{greyROwSWEight}{white}
	\renewcommand{\arraystretch}{1.5}
	\centering
	\begin{longtable}{ c c C{4cm}  c  c }
		
		\rowcolor{greySWEight}
		\textcolor{white}{\textbf{Versione}} & \textcolor{white}{\textbf{Data}} & \textcolor{white}{\textbf{Descrizione}} & \textcolor{white}{\textbf{Nominativo}} & \textcolor{white}{\textbf{Ruolo}}\\
		1.2.2 & 2019-02-25 & Ampliamento sezione 5.4 e 3.2.5.2 & Alberto Bacco & \reda{} \\
		
		1.2.1 & 2019-02-23 & Aggiunta sezione 3.2.5.8 Checkstyle & Sebastiano Caccaro & \reda{} \\		
		
		1.2.0 & 2019-02-20 & Aggiunta scelte tecnologiche 3.2.4.2, da 3.4.5.4 a 3.4.5.7, 4.3.1.4, 4.3.2.2, 4.4.6 e figlie & Sebastiano Caccaro & \reda{} \\	
		
		1.1.5 & 2019-02-20 & Modifica sezione 2 & Alberto Bacco & \reda{} \\
		
		1.1.4 & 2019-02-18 & Correzione errori grammatica, spostate sottosezioni di asana da 4.3 a 5.2, & Alberto Bacco & \reda{} \\
		
		1.1.3 & 2019-02-14 & Riorganizzazione e correzione errori sezione 5 & Enrico Muraro & \reda{} \\
		
		1.1.2 & 2019-02-03 & Modifica sottosezione 4.1.10, 4.3.1.4, 4.3.1.5, 4.3.1.6 & Alberto Bacco& \reda{} \\	
		
		1.1.1 & 2019-01-31 & Modifica struttura e contenuti sezione 3  & Damien Ciagola & \reda{} \\	
		
		1.1.0 & 2019-01-27 & Sezione Qualità 4.2 & Sebastiano Caccaro & \reda{} \\	
		
		1.0.1 & 2019-01-25 & Parziale ristrutturazione della struttura del documento & Sebastiano Caccaro & \reda{} \\		
		
		1.0.0 & 2019-01-11 & Approvazione per il rilascio & Sebastiano Caccaro & \Res{} \\
		
		0.9.0 & 2019-01-9 & Verifica finale & Francesco Corti & \ver{} \\
		
		0.9.0 & 2019-01-8 & Aggiunta lista di controllo & Gionata Legrottaglie & \reda{} \\
		
		0.8.0 & 2018-12-23 & Correzioni errori ortografici & Gionata Legrottaglie & \reda{} \\
		
		0.7.0 & 2018-12-20 & Verifica documento & Francesco Corti & \ver{}\\
		
		0.6.0 & 2018-12-18 & Aggiunta sottosezione 5.2.2.2, 5.2.2.3, 5.2.2.4 & Francesco Magarotto & \reda{} \\
		
		0.5.2 & 2018-12-16 & Modifica sezione 4.1.5.3 & Alberto Bacco & \reda{} \\
		
		0.5.2 & 2018-12-16 & Modifica sezione 4.1.5.3 & Alberto Bacco & \reda{} \\
		
		0.5.2 & 2018-12-16 & Aggiunte sottosezioni  & Alberto Bacco & \reda{} \\
		
		0.5.1 & 2018-12-15 & Aggiunte sottosezioni 5.3, 5.4, 5.5, 5.6, 5.7, 5.8 & Alberto Bacco & \reda{} \\
		
		0.5.0 & 2018-12-15 & Aggiunta sezione 5 e sottosezioni 5.1, 5.2 & Gionata Legrottaglie & \reda{} \\
		
		0.4.1 & 2018-12-11 & Aggiunta sezione 4.1.7.3.1 & Francesco Magarotto & \reda{} \\ 
		
		0.4.0 & 2018-12-10 & Aggiunte sottosezioni 4.1.5, 4.1.6, 4.1.7, 4.1.8 & Gionata Legrottaglie & \reda{} \\ 
		0.4.0 & 2018-12-09 & Aggiunta sezione 4 e sottosezioni 4.1.1, 4.1.2, 4.1.3, 4.1.4 & Gionata Legrottaglie & \reda{} \\ 
		
		0.3.1 & 2018-12-07 & Aggiunta sottosezione 3.2 & Gionata Legrottaglie & \reda{} \\ 
		
		0.3.0 & 2018-12-06 & Aggiunta sezione 3 e sottosezione 3.1 & Gionata Legrottaglie & \reda{} \\ 
		
		0.2.0 & 2018-12-05 & Aggiunti i riferimenti & Gionata Legrottaglie & \reda{} \\ 
		
		0.1.0 & 2018-11-30 & Aggiunta introduzione & Gionata Legrottaglie & \reda{} \\
		
		0.0.1 & 2018-11-28 & Creazione scheletro del documento & Gionata Legrottaglie & \reda{}\\
		
	\end{longtable}

}
\newpage
\tableofcontents
\newpage
\section{Introduzione}
	\subsection{Scopo del documento}
		% vecchia introduzione
% In questo documento è illustrata la {strategia}\ped{G} di {verifica}\ped{G} e 
% {validazione}\ped{G} del gruppo \gruppo . Tale strategia è fondamentale per dare una 
% misurazione oggettiva e quantificabile del livello di {qualità}\ped{G} di quanto viene 
% prodotto. \newline
% Ciò è vantaggioso sia per il gruppo \gruppo , che può facilmente individuare difetti 
% durante lo svolgimento del progetto, sia per il {committente}\ped{G}, che può costantemente 
% monitorare la qualità del prodotto in base a criteri oggettivi e prestabiliti.

% alternativa alla prima introduzione
In questo documento sono illustrate le {strategie}\ped{G} di {verifica}\ped{G}e 
{validazione}\ped{G} del gruppo \gruppo. 
Tale strategia ci si assicura la qualità dei processi, dei documenti e delle procedure 
utilizzate per gestire e sviluppare i risultati finali.
Lo scopo di questo documento è descrivere le informazioni necessarie per gestire efficacemente 
la qualità del progetto, dalla pianificazione alla consegna, comprendendo obiettivi 
di qualità, responsabilità, e l'approccio di gestione della qualità per 
garantire che gli obiettivi siano raggiunti.\newline
Ciò è vantaggioso sia per il gruppo \gruppo , che può facilmente individuare difetti 
durante lo svolgimento del progetto, sia per il {committente}\ped{G}, che può costantemente 
monitorare la qualità del prodotto in base a criteri oggettivi e prestabiliti.


	\subsection{Scopo del prodotto}
		Il progetto prevede la realizzazione di una piattaforma collaborativa di raccolta dati in cui gli utenti possano predisporre e/o svolgere piccoli esercizi di analisi grammaticale. Lo scopo è raccogliere dati relativi sia  agli esercizi predisposti, che al loro svolgimento da parte degli utenti. Sviluppatori e ricercatori utilizzeranno queste informazione per insegnare ad un elaboratore a svolgere i medesimi esercizi, mediante tecniche di apprendimento automatico.

%Questa parte è un copia-incolla cafonissimo dal capitolato della mivoc
%Siccome è proprio quello che vogliono, non mi sembrava il caso di andare a modifcarla
	\subsection{Glossario}
		
\section*{F}
\textbf{Freeling}: the library for pos-tagging developed by TALP Research Center written in C++;
\section*{P}
\textbf{Pos-tagging}: part-of-speech tagging, also called grammatical tagging or word-category disambiguation, is the process of marking up a word in a text (corpus) as corresponding to a particular part of speech; \\ 
\textbf{POJO}: Plain Old Java Object, is an ordinary Java object, not bound by any special restriction and not requiring any class path. In Spring it refers to a Java object (instance of definition) that isn't bogged down by framework extensions;
\section*{J}
\textbf{JSON}: JavaScript Object Notation, is a lightweight data-interchange format.  It is easy for humans to read and write. It is easy for machines to parse and generate.\\
\textbf{JWT}: JSON Web Token, a JSON-based open standard (RFC 7519) for creating access tokens that assert some number of claims;

	\subsection{Riferimenti}
		\subsubsection{Riferimenti normativi}
			\begin{itemize}
	\item \textbf{Norme di Progetto:} \NdP ;
	\item \textbf{Capitolato d'appalto C2: } Colletta \newline
		  \url{https://www.math.unipd.it/~tullio/IS-1/2018/Progetto/C2.pdf}.
\end{itemize}
		\subsubsection{Riferimenti informativi}
			\begin{itemize}
    \item Software Engineering (10th edition) - Ian Sommerville
    \item Slide "Gestione di Progetto", corso di Ingegneria del Software
          \newline \url{https://www.math.unipd.it/~tullio/IS-1/2018/Dispense/L06.pdf}
\end{itemize}
	\newpage
\section{Strategie di verifica}
	\subsection{Document goal}
The purpose of this document is to provide all the necessary information to extend, correct and improve Colletta.
There will be additional information regarding setting up the development environment to work in an environment that is as consistent as possible with that used
by the other members of group SWEight, but can be ignored if you only want to use part of the product.
This guide was written taking into account the Microsoft Windows and Linux operating systems. If other systems are used, compatibility issues may arise, even if it's unlikely. In this case refer to the git page. This document will grow as the product will be fully
developed.

\subsection{Product goal}
The purpose of the product is the creation of a collaborative data collection platform where users can prepare and/or perform small grammar exercises. 
The front-end of the system consists of a web application developed with React and Redux, while the back-end is a Spring Boot application written in Java, which will handle HTTP Requests sent from the front-end. 

\subsection{References}


\subsubsection{Installation references}

\begin{itemize}
\item \textbf{Git}: \url{https://git-scm.com/}
\item \textbf{Node.js}: \url{https://nodejs.org/en/}
\item \textbf{NPM}: \url{https://www.npmjs.com/}
\item \textbf{Oracle JDK}: \url{https://www.oracle.com/technetwork/java/javase/downloads/index.html}
\item \textbf{OpenJDK}: \url{https://openjdk.java.net/}
\item \textbf{Maven}: \url{https://maven.apache.org/}
\item \textbf{Lombok}: \url{https://projectlombok.org/}
\item \textbf{VSCode}: \url{https://code.visualstudio.com/} 

\end{itemize}

\subsubsection{Legal references}
\begin{itemize}
\item \textbf{MIT License}: \url{https://opensource.org/licenses/MIT}
\end{itemize}

%\subsubsection{Informative references}

	\subsection{Obiettivi di qualità}
		\subsubsection{Qualità di processo}
			Le metriche presentate in questa sezione monitorano lo stato dei processi del progetto analizzando l'uso che essi fanno di tempo e risorse finanziarie. Sono particolarmente utili per il \Res , che può quindi decidere di apportare modifiche alla pianificazione quando necessario.

\paragraph{MP001 - Schedule Variance}\mbox{}\\[0,3cm]
La Schedule Variance indica se una certa attività o processo è in anticipo, in pari, o in ritardo rispetto alla data di scadenza prevista\\[0,2cm]
\textbf{Parametri adottati:}
\begin{itemize}
	\item Range accettabile: $(-\infty , 3]$;
	\item Range ottimale: $(-\infty , 0]$.
\end{itemize}

\paragraph{MP002 - Budget Variance}\mbox{}\\[0,3cm]
La Budget Variance misura, ad una determinata data, lo scostamento fra quanto speso e quanto preventivato. \\[0,2cm]
\textbf{Parametri adottati:}
\begin{itemize}
	\item Range accettabile: $(-\infty , 9\%]$;
	\item Range ottimale: $(-\infty , 1\%]$.
\end{itemize}
		\subsubsection{Qualità di prodotto}
			Per garantire la qualità dei prodotti, viene adottato lo standard ISO/IEC 9126\footnote{ISO/IEC 9126: Vedi appendice \cref{app:ISO/IEC 9126}}. Quest'ultimo permette di monitorare la qualità del software, fornendo delle metriche per misurarla.\newline
Sono fissati i seguenti obiettivi:
\begin{itemize}
	\item La \textbf{documentazione} deve essere:
		\begin{itemize}
			\item Facilmente leggibile;
			\item Scritta in modo corretto, secondo le regole della lingua italiana.
		\end{itemize}
	\item Il \textbf{software} deve:
		\begin{itemize}
			\item Soddisfare i requisiti stabiliti nell'\AdR ;
			\item Garantire semplicità di utilizzo;			
			\item Garantire semplicità di manutenzione;
			\item Garantire affidabilità.
		\end{itemize}
		
\end{itemize}
	\subsection{Organizzazione????}
		\subsection{Pianficazione strategica temporale?????}
	\subsection{Risorse}
	\subsection{Responsabilità}
	\subsection{Misure e metriche}
		\subsubsection{Metriche processi}
		\subsubsection{Metriche documenti}
		\subsubsection{Metriche software}
\section{Gestione amministrativa della revisione}


\appendix
\addcontentsline{toc}{part}{Appendici}
\section{Qualità}
	\subsection{SPICE}
		\label{app:SPICE}
	\subsection{ISO/IEC 9126}
		\label{app:ISO/IEC 9126}
\section{Appendice - Esito Verifica}

\end{document}