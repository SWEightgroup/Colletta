%generare il pdf con il comando: pdflatex main.tex
\documentclass[a4paper, oneside, openany, dvipsnames, table]{article}
\usepackage{../template/SWEightStyle}
\usepackage{cleveref}
\usepackage{enumitem}
\usepackage{amsmath}

\newcommand{\Titolo}{Verbale Riunione 2018-12-12}

\newcommand{\Gruppo}{SWEight}

\newcommand{\ACapoRedazione}{Francesco Magarotto}

\newcommand{\Verifica}{Francesco Corti}

\newcommand{\Approvazione}{Sebastiano Caccaro}

\newcommand{\Distribuzione}{Vardanega Tullio \newline Cardin Riccardo \newline Gruppo SWEight}

\newcommand{\Uso}{Interno}

\newcommand{\NomeProgetto}{Colletta}

\newcommand{\Mail}{SWEightGroup@gmail.com}

\newcommand{\DescrizioneDoc}{Questo documento si occupa di riportare quanto discusso nella riunione del 12-12-2018}


\setcounter{tocdepth}{4}
\setcounter{secnumdepth}{4}

\begin{document}
\copertina{}
\definecolor{greySWEight}{RGB}{255, 71, 87}
\definecolor{greyROwSWEight}{RGB}{234, 234, 234}

\section*{Registro delle modifiche}
{
	\rowcolors{2}{greyROwSWEight}{white}
	\renewcommand{\arraystretch}{1.5}
	\centering
	\begin{longtable}{ c c  C{4cm}  c  c }
		
		\rowcolor{greySWEight}
		\textcolor{white}{\textbf{Versione}} & \textcolor{white}{\textbf{Data}} & \textcolor{white}{\textbf{Descrizione}} & \textcolor{white}{\textbf{Nominativo}} & \textcolor{white}{\textbf{Ruolo}}\\
		
		1.0.2 & 2019-03-02 & Aggiunti nuovi termini del documento Piano di Progetto & Isachi Gheorghe &\reda{}\\
		
		1.0.1 & 2019-02-23 & Verifica del documento &  Francesco Corti & \ver{}\\
		
		1.0.1 & 2019-02-20 & Aggiunti nuovi termini del documento Norme di Progetto & Isachi Gheorghe &\reda{}\\
		
		1.0.0 & 2019-01-09 & Approvazione & Sebastiano Caccaro & \Res{}\\
						
		0.1.1 & 2019-01-08 & Verifica del documento & Bacco Alberto & \ver{}\\
		
		0.1.1 & 2019-01-04 & Aggiunti termini del documento Norme di Progetto & Isachi Gheorghe &\reda{}\\
		
		0.1.0 & 2019-01-01 & Aggiunti termini del documento Analisi dei Requisiti & Isachi Gheorghe &\reda{}\\
		
		0.0.4 & 2018-12-29 & Verifica del documento & Bacco Alberto & \ver{}\\
				
		0.0.4 & 2018-12-27 & Aggiunti termini del documento Piano di Qualifica & Isachi Gheorghe &\reda{}\\
				
		0.0.3 & 2018-12-26 &Aggiunti termini del documento Piano di Progetto & Isachi Gheorghe & \reda{}\\
				
		0.0.2 & 2018-12-17 & Aggiunti termini del documento Studio di Fattibilità & Isachi Gheorghe &\reda{}\\
		
		0.0.1 & 2018-12-15 & Scheletro del glossario & Damien Ciagola & \reda{}\\
		
	\end{longtable}

}
\newpage
\tableofcontents
\newpage
\section{Introduzione}
========================================================\\
- CITARE IN QUALCHE MODO L'INCREMENTALITA' DEL DOCUMENTO\\
- RIEMPIRE SEZIONI VUOTE\\
- PREDISPORRE ALMENO SEZIONE TEST\\
=========================================================\\
	\subsection{Scopo del documento}
		In questo documento è illustrata la {strategia}\ped{G} di {verifica}\ped{G} e {validazione}\ped{G} del gruppo \gruppo . Tale strategia è fondamentale per dare una misurazione oggettiva e quantificabile del livello di {qualità}\ped{G} di quanto viene prodotto. \newline
Ciò è vantaggioso sia per il gruppo \gruppo , che può facilmente individuare difetti durante lo svolgimento del progetto, sia per il {committente}\ped{G}, che può costantemente monitorare la qualità del prodotto in base a criteri oggettivi e prestabiliti.


	\subsection{Scopo del prodotto}
		Lo scopo del progetto è realizzare una piattaforma collaborativa di raccolta dati in cui gli utenti possano predisporre
e/o svolgere piccoli esercizi di grammatica (per esempio esercizi di analisi grammaticale) e i dati raccolti
siano relativi sia agli esercizi predisposti che al loro svolgimento da parte degli utenti. I dati raccolti devono
essere utilizzabili da sviluppatori e ricercatori al fine di insegnare ad un elaboratore a svolgere i medesimi esercizi
mediante tecniche di apprendimento automatico supervisionato.

%Questa parte è un copia-incolla cafonissimo dal capitolato della mivoc
%Siccome è proprio quello che vogliono, non mi sembrava il caso di andare a modifcarla
	\subsection{Glossario}
		Nel documento è possibile incontrare termini tecnici, i quali potrebbero non essere immediatamente chiari al lettore. Per disambiguarne il significato, essi sono stati marcati con una \ped{G} a pedice e la loro definizione è reperibile nel glossario fornito separatamente.
	\subsection{Riferimenti}
		\subsubsection{Riferimenti normativi}
			\begin{itemize}
	\item \textbf{Norme di Progetto:} \NdP, §4.2 Qualità;
	\item \textbf{Capitolato d'appalto C2: } Colletta \newline
		  \url{https://www.math.unipd.it/~tullio/IS-1/2018/Progetto/C2.pdf}.
\end{itemize}
		\subsubsection{Riferimenti informativi}
			\begin{itemize}
	\item \textbf{Piano di Progetto:} \PdP;
	\item \textbf{Slide del corso di Ingegneria del Software:}\newline
		  		 \url{https://www.math.unipd.it/~tullio/IS-1/2018};
	\item \textbf{Ian Sommerville, Software Engineering, Nona edizione:}
		\begin{itemize}
		  	\item Capitolo 24: Quality management;
		  	\item Capitolo 26: Process improvement;
		\end{itemize}
	\item \textbf{Standard ISO/IEC 9126:}\newline
		  		  \url{https://it.wikipedia.org/wiki/ISO/IEC_9126}\newline
				  \url{https://en.wikipedia.org/wiki/ISO/IEC_9126};
	\item \textbf{Standard ISO/IEC 15504:}\newline
				  \url{https://en.wikipedia.org/wiki/ISO/IEC_15504}
	\item \textbf{VARI:};
	\item \textbf{ED:};
	\item \textbf{EVENTUALI:};
	\item \textbf{:};
	\item \textbf{:};
	\item \textbf{:};
	\item \textbf{:};
		
\end{itemize}
	\newpage
\section{Strategie di verifica}
	\subsection{Scopo del documento}
Il presente documento ha lo scopo di fornire agli sviluppatori uno specchietto informativo sul design strutturale e logico della piattaforma Colletta. Il documento sarà inoltre
corredato da diagrammi UML 2.X delle principali scelte prese dal gruppo SWEight e descriverà le tecnologie utilizzate nella realizzazione dell’applicazione.
\subsection{Scopo del prodotto}
Il prodotto da realizzare consta in un’applicazione web che fornisca uno strumento per creare e svolgere esercizi di analisi grammaticale, e al contempo né raccolga i risultati. I dati raccolti verranno impiegati dagli sviluppatori dell’azienda proponente come strumento per il miglioramento di algoritmi di {apprendimento automatico}\ped{G}. Nello specifico il prodotto verrà utilizzato da tre tipologie di utenti:
le/gli insegnanti che si occuperanno della creazione degli esercizi,
gli allievi che potranno svolgere gli esercizi e ottenere delle valutazioni e gli sviluppatori che filtreranno i dati secondo alcuni criteri, e infine li scaricheranno.\\Il prodotto si interfaccerà con un’applicazione di {PoS-tagging}\ped{G}, come {FreeLing}\ped{G}, a cui verrà delegata l’esecuzione dell’analisi grammaticale delle frasi.
\subsection{Glossario}
Al fine di rendere il documento il più comprensibile possibile e permetterne una rapida fruizione, viene allegato il \G{} in cui sono presenti i termini contraddistinti dal pedice G. Tali termini includono abbreviazioni, acronimi, termini di natura tecnica, oppure sono fonte di ambiguità e pertanto necessitano di una definizione che renda il loro significato inequivocabile. 
Ogni termine, solo alla prima occorrenza per documento, verrà contrassegnato con la dicitura sopra indicata e rimanderà alla medesima definizione nel \G{}.
\newpage
	\subsection{Obiettivi di qualità}
		\subsubsection{Qualità di processo}
			\paragraph{MP001 - Schedule Variance}\mbox{}\\[0,3cm]
\begin{table}[H]
    \centering
    \begin{tabular}{cccc}
        \rowcolor{greySWEight}
        \textcolor{white}{\textbf{Attività}} & 
        \textcolor{white}{\textbf{Abbreviazione}} &
        \textcolor{white}{\textbf{Valore Indice}}&
        \textcolor{white}{\textbf{Riscontro}}\\
		\textbf{Stesura Analisi dei Requisiti} & ADR & 1 & \textcolor{YellowOrange}{Accettabile}\\
		\textbf{Stesura Glossario} & GLO & -1 & \textcolor{ForestGreen}{Ottimale} \\
		\textbf{Stesura Piano di Progetto} & PDQ & 0 & \textcolor{ForestGreen}{Ottimale} \\
		\textbf{Stesura Piano di Qualifica} & PDP & 1 & \textcolor{ForestGreen}{Ottimale} \\
		\textbf{Stesura Norme di Progetto} & NDP & 1 & \textcolor{YellowOrange}{Accettabile} \\
		\textbf{Stesura Studio di Fattibilità} & SDF & 0 & \textcolor{YellowOrange}{Accettabile} \\

    \end{tabular}
    \caption{Schedule Variance nel periodo di Progettazione}
\end{table}
\begin{figure}[H]
    \centering
	\includegraphics[width=1\linewidth]{sez/App_Esito/Progettazione/graph/PR_SV.pdf}
	\caption{Schedule Variance nel periodo di Progettazione}
\end{figure}

\paragraph{MP002 - Budget Variance}\mbox{}\\[0,3cm]
\begin{table}[H]
    \centering
    \begin{tabular}{cccc}
        \rowcolor{greySWEight}
        \textcolor{white}{\textbf{Abbreviazione}} &
        \textcolor{white}{\textbf{Valore Indice}}&
        \textcolor{white}{\textbf{Valore in €}}&
        \textcolor{white}{\textbf{Riscontro}}\\
        BV & -2,91\% & -136 & \textcolor{ForestGreen}{Ottimale}\\
    \end{tabular}
    \caption{Budget Variance nel periodo di Progettazione}
\end{table}
\begin{figure}[H]
    \centering
	\includegraphics[height=4cm]{sez/App_Esito/Progettazione/graph/PR_BV.pdf}
	\caption{Budget Variance nel periodo di Progettazione}
\end{figure}
\begin{figure}[H]
	\centering
	\includegraphics[height=6cm]{sez/App_Esito/Progettazione/graph/PR_Storico_BV.pdf}
	\caption{Andamento del Budget Variance fino al periodo di Progettazione}
\end{figure}
		\subsubsection{Qualità di prodotto}
			Per garantire la qualità dei prodotti, viene adottato lo standard ISO/IEC 9126\footnote{ISO/IEC 9126: Vedi appendice \cref{app:ISO/IEC 9126}}. Quest'ultimo permette di monitorare la qualità del software, fornendo delle metriche per misurarla.
	\subsection{Organizzazione????}
		\subsection{Pianficazione strategica temporale?????}
	\subsection{Risorse}
	\subsection{Responsabilità}
	\subsection{Misure e metriche}
	\label{sec:metriche}
		\subsection{Scopo del documento}
Il presente documento ha lo scopo di fornire agli sviluppatori uno specchietto informativo sul design strutturale e logico della piattaforma Colletta. Il documento sarà inoltre
corredato da diagrammi UML 2.X delle principali scelte prese dal gruppo SWEight e descriverà le tecnologie utilizzate nella realizzazione dell’applicazione.
\subsection{Scopo del prodotto}
Il prodotto da realizzare consta in un’applicazione web che fornisca uno strumento per creare e svolgere esercizi di analisi grammaticale, e al contempo né raccolga i risultati. I dati raccolti verranno impiegati dagli sviluppatori dell’azienda proponente come strumento per il miglioramento di algoritmi di {apprendimento automatico}\ped{G}. Nello specifico il prodotto verrà utilizzato da tre tipologie di utenti:
le/gli insegnanti che si occuperanno della creazione degli esercizi,
gli allievi che potranno svolgere gli esercizi e ottenere delle valutazioni e gli sviluppatori che filtreranno i dati secondo alcuni criteri, e infine li scaricheranno.\\Il prodotto si interfaccerà con un’applicazione di {PoS-tagging}\ped{G}, come {FreeLing}\ped{G}, a cui verrà delegata l’esecuzione dell’analisi grammaticale delle frasi.
\subsection{Glossario}
Al fine di rendere il documento il più comprensibile possibile e permetterne una rapida fruizione, viene allegato il \G{} in cui sono presenti i termini contraddistinti dal pedice G. Tali termini includono abbreviazioni, acronimi, termini di natura tecnica, oppure sono fonte di ambiguità e pertanto necessitano di una definizione che renda il loro significato inequivocabile. 
Ogni termine, solo alla prima occorrenza per documento, verrà contrassegnato con la dicitura sopra indicata e rimanderà alla medesima definizione nel \G{}.
\newpage
		\subsubsection{Metriche processi}
			\paragraph{MP001 - Schedule Variance}\mbox{}\\[0,3cm]
\begin{table}[H]
    \centering
    \begin{tabular}{cccc}
        \rowcolor{greySWEight}
        \textcolor{white}{\textbf{Attività}} & 
        \textcolor{white}{\textbf{Abbreviazione}} &
        \textcolor{white}{\textbf{Valore Indice}}&
        \textcolor{white}{\textbf{Riscontro}}\\
		\textbf{Stesura Analisi dei Requisiti} & ADR & 1 & \textcolor{YellowOrange}{Accettabile}\\
		\textbf{Stesura Glossario} & GLO & -1 & \textcolor{ForestGreen}{Ottimale} \\
		\textbf{Stesura Piano di Progetto} & PDQ & 0 & \textcolor{ForestGreen}{Ottimale} \\
		\textbf{Stesura Piano di Qualifica} & PDP & 1 & \textcolor{ForestGreen}{Ottimale} \\
		\textbf{Stesura Norme di Progetto} & NDP & 1 & \textcolor{YellowOrange}{Accettabile} \\
		\textbf{Stesura Studio di Fattibilità} & SDF & 0 & \textcolor{YellowOrange}{Accettabile} \\

    \end{tabular}
    \caption{Schedule Variance nel periodo di Progettazione}
\end{table}
\begin{figure}[H]
    \centering
	\includegraphics[width=1\linewidth]{sez/App_Esito/Progettazione/graph/PR_SV.pdf}
	\caption{Schedule Variance nel periodo di Progettazione}
\end{figure}

\paragraph{MP002 - Budget Variance}\mbox{}\\[0,3cm]
\begin{table}[H]
    \centering
    \begin{tabular}{cccc}
        \rowcolor{greySWEight}
        \textcolor{white}{\textbf{Abbreviazione}} &
        \textcolor{white}{\textbf{Valore Indice}}&
        \textcolor{white}{\textbf{Valore in €}}&
        \textcolor{white}{\textbf{Riscontro}}\\
        BV & -2,91\% & -136 & \textcolor{ForestGreen}{Ottimale}\\
    \end{tabular}
    \caption{Budget Variance nel periodo di Progettazione}
\end{table}
\begin{figure}[H]
    \centering
	\includegraphics[height=4cm]{sez/App_Esito/Progettazione/graph/PR_BV.pdf}
	\caption{Budget Variance nel periodo di Progettazione}
\end{figure}
\begin{figure}[H]
	\centering
	\includegraphics[height=6cm]{sez/App_Esito/Progettazione/graph/PR_Storico_BV.pdf}
	\caption{Andamento del Budget Variance fino al periodo di Progettazione}
\end{figure}
		\subsubsection{Metriche documenti}
			\paragraph{Indice di Gulpease}\mbox{}\\[0,3cm]
\begin{table}[H]
	\centering
	\begin{tabular}{cccc}
	\rowcolor{greySWEight}
	\textcolor{white}{\textbf{Documento}} & 
	\textcolor{white}{\textbf{Abbreviazione}} &
	\textcolor{white}{\textbf{Valore Indice}}&
	\textcolor{white}{\textbf{Riscontro}}\\
	
	\textbf{Analisi dei Requisiti} & ADR & 58,84 & \textcolor{ForestGreen}{Ottimale} \\
	\textbf{Glossario} & GLO & 51,12 & \textcolor{YellowOrange}{Accettabile} \\
	\textbf{Piano di Progetto} & PDQ & 54,89 & \textcolor{YellowOrange}{Accettabile} \\
	\textbf{Piano di Qualifica} & PDP & 56,67 & \textcolor{ForestGreen}{Ottimale} \\
	\textbf{Norme di Progetto} & NDP & 55,3 & \textcolor{ForestGreen}{Ottimale} \\
	\textbf{Studio di Fattibilità} & SDF & 56,77 & \textcolor{ForestGreen}{Ottimale} \\

	\end{tabular}
	\caption{Indice di Gulpease nel periodo di Analisi}
\end{table}
\begin{figure}[H]
	\includegraphics[width=1\linewidth]{sez/App_Esito/graph/AN_Gulp.pdf}
	\caption{Indice di Gulpease nel periodo di Analisi}
\end{figure}

%\textcolor{ForestGreen}{Ottimale}
%\textcolor{YellowOrange}{Accettabile}
%\textcolor{BrickRed}{Non accettabile}
		\subsubsection{Metriche software}
			% METTI SOLO TABELLE E GRAFICI, SU RETROSPETTIVA FAI UNA TABELLA RIASSUNTIVA E COMMENTA I PROBLEMI
\paragraph{MS001 - Numero di metodi}\mbox{}\\[0,3cm]
\paragraph{MS002 - Numero di parametri}\mbox{}\\[0,3cm]
\paragraph{MS003 - Funzioni di interfaccia per package}\mbox{}\\[0,3cm]
\paragraph{MS004 - Complessità ciclomatica}\mbox{}\\[0,3cm]
\paragraph{MS005 - Campi dati per classe}\mbox{}\\[0,3cm]
\paragraph{MS006 - Commenti per linee di codice}\mbox{}\\[0,3cm]
    \begin{table}[H]
        \centering
        \begin{tabular}{ccccccc}
            \rowcolor{greySWEight}
            \textcolor{white}{\textbf{Codice}} &
            \textcolor{white}{\textbf{File analizzati}}&
            \textcolor{white}{\textbf{Totale righe}}&
            \textcolor{white}{\textbf{Totale righe commento}}&
            \textcolor{white}{\textbf{Percentuale}}&
            \textcolor{white}{\textbf{Riscontro}}\\
            \textbf{MS006} & 92 & 7524 & 945 & 12.56\% & \textcolor{YellowOrange}{Accettabile}\\
        \end{tabular}
        \caption{Totale commenti per linee di codice}
    \end{table}
    \begin{figure}[H]
        \centering
        \includegraphics[width=0.3\linewidth]{sez/App_Esito/Qualifica/graph/commenti.pdf}
        \caption{Percentuale commenti periodo di codifica}
    \end{figure}

\paragraph{MS007 - Code coverage}\mbox{}\\[0,3cm]
\paragraph{MS008 - Superamento test}\mbox{}\\[0,3cm]
\paragraph{MS009 - Soddisfacimento requisiti obbligatori}\mbox{}\\[0,3cm]
    \begin{table}[H]
        \centering
        \begin{tabular}{cccc}
        \rowcolor{greySWEight}
        \textcolor{white}{\textbf{Codice}} &
        \textcolor{white}{\textbf{Requisiti obbligatori soddisfatti}} &
        \textcolor{white}{\textbf{Riscontro}}\\
        \textbf{MS009}& 77.40\% & \textcolor{RubineRed}{non superato} \\

        \end{tabular}
        \caption{Requisit obbligatori soddisfatti nel periodo di pianificazione e codifica}
    \end{table}
    \begin{figure}[H]
        \centering
        \includegraphics[width=0.7\linewidth]{sez/App_Esito/Qualifica/graph/RequisitiObbligatori.pdf}
        \caption{Soddisfacimento requisiti obbligatori}
    \end{figure}

\paragraph{MS010 - Media di build Travis settimanali}\mbox{}\\[0,3cm]
    Estrapolando i dati dagli insights del sito di Travis è stata riscontrata una media di build settimanali di 70.75.

    \begin{table}[H]
        \centering
        \begin{tabular}{cccc}
        \rowcolor{greySWEight}
        \textcolor{white}{\textbf{Codice}} &
        \textcolor{white}{\textbf{Media}} &
        \textcolor{white}{\textbf{Riscontro}}\\
        \textbf{MS010}& 70.75 & \textcolor{ForestGreen}{Ottimale} \\
    
        \end{tabular}
        \caption{Media build settimanali nel periodo di pianificazione e codifica}
    \end{table}
    \begin{figure}[H]
        \centering
        \includegraphics[width=0.7\linewidth]{sez/App_Esito/Qualifica/graph/buildTravisSettimanale.pdf}
        \caption{Build di Travis settimanali, dal 2019-03-11 al 2019-04-07}
    \end{figure}

    \paragraph{MS011 - Percentuale build Travis superate }\mbox{}\\[0,3cm]
    Il valore minimo accettabile prefissato per le build superate era del 75\%, il valore riscontrato
    è 71.40\%, questo potrebbe implicare la presenza di errori. Dato che le percentuali delle
    build superate delle ultime due settimane corrisponde al 85.15\%, il valore finale
    trovato non rispetta i valori prestabiliti a causa della poca esperienza di codifica di inizio periodo.
    \begin{table}[H]
        \centering
        \begin{tabular}{cccc}
        \rowcolor{greySWEight}
        \textcolor{white}{\textbf{Codice}} &
        \textcolor{white}{\textbf{Percentuale build superate}} &
        \textcolor{white}{\textbf{Riscontro}}\\
        \textbf{MS011}& 71.40 & \textcolor{RubineRed}{non superato} \\
    
        \end{tabular}
        \caption{Build Travis superate nel periodo di pianificazione e codifica}
    \end{table}
    \begin{figure}[H]
        \centering
        \includegraphics[width=0.7\linewidth]{sez/App_Esito/Qualifica/graph/buildTravisGiornaliera.pdf}
        \caption{Build di Travis giornaliere, dal 2019-03-11 al 2019-04-07}
    \end{figure}
    \begin{figure}[H]
        \centering
        \includegraphics[width=0.7\linewidth]{sez/App_Esito/Qualifica/graph/buildTravisSuperate.pdf}
        \caption{Build di Travis superate}
    \end{figure}
		\subsubsection{Riassunto metriche}
			%\begin{table}[H]
	\rowcolors{2}{greyROwSWEight}{white}
	\renewcommand{\arraystretch}{1.5}
	%\centering
	\begin{longtable}{C{4cm} c c c}
	
	\rowcolor{greySWEight}
	\textcolor{white}{\textbf{Codice}} &
	\textcolor{white}{\textbf{Nome}} &
	\textcolor{white}{\textbf{Accettabile}} &
	\textcolor{white}{\textbf{Ottimale}}\\
	\endhead

	%%% ENTRY
	\textbf{MD001} &
	Indice Gulpease &
	$[40 , 100] $&
	$[55 , 100]$\\

	%%% ENTRY
	\textbf{MP001} &
	Schedule Variance &
	$]0 , 3] $&
	$(-\infty , 0]$\\
	
	%%% ENTRY
	\textbf{MP002} &
	Budget Variance &
	$]1\% , 9\%] $&
	$(-\infty , 1\%]$\\


	%%% ENTRY
	\textbf{MS001} &
	Numero di metodi &
	$]7 , 9] $&
	$[0 , 7]$\\
	
	%%% ENTRY
	\textbf{MS002} &
	Numero di parametri &
	$]5 , 8] $&
	$[0 , 5]$\\
	
	%%% ENTRY
	\textbf{MS003} &
	Funzioni di interfaccia per package &
	$]10 , 20] $&
	$[0 , 10]$\\
	
	%%% ENTRY
	\textbf{MS004} &
	Complessità ciclomatica &
	$]10 , 17] $&
	$[0 , 10]$\\
	
	%%% ENTRY
	\textbf{MS005} &
	Campi dati per classe &
	$]10 , 15] $&
	$[0 , 10]$\\
	
	%%% ENTRY
	\textbf{MS006} &
	Commenti per linee di codice &
	$[10\%, 15\%[ $&
	$[15\% , 100\%]$\\
	
	%%% ENTRY
	\textbf{MS007} &
	Code coverage &
	$[80\%, 90\%[$&
	$[90\%, 100\%]$\\

	
	%%% ENTRY
	\textbf{MS008} &
	Superamento test &
	$[100\%, 100\%]$&
	$[100\%, 100\%]$\\
	
	%%% ENTRY
	\textbf{MS009} &
	Soddisfacimento requisiti obbligatori &
	$[100\%, 100\%]$&
	$[100\%, 100\%]$\\
	
	%%% ENTRY	
	\textbf{MS010} &
	Media di build Travis settimanali &
	$[15,25[$&
	$[25,\infty)$\\
	
	%%% ENTRY
	\textbf{MS011} &
	Percentuale build Travis superate &
	$[75\%,85\%[$&
	$[85\%,100\%]$\\
	
	\rowcolor{white}
	\caption{Riassunto delle metriche}\\	
	\end{longtable}
\newpage
\section{Gestione amministrativa della revisione}


\appendix
\addcontentsline{toc}{part}{Appendici}
\newpage
\section{Qualità}
	\subsection{SPICE}
		\label{app:SPICE}
		Il modello ISO/IEC 1554, meglio conosciuto come SPICE (Software Process Improvement and Capability Determination), è lo standard di riferimento per valutare in modo oggettivo la qualità dei processi nello sviluppo del software. \newline
Sono definiti:
\begin{itemize}
	\item Una serie di \textbf{attributi} per ogni processo, che ne vanno a determinare la sua capability (capacità):
	\begin{itemize}
	    \item Process performance;
	    \item Performance management;
	    \item Work product management;
	    \item Process definition;
	    \item Process deployment;
	    \item Process measurement;
	    \item Process control;
	    \item Process innovation;
	    \item Process optimization;
	\end{itemize}
	
	Ognuno di questi attributi riceve una valutazione nella seguente scala:
	\begin{itemize}
	    \item \textbf{N: }Non raggiunto (0 - 15\%);
	    \item \textbf{P: }Parzialmente raggiunto (>15\% - 50\%);
	    \item \textbf{L: }Largamente raggiunto (>50\%- 85\%);
	    \item \textbf{F: }Pienamente raggiunto (>85\% - 100\%);
	\end{itemize}
	\item Sei \textbf{livelli di capacità} dei processi:
		\begin{enumerate}[start=0]
			\item Incompleto;
			\item Eseguito;
			\item Gestito;
			\item Stabilito;
			\item Predicibile;
			\item Ottimizzato;
		\end{enumerate}
	\item Delle linee guida per effettuare delle \textbf{stime}, eseguite tramite:
		\begin{itemize}
			\item \textbf{Processi di misurazione}, descritti nel \PdP ;
			\item \textbf{Modello di misurazione}, descritto in questo documento;
			\item \textbf{Strumenti di misurazione}, descritti nelle \NdP ;
		\end{itemize}
	\item Una serie di \textbf{competenze} che chi effettua misurazioni deve possedere. La mancanza di esperienza degli elementi del gruppo \gruppo , fa sì che nessun membro possieda queste skill, rendendo così impossibile la piena adesione allo standard. Tuttavia, ogni componente è chiamato a studiare SPICE e a applicare al meglio le indicazioni descritte in questo documento e nelle \NdP , al fine di perseguire un livello di qualità accettabile.
	
\end{itemize}

	\subsection{ISO/IEC 9126}
		\label{app:ISO/IEC 9126}
		Lo standard ISO/IEC 9126 stabilisce una serie di linee guida mirate al miglioramento delle qualità del software sviluppato.

\begin{figure}[H]
  \includegraphics[width=\linewidth]{sez/App_Qualita/grafico_9126.png}
  \caption{Rappresentazione grafica di ISO/IEC 9126 [Wikipedia]}
  \label{fig:9126}
\end{figure}

Come presentato in \autoref{fig:9126}, ISO/IEC 9126 prescrive indicazioni su:
\begin{itemize}
	\item \textbf{Qualità interna: }è misurata sul software non eseguibile, come, ad esempio il codice sorgente. Le misure effettuate permettono di avere una buona previsione della qualità esterna;
	\item \textbf{Qualità esterna: }è misurata tramite l'analisi dinamica su software eseguibile. Le misure effettuate permettono di avere una buona previsione della qualità in uso prodotto;
	\item \textbf{Qualità in uso: }definita in base all'esperienza utente. Sono da perseguire i seguenti obiettivi:
		\begin{itemize}
			\item Efficacia;
			\item Produttività;
			\item Soddisfazione;
			\item Sicurezza.
		\end{itemize}
\end{itemize}

ISO/IEC 9126 definisce inoltre una serie di requisiti da soddisfare per produrre software di qualità:
\begin{itemize}
	\item \textbf{Funzionalità: }capacità di un prodotto software di soddisfare le esigenze stabilite (vedi \AdR);
	\item \textbf{Affidabilità: }capacità di un prodotto di mantenere un determinato livello di prestazioni in date condizioni d'uso per un certo periodo;
	\item \textbf{Efficienza: }capacità di un prodotto software di eseguire il proprio compito minimizzando il numero di risorse usate;
	\item \textbf{Usabilità: }capacità del prodotto software di essere utilizzato, capito e studiato dall'utente a cui è rivolto;
	\item \textbf{Manutenibilità: }capacità del prodotto software di evolvere mediante modifiche, correzioni e miglioramenti;
	\item \textbf{Portabilità: }capacità del prodotto software di funzionare ed essere installato su più ambienti hardware e software.
\end{itemize}
\newpage
\section{Appendice - Esito Verifica}
	\label{app:misure}

\end{document}