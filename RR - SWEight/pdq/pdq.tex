%generare il pdf con il comando: pdflatex main.tex
\documentclass[a4paper, oneside, openany, dvipsnames, table]{article}
\usepackage{../template/SWEightStyle}
\usepackage{cleveref}

\newcommand{\Titolo}{Verbale Riunione 2018-12-12}

\newcommand{\Gruppo}{SWEight}

\newcommand{\ACapoRedazione}{Francesco Magarotto}

\newcommand{\Verifica}{Francesco Corti}

\newcommand{\Approvazione}{Sebastiano Caccaro}

\newcommand{\Distribuzione}{Vardanega Tullio \newline Cardin Riccardo \newline Gruppo SWEight}

\newcommand{\Uso}{Interno}

\newcommand{\NomeProgetto}{Colletta}

\newcommand{\Mail}{SWEightGroup@gmail.com}

\newcommand{\DescrizioneDoc}{Questo documento si occupa di riportare quanto discusso nella riunione del 12-12-2018}


\begin{document}
\copertina{}
\definecolor{greySWEight}{RGB}{255, 71, 87}
\definecolor{greyROwSWEight}{RGB}{234, 234, 234}

\section*{Registro delle modifiche}
{
	\rowcolors{2}{greyROwSWEight}{white}
	\renewcommand{\arraystretch}{1.5}
	\centering
	\begin{longtable}{ c c  C{4cm}  c  c }
		
		\rowcolor{greySWEight}
		\textcolor{white}{\textbf{Versione}} & \textcolor{white}{\textbf{Data}} & \textcolor{white}{\textbf{Descrizione}} & \textcolor{white}{\textbf{Nominativo}} & \textcolor{white}{\textbf{Ruolo}}\\
		
		1.0.2 & 2019-03-02 & Aggiunti nuovi termini del documento Piano di Progetto & Isachi Gheorghe &\reda{}\\
		
		1.0.1 & 2019-02-23 & Verifica del documento &  Francesco Corti & \ver{}\\
		
		1.0.1 & 2019-02-20 & Aggiunti nuovi termini del documento Norme di Progetto & Isachi Gheorghe &\reda{}\\
		
		1.0.0 & 2019-01-09 & Approvazione & Sebastiano Caccaro & \Res{}\\
						
		0.1.1 & 2019-01-08 & Verifica del documento & Bacco Alberto & \ver{}\\
		
		0.1.1 & 2019-01-04 & Aggiunti termini del documento Norme di Progetto & Isachi Gheorghe &\reda{}\\
		
		0.1.0 & 2019-01-01 & Aggiunti termini del documento Analisi dei Requisiti & Isachi Gheorghe &\reda{}\\
		
		0.0.4 & 2018-12-29 & Verifica del documento & Bacco Alberto & \ver{}\\
				
		0.0.4 & 2018-12-27 & Aggiunti termini del documento Piano di Qualifica & Isachi Gheorghe &\reda{}\\
				
		0.0.3 & 2018-12-26 &Aggiunti termini del documento Piano di Progetto & Isachi Gheorghe & \reda{}\\
				
		0.0.2 & 2018-12-17 & Aggiunti termini del documento Studio di Fattibilità & Isachi Gheorghe &\reda{}\\
		
		0.0.1 & 2018-12-15 & Scheletro del glossario & Damien Ciagola & \reda{}\\
		
	\end{longtable}

}
\newpage
\tableofcontents
\newpage
\section{Introduzione}
	\subsection{Scopo del documento}
		In questo documento è illustrata la {strategia}\ped{G} di {verifica}\ped{G} e {validazione}\ped{G} del gruppo \gruppo . Tale strategia è fondamentale per dare una misurazione oggettiva e quantificabile del livello di {qualità}\ped{G} di quanto viene prodotto. \newline
Ciò è vantaggioso sia per il gruppo \gruppo , che può facilmente individuare difetti durante lo svolgimento del progetto, sia per il {committente}\ped{G}, che può costantemente monitorare la qualità del prodotto in base a criteri oggettivi e prestabiliti.


	\subsection{Scopo del prodotto}
		Lo scopo del progetto è realizzare una piattaforma collaborativa di raccolta dati in cui gli utenti possano predisporre
e/o svolgere piccoli esercizi di grammatica (per esempio esercizi di analisi grammaticale) e i dati raccolti
siano relativi sia agli esercizi predisposti che al loro svolgimento da parte degli utenti. I dati raccolti devono
essere utilizzabili da sviluppatori e ricercatori al fine di insegnare ad un elaboratore a svolgere i medesimi esercizi
mediante tecniche di apprendimento automatico supervisionato.

%Questa parte è un copia-incolla cafonissimo dal capitolato della mivoc
%Siccome è proprio quello che vogliono, non mi sembrava il caso di andare a modifcarla
	\subsection{Glossario}
		Nel documento è possibile incontrare termini tecnici, i quali potrebbero non essere immediatamente chiari al lettore. Per disambiguarne il significato, essi sono stati marcati con una \ped{G} a pedice e la loro definizione è reperibile nel glossario fornito separatamente.
	\subsection{Riferimenti}
		\subsubsection{Riferimenti normativi}
			\begin{itemize}
	\item \textbf{Norme di Progetto:} \NdP, §4.2 Qualità;
	\item \textbf{Capitolato d'appalto C2: } Colletta \newline
		  \url{https://www.math.unipd.it/~tullio/IS-1/2018/Progetto/C2.pdf}.
\end{itemize}
		\subsubsection{Riferimenti informativi}
			\begin{itemize}
	\item \textbf{Piano di Progetto:} \PdP;
	\item \textbf{Slide del corso di Ingegneria del Software:}\newline
		  		 \url{https://www.math.unipd.it/~tullio/IS-1/2018};
	\item \textbf{Ian Sommerville, Software Engineering, Nona edizione:}
		\begin{itemize}
		  	\item Capitolo 24: Quality management;
		  	\item Capitolo 26: Process improvement;
		\end{itemize}
	\item \textbf{Standard ISO/IEC 9126:}\newline
		  		  \url{https://it.wikipedia.org/wiki/ISO/IEC_9126}\newline
				  \url{https://en.wikipedia.org/wiki/ISO/IEC_9126};
	\item \textbf{Standard ISO/IEC 15504:}\newline
				  \url{https://en.wikipedia.org/wiki/ISO/IEC_15504}
	\item \textbf{VARI:};
	\item \textbf{ED:};
	\item \textbf{EVENTUALI:};
	\item \textbf{:};
	\item \textbf{:};
	\item \textbf{:};
	\item \textbf{:};
		
\end{itemize}
	\newpage
\section{Strategie di verifica}
	\subsection{Scopo del documento}
Il presente documento ha lo scopo di fornire agli sviluppatori uno specchietto informativo sul design strutturale e logico della piattaforma Colletta. Il documento sarà inoltre
corredato da diagrammi UML 2.X delle principali scelte prese dal gruppo SWEight e descriverà le tecnologie utilizzate nella realizzazione dell’applicazione.
\subsection{Scopo del prodotto}
Il prodotto da realizzare consta in un’applicazione web che fornisca uno strumento per creare e svolgere esercizi di analisi grammaticale, e al contempo né raccolga i risultati. I dati raccolti verranno impiegati dagli sviluppatori dell’azienda proponente come strumento per il miglioramento di algoritmi di {apprendimento automatico}\ped{G}. Nello specifico il prodotto verrà utilizzato da tre tipologie di utenti:
le/gli insegnanti che si occuperanno della creazione degli esercizi,
gli allievi che potranno svolgere gli esercizi e ottenere delle valutazioni e gli sviluppatori che filtreranno i dati secondo alcuni criteri, e infine li scaricheranno.\\Il prodotto si interfaccerà con un’applicazione di {PoS-tagging}\ped{G}, come {FreeLing}\ped{G}, a cui verrà delegata l’esecuzione dell’analisi grammaticale delle frasi.
\subsection{Glossario}
Al fine di rendere il documento il più comprensibile possibile e permetterne una rapida fruizione, viene allegato il \G{} in cui sono presenti i termini contraddistinti dal pedice G. Tali termini includono abbreviazioni, acronimi, termini di natura tecnica, oppure sono fonte di ambiguità e pertanto necessitano di una definizione che renda il loro significato inequivocabile. 
Ogni termine, solo alla prima occorrenza per documento, verrà contrassegnato con la dicitura sopra indicata e rimanderà alla medesima definizione nel \G{}.
\newpage
	\subsection{Obiettivi di qualità}
		\subsubsection{Qualità di processo}
			\paragraph{MP001 - Schedule Variance}\mbox{}\\[0,3cm]
\begin{table}[H]
    \centering
    \begin{tabular}{cccc}
        \rowcolor{greySWEight}
        \textcolor{white}{\textbf{Attività}} & 
        \textcolor{white}{\textbf{Abbreviazione}} &
        \textcolor{white}{\textbf{Valore Indice}}&
        \textcolor{white}{\textbf{Riscontro}}\\
		\textbf{Stesura Analisi dei Requisiti} & ADR & 1 & \textcolor{YellowOrange}{Accettabile}\\
		\textbf{Stesura Glossario} & GLO & -1 & \textcolor{ForestGreen}{Ottimale} \\
		\textbf{Stesura Piano di Progetto} & PDQ & 0 & \textcolor{ForestGreen}{Ottimale} \\
		\textbf{Stesura Piano di Qualifica} & PDP & 1 & \textcolor{ForestGreen}{Ottimale} \\
		\textbf{Stesura Norme di Progetto} & NDP & 1 & \textcolor{YellowOrange}{Accettabile} \\
		\textbf{Stesura Studio di Fattibilità} & SDF & 0 & \textcolor{YellowOrange}{Accettabile} \\

    \end{tabular}
    \caption{Schedule Variance nel periodo di Progettazione}
\end{table}
\begin{figure}[H]
    \centering
	\includegraphics[width=1\linewidth]{sez/App_Esito/Progettazione/graph/PR_SV.pdf}
	\caption{Schedule Variance nel periodo di Progettazione}
\end{figure}

\paragraph{MP002 - Budget Variance}\mbox{}\\[0,3cm]
\begin{table}[H]
    \centering
    \begin{tabular}{cccc}
        \rowcolor{greySWEight}
        \textcolor{white}{\textbf{Abbreviazione}} &
        \textcolor{white}{\textbf{Valore Indice}}&
        \textcolor{white}{\textbf{Valore in €}}&
        \textcolor{white}{\textbf{Riscontro}}\\
        BV & -2,91\% & -136 & \textcolor{ForestGreen}{Ottimale}\\
    \end{tabular}
    \caption{Budget Variance nel periodo di Progettazione}
\end{table}
\begin{figure}[H]
    \centering
	\includegraphics[height=4cm]{sez/App_Esito/Progettazione/graph/PR_BV.pdf}
	\caption{Budget Variance nel periodo di Progettazione}
\end{figure}
\begin{figure}[H]
	\centering
	\includegraphics[height=6cm]{sez/App_Esito/Progettazione/graph/PR_Storico_BV.pdf}
	\caption{Andamento del Budget Variance fino al periodo di Progettazione}
\end{figure}
		\subsubsection{Qualità di prodotto}
			Per garantire la qualità dei prodotti, viene adottato lo standard ISO/IEC 9126\footnote{ISO/IEC 9126: Vedi appendice \cref{app:ISO/IEC 9126}}. Quest'ultimo permette di monitorare la qualità del software, fornendo delle metriche per misurarla.
	\subsection{Organizzazione????}
		\subsection{Pianficazione strategica temporale?????}
	\subsection{Risorse}
	\subsection{Responsabilità}
	\subsection{Misure e metriche}
		\subsubsection{Metriche processi}
		\subsubsection{Metriche documenti}
		\subsubsection{Metriche software}
\section{Gestione amministrativa della revisione}


\appendix
\addcontentsline{toc}{part}{Appendici}
\section{Qualità}
	\subsection{SPICE}
		\label{app:SPICE}
	\subsection{ISO/IEC 9126}
		\label{app:ISO/IEC 9126}
\section{Appendice - Esito Verifica}

\end{document}